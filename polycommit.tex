\documentclass{article}
\usepackage[operators,lambda,keys,sets,primitives,adversary,asymptotics,advantage]{cryptocode}
\usepackage{notations}
\usepackage{mdframed}
\usepackage{enumitem}
\usepackage{amsmath,amsthm,amssymb}
\usepackage[utf8x]{inputenc}
\usepackage[colorinlistoftodos]{todonotes}
\usepackage{xspace}
\usepackage[normalem]{ulem}
\usepackage{comment}

%Theorems
\theoremstyle{definition}
\newtheorem{definition}{Definition}
\newtheorem{theorem}{Theorem}
\newtheorem{lemma}{Lemma}
\newtheorem{assumption}{Assumption}
\newtheorem{fact}{Fact}
\newtheorem{corollary}{Corollary}

\newif\ifcomments
\commentstrue


\ifcomments
	\newcommand{\benedikt}[1]{{\textcolor{red}{[Benedikt: #1]}}}
	\newcommand{\ben}[1]{{\textcolor{green}{[Ben: #1]}}}
	\newcommand{\alan}[1]{{\todo[color=blue!40!white]{Alan: #1}}}
	\newcommand{\alaninline}[1]{{\todo[color=blue!20!white, inline]{Alan: #1}}}
	\else
	\newcommand{\benedikt}[1]{}
	\newcommand{\ben}[1]{}
	\newcommand{\alan}[1]{}
	\newcommand{\alaninline}[1]{}
	\fi

\begin{document}
\title{Transparent SNARKs from DARK Arguments}
\maketitle

\begin{abstract} 
We construct a new polynomial commitment scheme for multivariate polynomials over a finite field, with public-coin evaluation proofs that have logarithmic communication in the degree of the polynomial. The techniques are reminiscent of \emph{Diophantine Arguments of Knowledge} (Lipmaa, Asiacrypt'03), leveraging integer representations of polynomials and groups of unknown order. Security is shown from falsifiable assumptions that hold in generic groups. Moreover, the scheme does not require a trusted setup if instantiated with class groups. We apply this new cryptographic compiler to algebraic linear IOPs in order to obtain doubly-efficient public-coin IPs with succinct communication and witness-extended emulation for any NP relation. Allowing for linear preprocessing, the online verifier's work is logarithmic in the circuit complexity of the relation.

Concretely, compiling a QAP-based IOP results in quadratic prover time, but we obtain quasi-linear prover time when compiling instead the IOP employed in Sonic (Maller et. al., CCS 2019) based on bivariate Laurent polynomials. Applying the Fiat-Shamir transform in the random oracle model results in a transparent preprocessing zk-SNARK system with quasi-linear prover time, logarithmic proof size, and logarithmic verification time for arbitrary circuits, which we call \textsf{\textbf{Supersonic}}.\alan{If we decide to use \textsf{PLONK} instead ... SuperPLONK?}

\end{abstract} 

\section*{Notes for Writing}
\begin{table}
    \caption{Notation}
    \label{tab:notation}
    \centering
    \begin{tabular}{l|l}
        symbol & meaning   \\ \hline \hline
        {\bf polynomials} & \\ \hline
        $f \in \mathbb{F}_p[x]$ or $f \in \mathbb{Z}[x]$ & polynomial modulo $p$ or without modulo, depending on context \\
        $f_L, f_R \in \mathbb{F}_p[x]$ & first and second half of $f(x)$ \\
        $f_0, f_1, f_i \in \mathbb{F}_p$ & coefficients \emph{s.t.} $f(x) = \sum_{i=0}^d f_ix^i$ \\ \hline
        {\bf group elements} &  \\ \hline
        $\mathsf{g} \in \mathbb{G}$ & designated base element (the term ``generator'' is misleading) \\
        $\mathsf{c}, \mathsf{c}_0, \mathsf{c}_1 \in \mathbb{G}$ & commitments \\
        $\mathsf{c}_0^a \times \mathsf{c}_1^b, \mathsf{c}_0^a\mathsf{c}_1^b \, \textnormal{for} \, a, b \in \mathbb{Z}$ & multiplicative notation \\ 
    \end{tabular}
\end{table}

\section{Introduction}

A polynomial commitment scheme enables a prover to bind himself to a polynomial in much less bandwidth than transmitting all coefficients would require. A skeptical verifier can subsequently test the commitment for certain algebraic relations as though he were in possession of the polynomial's full description, except at a much smaller work cost. Indeed, polynomial commitments lie at the heart of a host of efficiently verifiable interactive proof systems.

Of particular interest to this paper are proof systems whereby the prover establishes the correct performance of an arbitrary computation (that may or may not involve secret information) in such a way that the communication or verification complexity scales asymptotically better than performing the computation naïvely. Without exception, these proof systems rely on a technique called \emph{arithmetization}: characterizing the computation in question as a collection of arithmetic operations over a finite field. The utility of polynomial commitments stems from their capacity to succinctly capture a canonical representation of such collections while retaining the algebraic properties that make arithmetization work in the first place.

The literature on proof systems for arbitrary computations focuses on two techniques to achieve polynomial commitments. First: Merkle trees --- here every leaf represents the polynomial's evaluation in a given point, and the Merkle root represents the commitment to the polynomial. The verifier needs to verify the authentication paths of selected points, which can be done in logarithmic space and time (as a function of the number of points). Second: groups equipped with bilinear maps --- in this case a structured reference string (SRS)\footnote{Previously known as \emph{common reference string}, CRS.} provides the values of all monomials up to a given degree when evaluated in an unknown point. By computing a weighted sum of these monomial values, the prover obtains the evaluation of his polynomial in the unknown point. The verifier performs the pairing operation to verify that multiplicative relations hold between committed polynomials.

This paper provides a third option for generating polynomial commitment schemes, namely by relying on groups of unknown order --- such as the group of integers with multiplication modulo an RSA modulus of unknown factorization, or the ideal class group of an order of an imaginary quadratic number field. These groups have seen relatively little adoption or even attention from the cryptographic community because the only known constructions thereof have subexponential attack algorithms. As a result, for a practical security level, elements of groups of unknown order typically require several hundreds of bytes to represent, in contrast to the tens of bytes needed for elements of elliptic curves for which no subexponential algorithms exist. 

Nevertheless, groups of unknown order provide a property that groups of known order, such as elliptic curve groups, cannot match: they enable homomorphic  commitments to an \emph{infinite} domain, namely the integers. Indeed, if the prover were capable of reducing a large integer to a smaller one without sacrificing the homomorphic properties, then he must know the group's order. The power of integer commitments was already noted by Lipmaa~\cite{} who characterizes proof systems arising therefrom as \emph{Diophantine} --- a reference to the family of languages for which such proof systems establish. Specifically, a set $S \subset \mathbb{Z}^n$ is called \emph{Diophantine} if it is the projection onto the first $n \leq m$ coordinates of the set of roots to a polynomial $P(x_1, \ldots, x_m) \in \mathbb{Z}[x_1, \ldots, x_m]$. Much more recently, Wesolowski produced a conceptually simple verifiable delay function (VDF) which builds on a proof of correct exponentiation in groups of unknown order. Building on this result, Boneh \emph{et al.} developed accumulators and vector commitments (with batch openings) from groups of unknown order~\cite{}. Both results have been received with great enthusiasm in the cryptocurrency community for their capacity to improve sustainability and scalability of blockchains.

\alaninline{Todo: \\
 - applications (trustless snarks etc) \\
 - implications (no unfalsifiable assumptions) \\
 - overview of techniques \\
 - related work}

\vspace{0.25cm}
\textsc{Contributions.} The contributions of this paper are divisible into three rubrics:
\begin{itemize}
    \item[] \textbf{Tools.} We start with an encoding scheme that represents polynomials over a prime field $\mathbb{F}_p$ as integers, by encoding the polynomial's coefficients into the integer's base-$q$ expansion. Adjoined with a group of unknown order and a designated base element $g \in \mathbb{G}$, this encoding scheme naturally gives rise to a polynomial commitment scheme that inherits its somewhat homomorphic properties. Next, we provide protocols for proving the correct evaluation of a committed polynomial, and showing that two polynomials have the same coefficients but flipped or rotated. We also present a protocol for extracting the $i$th coefficient, thereby promoting the commitment scheme to one that also provides vector commitment functionality. Building on this observation, we provide another protocol for showing that a commitment represents the inner product between two vectors of which at least one is represented by its vector commitment. Another protocol establishes that two vector commitments represent the same vector up to an arbitrary but known permutation of the coefficients.
    \item[] All the proof systems mentioned so far have logarithmic communication complexity and logarithmic verification time. Moreover, with the exception of the inner product proof and the permutation proof, the prover's complexity is quasi-linear. If one is willing to sacrifice this scalability for the prover, we also provide counterparts to all the above proofs with constant communication and verification complexity.
    \item[] \textbf{Applications.} To illustrate the usefulness and the versatility of the enumerated tools, we join them straightforwardly to construct a simple succinct non-interactive argument of knowledge (SNARK) based on quadratic arithmetic programs (QAPs). To the best of our knowledge, this is the first SNARK for circuits without trusted setup (when instantiated with the class group) or with an SRS whose size is independent of the circuit (when instantiated with the RSA group).\footnote{This classification takes note of the STARK proof system of Ben Sasson \emph{et al.}~\cite{C:BBHR19} whose verification time is polylogarithmic but as a function of the \emph{running time} of some program and not of any circuit; as well as of Hyrax~\cite{SP:WTTW17} and Spartan~\cite{eprint:Setty19}, which do apply to circuits but whose verification times are not polylogarithmic and thus fail to satisfy the definition of SNARKs as set forth in the paper that coined the term~\cite{JC:BCCGLRT17}.}
    \item[] \alan{deprecated} We follow up this conceptually simple QAP-based proof system with a survey of popular communication-efficient proof systems for arbitrary computations, in which we replace their constituent components with tools developed earlier in this paper. In this light we analyze Sonic, Spartan, Hyrax, Bulletproofs, and STARK. In all cases we find that using our techniques leads to different trade-offs; improving on some metrics while degrading others.
\end{itemize}

\subsection{Related Work}

Homomorphic integer commitment schemes based on the RSA group were first proposed by Fujisaki and Okamoto~\cite{C:FujOka97}, who also provide a protocol to prove that a list of committed integers satisfy a polynomial equation modulo an arbitrary positive integer as well as one for opening a commitment bit by bit. Damgård and Fujisaki~\cite{AC:DamFuj02} fix an issue with the soundness proof of that protocol and are the first to suggest using class groups of an imaginary quadratic order as a candidate group of unknown order. Around the same time, Lipmaa draws the link between zero-knowledge proofs constructed from integer commitment schemes and Diophantine complexity~\cite{AC:Lipmaa03b}. Much later, Couteau~\emph{et al.} study protocols derived from integer commitments specifically in the RSA group in order to lift their security proofs so as to require weaker assumptions; in the process they develop proofs for polynomial evaluation modulo a prime $\pi$ that is not initially known to the verifier, in addition to a proof showing that an integer $x$ lies in the range $\{a, \ldots, b\}$ by showing that $1+4(x-a)(b-x)$ decomposes as the sum of 3 squares~\cite{EC:CouPetPoi17}.

Spurred by the recent demand for cryptographic tools from the blockchain industry, Pietrzak~\cite{Pietrzak18} adds efficient verifiability to the RSA-based time lock puzzle due to Rivest, Shamir, and Wagner~\cite{RivShaWag96}, thereby obtaining a conceptually simple verifiable delay function (VDF). Wesolowski~\cite{EC:Wesolowski19} improves on this result by proposing a single-round protocol to prove correct exponentiation in groups of unknown order, down from a logarithmic (in the size of the exponent) number for Pietrzak's protocol. Boneh~\emph{et al.} generalize this protocol to arbitrary exponents (not just powers of 2), and adapt it for zero-knowledge and batching, providing the base tools for constructing efficient accumulators and vector commitment schemes~\cite{C:BonBunFis19}.

\section{Preliminaries}
\paragraph{Notation}
\begin{itemize}
\item Let $f(x) \in \mathbb{F}_p[x]$ be a polynomial of degree at most $N-1$ where $N$ is a power of two. The coefficients of $f(x)$ are denoted by $f_i$ such that $f(x) \stackrel{\triangle}{=} \sum_{i=0}^{N-1} f_i x^i$.
\item Throughout, $p$ is a prime of at least $\lambda$ bits, and $q \in \mathbb{N}$ be an integer with $q \gg p$.  $\primes$ denotes the set of primes less than $2^{\lambda \log(\lambda)}$. There are at least $2^\lambda$ primes in that set.
\item We work in a group $\mathbb{G}$ of unknown order together with one or many designated base elements $\gr{g}, \gr{h} \in \mathbb{G}$ with unknown order. 
%(It might be tempting refer to these elements as \emph{generators} but that terminology would imply that $\mathbb{G}$ is cyclic, which is not necessarily true. \ben{Every element is a generator of a cyclic subgroup; why make this comment at all?}\textcolor{blue}{Alan: true; moot point.})
We use multiplicative notation and use \textsf{sans-serif} font to indicate group elements, as opposed to integers, polynomials, or field elements.
\item Some protocols in this paper are between two parties, the prover and the verifier. We write $\boldsymbol{\it Protocol}(x;w) \rightarrow (y;z)$ to describe such a protocol with common input $x$, private input for the prover $w$, public output $y$, and private output for the prover $z$. We write $(y;z) \gets \boldsymbol{\it Protocol}(x;w)$ to denote the protocol's execution. Any of $x,w,y,z$ can consist of tuples of objects. This notation facilitates protocol composition and modular analysis.

\item We use the notation $\negl$ to denote negligible function $\textsf{negl}: \mathbb{N} \rightarrow [0,1]$ applied to the security parameter $\lambda$. 
\end{itemize}

\subsection{Assumptions}

The cryptographic compilers make extensive use of groups of unknown order, \emph{i.e.}, groups for which the order cannot be computed efficiently.
Concretely, we require groups for which two specific hardness assumptions hold.
First the strong RSA assumption\cite{CCS:CraSho99} which roughly states that it is hard to take \emph{arbitrary} roots of \emph{random} elements. Secondly, the much newer adaptive root assumption\cite{EC:Wesolowski19} which is the dual of the strong RSA assumption and states that it is hard to take \emph{random} roots of \emph{arbitrary} group elements. 
Both of these assumption hold in generic groups of unknown order\cite{genericunknown,C:BonBunFis19}, i.e. there are no efficient algorithms that only have black-box access to the group but are able to break these assumptions. 
It is an open research problem to show whether one of these assumption implies the other.


\begin{assumption}[Strong RSA Assumption]
The \defn{strong RSA Assumption} sates that no efficient adversary can compute any root of a given random group element. Specifically, it holds for $\ggen$ if for any probabilistic polynomial time adversary $\adv$:
\[
    \Pr\left[\gr{u}^\ell = \gr{g} \, \wedge \, \ell > 2:
    \begin{array}{l}
         \GG \leftarrow \ggen(\lambda)  \\
         \gr{g} \sample \GG \\
         (\gr{u}, \ell) \in \mathbb{G} \times \mathbb{N} \leftarrow \adv(\mathbb{G}, \gr{g}) \\
    \end{array}\right] \leq \negl \enspace .
\]
\end{assumption}
We note that some definitions of the strong RSA assumption additionally require that $\ell$ be a prime~\cite{EC:BarPfi97,journals/iacr/BonehBF18a}. We follow that of Cramer and Shoup~\cite{CCS:CraSho99}.

\begin{assumption}[Adaptive Root Assumption]
\label{assum:adaptiveroot}
We say that the \defn{adaptive root assumption} holds for $\ggen$ if 
there is no efficient adversary $(\adv_0,\adv_1)$ that succeeds 
in the following task.
First, $\adv_0$ outputs an element $\gr{w} \in \GG$ and some $\state$.
Then, a random prime $\ell$ in $\primes$ is chosen
and $\adv_1(\ell,\state)$ outputs $\gr{w}^{1/\ell} \in \GG$.
More precisely, for all efficient $(\adv_0,\adv_1)$:
\[           \advantage{AR}{(\adv_0,\adv_1)}\deq 
                \Pr\left[\gr{u}^\ell = \gr{w} \neq 1 \ : \ 
                \begin{array}{l}
                      \GG \sample \ggen(\lambda) \\ 
                      (\gr{w},\state) \sample \adv_0(\GG) \\
                      \ell \sample \primes \\ 
                      \gr{u} \gets \adv_1(\ell, \state)
                \end{array} 
        \right] \leq \negl.
\]
\end{assumption}

We additionally use two more assumptions, however both of them reduce to the Strong RSA and the Adaptive Root assumption.

The first assumption states that computing the order for \emph{any} element is hard. It reduces to the adaptive root assumption. Interestingly, it doesn't necessarily hold for all candidate groups of unknown order. In $\ZZ_n$ for composite $n$, the element $-1\in \ZZ_n$ has known order $2$. For other candidate groups such as class groups with fundamental discriminants or $\ZZ_n/\{-1,1\}$ we have no efficient algorithms for computing elements of known order. We note that if $n=p\cdot q$ for strong primes $p$ and $q$ and we operate in the group of quadratic residues of $\ZZ_n$ then the order assumption reduces to the strong RSA assumption.\benedikt{Suitable citation}
\begin{assumption}[Order assumption]
\label{assum:order}
	The order assumption holds for $\ggen$ if for any efficient adversary $\adv$:
\[        
                \Pr\left[\gr{w}\neq 1 \wedge \gr{w}^{\alpha}= 1: 
                \begin{array}{l} 
                      \GG \sample \ggen(\lambda) \\ 
                      (\gr{w},\alpha) \sample \adv(\GG) \\
                      \text{where } |\alpha|<2^{\poly{}}\in \ZZ\\
                      \text{and } \gr{w}\in \GG
                \end{array} 
        \right] \leq \negl \enspace .
\]
\end{assumption}
\begin{lemma}
\label{lem:ordertoadaptive}
	The adaptive root assumption implies the order assumption.
\end{lemma}
\begin{proof}
	We show that given an adversary $\adv_{\textsf{Order}}$ that breaks the order assumption we can construct with overwhelming probability $\adv_{\textsf{Adaptive Root}}$ that breaks the adaptive root assumption. We run $\adv_{\textsf{Order}}$ to get a $\gr{w}\neq 1\in \GG$ and $\alpha \in \ZZ$ such that $\gr{w}^{\alpha}=1$. To construct $\adv_{\textsf{Adaptive Root}}$, $\adv_{\textsf{Adaptive Root},0}$ outputs $(\gr{w},\alpha)$. The challenger generates a random challenge $\ell$. If $\gcd(\ell,\alpha)=1$ then $\adv_{\textsf{Adaptive Root},1}$ can compute $\beta\gets \ell^{-1} \bmod \alpha$ and output $\gr{u}\gets\gr{w}^{\beta}$. By construction $\gr{u}^{\ell}=\gr{w}$. The probability that $\gcd(\ell,\alpha)=1$ is overwhelming because $\gcd(\ell,\alpha)\neq 1 \implies \ell \not\vert \alpha$. This happens with negligible probability as $\ell$ is picked from a set of $2^\lambda$ primes and at most $\poly$ distinct primes can divide $\alpha$.
	\end{proof}
	
	
We additionally define the pseudo root assumption which is a generalization of the Strong RSA assumption. It reduces to the order assumption (and therefore to the adaptive root assumption) and the strong RSA assumption. Shoup\benedikt{Cite shoup strong rsa paper} showed that for the unknown order group of quadratic residues in $\ZZ_n$, where $n$ is the composite of two strong primes, that the pseudo root assumption reduces to just the strong RSA assumption.
\begin{assumption}[Pseudo root assumption]
\label{assum:fracroot}
The pseudo root assumption holds for $\ggen$ if for any efficient adversary $\adv$:
\[        
                \Pr\left[\gr{u}^\beta = \gr{g}^{\alpha} \wedge \vert\frac{\beta}{\gcd(\alpha,\beta)}|>2   : 
                \begin{array}{l} 
                      \GG \sample \ggen(\lambda) \\ 
                      \gr{g} \sample \GG \\
                      (\alpha, \beta, y) \sample \adv(\GG, \gr{g}) \\
                      \quad \textnormal{where} \, |\alpha|<2^{\poly}, |\beta|<2^{\poly} \in \ZZ \\
                      \quad \textnormal{and} \, \gr{u} \in \GG 
                \end{array} 
        \right] \leq \negl \enspace .
\]
\end{assumption}
\begin{lemma}
	The adaptive root assumption and the strong RSA assumption imply the pseudo root assumption.
\end{lemma}
\begin{proof}
	Given an adversary $\adv_{\textsf{PseudoRoot}}$ that succeeds for $\ggen$ we can construct either an adversary $\adv_{RSA}$ for the strong RSA assumption or an adversary $\adv_{\textsf{Order}}$ that breaks the order assumption for $\ggen$. As shown in Lemma \ref{lem:ordertoadaptive} the order assumption reduces to the adaptive root assumption with overwhelming probability. 
	We first generate a group of unknown order $\GG \sample \ggen(\lambda)$.
	Then we generate $\gr{g}\sample \GG$ as done in the \textsf{RSA} security definition.
	
	We now run the $\adv_{\textsf{PseudoRoot}}$ on input $\GG$ and $\gr{g}$ to generate a tuple $(\alpha,\beta,\gr{u})$ such that $\gr{u}^{\beta}=\gr{g}^{\alpha}$. Let $\gamma=\gcd(\alpha,\beta)$ and $\alpha'=\frac{\alpha}{\gamma}\in \ZZ$ and  $\beta'=\frac{\beta}{\gamma}$. Now either $\gr{g}^{\alpha'}=\gr{u}^{\beta'}$ or $\gr{g}^{\alpha'}/\gr{u}^{\beta'}$ is a non trivial element of order $\gamma$ which would directly break the order assumption. In that case we constructed $\adv_{\textsf{Order}}$ that outputs $(\gr{g}^{\alpha'}/\gr{u}^{\beta'},\gamma)$.
	
	Now assume otherwise, i.e. $\gr{g}^{\alpha'}=\gr{u}^{\beta'}$. By construction $\gcd(\alpha',\beta')=1$ and we can efficiently compute integers $a,b$ such that $a \alpha'+b \beta'=1$. By assumption on $\adv_{\textsf{PseudoRoot}}$ $\beta'$ is not $1$. Now let $\gr{w}\gets \gr{u}^{a}\gr{g}^{b}$. Note that $\gr{w}^{\alpha'\beta'}=\gr{g}^{\alpha'}$. So either $\gr{w}^{\beta'}=\gr{g}$ or $\gr{w}^{\beta'}/\gr{g}$ is a non-trivial element of order $\alpha'$. The first case breaks the strong RSA assumption, as we can construct $\adv_{\textsf{RSA}}$ that outputs $(\gr{w},\beta)$, and the second breaks the order assumption.
\end{proof}


\benedikt{Talk about specific groups of unknown order}

\subsection{Interactive Arguments of Knowledge}
Interactive arguments are \emph{interactive proofs}~\cite{GolMicRac89} in which security holds only against a computationally bounded prover. In an interactive argument for a relation $\mathcal{R}$, the prover convinces the verifier that it ``knows" a witness $w$ for a statement $x$ such that $(x, w) \in \mathcal{R}$. The standard definition of \emph{proofs of knowledge} (PoK) by Bellare and Goldreich~\cite{C:BelGol92} is based on the existence of an extractor machine $E$ that has oracle access to a malicious prover $P^*$, and if $P^*$ would cause the verifier to accept on input $x$ with high probability then $E$ outputs $w$ such that $(x, w) \in \mathcal{R}$ (with overwhelming probability). $E$ runs in expected polynomial time. This definition quantifies the success of $E$ over all inputs $x$, which unfortunately is problematic in the case of interactive  \emph{arguments},.

To illustrate one issue, if the interactive argument relies on a \emph{common reference string} (CRS) setup with secret trapdoor information (e.g. the factorization of an RSA modulus) then one of the inputs $x^*$ could leak the trapdoor to the prover. Any extractor should clearly fail on input $x^*$ while $P^*$ may succeed, hence the definition cannot be satisfied. This particular problem is fixed by requiring the adversary $P^*$ to generate the input $x$. If the trapdoor is exponentially hard to compute the polynomial time adversary will not be able to embed the trapdoor in $x$ except with negligible probability. (See Damg\r{a}rd and Fujisaki~\cite{AC:DamFuj02} for a broader discussion of the issues that arise when applying the standard PoK definition to interactive arguments).

\emph{Witness-extended emulation}~\cite{EC:Lindell03} strengthens the PoK notion so that the extractor outputs not only a witness but also a simulated transcript of the messages between the prover and verifier. This property is helpful for security analysis when a PoK is used as a subprotocol within a larger protocol (e.g. a PoK of a commitment opening within a NIZK for arbitrary circuits), in particularly in order to construct a simulator that needs to both obtain the adverssary's witness as well as simulate its view in the sub-protocol. Groth and Ishai~\cite{EC:GroIsh08} adapt Lindell's definition for interactive arguments of knowledge (AoK) in the CRS model. This is the AoK definition we will use in the present work.

\begin{definition} [Interactive Argument]
Let $(P, V)$ denote a pair of PPT interactive algorithms and $\textsf{Setup}$ denote a non-interactive setup algorithm that outputs public parameters $\params$ given a security parameter. Both $P$ and $V$ have access to $\params$. Let $\langle P(pp, x, w), V(pp, x) \rangle$ denote the output of $V$ on input $x$ after its interaction with $P$, who has witness $w$. The triple $(\textsf{Setup}, P, V)$ is called an argument for relation $\mathcal{R}$ if for all non-uniform PPT adversaries $\mathcal{A}$ the following properties hold: 

\begin{itemize}
\item \underline{Completeness}. 
\[
\Pr \left[
\begin{array}{c}
        (x, w) \not \in  \mathcal{R} \ \text{or} \\
         \ \langle P(\params, x, w), V(\params, x) \rangle = 1 \\
\end{array}
:
\begin{array}{c}
             \params \leftarrow \textsf{Setup}(1^\lambda) \\
             (x, w) \leftarrow \mathcal{A}(\params) \\
\end{array} 
\right]  \geq 1 - \negl 
 \]

\item \underline{Computational soundness}. 
\[
\Pr \left[
\begin{array}{c}
        \forall w \ (x, w) \not\in  \mathcal{R} \ \text{and} \\ 
         \langle \mathcal{A}(\params, x, \st), V(\params, x) \rangle = 1 
\end{array}
:
\begin{array}{c}
             \params \leftarrow \textsf{Setup}(1^\lambda) \\
             (x, \st) \leftarrow \mathcal{A}(\params) \\
\end{array}
        \right] \leq \negl
\]
\end{itemize} 
\end{definition} 

The interactive argument is called \textbf{public-coin} if all the verifier's messages are uniformly random values, independent of all prior messages and the setup parameters $\params$. 
We next recall the definition of witness-extended emulation for interactive arguments, which is a form of knowledge extraction.  

\begin{definition}[Witness-extended emulation~\cite{EC:GroIsh08}]\label{def:wee}
Given a public-coin interactive argument tuple $(\textsf{Setup}, P, V)$ and arbitrary prover algorithm $P^*$, let $\textsf{Record}(P^*, \params, x, \st)$ denote the message transcript between $P^*$ and $V$ on shared input $x$, initial prover state $\st$, and $\params$ generated by $\textsf{Setup}$. Furthermore, let $E^{\textsf{Record}(P^*, \params, x, \st)}$ denote an machine $E$ with a transcript oracle for this interaction that can be rewound to any round and run again on fresh verifier randomness. The tuple $(\textsf{Setup}, P, V)$ has witness-extended emulation if for every deterministic polynomial time $P^*$ there exists an expected polynomial time emulator $E$ such that for all non-uniform polynomial time adversaries $\mathcal{A}$ the following condition holds: 
\[
\Pr \left[
\mathcal{A}(\textsf{tr}) = 1
:
\begin{array}{c}
             \params \leftarrow \textsf{Setup}(1^\lambda) \\
             (x, \st) \leftarrow \mathcal{A}(\params) \\
             \textsf{tr} \leftarrow \textsf{Record}(P^*, \params, x, \st)
\end{array} 
\right] \approx
\]
\[
\Pr \left[
\begin{array}{c} 
\mathcal{A}(\textsf{tr}) = 1 \ \text{and} \\ 
\text{\textsf{tr} accepting} \Rightarrow \ (x, w) \in \mathcal{R}
\end{array} 
:
\begin{array}{c}
             \params \leftarrow \textsf{Setup}(1^\lambda) \\
             (x, \st) \leftarrow \mathcal{A}(\params) \\
(\textsf{tr}, w) \leftarrow E^{\textsf{Record}(P^*, \params, x, \st)}(\params, x)
\end{array}
\right]
\]

\end{definition}

%\alaninline{About the above definition: after reading [GI08] I think the adversary outputs a claim $x$ and a \emph{state} $\st$ (which may or may not be a witness). Otherwise you aren't considering adversaries that try to prove a claim without knowing a witness. I marked my proposed changes in blue because I'm not sure.}

\subsection{Commitment Schemes}

In defining the syntax of the various types of commitment schemes, we use the following convention with respect to public values (known to both the prover and the verifier) and secret ones (known only to the prover). In any list of arguments or returned tuple $(a, b, c; d, e)$ those variables listed before the semicolon are public, and those variables listed after it are secret. When there is no secret information, the semicolon is omitted.

\begin{definition}[Commitment scheme]
A (non-interactive) commitment scheme $\Gamma$ is a tuple $\Gamma = (\pro{Setup}, \pro{Commit}, \pro{Open})$ of PPT algorithms where:
\begin{itemize}
    \item $\pro{Setup}(1^\lambda) \rightarrow \params$ generates public parameters $\params$;
    \item $\pro{Commit}(\params; x) \rightarrow (c; r)$ takes a secret message $x$ and outputs a public commitment $c$ and (optionally) a secret opening hint $r$ (e.g. the randomness used in the computation).
    \item $\pro{Open}(\params, c, x, r) \rightarrow b \in \{0, 1\}$ verifies the opening of commitment $c$ to the message $x$ provided with the opening hint $r$. 
\end{itemize}

A commitment scheme $\Gamma$ is \defn{binding} if for all PPT adversaries $\adv$:
\[
    \Pr\left[
        b_0 = b_1 \neq 0 \, \wedge \, x_0 \neq x_1 \ : \
        \begin{array}{l}
             \params \gets \pro{Setup}(1^\lambda) \\
             (c, x_0, x_1, r_0, r_1) \gets \adv(\params) \\
             b_0 \gets \pro{Open}(\params, c, x_0, r_0) \\
             b_1 \gets \pro{Open}(\params, c, x_1, r_1) \\
        \end{array}
    \right] \leq \negl \enspace 
\]

%\ben{We don't use the hiding property so why present?} \alan{For the ZK Eval protocol, maybe. Not sure yet.}

A commitment scheme $\Gamma$ is \defn{hiding} if for all probabilistic polynomial time adversaries $\adv=(\adv_0,\adv_1)$,
\[
    \left|
        1 - 2\Pr\left[
            \hat{b} = b \ : \
        \begin{array}{l}
             \params \gets \pro{Setup}(1^\lambda) \\
             (\state, x_0, x_1) \gets \adv_0(\params) \\
             b \sample \{0,1\} \\
             (\gr{c}; *) \gets \pro{Commit}(\params; x_b) \\
             \hat{b} \gets \adv_1(\state, \gr{c})
        \end{array}
        \right]
    \right| \leq \negl \enspace .
\]
\end{definition}

We now extend the syntax to polynomial commitment schemes. The following definition generalizes that of Kate~\emph{et al.}~\cite{AC:KatZavGol10} to allow interactive evaluation proofs. It also stipulates that the polynomial's degree be an argument to the protocol, contrary to Kate~\emph{et al.} where the degree is known and fixed.

\begin{definition} (Polynomial commitment) 
A polynomial commitment scheme is a tuple of protocols $\Gamma = (\pro{Setup}, \pro{Commit}, \pro{Open}, \pro{Eval})$ where $(\pro{Setup},$ $\pro{Commit}, \pro{Open})$ is a binding commitment scheme for a message space $R[X]$ of polynomials over some ring $R$, and where
\begin{itemize}
    \item $\pro{Eval}(\params, c, z, y, d; f(X)) \rightarrow b \in \{0, 1\}$ is an interactive public-coin protocol between a PPT prover $P$ and verifier $V$. Both $P$ and $V$ have as input a commitment $c$, points $z, y \in R$ for the claimed input/output, and an integer $d$ for the degree. The prover additionally knows the opening of $c$ to a secret polynomial $f(X) \in R[X]$ with $\deg(f(X)) \leq d$. The protocol convinces the verifier that $f(z) = y$.
\end{itemize}

A polynomial commitment scheme is \defn{correct} if an honest committer can successfully convince the verifier of any evaluation. Specifically, if the prover is honest then for all polynomials $f(X) \in R[X]$ and all points $z \in R$,
\[
    \Pr\left[b = 1 \ : \ \begin{array}{l}
        \params \gets \setup(1^\lambda) \\
        (c; r) \gets \pro{Commit}(\params, f(X)) \\
        y \gets f(z) \\
        d \gets \deg(f(X)) \\
        b \gets \pro{Eval}(\params, c, z, y, d; f(X), r) \\
    \end{array} \right] = 1 \enspace .
\]

A polynomial commitment scheme is \defn{evaluation binding} if no efficient adversary can convince the verifier that the committed polynomial $f(X)$ evaluates to different values $y_0 \neq y_1 \in R$ in the same point $z \in R$. Let $b \gets \pro{Eval}_{\langle \adv_1 \leftrightarrow \mathsf{V}\rangle}(c, z, y, d, \st)$ denote the verifier's output in an execution of this protocol with adversarial prover $\adv_1$ on public inputs $c, z, y, d$ and private adversary state $\st$. (The adversary may or may not know a witness polynomial $f(X)$). Evaluation binding requires that for all probabilistic polynomial-time adversaries $\adv = (\adv_0, \adv_1)$,
\[
    \Pr\left[
         b_0 = b_1 \neq \bot \, \wedge \, y_0 \neq y_1 \ 
         : \
       \begin{array}{l}
            \params \gets \pro{Setup}(1^\lambda) \\
            (c, z, y_0, y_1, d_0, d_1, \st_0, \st_1) \gets \adv_0(\params) \\
            b_0 \gets \pro{Eval}_{\langle \adv_1 \leftrightarrow \mathsf{V}} \rangle(\params, c, z, y_0, d_0; \st_0) \\
            b_1 \gets \pro{Eval}_{\langle \adv_1 \leftrightarrow \mathsf{V}} \rangle(\params, c, z, y_1, d_1; \st_1) \\
        \end{array}
    \right] \leq \negl \enspace .
\]
\end{definition}

The syntax generalizes naturally to multivariate polynomial commitment schemes. Specifically, one obtains the syntax for an $m$-variate polynomial commitment scheme by replacing all occurrences of $x$ and $z$ by their $m$-dimensional vector counterparts, $\mathbf{x}$ and $\mathbf{z}$.


%\alaninline{Kate \emph{et al.} define a polynomial commitment scheme as a tuple of 6 algorithms $(\mathsf{Setup},\mathsf{Commit},\mathsf{Open},\mathsf{VerifyPoly},\mathsf{CreateWitness},\mathsf{VerifyEval})$. There is a difference between verifying a commitment and opening it. The proof-of-correct-evaluation is a ``witness''. Proving and verifying are given by non-interactive algorithms.}

%\alaninline{Sonic defines a polynomial commitment scheme as a tuple $(\mathsf{Commit}, \mathsf{Open}, \mathsf{pvC})$. The generation of the public parameters is not part of the polynomial commitment scheme. There is no special decommitment information; the $\mathsf{Commit}$ function takes a polynomial $f(x)$ and $\mathsf{Open}$ takes \emph{the same} polynomial. The algorithm $\mathsf{Open}$ outputs the evaluation $f(z)$ along with a proof; this proof is verified by $\mathsf{pcV}$.}


\paragraph{Knowledge of coefficients} In our application of polynomial commitments to constructing arguments of knowledge we also require the polynomial commitment to satisfy a \emph{knowledge} property. Informally, we require that any successful prover in the $\eval$ protocol must know a polynomial $f(X)$ such that $f(z) = y$ and $c$ is a commitment to $f(X)$. More formally, since $\eval$ is a public-coin interactive argument we define this knowledge property as a special case of witness-extended emulation (Definition~\ref{def:wee}). 

Define the following NP relation given $\params \leftarrow \pro{Setup}(1^\lambda)$: 
\[ 
\mathcal{R_\textsf{PolyCom}}(\params) = \left\{
\langle (c, z, y, d), (f(X), r) \rangle
: 
\begin{array}{l} 
f \in R[X] \ \text{is degree} \ d \ \text{and} \ f(z) = y \\ 
 \text{and} \ \pro{Open}(\params, c, f(X), r) = 1 \\
\end{array}
\right\}
\] 

The correctness definition above implies that if $\Gamma = (\pro{Setup}, \pro{Commit}, \pro{Open}, \pro{Eval})$ is \emph{correct} then $\eval$ is a correct interactive argument for $\mathcal{R_\textsf{PolyCom}}(\params)$, with overwhelming probability over the randomness of $\pro{Setup}$. We say that $\Gamma$ has \textbf{witness-extended emulation} if $\eval$ has witness-extended emulation as an interactive argument for $\mathcal{R_\textsf{PolyCom}}(\params)$. 

It is easy to see that witness-extended emulation implies evaluation binding when $(\pro{Setup}, \pro{Commit}, \pro{Open})$ is a binding commitment scheme. If the adversary succeeds in $\eval$ on both $(c, z, y_0, d_0)$ and $(c, z, y_1, d_1)$ for $y_0 \neq y_1$ or $d_0 \neq d_1$ then the emulator obtains two distinct witnesses $f(X) \neq f'(X)$ and such that $c$ is a valid commitment to both. This would contradict the binding property of the commitment scheme. 


%that evaluates to $y$ in $z$. This is already implied if the verifier can choose $z$, since opening the polynomial in $\deg(f(X))+1$ points determines it completely. However, we want an even stronger soundness notion, one that applies even when $z$ is not chosen by the verifier. This motivates the following definition of extractability.

%%OLD 
\if 0 
\begin{comment}
We additionally require a stronger extraction property that had in different variations been defined by Zhang~\emph{et al.}~\cite{SP:ZGKPP17} and Sonic~\cite{EPRINT:MBKM19}.
\begin{definition}[PolyCommit extraction]
We say a polynomial commitment scheme, consisting of algorithms $(\setup,\commit)$ and the interactive protocol $\eval$, is extractable if there exists a rewinding \benedikt{Define more properly}extractor $\extractor$ such that for all polynomial degrees $d=\poly\in \NN$ and for any efficient adversary $\adv$:
	\[        
                \Pr\left[\begin{array}{c}\eval_{\adv,\verifier}(\crs,C,z,y)=\text{"accept"}\\
                \wedge\\
                f(x)\gets\extractor^{<\eval_{\adv,\verifier}(\crs,C,z,y)>}(\crs,C)\in \FF_p[X]\\
                \wedge\\
               \commit(f(x))\neq C\vee f(z)\neq y \vee \deg(f)\neq d
                 \end{array}  : 
                \begin{array}{l} 
                      \crs \sample \setup(\lambda, d) \\
                      \tau \sample (0,1)^\lambda\\
                      (C,z,y)\sample \adv_1(\crs,\tau)]
                      
                \end{array} 
        \right] \leq \negl.
\]
\end{definition}
\end{comment}

\begin{definition}[extractability]
Let let $\Gamma$ be a polynomial commitment scheme, let $\adv = (\adv_0, \adv_1)$ be any probabilistic polynomial-time adversary attacking the scheme where in particular $\adv_0$ decides on a statement and $\adv_1$ tries to convince the verifier of it, and let $\extractor$ be a probabilistic polynomial-time black-box extractor algorithm that satisfies the following description.
\begin{itemize}
    \itemsep0em 
    \item $\extractor$ interacts with $\adv_1$ in accordance with the $\eval$ protocol syntax, where $\extractor$ assumes the role of the verifier and $\adv_1$ the role of the prover.
    \item $\extractor$ can rewind $\adv_1$ to any previous point in time, at which point the protocol will resume from that point onward. $\ext$ remembers the observed messages from previous execution branches.
    \item We write $f(x) \gets \pro{Eval}_{\langle \adv_1 \leftrightarrow \extractor\rangle}(c,z,y,d;\st)$ to denote an execution of this extraction procedure whereby $\extractor$ finally outputs $f(x)$.
\end{itemize}
We say the polynomial commitment scheme $\Gamma$ is \defn{extractable} if there is an extractor $\extractor$ for all adversaries $\adv = (\adv_0, \adv_1)$ that convince the honest verifier with noticeable probability\alan{Actually, I am not convinced that with this order of quantifiers, the text definition is equivalent as the formula definition.}, $\extractor$ outputs a polynomial $f(x) \in R[X]$ of degree at most $d$ that matches the commitment with overwhelming probability. Specifically, this means that
\[
    \Pr \left[
        b \neq \bot \, \wedge \, c' \neq c \ : \ 
        \begin{array}{l}
             \params \gets \pro{Setup}(1^\lambda) \\
             (c, z, y, d; \state) \gets \adv_0(\params) \\
             b \gets \pro{Eval}_{\langle \adv_1 \leftrightarrow \verifier\rangle}(c, z, y, d; \state) \\
             f(x) \gets \pro{Eval}_{\langle \adv_1 \leftrightarrow \extractor\rangle}(c, z, y, d; \state) \\
             (c'; *) \gets \pro{Commit}(\params; f(x))
        \end{array}
    \right] \leq \negl \enspace .
\]
\end{definition}

\alaninline{Todo: handwave why extractability as defined here implies witness-extended emulation.} 

\begin{lemma}[binding $\wedge$ extractability $\Rightarrow$ evaluation binding]
	A binding polynomial commitment scheme that satisfies extractability also satisfies evaluation binding.
\end{lemma}
\begin{proof}
Proof by contradiction. We assume that there exists an adversary $\adv_{\rm eval} = (\adv_{{\rm eval},0}, \adv_{{\rm eval},1})$ capable of breaking the evaluation-binding property with non-negligible probability. With this algorithm we construct another algorithm that breaks the binding or the extractability properties. The description of this constructed adversary follows.

Run $\adv_{{\rm eval}, 0}(\params)$ to obtain $(c, z, y_0, y_1, d_0, d_1, s_0, s_1)$. Then run the extractor to obtain the polynomials $f_0(x) \gets \eval_{\langle \adv_{{\rm eval}, 1} \leftrightarrow \extractor \rangle}(c, z, y_0, d_0; s_0)$ and $f_1(x) \gets \eval_{\langle \adv_{{\rm eval}, 1} \leftrightarrow \extractor \rangle}(c, z, y_1, d_1; s_1)$. The algorithm's final step depends on the property under consideration.

Let $(c_0; *) \gets \pro{Commit}(\params, f_0(x))$ and $(c_1; *) \gets \pro{Commit}(\params, f_1(x))$. Since $y_0 \neq y_1 \Rightarrow f_0(x) \neq f_1(x)$, the probability $\Pr[c_0 = c_1]$ is negligible due to the binding property. Specifically, the final step here consists of outputting $(c, f_0(x), f_1(x), f_0(x), f_1(x))$ and this would break the binding property.

The probability $\Pr[c_0 \neq c \, \vee \, c_1 \neq c]$ is also negligible, because either clause breaks the extractability property. Specifically, the final step would consist of outputting either $(c,z,y_0, \deg(f_0(x), \state)$ where $\state = f_0(x)$ or $(c, z, y_1, \deg(f_1(x)), \state)$ where $\state = f_1(x)$; the second half of the extractability adversary is identical to that of the evaluation binding adversary, in which case the extractor proceeds to extract $f_0(x)$ or $f_1(x)$, respectively.

The remaining option is that $\Pr[b \neq \bot]$ is non-negligible, but this would imply, contrary to our assumption, that the adversary $\adv_{\rm eval} = (\adv_{{\rm eval},0}, \adv_{{\rm eval},1})$ is not capable of breaking the evaluation binding property.
\end{proof}

\fi 

\subsection{Proofs of Exponentiation}
Wesolowski \cite{EC:Wesolowski19} in his work on efficient verifiable delay functions introduced a simple yet powerful proof of exponentiation in groups of unknown order. A prover can efficiently convince a verifier that a large exponentiation in a group of unknown order was done correctly. The verifier needs to do little more than read the exponent. If the exponent is smooth then the protocol takes constant space/time in the security parameter.
\benedikt{Describe PoE protocol}
Boneh et al. \cite{journals/iacr/BonehBF18a} built on top of PoE to also develop PoKEs. \benedikt{Finish this section}
\section{Protocol}
\label{sec:protocol}

\subsection{Polynomial Encoding}

	Consider the set $B_{q/2}\subset \ZZ$ of integers with absolute value of less than $q/2$. $B_{q/2}[X]$ is the polynomial ring of that set such that the polynomials have bounded coefficients. $p(q) \in \ZZ$ for $p(x)\in P_q$ is a unique encoding of the polynomial:
\begin{itemize}
	\item Domain $B_{q/2}[X]\subset \ZZ[X]$, Alphabet: $\ZZ$
	\item 	$f(x):=\sum_{i=0}^{\lfloor\log_q(y)\rfloor} f_i x^i$
	\item $\enc(f(x) \in X)\rightarrow  f(q) \in \ZZ$
	\item $\dec(y \in \ZZ): f_i \in [0,q-1]\gets \frac{y \mod q^{i+1}-y \mod q^{i}}{q^i} \forall i \in [0,\lfloor\log_q(y)\rfloor]$
	\item Fix negative case
\end{itemize}

\begin{fact}
	Under certain conditions\todo{todo: elaborate me}, polynomial encoding scheme is uniquely decodable.
\end{fact}

Note that the encoding has limited homomorphic properties. $\enc(g(x))+\enc(h(x))=\enc(g(x)+h(x))$ if $g(x)+h(x)\in P_q$, i,.e. all its coefficients are less than $q/2$ in absolute value.\ This is ensured if for example the coefficients of $g$ and $h$ are less than $q/4$. Additionally $\enc(g(x))\cdot \enc(h(x))=\enc(g(x)\cdot h(x))$ if $g(x)\cdot h(x)\in P_q$.

\subsection{Polynomial Commitment}

We now present our main result: a polynomial commitment scheme with an efficient evaluation protocol based on groups of unknown order. We use lower-case \textsf{sans-serif} font to denote group elements. To save space, we denote the prover by $\prover$ and the verifier by $\verifier$. The scheme is parameterized by three values that are independent of the setup: the prime $p$ that defines the field over which the polynomial is defined; the degree $d_{\it max}$ that provides an upper bound on the degree of the polynomials for which the scheme works; and the integer $q$ which is ``large enough'' power of two, where what ``large enough'' means depends on the context.

The commitment and opening protocols are straightforward. They simply raise the designated group element $\gr{g}$ to the integer encoding of the polynomial as described in the previous section. The evaluation protocol uses the limited homomorphic properties of this encoding scheme to reduce the statement to smaller versions of itself.

In every iteration of the evaluation protocol, the working polynomial $f(x)$ is split in half, the left part $f_L(x)$ and the right part $f_R(x)$ such that $f(x) = f_L(x) + f_R(x) \cdot x^{d'}$ where $d' = \left\lceil \frac{d+1}{2}\right\rceil - 1$ and $d = \mathrm{deg}(f(x))$. The evaluation $y = f(z) \mod p$ is also split into two halves, $y_L = f_L(z) \mod p$ and $y_R = f_R(z) \mod p$ such that $y = y_L + y_R \cdot z^{d'}$, and so is the commitment with $\gr{c}_L = \gr{g}^{f_L(q)}$, $\gr{c}_R = \gr{g}^{f_R(q)}$. The verifier infers $y_L$ from $y$ and $y_R$, and infers $\gr{c}_L$ from $\gr{c}$ and $\gr{c}_R^{q^{d'}}$ --- but to save verification work, the prover sends both $\gr{c}_R$ and $\gr{c}_R^{q^{d'}}$ and the two engage in a $\pro{PoE}$ to establish that the one commitment really is the correct power of the other. At this point, the verifier samples a randomizer $\alpha \sample \mathbb{F}_p$ and the protocol recurses to prove a folded version of the statement, namely that $\gr{c}' = \gr{c}_L^\alpha \times \gr{c}_R$ is a commitment to $f'(x)$ that evaluates to $y' = \alpha y_L + y_R$ in $z$ modulo $p$. In the last recursion step, the degree of the polynomial is zero, \emph{i.e.}, $f(x)$ is a constant; in this case the prover simply sends this constant $f(x) = f_0$ and the verifier checks that $f_0 = y \mod p$ along with $\gr{g}^{f_0} = \gr{c}$ and $0 \leq f_0 < q$. If all checks succeed, the verifier accepts.

This intuitive description omits special care afforded to the case where the degree of the polynomial is not one less than a power of two. To take care of this, the left polynomial is guaranteed to have odd degree (unless in the last step when it is zero), and the right half polynomial $f_R(x)$ is shifted to the right by one digit whenever its degree is even. The evaluation $y_R$ and commitment $\gr{c}_R$ are adapted accordingly.

\begin{figure}
\renewcommand{\figurename}{Scheme}
\caption{polynomial commitment scheme} \label{scheme:polynomial_commitment}
\noindent\begin{mdframed}[userdefinedwidth=\textwidth]
\begin{minipage}{\textwidth}
	\begin{flushleft}
	$\pro{Setup}(1^\secpar):$
		\begin{enumerate}[nolistsep]
			\item $ \GG \gets \ggen(\secpar)$
			\item $ \gr{g} \sample \GG$
			%\item $q \gets 2^k$ such that $q > (d+1) \cdot 2\cdot p^{\log_2(d+1)+1} $
			\item $\pcreturn \params = (\secpar,\GG,\gr{g})$
		\end{enumerate}
	$\pro{Commit}(\crs;f(x)):$ \pccomment{$f(x) \in \mathbb{Z}[x]$ with coefficients in $[0;p-1]$}
		\begin{enumerate}[nolistsep]
			\item $\gr{c} \gets g^{f(q)}$
			\item $\pcreturn (\gr{c};f(x))$
		\end{enumerate}
	$\pro{Open}(\crs,\gr{c},\bar{f}(x);f(x)):$ \pccomment{$f(x) \in \mathbb{Z}[x]$ but $\bar{f}(x) \in \mathbb{F}_p[x]$}
		\begin{enumerate}[nolistsep]
		    \item \prover sends $f(x)$ to \verifier.
			\item \verifier checks that $\gr{g}^{f(q)} = \gr{c}$
			\item \verifier checks that for all $i \in \{0, \ldots, d\}$, $0 \leq f_i < p^{d_{\it max}}$
			\item \verifier checks that $f(x) \equiv \bar{f}(x) \mod p$
			\item \pcif all checks pass \textbf{then} \pcreturn $\top$ \textbf{else} \pcreturn $\bot$
		\end{enumerate}
	$\pro{Eval}(\params, \gr{c}, z, y, d; f(x)) :$ \pccomment{$f(x) = \sum_{i=0}^d f_i x^i \in \mathbb{Z}[x]$}
	    \begin{enumerate}[nolistsep]
        \item \pcif $d=0$:
        \item \pcind[1] $\pcreturn \pro{Open}(\params, \gr{c}, y; f(x))$ \pccomment{$f(x) = f_0 \in \mathbb{Z}$}
        \item \pcelse: \pccomment{$d \geq 1$}
        \item \pcind[1] \prover and \verifier compute $d' \gets \lceil\frac{d+1}{2}\rceil - 1$
        \item \pcind[1] \prover computes $f_L(x) \gets \sum\limits_{i=0}^{d'} f_i x^i$ and $f_R(x)\gets\sum\limits_{i=0}^{d-d'-1} f_{d'+1+i} x^{i}$
        \item \pcind[1] \prover computes $y_L\gets f_L(z) \bmod p$ and $y_R\gets f_R(z)\bmod p$
        \item \pcind[1] \prover computes $\gr{c}_L \gets \gr{g}^{f_L(q)}$ and $\gr{c}_R \gets \gr{g}^{f_R(q)}$ and $\gr{c}_R^\star \gets \gr{c}_R^{q^{d'+1}}$
        \item \pcind[1] \prover sends $y_R, \gr{c}_R, \gr{c}_R^\star$ to \verifier
        \item \pcind[1] \verifier computes $y_L \gets y - y_R \cdot z^{d'+1}$ and $\gr{c}_L \gets \gr{c} \times \gr{c}_R^{\star \, -1}$
        \item \pcind[1] \prover and \verifier run $\pro{PoE}(\gr{c}_R, \gr{c}_R^\star, q^{d'+1})$
        \item \pcind[1] \verifier samples $\alpha \sample \FF_p$ and sends it to \prover
        \item \pcind[1] \pcif $d$ is odd $:$ \pccomment{degree of $f_R$ is odd}
        \item \pcind[1] \pcind[1] \prover and \verifier compute $y'\gets\alpha y_L +y_R \bmod p$ and $\gr{c}' \gets \gr{c}_L^\alpha \times \gr{c}_R$. 
        \item \pcind[1] \pcind[1] \prover computes $f'(x) \gets \alpha \cdot f_L(x) + f_R(x) \in \ZZ[x]$ 
        \item \pcind[1] \pcind[1] \prover and \verifier run $\pro{Eval}(\params, \gr{c}', z, y', d'; f'(x))$
        \item \pcind[1] \pcelse $:$ \pccomment{degree of $f_R$ is even}
        \item \pcind[1] \pcind[1] \prover and \verifier compute $y'\gets\alpha y_L + y_R \cdot z \bmod p$, $\gr{c}' \gets \gr{c}_L^\alpha \times \gr{c}_R^q$. 
        \item \pcind[1] \pcind[1] \prover computes $f'(x) \gets \alpha \cdot f_L(x) + f_R(x) \cdot x \in \ZZ[x]$ 
        \item \pcind[1] \pcind[1] \prover and \verifier run $\pro{Eval}(\params, \gr{c}', z, y', d'; f'(x))$
        \end{enumerate}
	\end{flushleft}
\end{minipage}
\end{mdframed}
\end{figure}

\alaninline{Todo: correctness.}

\begin{lemma}
	The polynomial commitment scheme (\ref{scheme:polynomial_commitment}) is binding if the adaptive root assumption holds.
\end{lemma}
\begin{proof}
    Assume that there is an adversary that breaks the binding property of the scheme. Specifically, assume that some probabilistic polynomial time algorithm $\adv$ takes $(\params)$ and outputs $\gr{c}, f(x), f'(x), \hat{f}(x), \hat{f'}(x)$ such that with high probability $\pro{Open}(\params, \gr{c}, f(x); \hat{f}(x)) = \pro{Open}(\params, \gr{c}, f'(x), \hat{f'}(x)) = \top$ and $f(x) \neq f'(x)$. We proceed to show that this implies a violation of the adaptive root assumption~\ref{assum:adaptiveroot}.
    
	Let $\hat{h}(x)=\hat{f}(x)-\hat{f'}(x) \in \ZZ[X]$; this is a polynomial of degree at most $\max(f(x), f'(x))$. Since $\gr{g}^{\hat{f}(q)}=\gr{g}^{\hat{f'}(q)}=\gr{c}$ we have that $\gr{g}^{\hat{h}(q)}=1$. Note that the coefficients of $\hat{f}(x)$ and $\hat{f'}(x)$ are all less than $q/2$ in absolute value. By triangle inequality we have that the coefficients of $\hat{h}(x)$ are less than $q$. We have $\hat{f}(x) \neq \hat{f'}(x)$ because this inequality still holds modulo $p$, otherwise the adversary fails to open the commitment to two \emph{different} values. So $\hat{h}$ is not the zero polynomial. This implies that $\hat{h}(q) \neq 0$ and is a multiple of the order of $\gr{g} \in \GG$. Construct an adversary that breaks the adaptive root assumption (Assumption \ref{assum:adaptiveroot}) by outputting $\gr{g}$ in the first round, and by using $\hat{h}(q)$ to compute the inverse of $\ell$ modulo $\hat{h}(q)$.
\end{proof}

\begin{theorem}
	The polynomial commitment scheme (Scheme \ref{scheme:polynomial_commitment}) instantiated using $\ggen$ is extractable if the adaptive root assumption and the strong RSA assumption hold for $\ggen$.
\end{theorem}
%\alan{Is it true that we don't need the order assumption?}
\begin{proof}
Suppose there is an adversary $\adv$ that breaks extractability with non-negligible probability $\gamma$. Assuming $\adv$ convinces the verifier with non-negligible probability, we proceed to build an algorithm violating the Fractional Root Assumption.

The proof works by induction on the sequence $d \mapsto d' = \left\lceil \frac{d+1}{2} \right\rceil - 1 \mapsto d'' = \left\lceil \frac{d'+1}{2} \right\rceil - 1 \mapsto \cdots \mapsto 0$ in reverse order. Specifically, at every step we unravel the innermost nesting of the recursion. We show that the following invariants hold at every iteration.
\begin{itemize}
    \itemsep0pt
    \item The coefficients of $f(x), f_L(x), f_R(x) \in \mathbb{Z}[x]$ lie in the interval $[-2^ip^{\lceil \log_2(d_{\it max} + 1) \rceil+i}; 2^ip^{\lceil \log_2(d_{\it max} + 1) \rceil+i}]$ in the $i$th induction step (starting at $i=0$).
    \item The polynomials $f_L(x)$ and $f_R(x)$ satisfy $f(x) = f_L(x) + f_R(x) \cdot x^{d'+1}$ and $\deg(f_L(x)) \leq d'$ and $\deg(f_R(x)) \leq d - d' - 1$.
    \item The commitment $\gr{c}$ satisfies $\gr{c} = \gr{g}^{f(q)}$.
    \item The commitments $\gr{c}, \gr{c}_L$, $\gr{c}_R$ and $\gr{c}_R^\star$ satisfy $\gr{c} = \gr{c}_L \times \gr{c}_R^\star$ and $\gr{c}_R^{q^{d'+1}} = \gr{c}_R^\star$ where $d' = \left\lceil \frac{d+1}{2} \right\rceil - 1$.
    \item The evaluation $y$ satisfies $f(z) = y \mod p$.
    \item The evaluations $y, y_R$ and $y_L$ satisfy $y = y_L + y_R \cdot z^{d'+1}$.
\end{itemize}

After the last step of the induction argument, the extractor is in possession of a degree-$d$ polynomial $f(x) \in \mathbb{Z}[x]$ such that $\gr{c} = \gr{g}^{f(q)}$ and $f(z) = y \mod p$. This situation contradicts the assumption that $\adv$ breaks extractability, so the extractor constructed here must fail at one of the identified points of possible failure --- any one of which leads to a violated assumption.

\underline{$d = 0$.} The extractor receives $\gr{c}$, $y = f(z) \mod p$ and $f_0 = f(x) \in \mathbb{Z}$. The checks performed in the $\pro{Open}$ protocol guarantee that the invariants are satisfied.

\underline{Induction step.} We denote by primes (\emph{e.g.}, $f'(x)$) the values of the previous iteration; variables without primes are those of the present one. The task is to show that if the invariants for $(f'(x), f_L'(x), f_R'(x), y', y_L', y_R', \gr{c}', \gr{c}_L', \gr{c}_R', {\gr{c}_R^{\star}}')$ are satisfied, then so are they satisfied for $(f(x), f_L(x), f_R(x), y, y_L, y_R, \gr{c}, \gr{c}_L, \gr{c}_R, \gr{c}_R^{\star})$. The extractor runs steps 11-19 with a different $\alpha''$, thereby obtaining the tuple $(f''(x), f_L''(x), f_R''(x), y'', y_L'', y_R'', \gr{c}'', \gr{c}_L'', \gr{c}_R'', {\gr{c}_R^{\star}}'')$. Let $b \in \{0,1\}$ indicate whether $d$ is odd, by assuming the value 1 when this is so.

Due to the way $\gr{c}', \gr{c}'', y', y''$ are constructed, we know that $\gr{c}'\gr{c}''^{-1} = \gr{c}_L^{\alpha' - \alpha''}$ as well as $y'-y'' = (\alpha' - \alpha'')y_L$, regardless of the value of $b$. The extractor uses $f'(x)-f''(x) = (\alpha' - \alpha'')f_L(x)$ to obtain $f_L(x)$. We distinguish three cases.
\begin{itemize}
    \itemsep0pt
    \item[1.] $(\alpha' - \alpha'')$ divides $f'(x) - f''(x)$. Then $f_L(x) = (f'(x) - f''(x)) / (\alpha' - \alpha'')$ is well defined and has coefficients bounded in absolute value by $2^ip^{\lceil \log_2(d_{\it max} + 1) \rceil+i}$. From here, the extractor computes $f_R(x) \leftarrow x^{-b}( f'(x) - \alpha' \cdot f_L(x) )$ with coefficients bounded by $2^{i+1}p^{\lceil \log_2(d_{\it max} + 1) \rceil+i+1}$ and $\gr{c}_R = \gr{g}^{f_R(q)}$ holds because $\gr{c}' = \gr{g}^{f'(q)} = \gr{g}^{\alpha' \cdot f_L(q) + q^b \cdot f_R(q)} = \gr{c}_L^{\alpha'} \cdot \gr{c}_R^{q^b}$. This suffices to establish that all the invariants are satisfied.
    \item[2.] $(\alpha' - \alpha'')$ does not divide $f'(x) - f''(x)$ but does divide $f'(q) - f''(q)$. The coefficients of $f'(x) - f''(x)$ are bounded. This implies that $f'(q) - f''(q)$ denotes an integer distinct from zero.
    \item[] Let $f_{L,q} = (f'(q) - f''(q))/(\alpha' - \alpha'')$ and let $\{\tilde{f}_i\}_{i=0}^{d'}$ be its base-$q$ expansion, \emph{i.e.}, $f_{L,q} = \sum_{i=0}^{d'} \tilde{f}_i q^i$ with all $\tilde{f}_i \in \{0,\ldots,q-1\}$. Also, note that $\gr{g}^{f_{L,q}} = \gr{c}_L$. The extractor obtains a polynomial with coefficients bounded by $p^{\lceil \log_2(d_{\it max} + 1) \rceil}$ by starting from $f_L(x) = \sum_{i=0}^{d'} \tilde{f}_i x^i$ and iterating over all powers $j$ from 0 to $d'-1$, and subtracting that unique multiple of $x^{j}(x-q)$ that sends the coefficient of $x^j$ to the interval $[-2^ip^{\lceil \log_2(d_{\it max} + 1) \rceil+i};2^ip^{\lceil \log_2(d_{\it max} + 1) \rceil+i}]$. There are two possible causes for failure.
    \begin{itemize}
        \item[2.1.] No polynomial $f_L(x) \in \mathbb{Z}[x]$ with bounded coefficients and satisfying $f_L(q) = f_{L,q}$ exists. Then at least one coefficient lies outside the interval $[-2^ip^{\lceil \log_2(d_{\it max} + 1) \rceil+i};2^ip^{\lceil \log_2(d_{\it max} + 1) \rceil+i}]$. As a result, with overwhelming probability, at least one coefficient of $\alpha \cdot f_L(x) + f_R(x)$ lies outside that interval for random $\alpha$. With overwhelming probability, the non-membership percolates to the terminal case where the coefficient $f_0$ is tested and this non-membership is observed, causing the verifier to reject the opening. (This coefficient shrinks only when $\alpha=0$ or $\alpha=1$, and either event occurs with negligible probability.) As a result, if such an $f_L(x)$ does not exist, then the prover cannot have more than a negligible chance of success, in contradiction to the requirement for breaking the evaluation binding property.
        \item[2.2.] The polynomial $f_L(x)$ does exist and it does have bounded coefficients, but $f(z) \neq y_L \mod p$. Then with overwhelming probability $\alpha \cdot f_L(z) + f_R(z) \neq \alpha \cdot y_L + y_R$. With overwhelming probability this inequality percolates to the terminal case where $f_0 \neq y \mod p$ and the prover fails to convince the verifier. As a result, if $f_L(z) \neq y_L \mod p$ then the prover cannot have more than a negligible chance of success, in contradiction to the requirement for breaking the evaluation binding property.
    \end{itemize}
    \item[] We may therefore assume that $f_L(x) \in \mathbb{Z}[x]$ with bounded coefficients exists, and satisfies both $f_L(q) = f_{L,q}$ and $f_L(z) = y \mod p$. The extractor then computes $f_R(x) \leftarrow x^{-b}(f'(x) - \alpha' \cdot f_L(x))$ which also has bounded coefficients. It follows that the induction invariants are satisfied.
    \item[3.] $(\alpha' - \alpha'')$ divides neither $f'(x) - f''(x)$ nor $f'(q) - f''(q)$. Then $\gr{c}_L$ is a $(\alpha' - \alpha'')$th root of $\gr{g}^{f'(q) - f''(q)} = \gr{c}'{\gr{c}''}^{-1}$ and outputting $(f'(q) - f''(q), \alpha'-\alpha'', \gr{c}_L)$ breaks the fractional root assumption.
\end{itemize}
This completes the inductive argument, which in turn completes the proof.
\end{proof}

{\it Note 1.} The extractability proof shows that the extractor is successful whenever the coefficients of the prover's polynomial are bounded in absolute value by $2^ip^{\lceil \log_2(d_{\it max} + 1) \rceil+i}$ or, after the final induction step, by $2^{\lceil \log_2(d_{\it max} + 1) \rceil}p^{2\lceil \log_2(d_{\it max} + 1) \rceil}$. In other words, prover who starts with a polynomial with at least one of the coefficients outside of this range, will succeed with negligible probability. However, the proof fails to cover what happens when he selects a polynomial within this range but outside $[0;p-1]$. As per correctness, his success is only guaranteed if all coefficients lie within this last range. Additionally this growing bound supplies in turn a lower bound on $q$. The coefficients that are too large in absolute value should percolate along with the recursion, remaining outside the bounds even after multiplication by a random $\alpha$. The set of possible coefficients must therefore be a factor $2^\lambda$ larger than the bounded interval. And therefore, $q > 2^{\lambda + \lceil \log_2(d_{\it max} + 1) \rceil}p^{2\lceil \log_2(d_{\it max} + 1) \rceil}$.

{\it Note 2.} The sequence of degrees followed by the inductive argument matches exactly the reversed sequence of degrees occurring in the protocol. As a result, the interactive proof establishes that $\deg(f(x)) \leq d_{\it max}$ and as such doubles as a range proof.

\begin{proof}
(SKETCH)
We prove the statement by showing that we can recursively either extract the encoding of an integer polynomial $f(x) \in \ZZ[X]$ of degree $d$ with bounded coefficients or a break of the pseudo root assumption for element $g$ or a non trivial element of known (random) order in $\GG$. We recurse over degree $\hat{d}$ up to the final degree $d$.
In each round the extracted witness is an integer $\hat{y}$ such that $\hat{y}=\enc(f(x))$ where the coefficients of $f(x)$ are less than $B$ in absolute value and the degree is at most $\hat{d}$ and such that $g^{\hat{y}}=C$. Also $f(z) \equiv y \mod p$.
If $\hat{d}=0$ then we can directly extract $f(x)=\hat{y}$ such that $\vert \hat{y} \vert < p^{\log_2(d+1)+1}$, $y=\hat{y}\mod p$, $f(x)=y\in \FF_p[X]$ and $g^{\hat{y}}=C$ as the witness. We proceed with $d=1$.
For $d>0$ we have $g^{\hat{y}}=C=C_0^{\alpha}C_1$. Rewinding once we get 
 $C=C_0^{\alpha}C_1=g^{\hat{y}}$ and $C'=C_0^{\alpha'}C_1=g^{\hat{y}'}$ for distinct $\alpha$ and $\alpha'$. 
 This gives us $C_0^{\alpha-\alpha'}=g^{\hat{y}-\hat{y}'}$. 
 Either $\alpha-\alpha' \not|~ \hat{y}-\hat{y}' $ which would directly break the pseudo root assumption or we can compute $D=g^{\frac{\hat{y}-\hat{y}'}{\alpha-\alpha'}}$. Either $D=C_0$ or $(D/C_0)^{\alpha-\alpha'}=1$, i.e. $D/C_0$ is an element of known order. In either of the cases the extraction succeeds and is completed. (Invoke order assumption).
 
 If $D=C_0$ then we have $\hat{y}_0=\frac{\hat{y}-\hat{y}'}{\alpha-\alpha'}$ and
 $g^{\hat{y}_0}=C_0$. In that case $C_1=g^{\hat{y}_1}=g^{\hat{y}-\alpha\hat{y}_0}$. The extractor has now successfully obtained $\hat{y}_0$ and $\hat{y}_1$
 
 If $|y|=|y'|<B$ then $|y_0|<  2 \cdot B$ and $|y_1|<|\hat{y}-\hat{y}+\alpha\hat{y}'|=|\alpha \hat{y}'|<\lambda \cdot B$. 
 We define $f(x)=\dec(y_0+x^{d} \cdot y_1)$ and with coefficients less than $2B$ in absolute value. 
 Note that $y_0=\hat{y}_0 \mod p=\dec(y_0)(z)$ and $y_1=\hat{y}_1 \mod p=\dec(y_1)(z)$. Since $y=\dec(\hat{y})(z)=\alpha \cdot \dec(\hat{y}_0)(z)+\dec(\hat{y}_1)(z)$ and $f(x)=\dec(y_0+x^{\hat{d}} \cdot y_1)$ we have that $y=y_0 +z^{\hat{d}/2}y_1 \mod p=f(z) \mod p$.
 Additionally $C=C_0^{\alpha}C_1=\commit(f(x))$ or we get a break of the adaptive root assumption.\benedikt{So this is technically a bit difficult because it's not actually a witeness}.
 
 The extractor recuses with the encoding of $f(x)$ and degree $\hat{d}'=\hat{d}\cdot 2$.
 
 Repeating this $\log_2(d+1)$ times we get a polynomial $f(x)$ of degree $d$ that has coefficients that are bounded by $(d+1) \cdot p^{\log_2(d+1)+1} <q/2$. 

\end{proof}

\begin{mdframed}[userdefinedwidth=0.8\textwidth]
\begin{minipage}{\textwidth}
	\begin{flushleft}
	\setup(\secpar,p,d,n):
		\begin{enumerate}[nolistsep]
			\item $ \GG \sample \ggen(\secpar)$
			\item $ g \sample \GG$
			\item $q\gets 2^k \text{s.t.} q>(d+1) \cdot 2\cdot p^{\log_2(d+1)+1} $
			\item $\pcreturn \crs=\{\secpar,p,\GG,g,q\}$
		\end{enumerate}
		$\commit(\crs,f(x)\in \ZZ[X])$ \pccomment{$f(x)$'s coefficients are \textcolor{blue}{in $[0,p-1]$}} 
		\begin{enumerate}[nolistsep]
			\item 	$f(q)\gets \enc(f(x)) \in \ZZ$
			\item \textcolor{blue}{$C \gets g^{f(q)}$}
			\item $\pcreturn \textcolor{blue}{C},f(q)$ \textcolor{blue}{Alan: why is including the degree necessary?}\benedikt{Probably commit should have a public and private output. I changed it but we need to think about this.}
		\end{enumerate}
		$\open(\crs,\textcolor{blue}{C},f(x)\in \ZZ[X])$ \pccomment{\textcolor{blue}{$f(x)$'s coefficients are in $[-q/2,q/2]$}} 
		\begin{enumerate}[nolistsep]
			\item $f(q)\gets \enc(f(x)) \in \ZZ$
			\item Prover sends $f(q),f(x)$ to verifier.
			\item Verifier checks that $\dec(f(q)) \mod p=f(x)$
			\item If $C=g^{f(q)}$ the verifier accepts.
		\end{enumerate}
	\end{flushleft}
	
\end{minipage}
\end{mdframed}

\begin{small}
 \begin{minipage}{1.1\textwidth}
\begin{mdframed}[userdefinedwidth=1\textwidth]  \label{prot:Opening}
	\noindent \underline{\textsf{Protocol \eval{} Information theoretic} (Polynomial evaluation)}\\
Inputs: $\enc(f(x)),z,y\in \FF_p,d \in \NN $\\
Witness: $f(x) \in \ZZ[X]$ of degree at most $d$ and with coefficients bounded by $p$ \textcolor{blue}{Alan: if it is bounded by $p$ then you cannot recurse, because the coefficients grow every step.}\footnote{We describe the protocol using an integer polynomial with bounded coefficient. This generalizes the case where the polynomial is in $\FF_p[X]$};\\ 
Claim: $f(x)\gets \dec(\enc(f(x))), y=f(z) \mod p$

\begin{enumerate}[nolistsep]
\item \pcif $d=0$:
\item \pcind[1] Prover sends $\hat{y}\gets \enc(f(x))$ to the verifier. 
\item \pcind[1] Verifier runs that $f(x)\gets \dec(\hat{y})$ and checks that  $f(z) \bmod p=y$, or rejects otherwise 
%\textcolor{blue}{Alan: not in the exponent of $p$?}, 
%right
\item \pcelse: 
\item \pcind[1] $d'\gets \frac{d+1}{2}-1$
\item \pcind[1] Prover computes $f_0(x)\gets\sum_{i=0}^{d'} f_i x^i\in \ZZ[x]$ and $f_1(x)\gets\sum_{i=0}^{d'} f_{d'+1+i} x^{i}\in \ZZ[x]$
\item \pcind[1] $y_0\gets f_0(z) \bmod p$, $y_1\gets f_1(z)\bmod p$
\item \pcind[1] Prover sends $y_0,y_1, \enc(f_0(x)),\enc(f_1(x))$ to the Verifier
\item \pcind[1] Verifier checks that $y_0+z^{d'+1} y_1=y\in \FF_p$ 
\item \pcind[1] Verifier checks that $\enc(f_0(x))+x^{d'+1}\cdot \enc(f_1(x))=\enc(f(x))$
\item \pcind[1] Verifier samples $\alpha \sample \FF_p$ and sends it to the prover
\item \pcind[1] Prover and Verifier compute $y'\gets\alpha y_0 +y_1 \bmod p$, $\enc(f'(x))\gets \alpha \cdot \enc(f_0(x)) +\enc(f_1(x))$. 
\item \pcind[1] Prover also computes $f'(x)\gets\alpha  \cdot f_0(x)+f_1(x) \in \ZZ[x]$ 
\item \pcind[1] Prover and Verifier run $\eval(\enc(f'(x),z,y',d';f'(x))$
\end{enumerate}
\end{mdframed}
\end{minipage}
\end{small}

\begin{small}
 \begin{minipage}{1.1\textwidth}
\begin{mdframed}[userdefinedwidth=1\textwidth]  \label{prot:Opening}
	\noindent \underline{\textsf{Protocol \eval} (Polynomial evaluation)}\\
\noindent Params: $\crs=\{\secpar,p,\GG,g,d_{\max},q\}$;\ \
Inputs: $C\in \GG,z,y\in \FF_p,d \in \NN $\\
Witness: $f(x) \in \ZZ[X]$ of degree at most $d$ and with coefficients bounded by $p$ \textcolor{blue}{Alan: if it is bounded by $p$ then you cannot recurse, because the coefficients grow every step.}\footnote{We describe the protocol using an integer polynomial with bounded coefficient. This generalizes the case where the polynomial is in $\FF_p[X]$};\\ 
Claim: $C=g^{f(q)},\deg(f)\leq d$ and $y=f(z) \mod p$

\begin{enumerate}[nolistsep]
\item \pcif $d=0$:
\item \pcind[1] Prover sends $f(x)$ to the verifier, $f(x)$ is a constant. 
\item \pcind[1] Verifier checks that $0\leq f(q)<  p^{\log_2(d+1)+1}$,  $f(z) \bmod p=y$ and $g^{f(q)}=C$, or rejects otherwise 
%\textcolor{blue}{Alan: not in the exponent of $p$?}, 
%right
\item \pcelse: 
\item \pcind[1] $d'\gets \frac{d+1}{2}-1$
\item \pcind[1] Prover computes $f_0(x)\gets\sum_{i=0}^{d'} f_i x^i\in \ZZ[x]$ and $f_1(x)\gets\sum_{i=0}^{d'} f_{d'+1+i} x^{i}\in \ZZ[x]$
\item \pcind[1] $y_0\gets f_0(z) \bmod p$, $y_1\gets f_1(z)\bmod p$, $C_0\gets g^{f_0(q)}$,$C_1\gets g^{f_1(q)}$
\item \pcind[1] Prover sends $y_0,y_1,C_0,C_1$ to the Verifier
\item \pcind[1] Verifier checks that $y_0+z^{d'+1} y_1=y\in \FF_p$ 
\item \pcind[1] Prover and Verifier engage in $\textsf{PoE}(C_1,C/C_0,q^{d'+1})$ to show that $C_0C_1^{(q^{d'+1})}=C$
\item \pcind[1] Verifier samples $\alpha \sample \FF_p$ and sends it to the prover
\item \pcind[1] Prover and Verifier compute $y'\gets\alpha y_0 +y_1 \bmod p$, $C'\gets C_0^{\alpha}C_1 \in \GG$. 
\item \pcind[1] Prover also computes $f'(x)\gets\alpha  \cdot f_0(x)+f_1(x) \in \ZZ[x]$ 
\item \pcind[1] Prover and Verifier run $\eval(C',z,y',d';f'(x))$
\end{enumerate}
\end{mdframed}
\end{minipage}
\end{small}

\subsection{Discussion and Optimizations}
\begin{enumerate}
	\item Precompute $g^{q^i}$ to use parallelism etc.
	\item SNARK without falsifiable assumptions
	\item First polynomial commitment scheme with constant size parameters
	\item join PoEs
	\item ZK
	\item early termination
\end{enumerate}

\subsection{Zero-Knowledge}

\subsection{Multivariate Commitment}
  We extend the Polynomial Commitment scheme to handle multivariate polynomials. The idea is to simply use higher degrees of $q$ to encode the polynomial. The protocol is linear in the number of variables and logarithmic in the total degree of the polynomial. For simplicity we only present a protocol for $n$-variate polynomials where the degree in each variable is $d$ and $d$ is a power of $2$. The protocol naturally extends to different degrees per variable.
We first define an encoding for multi-variate integer polynomials with small coefficients:
 \begin{itemize}
	\item Domain $B_{q/2}[X^n] \subset \ZZ[X^n]$, Alphabet: $\ZZ$, \\$f(x_0,\dots,x_{n-1})=\sum_{\vec{i} \in [0,d]^n} f_{\vec{i}} \prod_{j=1}^{n} x_j^{i_j}$
	\item $\enc(f(x_0,\dots,x_{n-1}) \in B_{q/2}[X^n]): f(q,q^{d+1},q^{(d+1)^2},\dots,q^{(d+1)^{n-1}})$
	\item $\textsf{Q-ary}(\vec{i}):\sum_{j=1}^{n} i_j\cdot (d+1)^{j-1}$
	\item $\dec(y \in \ZZ): f_{\vec{i}}=\frac{y \bmod q^{\textsf{Q-ary}(\vec{i})+1}-y \bmod q^{\textsf{Q-ary}(\vec{i})}}{q^{\textsf{Q-ary}(\vec{i})}} \forall \vec{i} \in [0,d]^n$
	\item \benedikt{Negative Coefficients?}

\end{itemize}
\begin{small}
 \begin{minipage}{1.1\textwidth}
\begin{mdframed}[userdefinedwidth=1\textwidth]  \label{prot:Opening}
	\noindent \underline{\textsf{Protocol \eval} (Polynomial evaluation)}\\
\noindent Params: $\crs=\{\secpar,p,\GG,g,d_{\max},n_{\max},q\}$;\ \
Inputs: $C\in \GG,z_1,\dots,z_n,y\in \FF_p,d \in \NN,n \in \NN $\\
Witness: $f(x_1,\dots,x_n) \in B_q[X^n]$ of degree at most d and with coefficients bounded by $p$\footnote{We describe the protocol using an integer polynomial with bounded coefficient. This generalizes the case where the polynomial is in $\FF_p[X^n]$};\\ 
Claim: $C=g^{f(q_1,q_2,\dots,q_n)}$ where $q_i=q^{(d_{\max}+1)^{i-1} },\deg(f)\leq d$ and $y=f(z_1,\dots,z_n) \mod p$

\begin{enumerate}[nolistsep]
\item \pcif $n=0 \wedge d=0$:
\item \pcind[1] Prover sends $f(x)$ to the verifier, $f(x)$ is a constant. 
\item \pcind[1] Verifier checks that $0\leq f(q)< p^{n_{\max}\cdot \log_2(d_{\max}+1)+1}$,  $f(z) \bmod p=y$ and $g^{f(q)}=C$ otherwise rejects
\item \pcelse: 
\item \pcind[1] $d'\gets \frac{d+1}{2}-1$
\item \pcind[1] Prover computes $f_0(x_1,\dots,x_n)\gets \sum_{\vec{i} \in [0,d_{\max}]^{n-1}}  \prod_{j=1}^{n-1} x_{j}^{i_j} \cdot(\sum_{k=0}^{d'} f_{\vec{i},k} x_{n}^k)\in \ZZ[X^n]$ and
$f_1(x_1,\dots,x_n)\gets \sum_{\vec{i} \in [0,d_{\max}]^{n-1}} \prod_{j=1}^{n-1} x_{j}^{i_j}(  \sum_{k=0}^{d'} f_{\vec{i},k+d'+1} x_{n}^{k})\in \ZZ[X^n]$
\item \pcind[1] $y_0\gets f_0(z_1,\dots,z_n) \bmod p$, $y_1\gets f_1(z_1,\dots,z_n)\bmod p$,\\ $C_0\gets g^{f_0(q_1,\dots,q_n)}$,$C_1\gets g^{f_1(q_1,\dots,q_n)}$
\item \pcind[1] Prover sends $y_0,y_1,C_0,C_1$ to the Verifier
\item \pcind[1] Verifier checks that $y_0+z_n^{d'+1} y_1=y\in \FF_p$ 
\item \pcind[1] Prover and Verifier engage in $\textsf{PoE}(C_1,C/C_0,q_n^{d'+1})$ to show that $C_0C_1^{q_n^{d'+1}}=C$
\item \pcind[1] Verifier samples $\alpha \sample \FF_p$ and sends it to the prover
\item \pcind[1] Prover and Verifier compute $y'\gets\alpha y_0 +y_1 \bmod p$, $C'\gets C_0^{\alpha}C_1 \in \GG$. \\Prover also computs $f'(x_1,\dots,x_n)\gets\alpha  \cdot f_0(x_1,\dots,x_n)+f_1(x_1,\dots,x_n) \in \ZZ[X^n]$ 
\item \pcind[1] \pcif $d=0$ and $n>0$: \pccomment{$f'$ is constant in $x_n$ so we write it as an $n-1$ variate polynomial.}
\item \pcind[2] $d'\gets d_{\max}$
\item \pcind[2] $n'\gets n-1$
\item \pcind[2] $\vec{z}\gets (z_1,\dots,z_{n-1})$
\item \pcind[1] Prover and Verifier run $\eval(C',\vec{z},y',d',n';f'(x))$
\end{enumerate}
\end{mdframed}
\end{minipage}
\end{small}

\subsection{Fix coefficients and negative degrees}
\begin{itemize}
	\item Several ways to do this.
\end{itemize}

\subsection{Range Proofs}

\subsection{Vector Commitments}

\subsection{Inner Product Argument}

\section{Security}

\begin{lemma}
	The pseudo root assumption holds in the generic group model.
\end{lemma}
\begin{lemma}
	The polynomial commitment scheme satisfies the opening binding property
\end{lemma}
\begin{proof}
	Assume $\hat{f}(x)$ and $\hat{f}'(x)$ are integer encodings of two distinct polynomials $f(x),f'(x) \in \FF_p[X]$. Assume that the adversary can open a commitment $C$ to both $f(x)$ and $f'(x)$. We show that we can use this adversary to break the pseudo root assumption (Assumption \ref{assum:fracroot}).
	
	 Then $\hat{h}(x)=\hat{f}(x)-\hat{f}'(x) \in \ZZ[X]$ is a polynomial of degree at most $d$. Since $g^{\hat{f}(q)}=g^{\hat{f}'(q)}=C$ we have that $g^{\hat{h}(q)}=1$. Note that the coefficients of $\hat{f}(x)$ and $\hat{f}'(x)$ are all less than $q/2$ in absolute value. By triangle inequality we have that $\hat{h}(x)$'s coefficients are less than $q$. $\hat{h}$ is by assumption not the zero polynomial. This implies that $\hat{h}(q)\neq 0$ is a multiple of the order of $g\in \GG$. This however can directly be used to break the adaptive root assumption (Assumption \ref{assum:adaptiveroot}) by taking the inverse of $\ell$ modulo $\hat{h}(q)$.
\end{proof}

\begin{theorem}
	The polynomial commitment scheme from Section \ref{sec:protocol} satisfies extraction under the Pseudo Root Assumption and the Order assumption. We show that we can either extract a pseudo root of $g$ or a witness \benedikt{Formally we should define witness extended emulation }
\end{theorem}
\begin{proof}
(SKETCH)
We prove the statement by showing that we can recursively either extract the encoding of an integer polynomial $f(x) \in \ZZ[X]$ of degree $d$ with bounded coefficients or a break of the pseudo root assumption for element $g$ or a non trivial element of known (random) order in $\GG$. We recurse over degree $\hat{d}$ up to the final degree $d$.
In each round the extracted witness is an integer $\hat{y}$ such that $\hat{y}=\enc(f(x))$ where the coefficients of $f(x)$ are less than $B$ in absolute value and the degree is at most $\hat{d}$ and such that $g^{\hat{y}}=C$. Also $f(z) \equiv y \mod p$.
If $\hat{d}=0$ then we can directly extract $f(x)=\hat{y}$ such that $\vert \hat{y} \vert < p^{\log_2(d+1)+1}$, $y=\hat{y}\mod p$, $f(x)=y\in \FF_p[X]$ and $g^{\hat{y}}=C$ as the witness. We proceed with $d=1$.
For $d>0$ we have $g^{\hat{y}}=C=C_0^{\alpha}C_1$. Rewinding once we get 
 $C=C_0^{\alpha}C_1=g^{\hat{y}}$ and $C'=C_0^{\alpha'}C_1=g^{\hat{y}'}$ for distinct $\alpha$ and $\alpha'$. 
 This gives us $C_0^{\alpha-\alpha'}=g^{\hat{y}-\hat{y}'}$. 
 Either $\alpha-\alpha' \not|~ \hat{y}-\hat{y}' $ which would directly break the pseudo root assumption or we can compute $D=g^{\frac{\hat{y}-\hat{y}'}{\alpha-\alpha'}}$. Either $D=C_0$ or $(D/C_0)^{\alpha-\alpha'}=1$, i.e. $D/C_0$ is an element of known order. In either of the cases the extraction succeeds and is completed. (Invoke order assumption).
 
 If $D=C_0$ then we have $\hat{y}_0=\frac{\hat{y}-\hat{y}'}{\alpha-\alpha'}$ and
 $g^{\hat{y}_0}=C_0$. In that case $C_1=g^{\hat{y}_1}=g^{\hat{y}-\alpha\hat{y}_0}$. The extractor has now successfully obtained $\hat{y}_0$ and $\hat{y}_1$
 
 If $|y|=|y'|<B$ then $|y_0|<  2 \cdot B$ and $|y_1|<|\hat{y}-\hat{y}+\alpha\hat{y}'|=|\alpha \hat{y}'|<\lambda \cdot B$. 
 We define $f(x)=\dec(y_0+x^{d} \cdot y_1)$ and with coefficients less than $2B$ in absolute value. 
 Note that $y_0=\hat{y}_0 \mod p=\dec(y_0)(z)$ and $y_1=\hat{y}_1 \mod p=\dec(y_1)(z)$. Since $y=\dec(\hat{y})(z)=\alpha \cdot \dec(\hat{y}_0)(z)+\dec(\hat{y}_1)(z)$ and $f(x)=\dec(y_0+x^{\hat{d}} \cdot y_1)$ we have that $y=y_0 +z^{\hat{d}/2}y_1 \mod p=f(z) \mod p$.
 Additionally $C=C_0^{\alpha}C_1=\commit(f(x))$ or we get a break of the adaptive root assumption.\benedikt{So this is technically a bit difficult because it's not actually a witeness}.
 
 The extractor recuses with the encoding of $f(x)$ and degree $\hat{d}'=\hat{d}\cdot 2$.
 
 Repeating this $\log_2(d+1)$ times we get a polynomial $f(x)$ of degree $d$ that has coefficients that are bounded by $(d+1) \cdot p^{\log_2(d+1)+1} <q/2$. 
 


\end{proof}

\begin{corollary}
	If $q<bla$ there exists an efficient adversary that can break the evaluation binding property of the polynomial commitment.
\end{corollary}
\benedikt{Figure out at what q we can attack the scheme. Probably needs to use negative coefficients and such.}

\section{Zero knowledge polynomial commitment} 
This section sketches how to make the polynomial commitment scheme zero knowledge. 

\paragraph{Commit} Let $g_1 \sample \GG$ be a random base distinct from $g$. 
The hiding polynomial commitment is $C \leftarrow g^{f(q)}g_1^r$ for $r \sample [-2^\lambda, 2^\lambda]$. 

\paragraph{Open} The opening of the entire polynomial is the same, but additionally gives the blinding factor $r$. 

\paragraph{Eval}

\begin{itemize}
\item In each recursive step we commit to polynomials $f_0$ and $f_1$ using the same hiding commitment scheme, where $f_0 + f_1 q^{d/2} = f$ as integer polynomials. 

\item Note that if $C_0 = g^{f_0(q)}g_1^{r_0}$ and $C_1 = g^{f_1(q)} g_1^{r_1}$ then $C_0 \cdot C_1^{q^{d/2}} = C \cdot g_1^{r'}$ where $r' = r_0 + q^{d/2} r_1$. The prover can give a non-interactive zk proof of this relation to the verifier using a sigma protocol. E.g., the prover provides $C_1' = C_1^{q^{d/2}}$ with a PoE, and then a zk-PoKE of $r'$ such that $g_1^{r'} = C_0 C_1' / C$. 

\item We could then recurse on $C_0^\alpha C_1$ which commits to $\alpha f_0 + f_1$ with the blinding factor $\alpha r_0 + r_1$. BUT we are not done yet, see next bullet point... 

\item The remaining problem is that the evaluation protocol opens $y_0 = f_0(z) \bmod p$ and $y_1 = f_1(z) \bmod p$, which is not zero knowledge. We need $y_0, y_1$ to be independently distributed subject to the constraint $y_0 + z^{d/2} y_1 = y \bmod p$, which the verifier checks. 

A solution is to modify $f_0$ and $f_1$ by adding constant terms $\alpha, \beta$ to each that cancel, i.e. $\alpha + z^{d/2} \beta = 0 \bmod p$, where $\alpha$ is uniformly distributed in $\ZZ_p$. This way the polynomials $f_0' = f_0 + \alpha$ and $f_1' = f_1 + \beta$ satisfy the relation $f_0'(z) + z^{d/2}f_1'(z) = f(z) \bmod p$. We end up revealing $y_0' = y_0 + \alpha \bmod p$ and $y_1' = y_1 + \beta \bmod p$, which is uniformly distributed in $\ZZ_p$ subject to $y_0' + y_1' = y \bmod p$. 

Finally, the prover needs to convince the verifier that it modified the $C_0$ and $C_1$ commitments appropriately. (It could not simply choose $f_0'$ and $f_1'$ in the first step because $f_0' + q^{d/2} f_1' \neq f$ as integer polynomials). 

However, the solution is still simple. The prover creates hiding commitments $C_\alpha$ to $\alpha$ and $C_\beta$ to $\beta$ and provides a zero-knowledge proof that $C_\alpha C_\beta ^{z^{d/2}}$ is a commitment to an integer multiple of $p$. This can be done efficiently through a combination of PoE and a PoKE. (Given $g^a$, to prove that $a = 0 \bmod p$ it suffices to provide $Q$ such that $Q^p = g^a$ and a PoKE for $Q$ base g. This can be made zero knowledge w/ the standard tricks). 

The protocol then proceeds on modified commitments $C_0' = C_0 C_\alpha$ and $C_1' = C_1 C_\beta$.

\end{itemize}

\section{Vector Commitment}


\subsection{Commitment Scheme}

In the following we denote by $(a_i)_{i=0}^{d-1} \in \mathbb{F}_p^{d}$ a vector of prime field elements. The vector commitment scheme is given by the following algorithms.
\begin{itemize}
\item $\mathsf{vcom} : \mathbb{F}_p^d \rightarrow \mathbb{G} \, , \quad (a_i)_{i=0}^{d-1} \mapsto g^{\sum_{i=0}^{d-1} a_iq^i} \enspace .$
\item $\mathsf{vopen} : \mathbb{G} \times \mathbb{Z} \rightarrow \mathbb{F}_p^d \cup \{\bot\} \, , $
\item[] $\phantom{\mathsf{vopen} :} (C, z = \sum_{i=0}^{d-1} z_iq^i) \mapsto \left\lbrace \begin{array}{ll}
(z_i \, \mathsf{mod} \, p)_{i=0}^{d-1} & \textnormal{\bf if } g^z = C \\
\quad \textnormal{where all } z_i \in \{0,\ldots,q-1\} & \\
\bot & \textnormal{\bf otherwise.}
\end{array} \right.$ 
\end{itemize}

Note: somewhat homomorphic properties: multiplication by constant, additivity. As long as coefficients don't overflow.

\subsection{Coordinate Extraction}

The following protocol enables the prover to extract a commitment to the $i$th component of the vector. Both Prover and Verifier know $i, g, C$. Only the prover knows an integer $z$ such that $g^z = C$ and corresponding to a vector $(a_j)_{j=0}^{d-1}$.
\begin{itemize}
\item Prover computes (or already knows) the $q$-ary expansion of $z$, \emph{i.e.}, $(z_j)_{j=0}^{d-1}$ such that $\sum_{j=0}^{d-1} z_j q^j = z$ and all $z_j \in \{0,\ldots, q-1\}$. He then sends to Verifier:
\begin{itemize}
\item $C_l = g^{\sum_{j=0}^{i-1} z_jq^j}, C_i = g^{z_i}, C_r = g^{\sum_{j=i+1}^{d-1} z_j q^{j-i-1}}$
\item $C_i^{q^i}, C_r^{q^{i+1}}$
\end{itemize}
\item Prover and Verifier run a proof of correct exponentiation to establish that $C_m^{q^i}, C_r^{q^{i+1}}$ were computed correctly.
\item Verifier checks that $C_l \times C_i^{q^i} \times C_r^{q^{i+1}} \stackrel{?}{=} C$ and aborts if false.
\item Prover and Verifier run a range proof to establish that the discrete logarithm of $C_l$ base $g$ is within the range $\{0, \ldots, q^i-1\}$.
\end{itemize}

Correctness, soundness, etc. (todo)

\subsection{Inner Product}
\label{section:inner_product}

The following protocol enables the prover to extract a commitment to the inner product $\mathbf{a}^\mathsf{T} \mathbf{s}$, where $\mathbf{a} = (a_i)_{i=0}^{d-1}$ is the vector to which $C$ is a commitment. The vector $\mathbf{s} \in \mathbb{F}_p^d$ is known to the verifier in the basic protocol, but later on we show how to hide this vector and simultaneously reduce the verifier's running time.
\begin{itemize}
\item Prover and Verifier flip $\mathbf{s}$ to obtain $\bar{\mathbf{s}} = (s_{d-1-i})_{i=0}^{d-1}$ and the matching integer encoding $z_{\bar{\mathbf{s}}} = \sum_{i=0}^{d-1} s_{d-1-i} q^i$.
\item Prover computes $C^{z_{\bar{\mathbf{s}}}}$ and sends this value to the verifier.
\item Prover and Verifier engage in a proof of correct exponentiation.
\item Prover and Verifier extract a commitment to coordinate $d$, which is exactly $\sum_{i=0}^{d-1} a_is_i$ modulo $p$.
\end{itemize}

Correctness, soundness, etc. (todo) Special attention for coefficient size.

Note that the Verifier must process all of $z_{\bar{\mathbf{s}}}$ in order to verify the exponentiation, which in particular is linear in $d$. However, it is possible to reduce this complexity and simultaneously hide the value of $z_{\bar{\mathbf{s}}}$. To do this, the prover must have committed to $z_{\bar{\mathbf{s}}}$ by sending $g^{z_{\bar{\mathbf{s}}}}$ (possibly with respect to a different base). At this point, a batched proof of knowledge of exponent establishes that the discrete logarithms of $C^{z_{\bar{\mathbf{s}}}}$ base $C$ and of $g^{z_{\bar{\mathbf{s}}}}$ base $g$ are equal.

\section{Illustration: QAP-based SNARK}

The next protocol describes an efficiently verifiable proof system for rank-one constraint satisfaction problems. Specifically, we start from a list of $m$ constraints of the form
\begin{equation} \label{equation:r1cs}
    \mathbf{a_i}^\mathsf{T} \mathbf{s} \times \mathbf{b}_i^\mathsf{T} \mathbf{s} = \mathbf{c_i}^\mathsf{T} \mathbf{s} \enspace ,
\end{equation}
where $\mathbf{s} \in \mathbb{F}_p^n$ is the secret witness and the $m$ triples $(\mathbf{a_i}, \mathbf{b_i}, \mathbf{c_i})_{i=0}^{m-1} \in \mathbb{F}_p^{3 \times m \times n}$ are the known parameters that define the constraints. Furthermore, $s_0 = 1$.

Translate this to a quadratic arithmetic program (QAP) by selecting $m$ arbitrary but different elements $\{e_0, \ldots, e_{m-1}\} \subset \mathbb{F}_p$ and defining $\mathbf{a}(x) \in \mathbb{F}^n[x]$ such that $\mathbf{a}(e_i) = \mathbf{a_i}$, and similarly for $\mathbf{b}(x)$ and $\mathbf{c}(x)$. Furthermore, set $h(x) = \prod_{i=0}^{m-1} (x-e_i)$. Then Equation~\ref{equation:r1cs} becomes
\begin{equation} \label{equation:qap_modular}
    \mathbf{a}(x)^\mathsf{T}\mathbf{s} \times \mathbf{b}(x)^\mathsf{T}\mathbf{s} \equiv \mathbf{c}(x)^\mathsf{T}\mathbf{s} \,\, \mathsf{mod} \,\, h(x) \enspace .
\end{equation}
Moreover, a prover knowledgeable of $\mathbf{s}$ can produce another polynomial $t(x)$ such that
\begin{equation} \label{equation:qap_explicit}
    \mathbf{a}(x)^\mathsf{T}\mathbf{s} \times \mathbf{b}(x)^\mathsf{T}\mathbf{s} = \mathbf{c}(x)^\mathsf{T}\mathbf{s} + t(x) \times h(x) \enspace .
\end{equation}

The proof establishes that the prover knows a vector $\mathbf{s}$ and a polynomial $h(x)$ such that Equation~\ref{equation:qap_explicit} is satisfied. Specifically:
\begin{itemize}
    \item Common input to Prover and Verifier: $A = \mathsf{vcom}(\mathbf{a}(x))$, $B = \mathsf{vcom}(\mathbf{b}(x))$, $C = \mathsf{vcom}(\mathbf{c}(x))$, $H = \mathsf{com}(h(x))$, and $D = \mathsf{vcom}((1 \, 0 \, \cdots \, 0)^\mathsf{T})$.
    \item Prover produces commitments $A_\mathbf{s} = \mathsf{com}(\mathbf{a}(x)^\mathsf{T} \mathbf{s})$, and similarly for $B_\mathbf{s}, C_\mathbf{s}$ with the inner product protocol of Section~\ref{section:inner_product}. Additionally, $D_\mathbf{s}$ is computed. All four inner product protocols are performed simultaneously, thereby establishing that the $\mathbf{s}$ used is the same in all four cases.
    \item Prover opens $D_\mathbf{s}$ to $1$, showing that $s_0$ is $1$.
    \item Prover multiplies $t(x)$ into $H$, thereby obtaining $H_t = \mathsf{com}(t(x) \times h(x))$.
    \item Prover and Verifier run a proof of knowledge of exponent.
    \item Prover multiplies $\mathbf{b}(x)^\mathsf{T} \mathbf{s}$ into $A_\mathbf{s}$, thereby obtaining $A_{\mathbf{s}B\mathbf{s}} = \mathsf{com}(\mathbf{a}(x)^\mathsf{T} \mathbf{s} \times \mathbf{b}(x)^\mathsf{T} \mathbf{s})$.
    \item Prover and Verifier run a proof of equal discrete logarithms showing that $A_{\mathbf{s}B\mathbf{s}}$ is to $A_\mathbf{s}$ as $B$ is to $g$.
    \item Verifier selects a random point, $z \xleftarrow{\$} \mathbb{F}_p$ and sends it to the prover.
    \item Prover and Verifier compute the weighted commitment $K = A_{\mathbf{s}B\mathbf{s}} \times C^{-1} \times H_T^{-1} = \mathsf{com}(\mathbf{a}(x)^\mathsf{T} \mathsf{s} \times \mathbf{b}(x)^\mathsf{T} \mathbf{s} - \mathbf{c}(x)^\mathsf{T} \mathbf{s} - h(x) \times t(x)) = \mathsf{com}(k(x))$
    \item Prover and Verifier run an evaluation proof establishing that $k(z) = 0$.
\end{itemize}

Note: we need to pay special attention to the size of the coefficients of $\mathbf{s}$ and of $t(x)$ are not too big. It is possible that the random selection of $z$ makes the prover who cheats by choosing larger coefficients overwhelmingly unlikely to succeed. Alternatively, we can devise a proof of small coefficients or something like that.

\section{Transparent Preprocessing SNARKs via Linear IOPs} 

\if 0 
All existing SNARK constructions can be viewed conceptually as consisting of an underlying information-theoretic statistically sound protocol that is then “cryptographically compiled” into one that achieves the desired efficiency properties (i.e. succinctness, non-interaction, etc) at the cost of \emph{computational soundness}. The information theoretic protocol is secure against unbounded provers whereas the compiled protocol is sound only against computationally bounded provers. In some cases zero-knowledge is also only achieved after compilation. This viewpoint has proved useful both as a modular method for constructing new proof systems as well as an analytical tool for classifying existing ones. 


\paragraph{CS proofs} The earliest construction of a succinct non-interactive argument system for NP, Micali’s ``CS proofs” \cite{CSproofs}, used random oracles and Merkle tree commitments to cryptographically compile classical PCPs via the Fiat-Shamir heuristic. In a PCP there is a verifier who has oracle access to a proof string and thus may query $q$ locations of the string in time O(q). The celebrated PCP theorem \cite{FOCS:ALMSS92} showed that any NP statement has a corresponding proof string of polynomial size, which the verifier only needs to check in O(1) locations in order to verify the statement with statistical soundness. In a CS proof, the first step is to build an interactive public-coin proof with succinct communication as first proposed by Kilian ~\cite{STOC:Kilian92} where the prover sends the verifier a Merkle tree commitment to the PCP string, receives the verifier’s public coin queries, and provides Merkle proofs to authenticate its answers to these queries. The second step is to make this non-interactive via Fiat-Shamir. However, this construction was purely theoretical due to the concrete inefficiency of classical PCPs. 

\paragraph{Short vs linear PCPs} These classical PCPs of polynomial length are called ``short PCPs''. Ishai, Kushilevitz, and Ostrovsky~\ref{CC:IKO07} gave the first communication efficient interactive argument that did not rely on commitments to short PCPs. The underlying information theoretic object in this construction is a \emph{linear} PCP, which is an oracle computing a linear function $\proofO: \FF^m \rightarrow \FF$, \emph{i.e.}, the answer to each query $\mathbf{q} \in \FF^m$ is the inner product $\langle \proofO, \mathbf{q} \rangle$. Their four-move succinct interactive argument uses linear homomorphic encryption to compile the linear PCP. The verification time in this construction is still linear and the prover time is quadratic due to the particular linear PCP instantiation based on Hadamard codes~\cite{FOCS:ALMSS92}. Gennaro, Gentry, Parno, and Raykova ~\cite{EC:GGPR13} were the first to present a concretely practical SNARK that reduced the prover time to $O(n log n)$ based on a more efficient instantiation of the linear PCP oracle, namely an encoding of the computation as a quadratic arithmetic program (QAP). The GGPR protocol (and followup improvements) were not initially described through the lens of linear PCPs, but were later adapted to this framework~\cite{TCC:BCIOP13,ES:SBVBPW13}. Bitansky~\emph{et al.}~\cite{TCC:BCIOP13} generalized this construction, showing how any linear PCP of a particular kind (QAPs being one example) could be combined with linear-only encodings to get a SNARK with sublinear verification time and linear time preprocessing. The preprocessing step in these constructions requires a trusted secret setup. 

\paragraph{IOPs} Interactive Oracle Proofs (IOPs)~\cite{TCC:BenChiSpo16,STOC:ReiRotRot16} combine IPs and PCPs: in each round of an IOP the verifier sends a message $m_i$ to the prover and the prover responds with a polynomial length proof oracle $\boldsymbol{\pi_i}$, which the verifier can query via random access. The verifier can continue to query this oracle in future rounds. In other words, each $\boldsymbol{\pi_i}$ is a PCP. Boneh~\emph{et al.} \cite{C:BBCGI19} introduced linear IOPs as the IOP extension of linear PCPs, where in each round the prover's message is a linear PCP oracle. The study of IOPs, and in particular interactive oracle proofs of proximity (IOPPs) based on Reed-Solomon codes, has led to more efficient SNARKs that extend the CS-proof paradigm ~\cite{ICALP:BBHR18}.

\fi 
%\paragraph{Linear IOPs} Another line of work … GKR based …. Can be viewed as linear IOP [IshaiCorrigan19]… In fact, linear IOPs capture all existing SNARK constructions, as they generalize both linear PCPs and short PCPs. Discuss STARK, Aurora, how it can be viewed as starting with a linear IOP where each proof oracle is a polynomial function and then turns it into a classical IOP by replacing each proof oracle with the evaluation of the polynomial at a linear number of points. This is so that they can apply weaker cryptographic compilers that don't require trusted setup (Merkle trees), but this underlying linear IOP could be compiled directly given a more advanced cryptographic compilation tool. 

%\paragraph{Our results} Present our polynomial commitment and inner product argument as cryptographic compilation techniques applied to linear IOPs. Introduce terminology of algebraic linear IOPs, where queries are derived by applying bounded-degree polynomials to verifier’s coins. Subclass of algebraic linear IOPs is Polynomial IOPs, where each oracle encodes a polynomial function of bounded degree and linear queries are all evaluations of the polynomial at a point. One way to see the connection to algebraic linear IOP is that each component of the query is derived via a bounded-degree monomial. 
%Explain a new view of QAP as one round linear IOP instead of linear PCP and how this yields a QAP-based SNARK without trusted setup. Provide general theorem for compiling linear IOPs of two kinds: algebraic linear IOPs (theorem generalizes QAP construction), Polynomial IOPs. In each give complexity of resulting preprocessing SNARK as a function of various parameters in the underlying linear IOP (number of rounds, etc). 

\subsection{Algebraic Linear IOPs} 

%In this section we define \emph{algebraic linear IOPs}.
An \emph{interactive oracle proof (IOP)}~\cite{TCC:BenChiSpo16,STOC:ReiRotRot16} is a multi-round interactive PCP: in each round of an IOP the verifier sends a message to the prover and the prover responds with a polynomial length proof, which the verifier can query via random access. A $t$-round $\ell$-query IOP has $t$ rounds of interaction in which the verifier makes exactly $\ell$ queries in each round. Linear IOPs~\cite{C:BBCGI19} are defined analogously except that in each round the prover sends a \emph{linear} PCP~\cite{CC:IKO07}, in which the prover sends a single proof vector $\proofO \in \FF^m$ %to some language membership assertion,
and the verifier makes $\emph{linear queries}$ to $\pi$. Specifically, the PCP gives the verifier access to an oracle that receives queries of the form $\mathbf{q} \in \FF^m$ and returns the inner product $\langle \proofO, \mathbf{q} \rangle$. 

Bitansky \emph{et al.} \cite{TCC:BCIOP13} defined a linear PCP to be of degree $(d_Q, d_V)$ if there is an explicit circuit of degree $d_Q$ that derives the query vector from the verifier's random coins, and an explicit circuit of degree $d_V$ that computes the verifier's decision from the query responses. %(An arithmetic circuit $C: \FF^\mu \rightarrow \FF^m$ has degree $d$ and arity $\mu$ if each $i$th component of the output is computed by an $\mu$-variate polynomial $p_i: \FF^\mu \rightarrow \FF$ of degree at most $d$.)
In a multi-query PCP, $d_Q$ refers to the maximum degree over all the independent circuits computing each query. Bitansky~\emph{et al.} called the linear PCP \emph{algebraic} for a security parameter $\lambda$ if it has degree $(\poly, \poly)$. The popular linear PCP based on \emph{Quadratic Arithmetic Programs} (QAPs) implicit in the GGPR protocol~\cite{EC:GGPR13} and follow-up works is an algebraic linear PCP with $d_Q \in O(m)$ and $d_V = 2$, where $m$ is the size of the witness.

For the purposes of the present work, we are only interested in the algebraic nature of the query circuit and not the verifier's decision circuit. Of particular interest are linear PCPs where each query-and-response interaction corresponds to the evaluation of a fixed $\mu$-variate degree $d$ polynomial at a query point in $\FF^\mu$. This description is equivalent to saying that the PCP is a vector of length $m = {d + \mu \choose \mu}$ and the query circuit is the vector of all $\mu$-variate monomials of degree at most $d$ (in some canonical order) applied to a point in $\FF^\mu$. We call this a $(\mu, d)$ \emph{Polynomial PCP} and define \emph{Polynomial IOPs} analogously. As we will explain, we are interested in Polynomial PCPs where $\mu \ll m$ because we can cryptographically compile them into succinct arguments using polynomial commitments, in the same way that Merkle trees are used to compile classical (point-) IOPs. %In fact, it is not important for the query point in $\FF^\mu$ to be random; it may be derived in a non-algebraic way from the verifier's random coins. Therefore, in order to view Polynomial PCPs as a special case of algebraic linear PCPs we must relax the definition: a linear PCP is $(\mu, d)$ algebraic if each query to a length $n$ oracle can be succinctly specified in $\FF^\mu$ and expanded into $\FF^n$ via a circuit $C : \FF^\mu \rightarrow \FF^n$ of degree $d$. 

In general, evaluating the query circuit for a linear PCP requires $\Omega(m)$ work. However, a general ``bootstrapping'' technique can reduce the work for the verifier: the prover expands the verifier's random coins into a full query vector, and then provides the verifier with a second PCP demonstrating that this expansion was computed correctly. It may also help to allow the verifier to perform $O(m)$ work in a one-time preprocessing stage (for instance, to check the correctness of a PCP oracle), enabling him to perform sublinear ``online'' work when verifying arbitrary PCPs later. We call this a \emph{preprocessing IOP}. In fact, we will see that any $t$-round $(\mu, d)$ algebraic linear IOP can be transformed into a $t+1$-round Polynomial IOP in which the verifier preprocesses $(\mu, d)$ Polynomial PCPs, at most one for each distinct query. 

%to be sublinear in $m$, a general strategy is to we would like to use the prover to derive the verifier's next query and provide a succinct \emph{succintly} prove   linear work. The arity and degree of the query circuit determine how easily the verifier can outsource affect how succinctly each query can be represented. Furthermore, the relevance of an algebraic query circuit is for succinctly representing the   of the Therefore, we will say that a linear PCP has degree $d$  

%are low (i.e. $\poly$) degree circuits that derive the query vector from the verifier's random coins and implement the verifier's decision algorithm based on the query responses. by an arithmetic circuit of low degree, i.e. each component of the query is the output of a polynomial function of 

% (i.e. $poly(\lambda)$). More precisely, in an algebraic degree $d$ PCP there is a query function $Q: \FF^\mu \rightarrow \FF^m$ such that each verifier query is of the form $\mathbf{q} \leftarrow Q(\mathbf{r})$, where $\mathbf{r}$ is uniformly sampled from $\FF^\mu$, and the function $Q$ is computable by a vector of $\mu$-variate polynomials of degree at most $d$. 

%A subclass of algebraic linear PCPs are \emph{polynomial PCPs}. A $\mu$-variate degree $d$ polynomial PCP oracle encodes a $\mu$-variate polynomial function of degree $d$, and all oracles queries return evaluations of the polynomial at points in $\FF^\mu$. If the proof oracle is represented by the coefficient vector $\proofO \in \FF^{d^\mu}$ then its evaluation on a point in $\FF^\mu$ can be viewed as a linear query $\mathbf{q} \in \FF^{d^\mu}$ (i.e. returning $\langle \proofO, \mathbf{q} \rangle$) with each component of $\mathbf{q}$ computed by a $\mu$-variate monomial of degree at most $d$. 

%We define algebraic linear IOPs analogously to algebraic linear PCPs. 

\paragraph{IOPPs and Polynomial IOPs} It turns out that with multiple rounds of interaction, it is also possible to implement Polynomial IOPs with classical IOPs, namely by using interactive oracle proofs of proximity (IOPPs)~\cite{STOC:ReiRotRot16,TCC:BenChiSpo16}. In particular, a univariate polynomial commitment scheme is implicit in the FRI (Fast Reed Solomon IOPP)~\cite{ICALP:BBHR18} protocol 
%and its improvement DEEP-FRI~\cite{ECCC:BGKS19}. 
This connection is also explicitly described in a recent preprint~\cite{MatterLabs}. Evaluating committed polynomials using these techniques results in an IOP with polylogarithmic communication complexity.\\

For completeness, we recall the formal definition of public-coin linear IOPs as well an algebraic restriction thereof. Since we are not interested in the possibly algebraic nature of the decision algorithm, we omit specifying the decision polynomial. From here onwards, we use algebraic linear IOP as shorthand for algebraic \emph{query} linear IOP. %Our results will only make use of public-coin IOPs.
%We relax the definition slightly to allow for non-uniform seed vectors. Specifically, in round $i$, the verifier's challenge $\mathit{coins}_i \in \{0,1\}^*$ is expanded into a query matrix $\mathbf{Q}_i$ in two steps. The first step generates a short query seed matrix $\boldsymbol{\sigma_i} \in \mathbb{F}^{\mu \times \ell}$, which is then mapped to via the circuit $C$ to the matrix $\mathbf{Q}_i$ in the second step. Only the circuit $C$ of the second step must have a low-degree polynomial representation for algebraic linear IOPs.

\begin{definition} [Public-coin linear IOP]
\label{def:linearIOP}
Let $\mathcal{R}$ be a binary relation and $\FF$ a finite field. A $t$-round $\ell$-query public-coin linear IOP for $\mathcal{R}$ over $\FF$ with soundness error $\epsilon$ and knowledge error $\delta$ and query length $\mathbf{m} = (m_1,...,m_t)$ consists of two stateful PPT algorithms, the \emph{prover} $\prover$, and the
\emph{verifier} $\verifier = (\qgen, \decider)$, where the verifier consists in turn of a public deterministic \emph{query generator} $\qgen$ and a
\emph{decision algorithm} $\decider$, that satisfy the following requirements:\\
 
\noindent \underline{Protocol syntax}. 
%There is a prover algorithm $P$, a query algorithm $Q$, and a verification algorithm $V$. 
For each $i$th round there is a prover state $\textsf{st}^\prover_i$ and a verifier state $\textsf{st}^\verifier_i$. For any common input $x$ and $\mathcal{R}$ witness $w$, at round 0 the states are $\textsf{st}^\prover_0 = (x, w)$ and $\textsf{st}^\verifier_0 = x$. 
In the $i$th round (starting at $i = 1$) the prover outputs a single\footnote{The prover may also output more than one proof oracle per round, however this doesn't add any power since two proof oracles of the same size may be viewed as a single (concatenated) oracle of twice the length.} proof oracle $\prover(\textsf{st}^\prover_{i-1}) \rightarrow \proofO_i \in \FF^{m_i}$. The verifier samples public random bits $\mathit{coins}_i \sample \{0,1\}^*$ and the query generator computes a query matrix from the verifier state and this randomness: $\qgen(\textsf{st}^\verifier_{i-1}, \mathit{coins}_i) \rightarrow \mathbf{Q}_i \in \mathbb{F}^{m_i \times \ell}$. The verifier obtains the linear oracle response vector $\proofO_i^\top \mathbf{Q}_i = \mathbf{a}_i \in \mathbb{F}^{1 \times \ell}$. The updated prover state is $\textsf{st}^\prover_i \leftarrow (\textsf{st}^\prover_{i-1}, \mathbf{Q}_i)$
and verifier state is $\textsf{st}^\verifier_i \leftarrow (\textsf{st}^\verifier_{i-1}, \mathit{coins}_i, \mathbf{a}_i)$
Finally, $\decider(\textsf{st}^\verifier_t)$ returns $1$ or $0$. \\ 

%\noindent \underline{Completeness}. For every $(x, w) \in \mathcal{R}$, if $(\prover, \verifier)$ follow the protocol then $\verifier(\textsf{st}^V_t)$ returns $1$ with probability 1. \\

%\noindent \underline{Soundness}. If $(x, w) \notin \mathcal{R}$ for every $w$, then for every prover algorithm $\prover^*$, the probability $\verifier(\textsf{st}^\verifier_t)$ returns $1$ is less than $\epsilon$. \\

%\noindent \underline{Knowledge}. There exists a PPT knowledge extractor $\mathcal{E}$ such that for any prover algorithm $\prover^*$ and every $x$, if $\verifier(\textsf{st}^\verifier_t)$ outputs 1 with probability greater than $\delta$ then $\mathcal{E}^{\prover^*}(x)$ outputs $w$ such that $(x, w) \in \mathcal{R}$ in expected polynomial time. $\mathcal{E}^{\prover^*}(x)$ receives the input $x$ and interacts with $\prover^*$ via rewinding, replaying with fresh verifier randomness, and recording the resulting transcripts.

%\alaninline{This notion $(\exists \mathcal{E} \forall \prover^*)$ is different from witness-extended emulation $(\forall \prover^* \exists \mathcal{E} \forall \adv)$. Is that by design?}

\noindent \underline{Argument of Knowledge.} As a proof system, $(\prover, \verifier)$ satisfies perfect completeness, soundness with respect to the relation $\mathcal{R}$ and with soundness error $\epsilon$, and witness-extended emulation with respect $\mathcal{R}$ with knowledge error $\delta$. \\

Furthermore, a linear IOP is \textbf{stateless} if for each $i \in [t]$, $\qgen(\textsf{st}^V_{i-1}, \mathit{coins}_i) = \qgen(i, \mathit{coins}_i)$.
It has \textbf{algebraic queries} if, additionally, for each $i \in [t]$, $\mathit{coins}_i$ is identifiable with a \emph{seed matrix} $\boldsymbol{\Sigma}_i \in \mathbb{F}^{\mu_i \times \ell}$ and the map $\boldsymbol{\Sigma}_i \xmapsto{\mathcal{Q}(i, \cdot)} \mathbf{Q}_i \in \mathbb{F}^{m_i \times \ell}$
is described by $\ell$ $\mu_i$-variate polynomial functions of degree at most $d = \poly$: $\vec{p}_1, \ldots, \vec{p}_\ell : \mathbb{F}^{\mu_i} \rightarrow \mathbb{F}^{m_i}$ such that for all $k \in [\ell]$, $p_k(\boldsymbol{\sigma}_{i,k}) = \mathbf{q}_{i,k}$, where $\boldsymbol{\sigma}_{i,k}$ and $\mathbf{q}_{i,k}$ denote the $k$th column of $\boldsymbol{\Sigma}_i$ and $\mathbf{Q}_i$, respectively.

%It is \textbf{input-oblivious} if $x$ is not included in the state passed to $Q$.
\end{definition} 


\begin{definition}[HVZK for public-coin linear IOPs]
Let $\textsf{View}_{\langle \prover(x, w), \verifier(x) \rangle}(\verifier)$ denote the view of the verifier in the $t$-round $\ell$-query interactive protocol described in Definition~\ref{def:linearIOP} on inputs $(x,w)$ with prover algorithm $\prover$ and verifier $\verifier$, consisting of all public-coin challenges and oracle outputs (this view is equivalent to the final state $\textsf{st}^\verifier_t$). The interactive protocol has \defn{$\delta$-statistical honest-verifier zero-knowledge} if there exists a probabilistic polynomial time algorithm $\mathcal{S}$ such that for every $(x, w) \in \mathcal{R}$, the distribution $\mathcal{S}(x)$ is $\delta$-close to $\textsf{View}_{\langle \prover(x, w), \verifier(x) \rangle}(\verifier)$ (as distributions over the randomness of $\prover$ and random public-coin challenges).
\end{definition}

We should like to define \emph{Polynomial IOP}s as a restriction of algebraic linear IOPs whereby in every round the $\ell$ $\mu_i$-variate polynomials $\vec{p}_1, \ldots, \vec{p}_\ell$ are identical and compute a vector of $m_i$ monomials in $\mathbf{X} = (X_1, \ldots, X_{\mu_i})$ in some canonical order. This restriction allows us to identify the verifier's queries $\boldsymbol{\sigma}_{i,k}$ with points in $\mathbb{F}^{\mu_i}$ and the linear oracle responses $\boldsymbol{\pi}_i^\mathsf{T} \boldsymbol{\sigma}_{i,k}$ as the value of a $\mu_i$-variate polynomial $\pi(\mathbf{X}) \in \mathbb{F}[\mathbf{X}]$ when evaluated in $\boldsymbol{\sigma}_{i,k}$. However, this natural restriction induces an annoying but important subtlety: it can make sense in some Polynomial IOPs for the verifier to make \emph{non-random} queries. For example, a verifier looking to check that a polynomial evaluates to zero in a given root will fail to ascertain this fact (with high probability) if he queries in a random point. The solution to this obstacle is to determine the seed matrix $\boldsymbol{\Sigma}_i$ not uniformly at random but as the output of some computation by the verifier. This non-uniform determination changes the relation between the verifier and the public query algorithm $\mathcal{Q}$. Previously, $\mathcal{Q}$ was the part of the verifier that produced the query; now $\mathcal{V}$ produces the query and $\mathcal{Q}$ is just a simple circuit that computes monomials and that both prover and verifier can evaluate. Moreover, as the following definition of Polynomial IOPs describes the query-response interactions in the language of polynomial evaluation rather than that of inner products between coefficient vector and monomial vector, there is no need to reference $\mathcal{Q}$ at all.

\begin{definition}[Polynomial IOP] 
Let $\mathcal{R}$ be a binary relation and $\FF$ a finite field. Let $\mathbf{X} = (X_1, \ldots, X_\mu)$ be a vector of indeterminates and let $m$ be the number of monomials in $\mu$ indeterminates of degree at most $d$. A $(\mu, d)$ Polynomial IOP for $\mathcal{R}$ over $\FF$ with soundness error $\epsilon$ and knowledge error $\delta$ consists of two stateful PPT algorithms, the \emph{prover} $\prover$, and the \emph{verifier} $\verifier$, that satisfy the following requirements:\\

\noindent \underline{Protocol syntax}. 
For each $i$th round there is a prover state $\textsf{st}^\prover_i$ and a verifier state $\textsf{st}^\verifier_i$. For any common input $x$ and $\mathcal{R}$ witness $w$, at round 0 the states are $\textsf{st}^\prover_0 = (x, w)$ and $\textsf{st}^\verifier_0 = x$. 
In the $i$th round (starting at $i = 1$) the prover outputs a single proof oracle $\prover(\textsf{st}^\prover_{i-1}) \rightarrow \pi(\mathbf{X}) \in \mathbb{F}[\mathbf{X}]$. The verifier computes a point list $\verifier(\st^\verifier_{i-1}) \rightarrow \boldsymbol{\Sigma}_i = (\boldsymbol{\sigma}_{i,1}, \ldots, \boldsymbol{\sigma}_{i, \ell}) \in \mathbb{F}^{\mu \times \ell}$. The verifier obtains the polynomial oracle response vector $(\pi_i(\boldsymbol{\sigma}_{i,1}), \ldots, \pi_i(\boldsymbol{\sigma}_{i, \ell})) = \mathbf{a}_i \in \mathbb{F}^{1 \times \ell}$. The updated prover state is $\textsf{st}^\prover_i \leftarrow (\textsf{st}^\prover_{i-1}, \boldsymbol{\Sigma}_i)$
and verifier state is $\textsf{st}^\verifier_i \leftarrow (\textsf{st}^\verifier_{i-1}, \boldsymbol{\Sigma}_i, \mathbf{a}_i)$. Finally, $\verifier(\textsf{st}^\verifier_t)$ returns $1$ or $0$. \\ 
%A public-coin $(\mu, d)$ Polynomial IOP is a public-coin algebraic linear IOP where in each query-response the inner product $\langle \boldsymbol{\pi}_i , \mathbf{q}_{i,k} \rangle$ corresponds to the evaluation of some polynomial $f \in \mathbb{F}[X_1, \ldots, X_\mu]$ of total degree at most $d$ in the seed matrix $\boldsymbol{\sigma_i}_k \in \mathbb{F}^{\mu}$ from which $\mathbf{q}_{i,k}$ was derived.

\noindent \underline{Argument of Knowledge.} As a proof system, $(\prover, \verifier)$ satisfies perfect completeness, soundness with respect to the relation $\mathcal{R}$ and with soundness error $\epsilon$, and witness-extended emulation with respect $\mathcal{R}$ with knowledge error $\delta$. \\

Furthermore, a Polynomial IOP is \defn{stateless} if for each $i \in [t]$, $\mathcal{V}(\st^\verifier_{i-1}) = \mathcal{V}(i)$. It is \defn{public coin} if, additionally, the randomized computation $\mathcal{V}(i) \rightarrow \boldsymbol{\Sigma}_i$ is equivalent to the derandomized computation $\mathit{coins}_i \sample \{0,1\}^*$ followed by $\mathcal{V}(i; \mathit{coins}_i) \rightarrow \boldsymbol{\Sigma}_i$ and where $\mathit{coins}_i$ is public and $\mathcal{V}(i; \mathit{coins}_i)$ is deterministic.
\end{definition}

It should be obvious that \emph{public coin} Polynomial IOPs are a subclass of public coin algebraic linear IOPs. The inclusion breaks, however, when the public coin property is dropped. We note that in this paper we characterize the underlying information-theoretical proof system for several argument systems for generating concise proofs as Polynomial IOPs and \emph{none} of them require secret coins\footnote{To be precise: we capture some trusted-setup SNARKs in our formalism, but the secret coins involved here are the result of the cryptographic compiler and not a property of the underlying IOP.}. The question remains whether Polynomial IOPs that are not public coin, have a compelling use case in practice.

\subsection{Polynomial IOP reductions} 

In this section we show that one can construct any algebraic linear IOP from a (multivariate) Polynomial IOP. This construction rests on two tools for univariate Polynomial IOPs that we cover first:
\begin{itemize}
    \item \emph{Coefficient queries.} The verifier verifies that an indicated coefficient of a Polynomial PCP has a given value.
    \item \emph{Inner products.} The verifier verifies that the inner product of the coefficient vectors of two Polynomial PCPs equals a given value.
\end{itemize}

\subsubsection{Coefficient queries}~\label{sec:opencoefficient} 
%For a polynomial $f \in \FF[X]$ let $f_i$ denote the $i$th coefficient. Given a Polynomial PCP $f$ of degree at most $d$ with coefficient vector $(f_1,...,f_d)$, 
The following is a $(1, d)$-Polynomial IOP for the statement $f_i = a$ with respect to a polynomial $f(X) = \sum_{j=0}^d f_j X^j$. %In the Polynomial IOP model each proof oracle sent to verifier is guaranteed to be a univariate polynomial of degree at most $d$; the verifier does not need to perform extra checks to ensure this.

\begin{itemize}

\item \emph{Prover}: Split $f(X)$ about the term $X^i$ into  $f_L(X)$ (of degree at most $i-1$) and $f_R(X)$ (of degree at most $d-i-1$) such that $f(X) = f_L(X) + a X^i + X^{i+1} f_R(X)$. Send polynomials $f_L(X)$ and $f_R(X)$. 

%The coefficients of $Compute commitments $c_R \leftarrow \pro{Commit}(\params, f_R(X))$ and $c_L \leftarrow \pro{Commit}(\params, f_L(X))$. Send $c_R$ and $c_L$ to the verifier. 

\item \emph{Verifier}: Sample uniform random  $\beta \sample \FF_p$ and query for $y_L \leftarrow f_L(\beta)$, $y_R \leftarrow f_R(\beta)$, and $y \leftarrow f(\beta)$. 
Check that $y = y_L + a \beta^i + \beta^{i+1} y_R \bmod p$ and return $0$ (abort) if not. Otherwise output $1$ (accept). 

%\item \emph{Prover}: Evaluate $y_R \leftarrow f_R(\beta)$, $y_L \leftarrow f_L(\beta)$, and $y \leftarrow f(\beta)$. Send $y_R, y_L, y$ to the verifier. 

%\item Prover and verifier run: 
%\begin {itemize} 
%\item  $\pro{Eval}(\params, c_R, \beta, y_R, d - i -1; f_R(X))$ 
%\item $\pro{Eval}(\params, c_L, \beta, y_L, i -1; f_L(X))$ 
%\item $\pro{Eval}(\params, c, \beta, y, d; f(X))$
%\end{itemize} 
%Verifier aborts and outputs $0$ if either subprotocol returns $0$. Otherwise it outputs $1$. 

\end{itemize}

%\textit{Knowledge extraction.} We need to show there is knowledge extractor that extracts $f$ with $i$th coefficient $a$.
The verifier only accepts given proof oracles for polynomials $f$, $f_L$, and $f_R$ in $\FF_p[X]$ of degree at most $d$, $i-1$ and $d-i-1$ such that $f(\beta) = f_L(\beta) + a\beta^i + \beta^{i+1} f_R(\beta)$ for random $\beta \sample \FF$. Via the Schwartz-Zippel lemma, if $f(X) \neq f_L(X) + aX^i + X^{i+1}f_R(X)$ then the verifier would accept with probability at most $d/|\FF|$, because the highest degree term in this equation is $X^{i+1} f_R(X)$ and its degree is at most $d$. This implies that $a$ is the $i$th coefficient of $f$. %The extractor obtains $f$ directly.

Note that this description assumes that the verifier is assured that the proof oracles for $f_L$ and $f_R$ have degrees $i-1$ and $d-i-1$, respectively. If no such assurance is given, then $f_L(X)$ should be shifted by $d-i+1$ digits. In particular, the proof oracle should $f_L^\star(X) = X^{d-i+1} f_L(X)$, in which case the verifier obtains the evaluation $y_L^\star = y_L \beta^{d-i+1}$ and tests $y = (\beta^{d-i+1})^{-1}y_L^\star + a \beta^i + \beta^{i+1} y_R$. This test admits false positives with probability at most $2d/|\mathbb{F}|$.

\subsubsection{Inner product}\label{sec:innerproduct}
The following is an IOP where the prover first sends two degree $d$ univariate polynomial oracles $f, g$ and proves to the verifier that $\langle \mathbf{f}, \mathbf{g}^r \rangle = a$ where $\mathbf{f}, \mathbf{g}$ denote the coefficient vectors of $f, g$ respectively and $\mathbf{g}^r$ is the reverse of $\mathbf{g}$. This argument is sufficient for our application to transforming algebraic linear IOPs into Polynomial IOPs. It is also possible to prove the inner product $\langle \mathbf{f}, \mathbf{g} \rangle$ by combining this IOP together with another one that probes the relation $g(X) = X^dg^r(X^{-1})$ in a random point $z \sample \mathbb{F} \backslash \{0\}$, and thereby shows that $\mathbf{g}$ and $\mathbf{g}^r$ have the same coefficients only reversed. We omit this more elaborate construction as it is not needed for any of our applications.

\begin{itemize}
\item \emph{Prover}: Sends proof oracles for $f(X)$, $g(X)$, and the degree $2d$ polynomial product $h(X) = f(X)\cdot g(X)$ to the verifier. 
\item \emph{Verifier}: Chooses $\beta \sample \FF$ and queries for $y_1 \leftarrow f(\beta)$, $y_2 \leftarrow g(\beta)$, and $y_3 \leftarrow h(\beta)$. Check that $y_1 y_2 = y_3$ and return $0$ (abort) if not.
\item Prover and verifier engage in the 1 round IOP (Section~\ref{sec:opencoefficient}) for proving that the $d$th coefficient (\emph{i.e.}, on term $X^d$) of $h(X)$ is equal to $a$. (Note that the proof oracles for this subprotocol can all be sent in the first round, so this does not add an additional round). %Abort and output $0$ if this fails, otherwise return 1 (accept). 
\end{itemize}

%\paragraph{Knowledge extraction}
Via Schwartz-Zippel, if $h(X) \neq f(X) \cdot g(X)$ then the verifier's check $y_1 y_2 = y_3$ at the random point $\beta$ fails with probability at least $(|\FF| - 2d)/|\FF|$. %The extractor directly obtains $h(X)$, and by definition the middle coefficient (\emph{i.e.}, on the monomial $X^d$) of $h(X)$ is the inner product $\langle \mathbf{f}, \mathbf{g}^r \rangle$. Combined with the the knowledge extraction property of the IOP for opening coefficients, it follows that this inner product is equal to $a$.
Observe that the middle coefficient of $h(X)$ is equal to $\sum_{i=0}^d f_i g_{d-i} = \sum_{i=0}^d f_i g^r_i = \langle \mathbf{f}, \mathbf{g}^r \rangle = a$.

\paragraph{Reducing algebraic linear IOPs to Polynomial IOPs} 
%The Polynomial IOP for inner products is a building block for the following theorem, which shows how to transform algebraic linear IOPs into Polynomial IOPs.
\label{sec:algebraicIOP}

\begin{theorem}\label{thm:algebraicIOPcompiler}
Any public-coin $t$-round stateless algebraic linear IOP can be implemented with a $t+1$-round Polynomial IOP. Suppose the original $\ell$-query IOP is $(\mu,d)$ algebraic with query length $(m_1,...,m_t)$ then the resulting Polynomial IOP has for each $i \in [t]$: $2\ell$ degree $m_i$ univariate oracles, $\ell$ pre-processed multivariate oracles of degree $d$ and $\mu+1$ variables, $\ell$ degree $2m_i$ univariate oracles %(for inner products), 
and $2\ell$ degree $2m_i$ univariate oracles. %(for coefficient openings). 
There is exactly one query to each oracle on a random point in $\FF$. The soundness loss of the transformation is $\negl$ for a sufficiently large field (\emph{i.e.}, whose cardinality is exponential in $\lambda$).
\end{theorem}

\begin{proof}
By definition of a $(\mu, d)$ algebraic linear IOP, in each $i$th round of the IOP there are $\ell$ query generation functions $\vec{p}_{i,1},\ldots,\vec{p}_{i,\ell}: \FF^\mu \rightarrow \FF^{m_i}$, where each $\vec{p}_{i,k}$ is a vector whose $j$th component is a $\mu$-variate degree-$d$ polynomial $p_{i,k,j}$. These polynomials are applied to a seed matrix $\boldsymbol{\sigma}_{i,k} \in \FF^\mu$ (which is identifiable with or derived from the verifier's $i$th round public-coin randomness $\mathit{coins}_i$); this evaluation produces $\vec{p}_{i,k}(\boldsymbol{\sigma}_{i,k}) = \mathbf{q}_{i,k} \in \mathbb{F}^{m_i}$ for all $k \in [\ell]$. The vectors $\mathbf{q}_{i,k}$ are the columns of the query matrix $\mathbf{Q}_i \in \mathbb{F}^{m_i \times \ell}$.

\paragraph{Preprocessed oracles} 
For each $i$th round, the prover and verifier preprocess $(\mu+1)$-variate degree-$d$ polynomial oracles. %(\emph{i.e.}, the prover sends these to the verifier and the verifier checks each at $(\mu+1)d+1$ points) for the the functions $P_1,\ldots,P_\ell$. 
%verifier and the verifier sends  uses $\Gamma$ to generate a commitment $c_{P_k}$ to $(p_{k,1},..., p_{k,m_i})$ as follows:
For each $k \in [\ell]$, the vector of polynomials $\vec{p}_{i,k} = (p_{i,k,1},\ldots,p_{i,k, m_i}) \in (\mathbb{F}[\mathbf{X}])^{m_i}$ with $\mathbf{X} = (X_1,\ldots,X_\mu)$ is encoded as a single polynomial in $\mu + 1$ variables as follows. Introduce a new indeterminate $Z$, and then define $\tilde{P}_{i,k}(\mathbf{X}, Z) := \sum_{j=1}^{m_i} p_{i,k,j}(\mathbf{X}) Z^j \in \mathbb{F}[\mathbf{X},Z]$.
The prover and verifier establish the oracle $\tilde{P}_{i,k}$, meaning that the verifier queries this oracle on enough points to be reassured that it is correct everywhere.

\paragraph{The transformed IOP} 
The original linear IOP is modified as follows. 

\begin{itemize}

\item Wherever the original IOP prover sends an oracle $\proofO_i$ of length $m_i$, the new prover sends a degree $m_i - 1$ univariate polynomial oracle $f_{\pi_i}$ whose coefficient vector is \emph{the reverse} of $\proofO_i$. 

\item Wherever the original IOP verifier makes $\ell$ queries within a round to a particular proof oracle $\proofO_i$, where queries are defined by query matrix $\mathbf{Q}_i \in \FF^{m_i \times \ell}$, consisting of column query vectors $(\mathbf{q}_{i,1},...,\mathbf{q}_{i,\ell})$, the new prover and verifier engage in the following interactive subprotocol for each $k \in [\ell]$ in order to replace the $k$th linear query $\langle \proofO_i, \mathbf{q}_{i,k} \rangle$: 

\begin{itemize}[nolistsep]
\item Verifier: Run the original IOP verifier to get the public coin seed matrix $\boldsymbol{\Sigma}_i$ and send it to the prover.
 \item Prover: Derive the query matrix $\mathbf{Q}_i$ from $\boldsymbol{\Sigma}_i$ using the polynomials $\vec{p}_{i,1}, \ldots, \vec{p}_{i, \ell}$. Send an oracle for the polynomial $F_{i,k}$ whose coefficient vector is $\mathbf{q}_{i,k}$. 
 \item Verifier: Sample uniform random $\beta \sample \FF$ and query both $F_{i,k}$ and $\tilde{P}_{i,k}$ (the $k$th preprocessed oracle for round $i$) at $\beta$ in order to check that $F_{i,k} (\beta) = \tilde{P}_{i,k}(\boldsymbol{\sigma}_{i,k}, \beta)$. If the check fails, abort and output 0.
 
% \item The prover evaluates $y_1 \leftarrow \mathbf{q}_k(\beta)$ and $y_2 \leftarrow \tilde{P}_k(\mathbf{r}_i, \beta)$ and sends these to the verifier. The verifier aborts if $y_1 \neq y_2$ and returns 0. 
% \item Assume at this point $y = y_1 = y_2$. The prover and verifier run $\eval(\params, c_{\mathbf{q}_k}, \beta, y, m_i; \mathbf{q}_k)$ and $\eval(\params, c_{P_k}, (\mathbf{r}, \beta), y, d; \tilde{P}_k)$. If the verifier returns 0 in either subprotocol, the verifier aborts and outputs 0. 
 
 \item Prover: Compute $a_{i,k} = \langle \proofO, \mathbf{q}_{i,k} \rangle$ and send $a_{i,k}$ to the verifier. 
 
 \item The prover and verifier run the inner product Polynomial IOP from Section~\ref{sec:innerproduct} on the oracles $F_{i,k}$ and $f_{\pi_i}$ to convince the verifier that $a_{i,k} = \langle \mathbf{q}_{i,k}, \proofO_i \rangle$. If the inner product subprotocol fails the verifier aborts and outputs 0.  
 \end{itemize}
 
\end{itemize}

If all substeps succeed, then the verifier obtains correct output of each oracle query; in other words, the responses are identical in the new and original IOP. These outputs are passed to the original verifier decision algorithm, which outputs 0 or 1.

\paragraph{Soundness and completeness} If the prover is honest then the verifier receives the same exact query-response pairs $(\mathbf{q}_{i,k}, a_{i,k})$ as the original IOP verifier and runs the same decision algorithm, and therefore the protocol inherits the completeness of the original IOP. As for soundness, an adversary who sends a polynomial oracle $F_{i,k}^*$ whose coefficient vector is \emph{not} $\mathbf{q}_{i,k}$, fails with overwhelming likelihood. To see this, note that since $\mathbf{q}_{i,k} = \vec{p}_{i,k}(\boldsymbol{\sigma}_k)$, the check that $F_{i,k} (\beta) = \tilde{P}_{i,k}(\boldsymbol{\sigma}_{i,k}, \beta)$ at a random $\beta$ fails with overwhelming probability by the Schwartz-Zippel lemma. Similarly, an adversary who provides an incorrect $a_{i,k}^* \neq \langle \proofO_i, \mathbf{q}_{i,k} \rangle$ fails the inner-product IOP with overwhelming probability.
%On the other hand, given an adversary who sends the honest $F_k$ in each round and succeeds with probability $\epsilon$, we derive an adversary who will succeed with the same probability $\epsilon$ in the original IOP.
Therefore, if the original IOP soundness error is $\epsilon$ then by a union bound the new soundness error is $\epsilon + \negl$. A similar composition argument follows for knowledge extraction.%  can be used to build an an adversary that that performs Therefore, any adversary who breaks soundness with non-negligible probability must send the honest $F_k$ in each query instance. It follows that this adversary can be used to build an adversary that breaks the soundness of the original IOP. 

\paragraph{Round complexity} The prover and verifier can first simulate the $t$-round original IOP  on the verifier's public-coin challenges, proceeding as if all queries were answered honestly. Wherever the original IOP prover would send an oracle for the vector $\proofO_i$ the prover sends $f_{\pi_i}$. Then, after the verifier has sent its final public coin challenge from the original IOP, there is one more round in which the prover sends all $F_{i, k}$ for the $k$th query vector in the $i$th round and all the purported answers $a_{i, k}$ to the $k$th query in the $i$th round. The prover and verifier engage in the protocol above to prove that these answers are correct. The inner product subprotocol for each $F_{i,k}$ with $f_{\pi_i}$ can be done in parallel with the check that $F_{i,k}(\beta) = \tilde{P}_{i,k}(\boldsymbol{\sigma}_{i,k}, \beta)$. Therefore, there is only one extra round.   
\end{proof}

\subsection{Compiling Polynomial IOPs} 
\label{subsec:compiling}
%We are now ready to present our main theorems. We formulate two separate theorems: the first pertains only to compiling Polynomial IOPs, and the second deals with more general stateless input-oblivious algebraic IOPs. The first result is more practical because it yields interactive arguments with quasi-linear prover time. In fact, there is a concrete instantiation of the Polynomial IOP (used in Sonic~\cite{Sonic}) that results in an interactive argument with both quasi-linear prover time and logarithmic communication/verification. The second result is less practical because it only guarantees polynomial prover time, but includes a much broader spectrum of concrete instantiations, including QAP-based IOPs. The prover time in a QAP-based instantiation is quadratic. These instantiations are discussed in more detail in Section~\ref{sec:instantiations}. \\

%\noindent \emph{Remark on $(\mu, d)$}: For simplicity, in our general theorems statements we consider $(\mu, d)$ Polynomial IOPs where every polynomial PCP oracle is a $\mu$-variate degree $d$ polynomial. This is without loss of generality because $(\mu, d)$ can be viewed as an upper bound on the variables and degree. Of course, when a Polynomial IOP involves a combination of oracles with different $\mu$ and $d$, the cryptographic compiler may indeed gain efficiency by taking advantage of this. We could formulate a more fine-grained theorem for $\{\mu_i, d_i\}$ Polynomial IOPs where the $i$th round PCP is a $\mu_i$-variate degree $d_i$ polynomial. 
%We similarly use a fixed $(\mu, d)$ upper bound on the variables/degree of the query generation polynomials in our theorem concerned with compiling algebraic linear IOPs. In Section~\ref{sec:instantiations} we discuss optimizations for concrete instantiations. 

%\subsubsection*{Compilation I: Polynomial IOP to IP} 
Let $\Gamma = (\pro{Setup}, \pro{Commit}, \pro{Open}, \pro{Eval})$ be a multivariate polynomial commitment scheme. Given any $t$-round Polynomial IOP for $\mathcal{R}$ over $\FF$, we construct an interactive protocol $\Pi = (\setup, \prover, \verifier)$ as follows. For clarity in our explanation, $\Pi$ consists of $t$ \emph{outer rounds} corresponding to the original IOP rounds and \emph{subrounds} where subprotocols may add additional rounds of interaction between outer rounds.
\begin{itemize}
\item $\setup$: Run $\params \leftarrow \pro{Setup}(1^\lambda)$
\item In any round where the IOP prover sends a $(\mu, d)$ polynomial proof oracle $\proofO: \FF^\mu \rightarrow \FF$, in the corresponding \emph{outer round} of $\Pi$, $\prover$ sends the commitment $c_{\proofO} \leftarrow \pro{Commit}(\params; \proofO)$
\item In any round where the IOP verifier makes an \emph{evaluation} query $\mathbf{z}$ to a $(\mu, d)$ polynomial proof oracle $\proofO$, in the corresponding \emph{outer round} of $\Pi$, insert an interactive execution of $\pro{Eval}(\params, c_\pi, \mathbf{z}, y, \mu, d; \proofO)$ between $\prover$ and $\verifier$, where $\proofO(\mathbf{z}) = y$. 
\end{itemize}

If $\verifier$ does not abort in any of these subprotocols, then it receives a simulated IOP transcript of oracle queries and responses. It runs the IOP verifier decision algorithm on this transcript and outputs the result.

\paragraph{Optimization: delayed evaluation} As an optimization to reduce round-complexity and enable batching techniques, all invocations of $\pro{Eval}$ can be delayed until the final round, and heuristically could be run in parallel. Delaying the evaluations until the final round does not affect our analysis. However, our analysis does not consider parallel execution of the $\pro{Eval}$ subprotocols. We assume the protocol transcript contains an isolated copy of each $\pro{Eval}$ instance and does not interleave messages or re-use randomness.
%We begin with a lemma on using polynomial commitments (that has witness-extended emulation) to compile a public-coin Polynomial IOP with negligible knowledge error into a public-coin interactive argument with witness-extended emulation. 

\begin{theorem}\label{thm:IOPcompiler}
If the polynomial commitment scheme $\Gamma$ has witness-extended emulation and the $t$-round Polynomial IOP for $\mathcal{R}$ has negligible knowledge error, then $\Pi$ is a public-coin interactive argument for $\mathcal{R}$ that has witness-extended emulation. The compilation also preserves HVZK if $\Gamma$ is hiding and $\eval$ is HVZK. 
\end{theorem}

The full proof is provided in \appendixphrase~\ref{sec:IOPcompilerproof}. 
HVZK is shown by a straightforward composition of the simulators for $\eval$ and the original IOP simulator. The emulator $E$ works as follows. Given the IP adversary $P'$, $E$ simulates an IOP adversary $P'_O$ by using the $\eval$ emulator $E_\eval$ to extract proof oracles (\emph{i.e.}, polynomials) from any commitment that $P'$ sends and subsequently opens at an evaluation point. We argue that $P'_O$ is successful whenever $P'$ is successful, with negligible loss. (The only events that cause $P'_O$ to fail when $P'$ succeeds is if $E_\eval$ fails to extract from a successful $\eval$ or $P'$ succesfully opens a commitment inconsistently with an extracted polynomial). $E$ then runs the IOP knowledge extractor with $P'_O$ to extract a witness for the input. 

%\ben{TODO perhaps provide some sketch here} 

\subsection{Concrete Instantiations} 
%We discuss concrete examples of Polynomial IOPs to which we can apply our polynomial commitment compiler.
We consider examples of Polynomial IOPs to which this compiler can be applied: \textsf{Sonic} and its improvement \textsf{PLONK}; \textsf{Spartan}~\cite{Spartan}, and the popular QAP of Gennaro~\emph{et al.}~\cite{EC:GGPR13}. 
\ifappendix
%To save space, the treatment of \textsf{Spartan} and QAP are deferred to Appendix~\ref{appendix:other_polynomial_iops}. We focus here on the \textsf{Sonic}/\textsf{PLONK} IOP.
\fi
%Section~\ref{sec:optimizations} discusses optimizations and performance estimates for a compilation of the Polynomial IOP introduced in \textsf{Sonic}~\cite{Sonic} and improved in \textsf{PLONK}~\cite{Plonk}. 

\subsubsection{Sonic} 
\textsf{Sonic} is zk-SNARK system that has a universal trusted setup, which produces a SRS of $n$ group elements that can be used to prove any statement represented as an arithmetic circuit with at most $n$ gates. The SRS can also be updated without re-doing the initial setup, for instance, to enable proving larger circuits, or to increase the distribution of trust. The result in \textsf{Sonic} was not presented using the language of IOPs. Furthermore, the result also relied on a special construction of polynomial commitments (a modification of Kate~\emph{et al.}~\cite{AC:KatZavGol10}) that forces the prover to commit to a Laurent polynomial with no constant term. Given our generic reduction from coefficient queries to evaluation queries (Section~\ref{sec:opencoefficient}), we re-characterize the main theorem of \textsf{Sonic} as follows: 

\begin{theorem}[Sonic Bivariate, \cite{Sonic}]
There exists a 2-round HVZK Polynomial IOP with preprocessing for any NP relation $\mathcal{R}$ (with arithmetic complexity $n$) that makes 1 query to a bivariate polynomial oracle of degree $n$ on each variable, and 6 queries to degree $n$ univariate oracles. The preprocessing verifier does $O(n)$ work to check the single bivariate oracle. 
\end{theorem}

The number of univariate queries increased from the original $3$ in \textsf{Sonic} (with special commitments) to $6$ with our generic coefficient query technique. 
%For uniform circuits\alan{As in, circuits that are the output of a polynomial-time Turing machine?}, the bivariate query is not necessary and can be evaluated efficiently by the verifier.\alan{This sentence does not make any sense to me.} 
If we were to compile the bivariate query directly using our multivariate commitment scheme this would result in $O(n^2)$ prover time (a bivariate polynomial with degree $n$ on each variable is converted to a univariate polynomial of degree roughly $n^2$). However, \textsf{Sonic} also provides a way to replace the bivariate polynomial with several degree $n$ univariate polynomials and more rounds of communication. 

\begin{theorem}[Sonic Univariate, \cite{Sonic}]\label{thm:sonic} 
There is a 5-round HVZK Polynomial IOP with preprocessing for any NP relation $\mathcal{R}$ (with arithmetic complexity $n$) that makes 27 queries overall to univariate degree $n$ polynomial oracles. The preprocessing verifier does $O(n)$ work to check $12$ univariate degree $n$ polynomials. 
\end{theorem}

The recent system \pro{PLONK} is an improvement on the underlying Polynomial IOP in \pro{Sonic}, and achieves the following: 

\begin{theorem}[PLONK,~\cite{Plonk}] 
There exists a 3-round HVZK Polynomial IOP with preprocessing for any NP relation $\mathcal{R}$ (with arithmetic complexity $n$) that makes $2$ queries overall to univariate degree $3n$ polynomial oracles. The preprocessing verifier does $O(n)$ work to check $9$ univariate degree $n$ polynomials.
\end{theorem}

Combining the \pro{Sonic} IOP with the new transparent polynomial compiler of Section~\ref{sec:protocol} gives the following result:  

%\ben{TODO: Add actual complexity for preprocessing and prover rather than quasilinear} 
\begin{theorem}
There exists an $O(\log n)$-round public-coin interactive argument of knowledge for any NP relation of arithmetic complexity $n$ that has $O(\log n)$ communication, $O(\log n)$ ``online" verification, quasilinear prover time, and a preprocessing step that is verifiable in quasilinear time. The argument of knowledge has witness-extended emulation assuming it is instantiated with a group $\GG$ for which the $r$-Strong RSA Assumption, and the Adaptive Root Assumption all hold. 
\end{theorem}
\begin{proof}
We apply the univariate polynomial commitment scheme from Section~\ref{sec:protocol} to the 5-round Polynomial IOP from Theorem~\ref{thm:sonic}. Denote this commitment scheme by $\Gamma = (\pro{Setup},$ $\pro{Commit}, \pro{Open}, \pro{Eval})$ 

The preprocessing requires running $\pro{Commit}$ on $12$ univariate degree $n$ polynomials, which involves a quasilinear number of group operations in the group of unknown order $\GG$ determined by $\pro{Setup}$. The prover sends a constant number of proof oracles of degree $n$ to the verifier, which also takes a quasilinear number of group operations. Finally, the 27 queries are replaced with at most $27$ invocations of $\eval$, which adds $O(\log n )$ rounds and has $O(\log n)$ communication. By Theorem~\ref{thm:polycommitsecurity} ($\Gamma$ has witness extended emulation) and Theorem~\ref{thm:IOPcompiler}, the compiled interactive argument has witness-extended emulation.
\end{proof}

%\ifappendix
%\else

\subsubsection{Spartan}
\textsf{Spartan}~\cite{Spartan} transforms an arbitrary circuit satisfaction problem into a Polynomial IOP based on an arithmetization technique developed by Blumberg~\emph{et al.} \cite{EPRINT:BTVW14}, which improved on the classical techniques of Babai, Fortnow, and Lund~\cite{BFL}. Specifically, satisfiability of a 2-fan-in arithmetic circuit on $n$ gates can be transformed into the expression: 
\begin{equation}\label{eqn:hypercubesum}
\sum_{x, y, z \in \{0,1\}^{\log n}} G(x, y, z) = 0
\end{equation} 
for a multilinear polynomial $G$ on $3 \log n$ variables over $\FF$. 
Furthermore, $G$ decomposes into the form: 
$$G(x,y,z) = A(x,y,z) F(x) + B(x, y, z) F(y) + C(x, y, z) F(y) F(z)$$
where $A, B, C,$ and $F$ are all multilinear poylnomials. The polynomials $A, B, C$ are derived from the arithmetic circuit defining the relation $\mathcal{R}$ and are input-independent. $F$ is degree $1$ with $\log n$ variables and is derived from a particular $(x, w) \in \mathcal{R}$. For uniform circuits, the verifier can evaluate $A, B, C$ locally in $O(\log n )$ time. The LFKN sum-check protocol~\cite{FOCS:LFKN90} is applied in order to prove Expression~\ref{eqn:hypercubesum} in a $3\log n$ round Polynomial IOP, where the prover's oracle consist of $Z$ and the low-degree polynomials sent in the sumcheck. Since the extra low-degree polynomials are constant size they can be read entirely by the verifier in constant time rather than via oracle access, and hence we ignore them in the total oracle count. The main result in Spartan can be summarized in our framework as follows: 

\begin{theorem}[Setty19]
There exists a $3 \log n$ round Polynomial IOP for any NP relation $\mathcal{R}$ computed by a \textbf{uniform} circuit with arithmetic complexity $n$, which makes three queries to a $\log n$-variate degree 1 polynomial oracle.  
\end{theorem}

Applying our multivariate compiler to the \textsf{Spartan} Polynomial IOP we obtain an $O(\log n)$-round public-coin interactive argument of knowledge for uniform circuits of size $n$. In our multivariate scheme (Section~\ref{sec:multivariate}), the $\log n$-variate degree 1 polynomial is tranformed into a univariate polynomial of degree $n$. With only three queries overall, the communication is just $6 \log n$ group elements and $6 \log n$ field elements. 

\subsubsection{QAPs} 

QAPs can be expressed as linear PCPs~\cite{TCC:BCIOP13,C:BCGTV13}. We review here how to express QAPs as a one round public-coin $(1, n)$ algebraic IOP. (This captures the satisfiability of any circuit with multiplicative complexity $n$, which is first translated to a system of quadratic equations over degree $n$ polynomials). Each linear query is computed by a vector of degree $n$ univariate polynomials evaluated at a random point chosen by the public-coin verifier. '

For illustration, will use the language \emph{satisfiability of rank-1 quadratic equations} over $\FF$ described by Ben-Sasson~\emph{et al.}~\cite{C:BCGTV13}. An instance of this language is defined by length $m+1$ polynomial vectors $A(X)$, $B(X)$, $C(X)$ such that the $i$th components $A_i(X)$, $B_i(X)$, $C_i(X)$ are all degree $n-1$ polynomials over $\FF_p[X]$ for $i \in [0,m]$, and $A_m = B_m = C_m$ is the degree $n$ polynomial $Z(X)$ that vanishes on a specified set of $n$ points in $\FF_p$. There is a length $m-1$ witness vector $\mathbf{w}$ whose first $\ell$ components are equal to the instance $\mathbf{x} \in \FF^\ell$, and a degree $n$ ``quotient" polynomial $H(X)$, such that the following constraint equation is satisfied: 
\begin{equation} \label{eqn:R1CS} 
\begin{split}
[(1, \mathbf{w}, \delta_1)^\top A(X)][(1, \mathbf{w}, \delta_2)^\top B(X)] 
- (1, \mathbf{w}, \delta_3)^\top C(X) = H(X)Z(X) \\ 
\ and \ (1,\mathbf{w})^\top (1,X,...,X^{\ell}, \mathbf{0}^{m- \ell -1}) = (1,\mathbf{x})^\top (1, X,...,X^{\ell})
\end{split} 
\end{equation} 
%\alan{What do $\delta_0, \delta_1, \delta_2$ do? Also, I'm not sure the dimensions work out.}

The deltas (i.e. $\delta_1, \delta_2, \delta_3$) are used as randomization terms for HVZK. 

\paragraph{QAP algebraic linear PCP} Equation~\ref{eqn:R1CS} is turned into a set of linear queries by evaluating the polynomials at a random point in $\FF$. Satisfaction of the equation evaluated at a random point implies satisfaction of the polynomial equation with error at most $2n / |\FF|$ by Schwartz-Zippel. Translated to an algebraic IOP, the prover sends a proof oracle $\proofO_w$ containing the vector $(1, \mathbf{w}, \delta_1, \delta_2, \delta_3)$ as well as a proof oracle $\proofO_h$ containing the coefficient vector of $H(X)$. A common proof oracle $\proofO_z$ is jointly established containing the coefficient vector of $Z(X)$. 

The verifier chooses a random point $\alpha \in \FF$ and makes four queries to $\proofO_w$, computed by the polynomial vectors $A(X), B(X), C(X)$ and $D(X) = (1, X,...,X^\ell, \mathbf{0}^{m- \ell -1})$. The verifier makes one query each to $\proofO_h$ and $\proofO_z$, which is the evaluation of $H(\alpha)$ and $Z(\alpha)$ respectively. The verifier obtains query responses $y_a, y_b, y_c, y_d, y_h, y_z$ and checks that $y_a \cdot y_b - y_c = y_h y_z$ and $y_d = \langle (1, \mathbf{x}), D(\alpha) \rangle$. 

\paragraph{Compiling QAP to public-coin argument} 

Following the compilation in Theorem~\ref{thm:algebraicIOPcompiler} (Section~\ref{sec:algebraicIOP}), the R1CS algebraic linear PCP can be transformed into a $2$-round Polynomial IOP. For simplicity, assume $m+3 < n$, where $m-1$ is the length of the witness and $n$ is the multiplicative complexity of the circuit. The preprocessing establishes three bivariate degree $n$ polynomials (\emph{i.e.}, encoding $A(X), B(X), C(X)$) and two univariate degree $n$ polynomials (\emph{i.e.}, encoding $Z(X)$ and $D(X)$). In the 2-round online phase the prover sends a degree $n$ univariate oracle for the witness vector $(1, \mathbf{w}, \delta_1, \delta_2, \delta_3)$, a degree $n$ univariate oracle for $H(X)$, four degree $n$ univariate oracles encoding linear PCP queries, four degree $2n$ univariate oracles encoding polynomial products, and eight degree $2n$ univariate oracles for opening inner products. The total number of polynomial oracle evaluation queries is $3$ bivariate degree $n$, $8$ univariate degree $2n$, and $7$ univariate degree $n$.

\begin{theorem}[QAP Polynomial IOP]
There exists a $2$-round Polynomial IOP with preprocessing for any NP relation $\mathcal{R}$ (with multiplicative complexity $n$) that makes $7$ queries to univariate degree $n$ oracles, $8$ queries to univariate degree $2n$ oracles, and $3$ queries to bivariate degree $n$ oracles.  
\end{theorem}
 
While theoretically intriguing, compiling the QAP-based IOP with our polynomial commitments of Section~\ref{sec:protocol} is less practical than compiling the \textsf{Sonic} IOP. While the R1CS Polynomial IOP has only $15$ univariate queries (compared to \pro{Sonic}'s $27$ queries to polynomials of approximately the same degree), the $3$ bivariate polynomial oracles take quadratic time to preprocess and open. Unfortunately, our polynomial commitment scheme does not take advantage of the sparsity of these bivariate polynomials. Furthermore, ignoring prover time complexity, the size of the bivariate $\eval$ proofs are twice as large as univariate $\eval$ proofs so the number of queries is effectively equivalent to $21$ univariate degree $n$ queries. 
%\fi




\section{Applications to Other Proof Systems}

\subsection{DARK-Sonic}

\subsection{DARK-Spartan}

\subsection{DARK-Bulletproofs}

\subsection{DARK-STARK}

\subsection{Comparison}

%\bibliography{cryptobib/references}
\bibliographystyle{alpha}
  \bibliography{cryptobib/crypto}

\appendix


\section{Protocols for Proving Permutations}


\subsection{Flip}

The following protocol establishes that two commitments, $c_a$ and $c_b$ represent polynomials $f_a, f_b \in \mathbb{F}_p$ (or vectors, for that matter) whose coefficients are flipped. Specifically, that $f_a = \sum_{i=0}^{d}f_i x^i$ for some coefficients $f_i$, and $f_b = \sum_{i=0}^df_ix^{d-i}$ for the same coefficients $f_i$.
Protocol:
\begin{itemize}
    \item Common knowledge: $c_a, c_b \in \mathbb{G}$.
    \item Verifier chooses $z \xleftarrow{\$} \mathbb{F}_p \backslash \{0\}$ and sends it to Prover.
    \item Prover and Verifier engage in proofs of correct evaluation producing $f_a(z)$ and $f_b(z^{-1})$, matching $c_a$ and $c_b$, respectively.
    \item Verifier checks that $f_a(z) \stackrel{?}{=} z^d f_b(z^{-1})$.
\end{itemize}

To see why it works, observe that $f_a(x) = x^df_b(x^{-1})$ and we can test this equation probabilistically by choosing a random $z \in \mathbb{F}_p \backslash \{0\}$ to evaluate $f_a$ and $f_b$ in. If $f_a$ is indeed the flipping of $f_b$ then the polynomial $F = f_a(x) - x^df_b(x^{-1})$ is identically zero; but otherwise it has at most $d$ zeros, and so the inequality will be exposed with overwhelming probability $(p-1-d)/(p-1)$.

\subsection{Rotation}

A similar observation gives rise to a proof of correct rotation. If $f(x) = \sum_{i=0}^d f_i x^i \in \mathbb{F}_p$ and $p(x) = \sum_{i=0}^d f_{i+r \, \mathsf{mod} \, d+1} x^i \in \mathbb{F}_p$ are polynomials consisting of the same coefficients but rotated by $r$ positions, then $p(x) = x^r f(x) \, \mathsf{mod} \, x^r - 1$ in all points. More explicitly, $p(x) = x^r f(x) + k(x) (x^r - 1)$ for some $k(x) \in \mathbb{F}_p$. The verifier can test this relation probabilistically.

\begin{itemize}
    \item Common knowledge: $c_f, c_p \in \mathbb{G}$ --- commitments to $f(x)$ and $p(x)$, respectively. Secret knowledge for the prover $f(x), p(x)$.
    \item Prover computes $k(x) = (p(x) - x^r f(x)) / x^r - 1$ and sends the commitment $c_k = g^{\hat{k}(q)}$ to it to Verifier.
    \item Verifier chooses a random point $z \xleftarrow{\$} \mathbb{F}_p$ and sends it to Prover.
    \item Prover and Verifier engage in proofs of correct evaluation for $f(z)$, $p(z)$, and $k(z)$.
    \item Verifier checks that $z^r f(z) + k(z) (z^r-1) = p(z)$.
\end{itemize}

\subsection{Generic Permutation}

The following protocol establishes that two polynomial commitments have the same coefficients but permuted according to a known permutation $\sigma : \{0,\ldots,d\} \rightarrow \{0,\ldots,d\}$. Specifically, $c_f$ is a commitment to $f(x) = \sum_{i=0}^d f_i x^i$ and $c_p$ is a commitment to $p(x) = \sum_{i=0}^d f_{\sigma(i)} x^i$. The proof makes use of the relation $p(x) = x^{\sigma(0)} f(x^{d+1}) - d(x)$ where $d(x) = \sum_{i=1}^d f_i (x^{i(d+1) + \sigma(0)} - x^{\sigma(i)})$. As $d(x)$ relies on the coefficients of $f(x)$, it is important to establish that $d(x)$ is correctly formed.

\begin{itemize}
    \item Common knowledge: $c_f, c_p, \sigma$ -- commitment to $f(x)$, commitment to $p(x)$, and permutation of coefficients.
    \item Secret knowledge for Prover: $f(x), p(x) \in \mathbb{F}_p$.
    \item Prover computes $n = \hat{f}(q^{d^2})$ and sends $c_n = g^n$ to Verifier.
    \item Prover and Verifier run an inner product protocol between $c_n$ and $(q^{di})_{i=0}^d$, and again between the result and $(q^i)_{i=0}^d$. This establishes that $n$ has the same coefficients as $c_f$ but spaced differently.
    \item Prover and Verifier run an inner product protocol between $c_n$ and $(q^{(i(d+1) + \sigma(0)} - q^{\sigma(i)})_{i=0}^d$ to compute $c_d$, the commitment to $d(x)$ that is well-formed wrt. $f(x)$.
    \item Verifier chooses a random point $z \xleftarrow{\$} \mathbb{F}_p$.
    \item Prover and Verifier engage in proofs of correct evaluation for $f(z), p(z), d(z)$.
    \item Verifier checks that $p(z) = z^{\sigma(0)} f(z^{d+1}) - d(z)$.
\end{itemize}


\end{document}
