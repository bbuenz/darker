\documentclass{article}
\usepackage[operators,lambda,keys,sets,primitives,adversary,asymptotics,advantage]{cryptocode}
\usepackage{notations}
\usepackage{mdframed}
\usepackage{enumitem}
\usepackage{amsmath,amsthm,amssymb}
\usepackage[utf8x]{inputenc}
\usepackage[colorinlistoftodos]{todonotes}

%Theorems
\newtheorem{definition}{Definition}
\newtheorem{theorem}{Theorem}
\newtheorem{lemma}{Lemma}
\newtheorem{assumption}{Assumption}
\newtheorem{fact}{Fact}
\newtheorem{corollary}{Corollary}

\newif\ifcomments
\commentstrue


\ifcomments
	\newcommand{\benedikt}[1]{{\textcolor{red}{[Benedikt: #1]}}}
	\newcommand{\alan}[1]{{\todo[color=blue!40!white]{Alan: #1}}}
	\newcommand{\alaninline}[1]{{\todo[color=blue!20!white, inline]{Alan: #1}}}
	\else
	\newcommand{\benedikt}[1]{}
	\newcommand{\alan}[1]{}
	\newcommand{\alaninline}[1]{}
	\fi

\begin{document}
\title{DARK Proof Systems}
\maketitle

\section*{Notes for Writing}
\begin{table}
    \caption{Notation}
    \label{tab:notation}
    \centering
    \begin{tabular}{l|l}
        symbol & meaning   \\ \hline \hline
        {\bf polynomials} & \\ \hline
        $f \in \mathbb{F}_p[x]$ & polynomial modulo $p$ \\
        $f_{(0)}, f_{(1)} \in \mathbb{F}_p[x]$ & first and second half of $f(x)$ \\
        $f_0, f_1, f_i \in \mathbb{F}_p$ & coefficients \emph{s.t.} $f(x) = \sum_{i=0}^d f_ix^i$ \\
        $\hat{f}(x) \in \mathbb{Z}[x]$ & polynomial with integer coefficients, typically in $\{0,\ldots,p-1\}$ \\ 
        $\hat{f}(q) \in \mathbb{Z}$ & integer encoding of a polynomial \\ \hline
        {\bf group elements} & (proposal by Alan) \\ \hline
        $\mathsf{g} \in \mathbb{G}$ & designated base element (the term ``generator'' is misleading) \\
        $\mathsf{c}, \mathsf{c}_0, \mathsf{c}_1 \in \mathbb{G}$ & commitments \\
        $\mathsf{c}_0^a \times \mathsf{c}_1^b, \mathsf{c}_0^a\mathsf{c}_1^b \, \textnormal{for} \, a, b \in \mathbb{Z}$ & multiplicative notation \\ \hline
%        {\bf schemes} & (proposal by Alan) {actually, I don't really like this anymore} \\ \hline
%        $\mathbf{Commit}_{\mathbb{A}} : \mathbb{A} \rightarrow \mathbb{G}$ & commit to an element from algebra $\mathbb{A}$. \\
%         & This notation allows stacking algebras, $\emph{e.g.}$, $\mathbf{Commit}_{(\mathbb{F}_p[x]_{\leq d})^n}$ \\
%        $\mathbf{Open}_\mathbb{A} : \{0,1\}^* \times \mathbb{G} \rightarrow \mathbb{A} \cup \{\bot\}$ & open a commitment by providing the decommitment information \\
%         & (in the basic scheme: an integer) and the commitment; if the pro- \\
%         & vided input is invalid, the function outputs $\bot$.\\
    \end{tabular}
\end{table}

\subsection*{Structure:} \alan{To be updated based on Benedikt's notes.}
\begin{itemize}
    \item introduction
    \item tools
    \begin{itemize}
        \item syntax of poly/vector commitment
        \item construction: poly/vector commitment
        \item eval (log / const size)
        \item coeff extract (log / const size)
        \item inner product (log/ const size)
        \item permutations: flip, rotate, generic
    \end{itemize}
    \item illustration: simple QAP
    \item proof systems
    \begin{itemize}
        \item short paragraph for each of:
        \item sonic
        \item spartan
        \item bulletproof
        \item stark
        \item other?
        \item[+] comparison
    \end{itemize}
\end{itemize}

\clearpage

\section{Introduction}

A polynomial commitment scheme enables a prover to bind himself to a polynomial in much less bandwidth than transmitting all coefficients would require. A skeptical verifier can subsequently test the commitment for certain algebraic relations as though he were in possession of the polynomial's full description, except at a much smaller work cost. Indeed, polynomial commitments lie at the heart of a host of efficiently verifiable interactive proof systems.

Of particular interest to this paper are proof systems whereby the prover establishes the correct performance of an arbitrary computation (that may or may not involve secret information) in such a way that the communication or verification complexity scales asymptotically better than performing the computation naïvely. Without exception, these proof systems rely on a technique called \emph{arithmetization}: characterizing the computation in question as a collection of arithmetic operations over a finite field. The utility of polynomial commitments stems from their capacity to succinctly capture a canonical representation of such collections while retaining the algebraic properties that make arithmetization work in the first place.

The literature on proof systems for arbitrary computations focuses on two techniques to achieve polynomial commitments. First: Merkle trees --- here every leaf represents the polynomial's evaluation in a given point, and the Merkle root represents the commitment to the polynomial. The verifier operates by opening selected points, which can be done in logarithmic space and time (as a function of the number of points). Second: groups equipped with bilinear maps --- in this case a structured reference string (SRS)\footnote{Previously known as \emph{common reference string}, CRS.} provides the values of all monomials up to a given degree when evaluated in an unknown point. By computing a weighted sum of these monomial values, the prover obtains the evaluation of his polynomial in the unknown point. The verifier performs the pairing operation to verify that multiplicative relations hold between committed polynomials.

This paper provides a third option for generating polynomial commitment schemes, namely by relying on groups of unknown order --- such as the group of integers with multiplication modulo an RSA modulus of unknown factorization, or the ideal class group of an order of an imaginary quadratic number field. These groups have seen relatively little adoption or even attention from the cryptographic community because the only known constructions thereof have subexponential attack algorithms. As a result, for a practical security level, elements of groups of unknown order typically require several hundreds of bytes to represent, in contrast to the tens of bytes needed for elements of elliptic curves for which no subexponential algorithms exist. 

Nevertheless, groups of unknown order provide a property that groups of known order, such as elliptic curve groups, cannot match: they enable homomorphic  commitments to an \emph{infinite} domain, namely the integers. Indeed, if the prover were capable of reducing a large integer to a smaller one without sacrificing the homomorphic properties, then he must know the group's order. The power of integer commitments was already noted by Lipmaa~\cite{} who characterizes proof systems arising therefrom as \emph{Diophantine} --- a reference to the family of languages for which such proof systems establish. Specifically, a set $S \subset \mathbb{Z}^n$ is called \emph{Diophantine} if it is the projection onto the first $n \leq m$ coordinates of the set of roots to a polynomial $P(x_1, \ldots, x_m) \in \mathbb{Z}[x_1, \ldots, x_m]$. Much more recently, Wesolowski produced a conceptually simple verifiable delay function (VDF) which builds on a proof of correct exponentiation in groups of unknown order. Building on this result, Boneh \emph{et al.} developed accumulators and vector commitments (with batch openings) from groups of unknown order~\cite{}. Both results have been received with great enthusiasm in the cryptocurrency community for their capacity to improve sustainability and scalability of blockchains.

\alaninline{Todo: \\
 - applications (trustless snarks etc) \\
 - implications (no unfalsifiable assumptions) \\
 - overview of techniques \\
 - related work}

\vspace{0.25cm}
\textsc{Contributions.} The contributions of this paper are divisible into three rubrics:
\begin{itemize}
    \item[] \textbf{Tools.} We start with an encoding scheme that represents polynomials over a prime field $\mathbb{F}_p$ as integers, by encoding the polynomial's coefficients into the integer's base-$q$ expansion. Adjoined with a group of unknown order and a designated base element $g \in \mathbb{G}$, this encoding scheme naturally gives rise to a polynomial commitment scheme that inherits its somewhat homomorphic properties. Next, we provide protocols for proving the correct evaluation of a committed polynomial, and showing that two polynomials have the same coefficients but flipped or rotated. We also present a protocol for extracting the $i$th coefficient, thereby promoting the commitment scheme to one that also provides vector commitment functionality. Building on this observation, we provide another protocol for showing that a commitment represents the inner product between two vectors of which at least one is represented by its vector commitment. Another protocol establishes that two vector commitments represent the same vector up to an arbitrary but known permutation of the coefficients.
    \item[] All the proof systems mentioned so far have logarithmic communication complexity and logarithmic verification time. Moreover, with the exception of the inner product proof and the permutation proof, the prover's complexity is quasi-linear. If one is willing to sacrifice this scalability for the prover, we also provide counterparts to all the above proofs with constant communication and verification complexity.
    \item[] \textbf{Applications.} To illustrate the usefulness and the versatility of the enumerated tools, we join them straightforwardly to construct a simple succinct non-interactive argument of knowledge (SNARK) based on quadratic arithmetic programs (QAPs). To the best of our knowledge, this is the first SNARK for circuits without trusted setup (when instantiated with the class group) or with an SRS whose size is independent of the circuit (when instantiated with the RSA group).\footnote{This classification takes note of the STARK proof system~\cite{C:BBHR19} whose verification time is polylogarithmic but as a function of some program's running time and not of any circuit; as well as of Hyrax~\cite{SP:WTTW17} and Spartan~\cite{eprint:Setty19}, which do apply to circuits but whose verification times are not polylogarithmic and thus fail to satisfy the definition of SNARKs as set forth in the paper that coined the term~\cite{JC:BCCGLRT17}.}
    \item[] We follow up this conceptually simple QAP-based proof system with a survey of popular communication-efficient proof systems for arbitrary computations, in which we replace their constituent components with tools developed earlier in this paper. In this light we analyze Sonic, Spartan, Hyrax, Bulletproofs, and STARK. In all cases we find that using our techniques leads to different trade-offs; improving on some metrics while degrading others.
\end{itemize}

\subsection{Related Work}

Homomorphic integer commitment schemes based on the RSA group were first proposed by Fujisaki and Okamoto~\cite{C:FujOka97}, who also provide a protocol to prove that a list of committed integers satisfy a polynomial equation modulo an arbitrary positive integer as well as one for opening a commitment bit by bit. Damgård and Fujisaki fix an issue with the soundness proof and are the first to suggest using class groups of an imaginary quadratic order as a candidate group of unknown order~\cite{AC:DamFuj02}. Lipmaa draws the link between zero-knowledge proofs constructed from integer commitment schemes and Diophantine complexity~\cite{AC:Lipmaa03b}. Couteau~\emph{et al.} study protocols derived from integer commitments specifically in the RSA group in order to lift their security proofs so as to require weaker assumptions; in the process they develop proofs for polynomial evaluation modulo a prime $\pi$ that is not initially known to the verifier, in addition to a proof showing that an integer $x$ lies in the range $\{a, \ldots, b\}$ by showing that $1+4(x-a)(b-x)$ decomposes as the sum of 3 squares~\cite{EC:CouPetPoi17}.

Pietrzak~\cite{Pietrzak18} adds efficient verifiability to the RSA-based time lock puzzle due to Rivest, Shamir, and Wagner~\cite{RivShaWag96}, thereby obtaining a conceptually simple verifiable delay function (VDF). Wesolowski~\cite{EC:Wesolowski19} proposes a single-round protocol to prove correct exponentiation in groups of unknown order, down from a logarithmic (in the size of the exponent) number for Pietrzak's protocol. Boneh~\emph{et al.} generalize this protocol to arbitrary exponents (not just powers of 2), and adapt it for zero-knowledge and batching, providing the base tools for constructing efficient accumulators and vector commitment schemes~\cite{C:BonBunFis19}.

\section{Preliminaries}
\paragraph{Notation}
\begin{itemize}
\item Let $f(x) \in \mathbb{F}_p[x]$ be a polynomial of degree at most $N-1$ where $N$ is a power of two. The coefficients of $f(x)$ are denoted by $f_i$ such that $f(x) \stackrel{\triangle}{=} \sum_{i=0}^{N-1} f_i x^i$.
\item Throughout, $p$ is a prime of at least $\lambda$ bits, and $q \in \mathbb{N}$ be an integer with $q \gg p$.
\item We work in a group $\mathbb{G}$ of unknown order together with one or many designated base elements $\gr{g}, \gr{h} \in \mathbb{G}$ with unknown order. (It might be tempting refer to these elements as \emph{generators} but that terminology would imply that $\mathbb{G}$ is cyclic, which is not necessarily true.) We use multiplicative notation and use \textsf{sans-serif} font to indicate group elements, as opposed to integers, polynomials, or field elements.
\item All protocols in this paper are between two parties, the prover and the verifier. We write $\boldsymbol{\it Protocol}(x;w) \rightarrow (y;z)$ to describe such a protocol with common input $x$, private input for the prover $w$, public output $y$, and private output for the prover $z$. We write $(y;z) \gets \boldsymbol{\it Protocol}(x;w)$ to denote the protocol's execution. Any of $x,w,y,z$ can consist of tuples of objects. This notation facilitates protocol composition and modular analysis.
\end{itemize}

\subsection{Assumptions}
%\alaninline{Fractional/pseudo root assumption is deprecated}
The cryptographic compilers make extensive use of groups of unknown order, \emph{i.e.}, groups for which the order cannot be computed efficiently.
Concretely, we require groups for which two specific hardness assumptions hold.
First, the strong RSA assumption which roughly states that it is hard to take \emph{arbitrary} roots of \emph{random} elements. Second, the much newer adaptive root assumption\cite{EC:Wesolowski19} which is the dual of the strong RSA assumption and states that it is hard to take \emph{random} roots of \emph{arbitrary} group elements. Both of these assumption hold in generic groups of unknown order\cite{C:Bach88}.
%In our proof we also rely on two a
%fractional root assumption which is a generalization of the strong RSA assumption. The assumption states that an adversary cannot compute fractional roots of random group elements. B\"unz, Boneh and Fisch \cite{journals/iacr/BonehBF18a} show that this assumption is satisfied in the generic group model.  
%\begin{assumption}[Pseudo root assumption]
%\label{assum:fracroot}
%The pseudo root assumption holds for $\ggen$ if for any efficient adversary $\adv$:
%\[        
%                \Pr\left[y^\beta = g^{\alpha} \wedge  \beta \not\vert~\alpha   : 
%                \begin{array}{l} 
%                      \GG \sample \ggen(\lambda) \\ 
%                      g \sample \GG \\
%                      (\alpha, \beta, y) \sample \adv(\GG, g) \\
%                      \quad \textnormal{where} \, \alpha, \beta \in \ZZ \, \textnormal{and} \, y \in \GG 
%                \end{array} 
%        \right] \leq \negl \enspace .
%\]
%\end{assumption}
\begin{assumption}[Strong RSA Assumption]
The \defn{strong RSA Assumption} sates that no efficient adversary can compute any root of a given random group element. Specifically, it holds for $\ggen$ if for any probabilistic polynomial time adversary $\adv$:
\[
    \advantage{SRA}{\adv}\deq 
    \Pr\left[\gr{u}^\ell = \gr{g} \, \wedge \, \ell > 1
    :
    \begin{array}{l}
         \GG \leftarrow \ggen(\lambda)  \\
         \gr{g} \sample \GG \\
         (\gr{u}, \ell) \in \mathbb{G} \times \mathbb{N} \leftarrow \adv(\mathbb{G}, \gr{g}) \\
    \end{array}\right] \leq \negl \enspace .
\]
\end{assumption}
We note that some definitions of the strong RSA assumption additionally require that $\ell$ be a prime~\cite{EC:BarPfi97,journals/iacr/BonehBF18a}. We follow that of Cramer and Shoup~\cite{CCS:CraSho99}.

\begin{assumption}[Adaptive Root Assumption]
\label{assum:adaptiveroot}
We say that the \defn{adaptive root assumption} holds for $\ggen$ if 
there is no efficient adversary $(\adv_0,\adv_1)$ that succeeds 
in the following task.
First, $\adv_0$ outputs an element $\gr{w} \in \GG$ and some $\state$.
Then, a random prime $\ell$ in $\primes$ is chosen
and $\adv_1(\ell,\state)$ outputs $\gr{w}^{1/\ell} \in \GG$.
More precisely, for all efficient $(\adv_0,\adv_1)$:
\[           \advantage{AR}{(\adv_0,\adv_1)}\deq 
                \Pr\left[\gr{u}^\ell = \gr{w} \neq 1 \ : \ 
                \begin{array}{l}
                      \GG \gets \ggen(\lambda) \\ 
                      (\gr{w},\state) \sample \adv_0(\GG) \\
                      \ell \sample \primes \\ 
                      \gr{u} \gets \adv_1(\ell, \state)
                \end{array} 
        \right] \leq \negl.
\]
\end{assumption}

\subsection{Commitment Schemes}

We adapt the definition of a commitment schemes to our protocol notation.

\begin{definition}[commitment scheme]
A \defn{commitment scheme} $\Gamma$ is a tuple $\Gamma = (\pro{Setup}, \pro{Commit}, \pro{Open})$ of protocols where
\begin{itemize}
    \item $\pro{Setup}(1^\lambda; \varnothing) \rightarrow (\params; \varnothing)$ generates public parameters $\params$;
    \item $\pro{Commit}(\varnothing; x) \rightarrow (c; d)$ commits to a message $x$ and outputs a commitment $c$ and decommitment information $d$;
    \item $\pro{Open}(c;d) \rightarrow (y; \varnothing)$ decommits $c$ with secret decommitment information $d$ to produce $y = x$ if the verifier's check was successful, or $y = \bot$ if not.
\end{itemize}
We say a commitment scheme $\Gamma$ is \defn{binding} if for all probabilistic polynomial time adversaries $\adv$,
\[
    \advantage{bnd}{\adv, \Gamma} \deq 
    \Pr\left[
        \bot \neq x_0 \neq x_1 \ : \
        \begin{array}{l}
             (\params; \varnothing) \gets \pro{Setup}(1^\lambda; \varnothing ) \\
             (c, d_0, d_1) \gets \adv(\params) \\
             (x_0; \varnothing) \gets \pro{Open}(c; d_0) \\
             (x_1; \varnothing) \gets \pro{Open}(c; d_1) \\
        \end{array}
    \right] \leq \negl \enspace .
\]
We say a commitment scheme $\Gamma$ is \defn{hiding} if for all probabilistic polynomial time adversaries $\adv=(\adv_0,\adv_1)$,
\[
    \advantage{dist}{(\adv_0, \adv_1), \Gamma} \deq
    \left|
        1 - 2\Pr\left[
            \hat{b} = b \ : \
        \begin{array}{l}
             (\params; \varnothing) \gets \pro{Setup}(1^\lambda; \varnothing ) \\
             (\state, x_0, x_1) \gets \adv_0(\params) \\
             b \sample \{0,1\} \\
             (c; \varnothing) \gets \pro{Commit}(\varnothing; x_b) \\
             \hat{b} \gets \adv_1(\state, c)
        \end{array}
        \right]
    \right| \leq \negl \enspace .
\]
\end{definition}

We now define a security games for polynomial commitments

\alaninline{Before we state the definitions of the security properties, it makes sense to introduce the syntax first. This syntax will probably be a point of ongoing discussion because this syntax has to be a) compatible with our protocol; and b) compatible with our adaptation of Sonic (and esp.\ its proof).}
\alaninline{Here's a point we can start with. There are two options for the syntax of opening: \\
1. The output of $\open$ is a single bit, depending on whether the provided inputs are valid. The opened polynomial should be one of the input arguments. \\
2. The output of $\open$ is either the opened polynomial itself (if all inputs are valid), or $\bot$ if the inputs are invalid. \\
While option (2) is less common, makes more sense for our scheme. The reason is because the prover needs to know the correct encoding of $f(x)$ as an integer in addition to $f(x)$ itself. Furthermore, this choice forces us to take explicit care of the integer encoding, which is a) beneficial for implementers, and b) more likely than the alternative to catch errors related to the integer encoding (overflow but also other types of errors).}

\alaninline{Kate \emph{et al.} define a polynomial commitment scheme as a tuple of 6 algorithms $(\mathsf{Setup},\mathsf{Commit},\mathsf{Open},\mathsf{VerifyPoly},\mathsf{CreateWitness},\mathsf{VerifyEval})$. There is a difference between verifying a commitment and opening it. The proof-of-correct-evaluation is a ``witness''. Proving and verifying are given by non-interactive algorithms.}
\alaninline{Sonic defines a polynomial commitment scheme as a tuple $(\mathsf{Commit}, \mathsf{Open}, \mathsf{pvC})$. The generation of the public parameters is not part of the polynomial commitment scheme. There is no special decommitment information; the $\mathsf{Commit}$ function takes a polynomial $f(x)$ and $\mathsf{Open}$ takes \emph{the same} polynomial. The algorithm $\mathsf{Open}$ outputs the evaluation $f(z)$ along with a proof; this proof is verified by $\mathsf{pcV}$.}

\begin{definition}[Polynomial Commitment see \cite{AC:KatZavGol10} ]
We say a polynomial commitment scheme, consisting of algorithms $(\setup,\commit)$ and the interactive protocols $\open$ and $\eval$ is secure interactive if it satisfies the following properties:
\paragraph{Correctness}
For an honest prover $\prover$ and a public coin honest verifier $\verifier$, is correct if for all polynomials $f(x)\in \ZZ_p[X]$ with degree polynomial in $\lambda$ and evaluation points $z\in \FF_p$:
	\[        
                \Pr\left[\begin{array}{c}\open_{\prover,\verifier}(\crs,f(x),C)=\text{"accept"}\\
                \eval_{\prover,\verifier}(\crs,C,z,y;f(x))=\text{"accept"}
                \end{array}  : 
                \begin{array}{l} 
                      \crs \sample \setup(\lambda, \deg(p(x)) \\
                      C\gets \commit(\crs,f(x))]\\
               z\sample \FF_p\\
                      y\gets f(z) \in \FF_p
                \end{array} 
        \right] =1.
\]
\benedikt{ok to mesh them together?}
\paragraph{Opening Binding}
\alaninline{This property is called ``polynomial binding'' in Kate \emph{et al.}}
We say a polynomial commitment scheme, consisting of algorithms $(\setup,\commit)$ and the interactive protocol $\open$, is opening binding if for all polynomial degrees $d=\poly\in \NN$ and for any efficient adversary $\adv$ and honest verifier $\verifier$:

	\[        
                \Pr\left[\begin{array}{c}\open_{\adv,\verifier}(\crs,C,f(x))=\text{"accept"}\\
                \wedge\\
           \open_{\adv,\verifier}(\crs,C,g(x))=\text{"accept"}\\
                \wedge\\
                f(x)\neq g(x) \vee \deg(f(x))\neq d\end{array}  : 
                \begin{array}{l} 
                      \crs \sample \setup(\lambda, d) \\
                      (C,z,f(x),g(x))\sample \adv(\crs)]
                \end{array} 
        \right] \leq \negl.
\]
\paragraph{Evaluation Binding}
	\[        
                \Pr\left[\begin{array}{c}\eval_{\adv,\verifier}(\crs,z,y_0,C)=\text{"accept"}\\
                \wedge\\
                \eval_{\adv,\verifier}(\crs,z,y_1,C)=\text{"accept"}\\
                \wedge\\
                y_0\neq y_1\end{array}  : 
                \begin{array}{l} 
                      \crs \sample \setup(\lambda, d) \\
                      (C,z,y_0,y_1)\sample \adv(\crs)]
                \end{array} 
        \right] \leq \negl.
\]
\end{definition}
Note that technically a complete $\eval$ protocol suffices as one can always open a polynomial by evaluating it at $\deg(f(x))+1$ points.

We additionally require a stronger extraction property that had in different variations been defined by \cite{SP:ZGKPP17} and \cite{EPRINT:MBKM19}.
\begin{definition}[PolyCommit extraction]
We say a polynomial commitment scheme, consisting of algorithms $(\setup,\commit)$ and the interactive protocol $\eval$, is extractable if there exists a rewinding \benedikt{Define more properly}extractor $\extractor$ such that for all polynomial degrees $d=\poly\in \NN$ and for any efficient adversary $\adv$:
	\[        
                \Pr\left[\begin{array}{c}\eval_{\adv,\verifier}(\crs,C,z,y)=\text{"accept"}\\
                \wedge\\
                f(x)\gets\extractor^{<\eval_{\adv,\verifier}(\crs,C,z,y)>}(\crs,C)\in \FF_p[X]\\
                \wedge\\
               \commit(f(x))\neq C\vee f(z)\neq y \vee \deg(f)\neq d
                 \end{array}  : 
                \begin{array}{l} 
                      \crs \sample \setup(\lambda, d) \\
                      \tau \sample (0,1)^\lambda\\
                      (C,z,y)\sample \adv_1(\crs,\tau)]
                      
                \end{array} 
        \right] \leq \negl.
\]
\end{definition}
\begin{lemma}[Extraction soundness]
	A polynomial commitment scheme that satisfies opening binding as well as extraction also satisfies evaluation binding.
\end{lemma}
\begin{proof}
	(SKETCH) Assume an adversary $\adv_{\eval}$ can break the evaluation binding property with non negligible probability $\gamma$. Using the extractor $\extractor$ we will construct an adversary $\adv_\open$ that will break the opening binding property with non-negligible probability. Run $\adv_\eval$ to get $(C,z,y_1,y_2)$. Now using $\ext$ we extract polynomials $f(x)$ and $g(x)$ of degree $d$ with all but negligible probability. Note that $f(z)=y_1$ and $g(z)=y_2$. This means $f(x)\neq g(x)$ but since $C=\commit(f(x))=\commit(g(x))$ we have a break of the opening binding property.
\end{proof}


\subsection{Proofs of Exponentiation}
Wesolowski \cite{EC:Wesolowski19} in his work on efficient verifiable delay functions introduced a simple yet powerful proof of exponentiation in groups of unknown order. A prover can efficiently convince a verifier that a large exponentiation in a group of unknown order was done correctly. The verifier needs to do little more than read the exponent. If the exponent is smooth then the protocol takes constant space/time in the security parameter.
\benedikt{Describe PoE protocol}
Boneh et al. \cite{journals/iacr/BonehBF18a} built on top of PoE to also develop PoKEs. \benedikt{Finish this section}
\section{Protocol}
\label{sec:protocol}

\subsection{Polynomial Encoding}

	Consider the set $B_{q/2}\subset \ZZ$ of integers with absolute value of less than $q/2$. $B_{q/2}[X]$ is the polynomial ring of that set such that the polynomials have bounded coefficients. $p(q) \in \ZZ$ for $p(x)\in P_q$ is a unique encoding of the polynomial:
\begin{itemize}
	\item Domain $B_{q/2}[X]\subset \ZZ[X]$, Alphabet: $\ZZ$
	\item 	$f(x):=\sum_{i=0}^{\lfloor\log_q(y)\rfloor} f_i x^i$
	\item $\enc(f(x) \in X)\rightarrow  f(q) \in \ZZ$
	\item $\dec(y \in \ZZ): f_i \in [0,q-1]\gets \frac{y \mod q^{i+1}-y \mod q^{i}}{q^i} \forall i \in [0,\lfloor\log_q(y)\rfloor]$
	\item Fix negative case
\end{itemize}

\begin{fact}
	The polynomial encoding scheme is uniquely decodable.
\end{fact}

Note that the encoding has limited homomorphic properties. $\enc(g(x))+\enc(h(x))=\enc(g(x)+h(x))$ if $g(x)+h(x)\in P_q$, i,.e. all its coefficients are less than $q/2$ in absolute value.\ This is ensured if for example the coefficients of $g$ and $h$ are less than $q/4$. Additionally $\enc(g(x))\cdot \enc(h(x))=\enc(g(x)\cdot h(x))$ if $g(x)\cdot h(x)\in P_q$.
\subsection{Polynomial Commitment}
 For now we assume that the degree $d$ is always a power of two. We describe the commitment scheme for integer polynomials with coefficients bounded in absolute value by $p$. The scheme naturally extends for polynomials in $\FF_p$.

\begin{mdframed}[userdefinedwidth=0.8\textwidth]
\begin{minipage}{\textwidth}
	\begin{flushleft}
	\setup(\secpar,p,d,n):
		\begin{enumerate}[nolistsep]
			\item $ \GG \sample \ggen(\secpar)$
			\item $ g \sample \GG$
			\item $q\gets 2^k \text{s.t.} q>(d+1) \cdot 2\cdot p^{\log_2(d+1)+1} $
			\item $\pcreturn \crs=\{\secpar,p,\GG,g,q\}$
		\end{enumerate}
		$\commit(\crs,f(x)\in \ZZ[X])$ \pccomment{$f(x)$'s coefficients are \textcolor{blue}{in $[0,p-1]$}} 
		\begin{enumerate}[nolistsep]
			\item 	$f(q)\gets \enc(f(x)) \in \ZZ$
			\item \textcolor{blue}{$C \gets g^{f(q)}$}
			\item $\pcreturn \textcolor{blue}{C},f(q)$ \textcolor{blue}{Alan: why is including the degree necessary?}\benedikt{Probably commit should have a public and private output. I changed it but we need to think about this.}
		\end{enumerate}
		$\open(\crs,\textcolor{blue}{C},f(x)\in \ZZ[X])$ \pccomment{\textcolor{blue}{$f(x)$'s coefficients are in $[-q/2,q/2]$}} 
		\begin{enumerate}[nolistsep]
			\item $f(q)\gets \enc(f(x)) \in \ZZ$
			\item Prover sends $f(q),f(x)$ to verifier.
			\item Verifier checks that $\dec(f(q)) \mod p=f(x)$
			\item If $C=g^{f(q)}$ the verifier accepts.
		\end{enumerate}
	\end{flushleft}
	
\end{minipage}
\end{mdframed}

\begin{small}
 \begin{minipage}{1.1\textwidth}
\begin{mdframed}[userdefinedwidth=1\textwidth]  \label{prot:Opening}
	\noindent \underline{\textsf{Protocol \eval{} Information theoretic} (Polynomial evaluation)}\\
Inputs: $\enc(f(x)),z,y\in \FF_p,d \in \NN $\\
Witness: $f(x) \in \ZZ[X]$ of degree at most $d$ and with coefficients bounded by $p$ \textcolor{blue}{Alan: if it is bounded by $p$ then you cannot recurse, because the coefficients grow every step.}\footnote{We describe the protocol using an integer polynomial with bounded coefficient. This generalizes the case where the polynomial is in $\FF_p[X]$};\\ 
Claim: $f(x)\gets \dec(\enc(f(x))), y=f(z) \mod p$

\begin{enumerate}[nolistsep]
\item \pcif $d=0$:
\item \pcind[1] Prover sends $\hat{y}\gets \enc(f(x))$ to the verifier. 
\item \pcind[1] Verifier runs that $f(x)\gets \dec(\hat{y})$ and checks that  $f(z) \bmod p=y$, or rejects otherwise 
%\textcolor{blue}{Alan: not in the exponent of $p$?}, 
%right
\item \pcelse: 
\item \pcind[1] $d'\gets \frac{d+1}{2}-1$
\item \pcind[1] Prover computes $f_0(x)\gets\sum_{i=0}^{d'} f_i x^i\in \ZZ[x]$ and $f_1(x)\gets\sum_{i=0}^{d'} f_{d'+1+i} x^{i}\in \ZZ[x]$
\item \pcind[1] $y_0\gets f_0(z) \bmod p$, $y_1\gets f_1(z)\bmod p$
\item \pcind[1] Prover sends $y_0,y_1, \enc(f_0(x)),\enc(f_1(x))$ to the Verifier
\item \pcind[1] Verifier checks that $y_0+z^{d'+1} y_1=y\in \FF_p$ 
\item \pcind[1] Verifier checks that $\enc(f_0(x))+x^{d'+1}\cdot \enc(f_1(x))=\enc(f(x))$
\item \pcind[1] Verifier samples $\alpha \sample \FF_p$ and sends it to the prover
\item \pcind[1] Prover and Verifier compute $y'\gets\alpha y_0 +y_1 \bmod p$, $\enc(f'(x))\gets \alpha \cdot \enc(f_0(x)) +\enc(f_1(x))$. 
\item \pcind[1] Prover also computes $f'(x)\gets\alpha  \cdot f_0(x)+f_1(x) \in \ZZ[x]$ 
\item \pcind[1] Prover and Verifier run $\eval(\enc(f'(x),z,y',d';f'(x))$
\end{enumerate}
\end{mdframed}
\end{minipage}
\end{small}

\begin{small}
 \begin{minipage}{1.1\textwidth}
\begin{mdframed}[userdefinedwidth=1\textwidth]  \label{prot:Opening}
	\noindent \underline{\textsf{Protocol \eval} (Polynomial evaluation)}\\
\noindent Params: $\crs=\{\secpar,p,\GG,g,d_{\max},q\}$;\ \
Inputs: $C\in \GG,z,y\in \FF_p,d \in \NN $\\
Witness: $f(x) \in \ZZ[X]$ of degree at most $d$ and with coefficients bounded by $p$ \textcolor{blue}{Alan: if it is bounded by $p$ then you cannot recurse, because the coefficients grow every step.}\footnote{We describe the protocol using an integer polynomial with bounded coefficient. This generalizes the case where the polynomial is in $\FF_p[X]$};\\ 
Claim: $C=g^{f(q)},\deg(f)\leq d$ and $y=f(z) \mod p$

\begin{enumerate}[nolistsep]
\item \pcif $d=0$:
\item \pcind[1] Prover sends $f(x)$ to the verifier, $f(x)$ is a constant. 
\item \pcind[1] Verifier checks that $0\leq f(q)<  p^{\log_2(d+1)+1}$,  $f(z) \bmod p=y$ and $g^{f(q)}=C$, or rejects otherwise 
%\textcolor{blue}{Alan: not in the exponent of $p$?}, 
%right
\item \pcelse: 
\item \pcind[1] $d'\gets \frac{d+1}{2}-1$
\item \pcind[1] Prover computes $f_0(x)\gets\sum_{i=0}^{d'} f_i x^i\in \ZZ[x]$ and $f_1(x)\gets\sum_{i=0}^{d'} f_{d'+1+i} x^{i}\in \ZZ[x]$
\item \pcind[1] $y_0\gets f_0(z) \bmod p$, $y_1\gets f_1(z)\bmod p$, $C_0\gets g^{f_0(q)}$,$C_1\gets g^{f_1(q)}$
\item \pcind[1] Prover sends $y_0,y_1,C_0,C_1$ to the Verifier
\item \pcind[1] Verifier checks that $y_0+z^{d'+1} y_1=y\in \FF_p$ 
\item \pcind[1] Prover and Verifier engage in $\textsf{PoE}(C_1,C/C_0,q^{d'+1})$ to show that $C_0C_1^{(q^{d'+1})}=C$
\item \pcind[1] Verifier samples $\alpha \sample \FF_p$ and sends it to the prover
\item \pcind[1] Prover and Verifier compute $y'\gets\alpha y_0 +y_1 \bmod p$, $C'\gets C_0^{\alpha}C_1 \in \GG$. 
\item \pcind[1] Prover also computes $f'(x)\gets\alpha  \cdot f_0(x)+f_1(x) \in \ZZ[x]$ 
\item \pcind[1] Prover and Verifier run $\eval(C',z,y',d';f'(x))$
\end{enumerate}
\end{mdframed}
\end{minipage}
\end{small}

\subsection{Discussion and Optimizations}
\begin{enumerate}
	\item Precompute $g^{q^i}$ to use parallelism etc.
	\item SNARK without falsifiable assumptions
	\item First polynomial commitment scheme with constant size parameters
	\item Instead of sending $C_0$ and $C_1$ the prover can simply send $C_0$ and compute $C_1=C/(C_0^{q^{\frac{d}{2}}})$. Using a non-interactive PoE this means that $C_1=C/(Q^{\ell}g^r)$
\end{enumerate}

\subsection{Multivariate Commitment}
  We extend the Polynomial Commitment scheme to handle multivariate polynomials. The idea is to simply use higher degrees of $q$ to encode the polynomial. The protocol is linear in the number of variables and logarithmic in the total degree of the polynomial. For simplicity we only present a protocol for $n$-variate polynomials where the degree in each variable is $d$ and $d$ is a power of $2$. The protocol naturally extends to different degrees per variable.
We first define an encoding for multi-variate integer polynomials with small coefficients:
 \begin{itemize}
	\item Domain $B_{q/2}[X^n] \subset \ZZ[X^n]$, Alphabet: $\ZZ$, \\$f(x_0,\dots,x_{n-1})=\sum_{\vec{i} \in [0,d]^n} f_{\vec{i}} \prod_{j=1}^{n} x_j^{i_j}$
	\item $\enc(f(x_0,\dots,x_{n-1}) \in B_{q/2}[X^n]): f(q,q^{d+1},q^{(d+1)^2},\dots,q^{(d+1)^{n-1}})$
	\item $\textsf{Q-ary}(\vec{i}):\sum_{j=1}^{n} i_j\cdot (d+1)^{j-1}$
	\item $\dec(y \in \ZZ): f_{\vec{i}}=\frac{y \bmod q^{\textsf{Q-ary}(\vec{i})+1}-y \bmod q^{\textsf{Q-ary}(\vec{i})}}{q^{\textsf{Q-ary}(\vec{i})}} \forall \vec{i} \in [0,d]^n$
	\item \benedikt{Negative Coefficients?}

\end{itemize}
\begin{small}
 \begin{minipage}{1.1\textwidth}
\begin{mdframed}[userdefinedwidth=1\textwidth]  \label{prot:Opening}
	\noindent \underline{\textsf{Protocol \eval} (Polynomial evaluation)}\\
\noindent Params: $\crs=\{\secpar,p,\GG,g,d_{\max},n_{\max},q\}$;\ \
Inputs: $C\in \GG,z_1,\dots,z_n,y\in \FF_p,d \in \NN,n \in \NN $\\
Witness: $f(x_1,\dots,x_n) \in B_q[X^n]$ of degree at most d and with coefficients bounded by $p$\footnote{We describe the protocol using an integer polynomial with bounded coefficient. This generalizes the case where the polynomial is in $\FF_p[X^n]$};\\ 
Claim: $C=g^{f(q_1,q_2,\dots,q_n)}$ where $q_i=q^{(d_{\max}+1)^{i-1} },\deg(f)\leq d$ and $y=f(z_1,\dots,z_n) \mod p$

\begin{enumerate}[nolistsep]
\item \pcif $n=0 \wedge d=0$:
\item \pcind[1] Prover sends $f(x)$ to the verifier, $f(x)$ is a constant. 
\item \pcind[1] Verifier checks that $0\leq f(q)< p^{n_{\max}\cdot \log_2(d_{\max}+1)+1}$,  $f(z) \bmod p=y$ and $g^{f(q)}=C$ otherwise rejects
\item \pcelse: 
\item \pcind[1] $d'\gets \frac{d+1}{2}-1$
\item \pcind[1] Prover computes $f_0(x_1,\dots,x_n)\gets \sum_{\vec{i} \in [0,d_{\max}]^{n-1}}  \prod_{j=1}^{n-1} x_{j}^{i_j} \cdot(\sum_{k=0}^{d'} f_{\vec{i},k} x_{n}^k)\in \ZZ[X^n]$ and
$f_1(x_1,\dots,x_n)\gets \sum_{\vec{i} \in [0,d_{\max}]^{n-1}} \prod_{j=1}^{n-1} x_{j}^{i_j}(  \sum_{k=0}^{d'} f_{\vec{i},k+d'+1} x_{n}^{k})\in \ZZ[X^n]$
\item \pcind[1] $y_0\gets f_0(z_1,\dots,z_n) \bmod p$, $y_1\gets f_1(z_1,\dots,z_n)\bmod p$,\\ $C_0\gets g^{f_0(q_1,\dots,q_n)}$,$C_1\gets g^{f_1(q_1,\dots,q_n)}$
\item \pcind[1] Prover sends $y_0,y_1,C_0,C_1$ to the Verifier
\item \pcind[1] Verifier checks that $y_0+z_n^{d'+1} y_1=y\in \FF_p$ 
\item \pcind[1] Prover and Verifier engage in $\textsf{PoE}(C_1,C/C_0,q_n^{d'+1})$ to show that $C_0C_1^{q_n^{d'+1}}=C$
\item \pcind[1] Verifier samples $\alpha \sample \FF_p$ and sends it to the prover
\item \pcind[1] Prover and Verifier compute $y'\gets\alpha y_0 +y_1 \bmod p$, $C'\gets C_0^{\alpha}C_1 \in \GG$. \\Prover also computs $f'(x_1,\dots,x_n)\gets\alpha  \cdot f_0(x_1,\dots,x_n)+f_1(x_1,\dots,x_n) \in \ZZ[X^n]$ 
\item \pcind[1] \pcif $d=0$ and $n>0$: \pccomment{$f'$ is constant in $x_n$ so we write it as an $n-1$ variate polynomial.}
\item \pcind[2] $d'\gets d_{\max}$
\item \pcind[2] $n'\gets n-1$
\item \pcind[2] $\vec{z}\gets (z_1,\dots,z_{n-1})$
\item \pcind[1] Prover and Verifier run $\eval(C',\vec{z},y',d',n';f'(x))$
\end{enumerate}
\end{mdframed}
\end{minipage}
\end{small}

\subsection{Fix coefficients and negative degrees}
\begin{itemize}
	\item Several ways to do this.
\end{itemize}

\subsection{Range Proofs}

\section{Security}

\begin{lemma}
	The pseudo root assumption holds in the generic group model.
\end{lemma}
\begin{lemma}
	The polynomial commitment scheme satisfies the opening binding property
\end{lemma}
\begin{proof}
	Assume $\hat{f}(x)$ and $\hat{f}'(x)$ are integer encodings of two distinct polynomials $f(x),f'(x) \in \FF_p[X]$. Assume that the adversary can open a commitment $C$ to both $f(x)$ and $f'(x)$. We show that we can use this adversary to break the pseudo root assumption (Assumption \ref{assum:fracroot}).
	
	 Then $\hat{h}(x)=\hat{f}(x)-\hat{f}'(x) \in \ZZ[X]$ is a polynomial of degree at most $d$. Since $g^{\hat{f}(q)}=g^{\hat{f}'(q)}=C$ we have that $g^{\hat{h}(q)}=1$. Note that the coefficients of $\hat{f}(x)$ and $\hat{f}'(x)$ are all less than $q/2$ in absolute value. By triangle inequality we have that $\hat{h}(x)$'s coefficients are less than $q$. $\hat{h}$ is by assumption not the zero polynomial. This implies that $\hat{h}(q)\neq 0$ is a multiple of the order of $g\in \GG$. This however can directly be used to break the adaptive root assumption (Assumption \ref{assum:adaptiveroot}) by taking the inverse of $\ell$ modulo $\hat{h}(q)$.
\end{proof}

\begin{theorem}
	The polynomial commitment scheme from Section \ref{sec:protocol} satisfies extraction under the Pseudo Root Assumption and the Order assumption. We show that we can either extract a pseudo root of $g$ or a witness \benedikt{Formally we should define witness extended emulation }
\end{theorem}
\begin{proof}
(SKETCH)
We prove the statement by showing that we can recursively either extract the encoding of an integer polynomial $f(x) \in \ZZ[X]$ of degree $d$ with bounded coefficients or a break of the pseudo root assumption for element $g$ or a non trivial element of known (random) order in $\GG$. We recurse over degree $\hat{d}$ up to the final degree $d$.
In each round the extracted witness is an integer $\hat{y}$ such that $\hat{y}=\enc(f(x))$ where the coefficients of $f(x)$ are less than $B$ in absolute value and the degree is at most $\hat{d}$ and such that $g^{\hat{y}}=C$. Also $f(z) \equiv y \mod p$.
If $\hat{d}=0$ then we can directly extract $f(x)=\hat{y}$ such that $\vert \hat{y} \vert < p^{\log_2(d+1)+1}$, $y=\hat{y}\mod p$, $f(x)=y\in \FF_p[X]$ and $g^{\hat{y}}=C$ as the witness. We proceed with $d=1$.
For $d>0$ we have $g^{\hat{y}}=C=C_0^{\alpha}C_1$. Rewinding once we get 
 $C=C_0^{\alpha}C_1=g^{\hat{y}}$ and $C'=C_0^{\alpha'}C_1=g^{\hat{y}'}$ for distinct $\alpha$ and $\alpha'$. 
 This gives us $C_0^{\alpha-\alpha'}=g^{\hat{y}-\hat{y}'}$. 
 Either $\alpha-\alpha' \not|~ \hat{y}-\hat{y}' $ which would directly break the pseudo root assumption or we can compute $D=g^{\frac{\hat{y}-\hat{y}'}{\alpha-\alpha'}}$. Either $D=C_0$ or $(D/C_0)^{\alpha-\alpha'}=1$, i.e. $D/C_0$ is an element of known order. In either of the cases the extraction succeeds and is completed. (Invoke order assumption).
 
 If $D=C_0$ then we have $\hat{y}_0=\frac{\hat{y}-\hat{y}'}{\alpha-\alpha'}$ and
 $g^{\hat{y}_0}=C_0$. In that case $C_1=g^{\hat{y}_1}=g^{\hat{y}-\alpha\hat{y}_0}$. The extractor has now successfully obtained $\hat{y}_0$ and $\hat{y}_1$
 
 If $|y|=|y'|<B$ then $|y_0|<  2 \cdot B$ and $|y_1|<|\hat{y}-\hat{y}+\alpha\hat{y}'|=|\alpha \hat{y}'|<\lambda \cdot B$. 
 We define $f(x)=\dec(y_0+x^{d} \cdot y_1)$ and with coefficients less than $2B$ in absolute value. 
 Note that $y_0=\hat{y}_0 \mod p=\dec(y_0)(z)$ and $y_1=\hat{y}_1 \mod p=\dec(y_1)(z)$. Since $y=\dec(\hat{y})(z)=\alpha \cdot \dec(\hat{y}_0)(z)+\dec(\hat{y}_1)(z)$ and $f(x)=\dec(y_0+x^{\hat{d}} \cdot y_1)$ we have that $y=y_0 +z^{\hat{d}/2}y_1 \mod p=f(z) \mod p$.
 Additionally $C=C_0^{\alpha}C_1=\commit(f(x))$ or we get a break of the adaptive root assumption.\benedikt{So this is technically a bit difficult because it's not actually a witeness}.
 
 The extractor recuses with the encoding of $f(x)$ and degree $\hat{d}'=\hat{d}\cdot 2$.
 
 Repeating this $\log_2(d+1)$ times we get a polynomial $f(x)$ of degree $d$ that has coefficients that are bounded by $(d+1) \cdot p^{\log_2(d+1)+1} <q/2$. 
 


\end{proof}

\begin{corollary}
	If $q<bla$ there exists an efficient adversary that can break the evaluation binding property of the polynomial commitment.
\end{corollary}
\benedikt{Figure out at what q we can attack the scheme. Probably needs to use negative coefficients and such.}

  \bibliography{references,cryptobib/abbrev0,cryptobib/crypto}

\section{Zero knowledge polynomial commitment} 
This section sketches how to make the polynomial commitment scheme zero knowledge. 

\paragraph{Commit} Let $g_1 \sample \GG$ be a random base distinct from $g$. 
The hiding polynomial commitment is $C \leftarrow g^{f(q)}g_1^r$ for $r \sample [-2^\lambda, 2^\lambda]$. 

\paragraph{Open} The opening of the entire polynomial is the same, but additionally gives the blinding factor $r$. 

\paragraph{Eval}

\begin{itemize}
\item In each recursive step we commit to polynomials $f_0$ and $f_1$ using the same hiding commitment scheme, where $f_0 + f_1 q^{d/2} = f$ as integer polynomials. 

\item Note that if $C_0 = g^{f_0(q)}g_1^{r_0}$ and $C_1 = g^{f_1(q)} g_1^{r_1}$ then $C_0 \cdot C_1^{q^{d/2}} = C \cdot g_1^{r'}$ where $r' = r_0 + q^{d/2} r_1$. The prover can give a non-interactive zk proof of this relation to the verifier using a sigma protocol. E.g., the prover provides $C_1' = C_1^{q^{d/2}}$ with a PoE, and then a zk-PoKE of $r'$ such that $g_1^{r'} = C_0 C_1' / C$. 

\item We could then recurse on $C_0^\alpha C_1$ which commits to $\alpha f_0 + f_1$ with the blinding factor $\alpha r_0 + r_1$. BUT we are not done yet, see next bullet point... 

\item The remaining problem is that the evaluation protocol opens $y_0 = f_0(z) \bmod p$ and $y_1 = f_1(z) \bmod p$, which is not zero knowledge. We need $y_0, y_1$ to be independently distributed subject to the constraint $y_0 + z^{d/2} y_1 = y \bmod p$, which the verifier checks. 

A solution is to modify $f_0$ and $f_1$ by adding constant terms $\alpha, \beta$ to each that cancel, i.e. $\alpha + z^{d/2} \beta = 0 \bmod p$, where $\alpha$ is uniformly distributed in $\ZZ_p$. This way the polynomials $f_0' = f_0 + \alpha$ and $f_1' = f_1 + \beta$ satisfy the relation $f_0'(z) + z^{d/2}f_1'(z) = f(z) \bmod p$. We end up revealing $y_0' = y_0 + \alpha \bmod p$ and $y_1' = y_1 + \beta \bmod p$, which is uniformly distributed in $\ZZ_p$ subject to $y_0' + y_1' = y \bmod p$. 

Finally, the prover needs to convince the verifier that it modified the $C_0$ and $C_1$ commitments appropriately. (It could not simply choose $f_0'$ and $f_1'$ in the first step because $f_0' + q^{d/2} f_1' \neq f$ as integer polynomials). 

However, the solution is still simple. The prover creates hiding commitments $C_\alpha$ to $\alpha$ and $C_\beta$ to $\beta$ and provides a zero-knowledge proof that $C_\alpha C_\beta ^{z^{d/2}}$ is a commitment to an integer multiple of $p$. This can be done efficiently through a combination of PoE and a PoKE. (Given $g^a$, to prove that $a = 0 \bmod p$ it suffices to provide $Q$ such that $Q^p = g^a$ and a PoKE for $Q$ base g. This can be made zero knowledge w/ the standard tricks). 

The protocol then proceeds on modified commitments $C_0' = C_0 C_\alpha$ and $C_1' = C_1 C_\beta$.

\end{itemize}

\section{Vector Commitment}


\subsection{Commitment Scheme}

In the following we denote by $(a_i)_{i=0}^{d-1} \in \mathbb{F}_p^{d}$ a vector of prime field elements. The vector commitment scheme is given by the following algorithms.
\begin{itemize}
\item $\mathsf{vcom} : \mathbb{F}_p^d \rightarrow \mathbb{G} \, , \quad (a_i)_{i=0}^{d-1} \mapsto g^{\sum_{i=0}^{d-1} a_iq^i} \enspace .$
\item $\mathsf{vopen} : \mathbb{G} \times \mathbb{Z} \rightarrow \mathbb{F}_p^d \cup \{\bot\} \, , $
\item[] $\phantom{\mathsf{vopen} :} (C, z = \sum_{i=0}^{d-1} z_iq^i) \mapsto \left\lbrace \begin{array}{ll}
(z_i \, \mathsf{mod} \, p)_{i=0}^{d-1} & \textnormal{\bf if } g^z = C \\
\quad \textnormal{where all } z_i \in \{0,\ldots,q-1\} & \\
\bot & \textnormal{\bf otherwise.}
\end{array} \right.$ 
\end{itemize}

Note: somewhat homomorphic properties: multiplication by constant, additivity. As long as coefficients don't overflow.

\subsection{Coordinate Extraction}

The following protocol enables the prover to extract a commitment to the $i$th component of the vector. Both Prover and Verifier know $i, g, C$. Only the prover knows an integer $z$ such that $g^z = C$ and corresponding to a vector $(a_j)_{j=0}^{d-1}$.
\begin{itemize}
\item Prover computes (or already knows) the $q$-ary expansion of $z$, \emph{i.e.}, $(z_j)_{j=0}^{d-1}$ such that $\sum_{j=0}^{d-1} z_j q^j = z$ and all $z_j \in \{0,\ldots, q-1\}$. He then sends to Verifier:
\begin{itemize}
\item $C_l = g^{\sum_{j=0}^{i-1} z_jq^j}, C_i = g^{z_i}, C_r = g^{\sum_{j=i+1}^{d-1} z_j q^{j-i-1}}$
\item $C_i^{q^i}, C_r^{q^{i+1}}$
\end{itemize}
\item Prover and Verifier run a proof of correct exponentiation to establish that $C_m^{q^i}, C_r^{q^{i+1}}$ were computed correctly.
\item Verifier checks that $C_l \times C_i^{q^i} \times C_r^{q^{i+1}} \stackrel{?}{=} C$ and aborts if false.
\item Prover and Verifier run a range proof to establish that the discrete logarithm of $C_l$ base $g$ is within the range $\{0, \ldots, q^i-1\}$.
\end{itemize}

Correctness, soundness, etc. (todo)

\subsection{Inner Product}
\label{section:inner_product}

The following protocol enables the prover to extract a commitment to the inner product $\mathbf{a}^\mathsf{T} \mathbf{s}$, where $\mathbf{a} = (a_i)_{i=0}^{d-1}$ is the vector to which $C$ is a commitment. The vector $\mathbf{s} \in \mathbb{F}_p^d$ is known to the verifier in the basic protocol, but later on we show how to hide this vector and simultaneously reduce the verifier's running time.
\begin{itemize}
\item Prover and Verifier flip $\mathbf{s}$ to obtain $\bar{\mathbf{s}} = (s_{d-1-i})_{i=0}^{d-1}$ and the matching integer encoding $z_{\bar{\mathbf{s}}} = \sum_{i=0}^{d-1} s_{d-1-i} q^i$.
\item Prover computes $C^{z_{\bar{\mathbf{s}}}}$ and sends this value to the verifier.
\item Prover and Verifier engage in a proof of correct exponentiation.
\item Prover and Verifier extract a commitment to coordinate $d$, which is exactly $\sum_{i=0}^{d-1} a_is_i$ modulo $p$.
\end{itemize}

Correctness, soundness, etc. (todo) Special attention for coefficient size.

Note that the Verifier must process all of $z_{\bar{\mathbf{s}}}$ in order to verify the exponentiation, which in particular is linear in $d$. However, it is possible to reduce this complexity and simultaneously hide the value of $z_{\bar{\mathbf{s}}}$. To do this, the prover must have committed to $z_{\bar{\mathbf{s}}}$ by sending $g^{z_{\bar{\mathbf{s}}}}$ (possibly with respect to a different base). At this point, a batched proof of knowledge of exponent establishes that the discrete logarithms of $C^{z_{\bar{\mathbf{s}}}}$ base $C$ and of $g^{z_{\bar{\mathbf{s}}}}$ base $g$ are equal.

\subsection{Protocols for Proving Permutations}


\subsubsection{Flip}

The following protocol establishes that two commitments, $c_a$ and $c_b$ represent polynomials $f_a, f_b \in \mathbb{F}_p$ (or vectors, for that matter) whose coefficients are flipped. Specifically, that $f_a = \sum_{i=0}^{d}f_i x^i$ for some coefficients $f_i$, and $f_b = \sum_{i=0}^df_ix^{d-i}$ for the same coefficients $f_i$.
Protocol:
\begin{itemize}
    \item Common knowledge: $c_a, c_b \in \mathbb{G}$.
    \item Verifier chooses $z \xleftarrow{\$} \mathbb{F}_p \backslash \{0\}$ and sends it to Prover.
    \item Prover and Verifier engage in proofs of correct evaluation producing $f_a(z)$ and $f_b(z^{-1})$, matching $c_a$ and $c_b$, respectively.
    \item Verifier checks that $f_a(z) \stackrel{?}{=} z^d f_b(z^{-1})$.
\end{itemize}

To see why it works, observe that $f_a(x) = x^df_b(x^{-1})$ and we can test this equation probabilistically by choosing a random $z \in \mathbb{F}_p \backslash \{0\}$ to evaluate $f_a$ and $f_b$ in. If $f_a$ is indeed the flipping of $f_b$ then the polynomial $F = f_a(x) - x^df_b(x^{-1})$ is identically zero; but otherwise it has at most $d$ zeros, and so the inequality will be exposed with overwhelming probability $(p-1-d)/(p-1)$.

\subsubsection{Rotation}

A similar observation gives rise to a proof of correct rotation. If $f(x) = \sum_{i=0}^d f_i x^i \in \mathbb{F}_p$ and $p(x) = \sum_{i=0}^d f_{i+r \, \mathsf{mod} \, d+1} x^i \in \mathbb{F}_p$ are polynomials consisting of the same coefficients but rotated by $r$ positions, then $p(x) = x^r f(x) \, \mathsf{mod} \, x^r - 1$ in all points. More explicitly, $p(x) = x^r f(x) + k(x) (x^r - 1)$ for some $k(x) \in \mathbb{F}_p$. The verifier can test this relation probabilistically.

\begin{itemize}
    \item Common knowledge: $c_f, c_p \in \mathbb{G}$ --- commitments to $f(x)$ and $p(x)$, respectively. Secret knowledge for the prover $f(x), p(x)$.
    \item Prover computes $k(x) = (p(x) - x^r f(x)) / x^r - 1$ and sends the commitment $c_k = g^{\hat{k}(q)}$ to it to Verifier.
    \item Verifier chooses a random point $z \xleftarrow{\$} \mathbb{F}_p$ and sends it to Prover.
    \item Prover and Verifier engage in proofs of correct evaluation for $f(z)$, $p(z)$, and $k(z)$.
    \item Verifier checks that $z^r f(z) + k(z) (z^r-1) = p(z)$.
\end{itemize}

\subsubsection{Generic Permutation}

The following protocol establishes that two polynomial commitments have the same coefficients but permuted according to a known permutation $\sigma : \{0,\ldots,d\} \rightarrow \{0,\ldots,d\}$. Specifically, $c_f$ is a commitment to $f(x) = \sum_{i=0}^d f_i x^i$ and $c_p$ is a commitment to $p(x) = \sum_{i=0}^d f_{\sigma(i)} x^i$. The proof makes use of the relation $p(x) = x^{\sigma(0)} f(x^{d+1}) - d(x)$ where $d(x) = \sum_{i=1}^d f_i (x^{i(d+1) + \sigma(0)} - x^{\sigma(i)})$. As $d(x)$ relies on the coefficients of $f(x)$, it is important to establish that $d(x)$ is correctly formed.

\begin{itemize}
    \item Common knowledge: $c_f, c_p, \sigma$ -- commitment to $f(x)$, commitment to $p(x)$, and permutation of coefficients.
    \item Secret knowledge for Prover: $f(x), p(x) \in \mathbb{F}_p$.
    \item Prover computes $n = \hat{f}(q^{d^2})$ and sends $c_n = g^n$ to Verifier.
    \item Prover and Verifier run an inner product protocol between $c_n$ and $(q^{di})_{i=0}^d$, and again between the result and $(q^i)_{i=0}^d$. This establishes that $n$ has the same coefficients as $c_f$ but spaced differently.
    \item Prover and Verifier run an inner product protocol between $c_n$ and $(q^{(i(d+1) + \sigma(0)} - q^{\sigma(i)})_{i=0}^d$ to compute $c_d$, the commitment to $d(x)$ that is well-formed wrt. $f(x)$.
    \item Verifier chooses a random point $z \xleftarrow{\$} \mathbb{F}_p$.
    \item Prover and Verifier engage in proofs of correct evaluation for $f(z), p(z), d(z)$.
    \item Verifier checks that $p(z) = z^{\sigma(0)} f(z^{d+1}) - d(z)$.
\end{itemize}

\section{Illustration: QAP-based SNARK}

The next protocol describes an efficiently verifiable proof system for rank-one constraint satisfaction problems. Specifically, we start from a list of $m$ constraints of the form
\begin{equation} \label{equation:r1cs}
    \mathbf{a_i}^\mathsf{T} \mathbf{s} \times \mathbf{b}_i^\mathsf{T} \mathbf{s} = \mathbf{c_i}^\mathsf{T} \mathbf{s} \enspace ,
\end{equation}
where $\mathbf{s} \in \mathbb{F}_p^n$ is the secret witness and the $m$ triples $(\mathbf{a_i}, \mathbf{b_i}, \mathbf{c_i})_{i=0}^{m-1} \in \mathbb{F}_p^{3 \times m \times n}$ are the known parameters that define the constraints. Furthermore, $s_0 = 1$.

Translate this to a quadratic arithmetic program (QAP) by selecting $m$ arbitrary but different elements $\{e_0, \ldots, e_{m-1}\} \subset \mathbb{F}_p$ and defining $\mathbf{a}(x) \in \mathbb{F}^n[x]$ such that $\mathbf{a}(e_i) = \mathbf{a_i}$, and similarly for $\mathbf{b}(x)$ and $\mathbf{c}(x)$. Furthermore, set $h(x) = \prod_{i=0}^{m-1} (x-e_i)$. Then Equation~\ref{equation:r1cs} becomes
\begin{equation} \label{equation:qap_modular}
    \mathbf{a}(x)^\mathsf{T}\mathbf{s} \times \mathbf{b}(x)^\mathsf{T}\mathbf{s} \equiv \mathbf{c}(x)^\mathsf{T}\mathbf{s} \,\, \mathsf{mod} \,\, h(x) \enspace .
\end{equation}
Moreover, a prover knowledgeable of $\mathbf{s}$ can produce another polynomial $t(x)$ such that
\begin{equation} \label{equation:qap_explicit}
    \mathbf{a}(x)^\mathsf{T}\mathbf{s} \times \mathbf{b}(x)^\mathsf{T}\mathbf{s} = \mathbf{c}(x)^\mathsf{T}\mathbf{s} + t(x) \times h(x) \enspace .
\end{equation}

The proof establishes that the prover knows a vector $\mathbf{s}$ and a polynomial $h(x)$ such that Equation~\ref{equation:qap_explicit} is satisfied. Specifically:
\begin{itemize}
    \item Common input to Prover and Verifier: $A = \mathsf{vcom}(\mathbf{a}(x))$, $B = \mathsf{vcom}(\mathbf{b}(x))$, $C = \mathsf{vcom}(\mathbf{c}(x))$, $H = \mathsf{com}(h(x))$, and $D = \mathsf{vcom}((1 \, 0 \, \cdots \, 0)^\mathsf{T})$.
    \item Prover produces commitments $A_\mathbf{s} = \mathsf{com}(\mathbf{a}(x)^\mathsf{T} \mathbf{s})$, and similarly for $B_\mathbf{s}, C_\mathbf{s}$ with the inner product protocol of Section~\ref{section:inner_product}. Additionally, $D_\mathbf{s}$ is computed. All four inner product protocols are performed simultaneously, thereby establishing that the $\mathbf{s}$ used is the same in all four cases.
    \item Prover opens $D_\mathbf{s}$ to $1$, showing that $s_0$ is $1$.
    \item Prover multiplies $t(x)$ into $H$, thereby obtaining $H_t = \mathsf{com}(t(x) \times h(x))$.
    \item Prover and Verifier run a proof of knowledge of exponent.
    \item Prover multiplies $\mathbf{b}(x)^\mathsf{T} \mathbf{s}$ into $A_\mathbf{s}$, thereby obtaining $A_{\mathbf{s}B\mathbf{s}} = \mathsf{com}(\mathbf{a}(x)^\mathsf{T} \mathbf{s} \times \mathbf{b}(x)^\mathsf{T} \mathbf{s})$.
    \item Prover and Verifier run a proof of equal discrete logarithms showing that $A_{\mathbf{s}B\mathbf{s}}$ is to $A_\mathbf{s}$ as $B$ is to $g$.
    \item Verifier selects a random point, $z \xleftarrow{\$} \mathbb{F}_p$ and sends it to the prover.
    \item Prover and Verifier compute the weighted commitment $K = A_{\mathbf{s}B\mathbf{s}} \times C^{-1} \times H_T^{-1} = \mathsf{com}(\mathbf{a}(x)^\mathsf{T} \mathsf{s} \times \mathbf{b}(x)^\mathsf{T} \mathbf{s} - \mathbf{c}(x)^\mathsf{T} \mathbf{s} - h(x) \times t(x)) = \mathsf{com}(k(x))$
    \item Prover and Verifier run an evaluation proof establishing that $k(z) = 0$.
\end{itemize}

Note: we need to pay special attention to the size of the coefficients of $\mathbf{s}$ and of $t(x)$ are not too big. It is possible that the random selection of $z$ makes the prover who cheats by choosing larger coefficients overwhelmingly unlikely to succeed. Alternatively, we can devise a proof of small coefficients or something like that.

\begin{small}
\noindent\begin{minipage}{\textwidth}
\begin{mdframed}[userdefinedwidth=1\textwidth]  \label{prot:QAP}
\newcommand{\qap}{$\mathsf{QAP}$}
	\noindent \underline{\textsf{Protocol \qap} (R1CS)}\\
\noindent Params: $\secpar,p,\GG, \mathsf{g}, d_{\max},q, m, n, ((\mathbf{a}_j, \mathbf{b}_j, \mathbf{c}_j))_{j=0}^{m-1}$; \\
Common Inputs: $\mathsf{c}_u, \mathsf{c}_v, \mathsf{c}_w \in \mathbb{G}$ which are commitments to $\mathbf{u}(x), \mathbf{v}(x), \mathbf{w}(x), \mathbf{d}(x)$; \\
Witness: $\mathbf{s} \in \mathbb{F}_p^n$; \\ 
Claim: Prover knows a witness $\mathbf{s} \in \mathbb{F}_p^n$ such that $s_0 = 1$ and $\forall j \in \{0,\ldots,m-1\} \, . \, \mathbf{a}_j^\mathsf{T} \mathbf{s} \times \mathbf{b}_j^\mathsf{T} \mathbf{s} = \mathbf{c}_j^\mathsf{T} \mathbf{s}$.

\begin{enumerate}[nolistsep]
\item Prover and Verifier run protocol \\ $\mathsf{BatchInnerProduct}((\mathsf{c}_u, \mathsf{c}_v, \mathsf{c}_w, \mathsf{g}); (\mathbf{u}(x), \mathbf{v}(x), \mathbf{w}(x), \mathbf{d}(x)), \mathbf{s})$:
\item \pcind[1] Prover computes $z_s \gets \sum_{j=0}^{n-1} q_y^j s_{n-1-j}$
\item \pcind[1] Prover computes $\mathsf{c}_{us} \gets \mathsf{c}_u^{z_s}$,  $\mathsf{c}_{vs} \gets \mathsf{c}_v^{z_s}$,  $\mathsf{c}_{ws} \gets \mathsf{c}_w^{z_s}$,  $\mathsf{c}_{s} \gets \mathsf{g}^{z_s}$ and sends $(\mathsf{c}_{us}, \mathsf{c}_{vs}, \mathsf{c}_{ws}, \mathsf{c}_{s})$ to Verifier.
\item[] \pcind[1] \pccomment{Note that $\mathsf{c}_{us} = \left(\mathsf{g}^{\sum_{i=0}^{n-1} q_y^i u_i(q_x)}\right)^{\sum_{j=0}^{n-1}q_y^j s_{n-1-j}} = \mathsf{g}^{\sum_{i=0}^{2d-1} q_y^i \left(\sum_{j=0}^{i+1} u_{n-j+1} s_j \right)}$.}
\item (work in progress....)
\end{enumerate}
\end{mdframed}
\end{minipage}
\end{small}

\section{Applications to Other Proof Systems}

\subsection{DARK-Sonic}

\subsection{DARK-Spartan}

\subsection{DARK-Bulletproofs}

\subsection{DARK-STARK}

\subsection{Comparison}

\end{document}
