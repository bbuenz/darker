\documentclass{article}
\usepackage[operators,lambda,keys,sets,primitives,adversary,asymptotics]{cryptocode}
\usepackage{notations}
\usepackage{mdframed}
\usepackage{enumitem}
\usepackage{amsmath,amsthm}
\usepackage[colorinlistoftodos]{todonotes}

%Theorems
\newtheorem{definition}{Definition}
\newtheorem{theorem}{Theorem}
\newtheorem{lemma}{Lemma}
\newtheorem{assumption}{Assumption}
\newtheorem{fact}{Fact}
\newtheorem{corollary}{Corollary}

\newif\ifcomments
\commentstrue


\ifcomments
	\newcommand{\benedikt}[1]{{\textcolor{red}{[Benedikt: #1]}}}
	\newcommand{\alan}[1]{{\todo[color=blue!40!white]{Alan: #1}}}
	\newcommand{\alaninline}[1]{{\todo[color=blue!20!white, inline]{Alan: #1}}}
	\else
	\newcommand{\benedikt}[1]{}
	\newcommand{\alan}[1]{}
	\newcommand{\alaninline}[1]{}
	\fi

\begin{document}
\title{RSA-based Polynomial Commitment}
\maketitle

\section{Introduction}

interactive proof systems for arbitrary computation

 - zero-knowledge proofs
 - arithmetization
 - polynomial relations
 
circuit model: QAP and Groth16 NILP

 - construction
 - used algebra
 - advantages and disadvantages
 
Turing model: STARK

 - construction
 - used algebra
 - advantages and disadvantages

Groups of unknown order

 - history
 - vdfs / accumulators
 - construction: RSA
 - construction class group

Our contribution:

 - polynomial commitments with groups of unknown order
 - efficiently verifiable evaluation proofs
 - interactive proofs for arbitrary computation based on SONIC
 -- constant-size SRS
 -- no trusted setup (class group)
 -- trusted setup independent of circuit (RSA)

Related work:
	
	-Kate et al.
	-Sonic
\section{Preliminaries}
\paragraph{Notation}
\begin{itemize}
\item Let $f(x) \in \mathbb{F}_p[x]$ be a polynomial of degree at most $N-1$ where $N$ is a power of two. The coefficients of $f(x)$ are denoted by $f_i$ such that $f(x) \stackrel{\triangle}{=} \sum_{i=0}^{N-1} f_i x^i$.
\item $p$ is a prime.
\item We work in a group $\mathbb{G}$ of unknown order (\emph{e.g.} an ideal class group) with a designated base element $g \in \mathbb{G}$ with unknown order. (It might be tempting refer to this element as the \emph{generator} but that terminology would imply that $\mathbb{G}$ is cyclic, which is not necessarily true.) We use multiplicative notation.
\item Let $q \in \mathbb{N}$ be an integer with $q \gg p$.
\item $\textbf{Protocol}_{A,B}(x;w)$ denotes an interactive public coin protocol with common input $x$ and B's private input $w$
\end{itemize}

\subsection{Assumptions}
The security of the scheme relies on the fractional root assumption which is a generalization of the strong RSA assumption. The assumption\alaninline{``this assumption'' is the strong RSA assumption or the fractional root assumption?} states that an adversary cannot compute fractional roots of random group elements. B\"unz, Boneh and Fisch \cite{journals/iacr/BonehBF18a} show that this assumption is satisfied in the generic group model.  
\begin{assumption}[Pseudo root assumption]
\label{assum:fracroot}
The pseudo root assumption holds for $\ggen$ if for any efficient adversary $\adv$:
\[        
                \Pr\left[y^\beta = g^{\alpha} \wedge  \beta \not\vert~ \alpha   : 
                \begin{array}{l} 
                      \GG \sample \ggen(\lambda) \\ 
                      g \sample \GG \\
                      (\alpha, \beta, y) \sample \adv(\GG, g) \\
                      \quad \textnormal{where} \, \alpha, \beta \in \ZZ \, \textnormal{and} \, y \in \GG 
                \end{array} 
        \right] \leq \negl.
\]
\end{assumption}
We now define a security games for polynomial commitments

\subsection{Polynomial Commitments}

\alaninline{Before we state the definitions of the security properties, it makes sense to introduce the syntax first. This syntax will probably be a point of ongoing discussion because this syntax has to be a) compatible with our protocol; and b) compatible with our adaptation of Sonic (and esp. its proof).}
\alaninline{Here's a point we can start with. There are two options for the syntax of opening: \\
1. The output of $\open$ is a single bit, depending on whether the provided inputs are valid. The opened polynomial should be one of the input arguments. \\
2. The output of $\open$ is either the opened polynomial itself (if all inputs are valid), or $\bot$ if the inputs are invalid. \\
While option (2) is less common, makes more sense for our scheme. The reason is because the prover needs to know the correct encoding of $f(x)$ as an integer in addition to $f(x)$ itself. Furthermore, this choice forces us to take explicit care of the integer encoding, which is a) beneficial for implementers, and b) more likely than the alternative to catch errors related to the integer encoding (overflow but also other types of errors).}
\alaninline{Kate \emph{et al.} define a polynomial commitment scheme as a tuple of 6 algorithms $(\mathsf{Setup},\mathsf{Commit},\mathsf{Open},\mathsf{VerifyPoly},\mathsf{CreateWitness},\mathsf{VerifyEval})$. There is a difference between verifying a commitment and opening it. The proof-of-correct-evaluation is a ``witness''. Proving and verifying are given by non-interactive algorithms.}
\alaninline{Sonic defines a polynomial commitment scheme as a tuple $(\mathsf{Commit}, \mathsf{Open}, \mathsf{pvC})$. The generation of the public parameters is not part of the polynomial commitment scheme. There is no special decommitment information; the $\mathsf{Commit}$ function takes a polynomial $f(x)$ and $\mathsf{Open}$ takes \emph{the same} polynomial. The algorithm $\mathsf{Open}$ outputs the evaluation $f(z)$ along with a proof; this proof is verified by $\mathsf{pcV}$.}

\begin{definition}[Polynomial Commitment see \cite{AC:KatZavGol10} ]
We say a polynomial commitment scheme, consisting of algorithms $(\setup,\commit)$ and the interactive protocols $\open$ and $\eval$ is secure interactive if it satisfies the following properties:
\paragraph{Correctness}
For an honest prover $\prover$ and a public coin honest verifier $\verifier$, is correct if for all polynomials $f(x)\in \ZZ_p[X]$ with degree polynomial in $\lambda$ and evaluation points $z\in \FF_p$:
	\[        
                \Pr\left[\begin{array}{c}\open_{\prover,\verifier}(\crs,f(x),C)=\text{"accept"}\\
                \eval_{\prover,\verifier}(\crs,C,z,y;f(x))=\text{"accept"}
                \end{array}  : 
                \begin{array}{l} 
                      \crs \sample \setup(\lambda, \deg(p(x)) \\
                      C\gets \commit(\crs,f(x))]\\
               z\sample \FF_p\\
                      y\gets f(z) \in \FF_p
                \end{array} 
        \right] =1.
\]
\benedikt{ok to mesh them together?}
\paragraph{Opening Binding}
\alaninline{This property is called ``polynomial binding'' in Kate \emph{et al.}}
We say a polynomial commitment scheme, consisting of algorithms $(\setup,\commit)$ and the interactive protocol $\open$, is opening binding if for all polynomial degrees $d=\poly\in \NN$ and for any efficient adversary $\adv$ and honest verifier $\verifier$:

	\[        
                \Pr\left[\begin{array}{c}\open_{\adv,\verifier}(\crs,C,f(x))=\text{"accept"}\\
                \wedge\\
           \open_{\adv,\verifier}(\crs,C,g(x))=\text{"accept"}\\
                \wedge\\
                f(x)\neq g(x) \vee \deg(f(x))\neq d\end{array}  : 
                \begin{array}{l} 
                      \crs \sample \setup(\lambda, d) \\
                      (C,z,f(x),g(x))\sample \adv(\crs)]
                \end{array} 
        \right] \leq \negl.
\]
\paragraph{Evaluation Binding}
	\[        
                \Pr\left[\begin{array}{c}\eval_{\adv,\verifier}(\crs,z,y_0,C)=\text{"accept"}\\
                \wedge\\
                \eval_{\adv,\verifier}(\crs,z,y_1,C)=\text{"accept"}\\
                \wedge\\
                y_0\neq y_1\end{array}  : 
                \begin{array}{l} 
                      \crs \sample \setup(\lambda, d) \\
                      (C,z,y_0,y_1)\sample \adv(\crs)]
                \end{array} 
        \right] \leq \negl.
\]
\end{definition}
Note that technically a complete $\eval$ protocol suffices as one can always open a polynomial by evaluating it at $\deg(f(x))+1$ points.

We additionally require a stronger extraction property that had in different variations been defined by \cite{SP:ZGKPP17} and \cite{EPRINT:MBKM19}.
\begin{definition}[PolyCommit extraction]
We say a polynomial commitment scheme, consisting of algorithms $(\setup,\commit)$ and the interactive protocol $\eval$, is extractable if there exists a rewinding \benedikt{Define more properly}extractor $\extractor$ such that for all polynomial degrees $d=\poly\in \NN$ and for any efficient adversary $\adv$:
	\[        
                \Pr\left[\begin{array}{c}\eval_{\adv,\verifier}(\crs,C,z,y)=\text{"accept"}\\
                \wedge\\
                f(x)\gets\extractor^{<\eval_{\adv,\verifier}(\crs,C,z,y)>}(\crs,C)\in \FF_p[X]\\
                \wedge\\
               \commit(f(x))\neq C\vee f(z)\neq y \vee \deg(f)\neq d
                 \end{array}  : 
                \begin{array}{l} 
                      \crs \sample \setup(\lambda, d) \\
                      \tau \sample (0,1)^\lambda\\
                      (C,z,y)\sample \adv_1(\crs,\tau)]
                      
                \end{array} 
        \right] \leq \negl.
\]
\end{definition}
\begin{lemma}[Extraction soundness]
	A polynomial commitment scheme that satisfies opening binding as well as extraction also satisfies evaluation binding.
\end{lemma}
\begin{proof}
	(SKETCH) Assume an adversary $\adv_{\eval}$ can break the evaluation binding property with non negligible probability $\gamma$. Using the extractor $\extractor$ we will construct an adversary $\adv_\open$ that will break the opening binding property with non-negligible probability. Run $\adv_\eval$ to get $(C,z,y_1,y_2)$. Now using $\ext$ we extract polynomials $f(x)$ and $g(x)$ of degree $d$ with all but negligible probability. Note that $f(z)=y_1$ and $g(z)=y_2$. This means $f(x)\neq g(x)$ but since $C=\commit(f(x))=\commit(g(x))$ we have a break of the opening binding property.
	
	\alaninline{What remains to be shown is that the success probability of the extractor is also non-negligible.}
\end{proof}
\section{Protocol}
\label{sec:protocol}

\subsection{Polynomial Encoding}

	Consider the set $P_q\subset \ZZ[X]$ of integer polynomials with coefficients whose absolute value is less than $q/2$. $p(q) \in \ZZ$ for $p(x)\in P_q$ is a unique encoding of the polynomial:
\begin{itemize}
	\item Domain $P_q \subset \ZZ[X]$, Alphabet: $\ZZ$
	\item $\enc(p(x) \in X): p(q)$
	\item $\dec(y \in \ZZ): p_i=\frac{y \mod q^{i+1}-y \mod q^{i}}{q^i} \forall i \in [0,\lfloor\log_q(y)\rfloor]$\\
	$p(x)=\sum_{i=0}^{\lfloor\log_q(y)\rfloor} p_i x^i$
\end{itemize}

\begin{fact}
	The polynomial encoding scheme is uniquely decodable.
\end{fact}

Note that the encoding has limited homomorphic properties. $\enc(g(x))+\enc(h(x))=\enc(g(x)+h(x))$ if $g(x)+h(x)\in P_q$, i,.e. all its coefficients are less than $q/2$ in absolute value. This is ensured if for example the coefficients of $g$ and $h$ are less than $q/4$. Additionally $\enc(g(x))\cdot \enc(h(x))=\enc(g(x)\cdot h(x))$ if $g(x)\cdot h(x)\in P_q$.
\subsection{Polynomial Commitment}
 For now we assume that the degree $d$ is always a power of two

\begin{mdframed}[userdefinedwidth=0.8\textwidth]
\begin{minipage}{\textwidth}
	\begin{flushleft}
	\setup(\secpar,p,d):
		\begin{enumerate}[nolistsep]
			\item $ \GG \sample \ggen(\secpar)$
			\item $ g \sample \GG$
			\item $q\gets 2^k \text{s.t.} q>(d+1) \cdot 2\cdot p^{\log_2(d+1)+1} $
			\item $\pcreturn \crs=\{\secpar,p,\GG,g,q\}$
		\end{enumerate}
		$\commit(\crs,p(x)\in \FF_p[X])$
		\begin{enumerate}[nolistsep]
			\item 	$\hat{f}(x)\gets f(x) \in \ZZ[X]$
			\item $y\gets \hat{f}(q)$
			\item $\pcreturn g^y,\deg(f)$
		\end{enumerate}
		$\open(\crs,p(x)\in \FF_p[X])$
		\begin{enumerate}[nolistsep]
			\item Prover computes the integer encoding of $p$ $\hat{f}(x)\gets p(x) \in \ZZ[X]$
			\item Prover sends $\hat{p}(x),p(x)$ to verifier.
			\item Verifier checks that $\hat{f}(x) \mod p=f(x)$
			\item Verifier checks that $\deg(f(x))=d$ and that for each coefficient $f_i$, $|f_i|< q/2$
			\item If $C=g^{f(q)}$ the verifier accepts.
		\end{enumerate}
	\end{flushleft}
	
\end{minipage}
\end{mdframed}
\begin{small}
 \begin{minipage}{1.1\textwidth}
\begin{mdframed}[userdefinedwidth=1\textwidth]  \label{prot:Opening}
$\eval(\crs,C,d,z,y;f(x))$\\
Evaluation protocol $\eval$ at point $y=f(z)\in \FF_p$. Degree $d$ Commitment $C$. $f(x)=\sum_{i=0}^{d} f_i x^{i}$
\begin{enumerate}[nolistsep]
\item \pcif $d=0$:
\item \pcind[1] Prover computes $\hat{y}=\hat{f} \in \ZZ$, i.e. $\hat{f}(x)$ is a constant. 
\item \pcind[1] Prover sends $\hat{y}$ to the verifier.
\item \pcind[1] Verifier checks that $0\leq\hat{y}< (\log_2(d+1)+1)\cdot p$,  $\hat{y} \bmod p=y$ and $g^{\hat{y}}=C$ otherwise rejects
\item \pcelse: \benedikt{Doesn't quite work for $d=1$}
\item \pcind[1] Prover computes $f_0(x)=\sum_{i=0}^{d/2-1} f_i x^i\in \ZZ$ and $f_1(x)=\sum_{i=0}^{d/2-1} f_{d/2+i} x^{i}\in \ZZ$
\item \pcind[1] $y_0=f_0(z) \bmod p$, $y_1=f_1(z)\bmod p$, $C_0\gets\commit(f_0)$,$C_1\gets\commit(f_1)$
\item \pcind[1] Prover sends $y_0,y_1,C_0,C_1$ to the Verifier
\item \pcind[1] Verifier checks that $y_0+z^{d/2} y_1=y\in \FF_p$ and that $C_0C_1^{q^{d/2}}=C$\footnote{Using $\textsf{PoE}(C_1,C/C_0,q^{d/2})$}
\item \pcind[1] Verifier samples $\alpha \sample \FF_p$ and sends it to the prover
\item \pcind[1] Prover and Verifier compute $y'\gets\alpha y_0 +y_1 \bmod p$, $C'\gets C_0^{\alpha}C_1 \in \GG$. \\Prover also computs $f'(x)\gets\alpha  \cdot \hat{f}_0(x)+\hat{f}_1(x) \in \ZZ[X]$ 
\item \pcind[1] Prover and Verifier run $\eval(y',C',d/2,z,y';f'(x))$
\end{enumerate}
\end{mdframed}
\end{minipage}
\end{small}

\section{Security}


\begin{lemma}
	The pseudo root assumption holds in the generic group model.
\end{lemma}
\begin{lemma}
	The polynomial commitment scheme satisfies the opening binding property
\end{lemma}
\begin{proof}
	Assume $\hat{f}(x)$ and $\hat{g}(x)$ are integer encodings of two distinct polynomials $f(x),p(x) \in \FF_p[X]$. Then $\hat{h}(x)=\hat{f}(x)-\hat{g}(x) \in \ZZ[X]$ is a polynomial of degree at most $d$. Since $g^{\hat{g}(q)}=g^{\hat{f}(q)}=C$ we have that $g^{\hat{h}(q)}=1$\benedikt{Change generator base name}. Note that the coefficients of $g(x)$ and $h(x)$ are all less than $q/2$ in absolute value. By triangle inequality we have that $\hat{h}(x)$ are less than $q$. $\hat{h}$ is by assumption not the zero polynomial. This implies that $\hat{h}(q)\neq 0$ is a multiple of the order of $g\in \GG$. This however can directly be used to break the pseudo root assumption (Assumption \ref{assum:fracroot}) by taking the inverse of an integer mod $h(q)$.
\end{proof}

\begin{theorem}
	The polynomial commitment scheme from Section \ref{sec:protocol} satisfies extraction under the Pseudo Root Assumption and the Order assumption. We show that we can either extract a pseudo root of $g$ or a witness \benedikt{Formally we should define witness extended emulation }
\end{theorem}
\begin{proof}
(SKETCH)
We prove the statement by showing that we can recursively either extract the encoding of an integer polynomial $f(x) \in \ZZ[X]$ of degree $d$ with bounded coefficients or a break of the pseudo root assumption for element $g$ or a non trivial element of known (random) order in $\GG$. We recurse over degree $\hat{d}$ up to the final degree $d$.
In each round the extracted witness is an integer $\hat{y}$ such that $\hat{y}=\enc(f(x))$ where the coefficients of $f(x)$ are less than $B$ in absolute value and the degree is at most $\hat{d}$ and such that $g^{\hat{y}}=C$. Also $f(z) \equiv y \mod p$.
If $\hat{d}=0$ then we can directly extract $f(x)=\hat{y}$ such that $\vert \hat{y} \vert < p^{\log_2(d+1)+1}$, $y=\hat{y}\mod p$, $f(x)=y\in \FF_p[X]$ and $g^{\hat{y}}=C$ as the witness. We proceed with $d=1$.
For $d>0$ we have $g^{\hat{y}}=C=C_0^{\alpha}C_1$. Rewinding once we get 
 $C=C_0^{\alpha}C_1=g^{\hat{y}}$ and $C'=C_0^{\alpha'}C_1=g^{\hat{y}'}$ for distinct $\alpha$ and $\alpha'$. 
 This gives us $C_0^{\alpha-\alpha'}=g^{\hat{y}-\hat{y}'}$. 
 Either $\alpha-\alpha' \not|~ \hat{y}-\hat{y}' $ which would directly break the pseudo root assumption or we can compute $D=g^{\frac{\hat{y}-\hat{y}'}{\alpha-\alpha'}}$. Either $D=C_0$ or $(D/C_0)^{\alpha-\alpha'}=1$, i.e. $D/C_0$ is an element of known random order. In either of the cases the extraction succeeds and is completed.
 If $D=C_0$ then we have $\hat{y}_0=\frac{\hat{y}-\hat{y}'}{\alpha-\alpha'}$ and
 $g^{\hat{y}_0}=C_0$. In that case $C_1=g^{\hat{y}_1}=g^{\hat{y}-\hat{y}_0}$. If $|y|=|y'|<B$ then $|y_0|<  2 \cdot B$ and $|y_1|<|\hat{y}-\hat{y}+\hat{y}'|=|\hat{y}'|<B$. 
 We define $f(x)=\dec(y_0+x^{d} \cdot y_1)$ and with coefficients less than $2B$ in absolute value. 
 Note that $y_0=\hat{y}_0 \mod p=\dec{y_0}(z)$ and $y_1=\hat{y}_1 \mod p=\dec{y_1}(z)$. Since $y=\dec{\hat{y}}(z)=\alpha \cdot \dec{\hat{y}_0(z)}+\dec{\hat{y}_1(z)}$ and $f(x)=\dec(y_0+x^{\hat{d}} \cdot y_1)$ we have that $y=y_0 +z^{\hat{d}/2}y_1 \mod p=f(z) \mod p$.
 Additionally $C=C_0^{\alpha}C_1=\commit(f(x))$ or we get a break of the adaptive root assumption.\benedikt{So this is technically a bit difficult because it's not actually a witeness}.
 
 The extractor recuses with the encoding of $f(x)$ and degree $\hat{d}'=\hat{d}\cdot 2$.
 
 Repeating this $\log_2(d+1)$ times we get a polynomial $f(x)$ of degree $d$ that has coefficients that are bounded by $(d+1) \cdot p^{\log_2(d+1)+1} <q/2$. 
 


\end{proof}

\begin{corollary}
	If $q<bla$ there exists an efficient adversary that can break the evaluation binding property of the polynomial commitment.
\end{corollary}
\benedikt{Figure out at what q we can attack the scheme. Probably needs to use negative coefficients and such.}
\section{Supersonic: A SNARK with trustless, constant size CRS}
We build \emph{supersonic} by instantiating the Sonic SNARK using our trustless setup polynomial commitment scheme.
  \bibliography{references,cryptobib/abbrev0,cryptobib/crypto}

\end{document}
