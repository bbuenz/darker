\documentclass{article}
\usepackage[operators,lambda,keys,sets,primitives,adversary,asymptotics]{cryptocode}
\usepackage{notations}
\usepackage{mdframed}
\usepackage{enumitem}
\usepackage{amsmath,amsthm}
\usepackage[colorinlistoftodos]{todonotes}

%Theorems
\newtheorem{definition}{Definition}
\newtheorem{theorem}{Theorem}
\newtheorem{lemma}{Lemma}
\newtheorem{assumption}{Assumption}
\newtheorem{fact}{Fact}
\newtheorem{corollary}{Corollary}

\newif\ifcomments
\commentstrue


\ifcomments
	\newcommand{\benedikt}[1]{{\textcolor{red}{[Benedikt: #1]}}}
	\newcommand{\alan}[1]{{\todo[color=blue!40!white]{Alan: #1}}}
	\newcommand{\alaninline}[1]{{\todo[color=blue!20!white, inline]{Alan: #1}}}
	\else
	\newcommand{\benedikt}[1]{}
	\newcommand{\alan}[1]{}
	\newcommand{\alaninline}[1]{}
	\fi

\begin{document}
\title{DARK Proof Systems}
\maketitle

\section*{Notes for Writing}
\begin{table}
    \caption{Notation}
    \label{tab:notation}
    \centering
    \begin{tabular}{l|l}
        symbol & meaning   \\ \hline \hline
        {\bf polynomials} & \\ \hline
        $f \in \mathbb{F}_p[x]$ & polynomial modulo $p$ \\
        $f_{(0)}, f_{(1)} \in \mathbb{F}_p[x]$ & first and second half of $f(x)$ \\
        $f_0, f_1, f_i \in \mathbb{F}_p$ & coefficients \emph{s.t.} $f(x) = \sum_{i=0}^d f_ix^i$ \\
        $\hat{f}(x) \in \mathbb{Z}[x]$ & polynomial with integer coefficients, typically in $\{0,\ldots,p-1\}$ \\ 
        $\hat{f}(q) \in \mathbb{Z}$ & integer encoding of a polynomial \\ \hline
        {\bf group elements} & (proposal by Alan) \\ \hline
        $\mathsf{g} \in \mathbb{G}$ & designated base element (the term ``generator'' is misleading) \\
        $\mathsf{c}, \mathsf{c}_0, \mathsf{c}_1 \in \mathbb{G}$ & commitments \\
        $\mathsf{c}_0^a \times \mathsf{c}_1^b, \, a, b \in \mathbb{Z}$ & multiplicative notation \\ \hline
        {\bf schemes} & (proposal by Alan) {actually, I don't really like this anymore} \\ \hline
        $\mathbf{Commit}_{\mathbb{A}} : \mathbb{A} \rightarrow \mathbb{G}$ & commit to an element from algebra $\mathbb{A}$. \\
         & This notation allows stacking algebras, $\emph{e.g.}$, $\mathbf{Commit}_{(\mathbb{F}_p[x]_{\leq d})^n}$ \\
        $\mathbf{Open}_\mathbb{A} : \{0,1\}^* \times \mathbb{G} \rightarrow \mathbb{A} \cup \{\bot\}$ & open a commitment by providing the decommitment information \\
         & (in the basic scheme: an integer) and the commitment; if the pro- \\
         & vided input is invalid, the function outputs $\bot$.\\
    \end{tabular}
\end{table}

\subsection*{Structure:}
\begin{itemize}
    \item introduction
    \item tools
    \begin{itemize}
        \item syntax of poly/vector commitment
        \item construction: poly/vector commitment
        \item eval (log / const size)
        \item coeff extract (log / const size)
        \item inner product
        \item flip
    \end{itemize}
    \item illustration: simple QAP
    \item proof systems
    \begin{itemize}
        \item short paragraph for each of:
        \item sonic
        \item spartan
        \item bulletproof
        \item stark
        \item other?
        \item[+] comparison
    \end{itemize}
\end{itemize}

\section{Introduction}

interactive proof systems for arbitrary computation

 - zero-knowledge proofs
 - arithmetization
 - polynomial relations
 
circuit model: QAP and Groth16 NILP

 - construction
 - used algebra
 - advantages and disadvantages
 
Turing model: STARK

 - construction
 - used algebra
 - advantages and disadvantages

Groups of unknown order

 - history
 - vdfs / accumulators
 - construction: RSA
 - construction class group

Our contribution:

 - polynomial commitments with groups of unknown order
 - efficiently verifiable evaluation proofs
 - interactive proofs for arbitrary computation based on SONIC
 -- constant-size SRS
 -- no trusted setup (class group)
 -- trusted setup independent of circuit (RSA)

Related work:
	
	-Kate et al.
	-Sonic
\section{Preliminaries}
\paragraph{Notation}
\begin{itemize}
\item Let $f(x) \in \mathbb{F}_p[x]$ be a polynomial of degree at most $N-1$ where $N$ is a power of two. The coefficients of $f(x)$ are denoted by $f_i$ such that $f(x) \stackrel{\triangle}{=} \sum_{i=0}^{N-1} f_i x^i$.
\item $p$ is a prime.
\item We work in a group $\mathbb{G}$ of unknown order (\emph{e.g.} an ideal class group) with a designated base element $g \in \mathbb{G}$ with unknown order. (It might be tempting refer to this element as the \emph{generator} but that terminology would imply that $\mathbb{G}$ is cyclic, which is not necessarily true.) We use multiplicative notation.
\item Let $q \in \mathbb{N}$ be an integer with $q \gg p$.
\item $\textbf{Protocol}_{A,B}(x;w)$ denotes an interactive public coin protocol with common input $x$ and B's private input $w$
\end{itemize}

\subsection{Assumptions}
The security of the scheme relies on the fractional root assumption which is a generalization of the strong RSA assumption. The assumption\alaninline{``this assumption'' is the strong RSA assumption or the fractional root assumption?} states that an adversary cannot compute fractional roots of random group elements. B\"unz, Boneh and Fisch \cite{journals/iacr/BonehBF18a} show that this assumption is satisfied in the generic group model.  
\begin{assumption}[Pseudo root assumption]
\label{assum:fracroot}
The pseudo root assumption holds for $\ggen$ if for any efficient adversary $\adv$:
\[        
                \Pr\left[y^\beta = g^{\alpha} \wedge  \beta \not\vert~ \alpha   : 
                \begin{array}{l} 
                      \GG \sample \ggen(\lambda) \\ 
                      g \sample \GG \\
                      (\alpha, \beta, y) \sample \adv(\GG, g) \\
                      \quad \textnormal{where} \, \alpha, \beta \in \ZZ \, \textnormal{and} \, y \in \GG 
                \end{array} 
        \right] \leq \negl.
\]
\end{assumption}
We now define a security games for polynomial commitments

\subsection{Polynomial Commitments}

\alaninline{Before we state the definitions of the security properties, it makes sense to introduce the syntax first. This syntax will probably be a point of ongoing discussion because this syntax has to be a) compatible with our protocol; and b) compatible with our adaptation of Sonic (and esp. its proof).}
\alaninline{Here's a point we can start with. There are two options for the syntax of opening: \\
1. The output of $\open$ is a single bit, depending on whether the provided inputs are valid. The opened polynomial should be one of the input arguments. \\
2. The output of $\open$ is either the opened polynomial itself (if all inputs are valid), or $\bot$ if the inputs are invalid. \\
While option (2) is less common, makes more sense for our scheme. The reason is because the prover needs to know the correct encoding of $f(x)$ as an integer in addition to $f(x)$ itself. Furthermore, this choice forces us to take explicit care of the integer encoding, which is a) beneficial for implementers, and b) more likely than the alternative to catch errors related to the integer encoding (overflow but also other types of errors).}
\alaninline{Kate \emph{et al.} define a polynomial commitment scheme as a tuple of 6 algorithms $(\mathsf{Setup},\mathsf{Commit},\mathsf{Open},\mathsf{VerifyPoly},\mathsf{CreateWitness},\mathsf{VerifyEval})$. There is a difference between verifying a commitment and opening it. The proof-of-correct-evaluation is a ``witness''. Proving and verifying are given by non-interactive algorithms.}
\alaninline{Sonic defines a polynomial commitment scheme as a tuple $(\mathsf{Commit}, \mathsf{Open}, \mathsf{pvC})$. The generation of the public parameters is not part of the polynomial commitment scheme. There is no special decommitment information; the $\mathsf{Commit}$ function takes a polynomial $f(x)$ and $\mathsf{Open}$ takes \emph{the same} polynomial. The algorithm $\mathsf{Open}$ outputs the evaluation $f(z)$ along with a proof; this proof is verified by $\mathsf{pcV}$.}

\begin{definition}[Polynomial Commitment see \cite{AC:KatZavGol10} ]
We say a polynomial commitment scheme, consisting of algorithms $(\setup,\commit)$ and the interactive protocols $\open$ and $\eval$ is secure interactive if it satisfies the following properties:
\paragraph{Correctness}
For an honest prover $\prover$ and a public coin honest verifier $\verifier$, is correct if for all polynomials $f(x)\in \ZZ_p[X]$ with degree polynomial in $\lambda$ and evaluation points $z\in \FF_p$:
	\[        
                \Pr\left[\begin{array}{c}\open_{\prover,\verifier}(\crs,f(x),C)=\text{"accept"}\\
                \eval_{\prover,\verifier}(\crs,C,z,y;f(x))=\text{"accept"}
                \end{array}  : 
                \begin{array}{l} 
                      \crs \sample \setup(\lambda, \deg(p(x)) \\
                      C\gets \commit(\crs,f(x))]\\
               z\sample \FF_p\\
                      y\gets f(z) \in \FF_p
                \end{array} 
        \right] =1.
\]
\benedikt{ok to mesh them together?}
\paragraph{Opening Binding}
\alaninline{This property is called ``polynomial binding'' in Kate \emph{et al.}}
We say a polynomial commitment scheme, consisting of algorithms $(\setup,\commit)$ and the interactive protocol $\open$, is opening binding if for all polynomial degrees $d=\poly\in \NN$ and for any efficient adversary $\adv$ and honest verifier $\verifier$:

	\[        
                \Pr\left[\begin{array}{c}\open_{\adv,\verifier}(\crs,C,f(x))=\text{"accept"}\\
                \wedge\\
           \open_{\adv,\verifier}(\crs,C,g(x))=\text{"accept"}\\
                \wedge\\
                f(x)\neq g(x) \vee \deg(f(x))\neq d\end{array}  : 
                \begin{array}{l} 
                      \crs \sample \setup(\lambda, d) \\
                      (C,z,f(x),g(x))\sample \adv(\crs)]
                \end{array} 
        \right] \leq \negl.
\]
\paragraph{Evaluation Binding}
	\[        
                \Pr\left[\begin{array}{c}\eval_{\adv,\verifier}(\crs,z,y_0,C)=\text{"accept"}\\
                \wedge\\
                \eval_{\adv,\verifier}(\crs,z,y_1,C)=\text{"accept"}\\
                \wedge\\
                y_0\neq y_1\end{array}  : 
                \begin{array}{l} 
                      \crs \sample \setup(\lambda, d) \\
                      (C,z,y_0,y_1)\sample \adv(\crs)]
                \end{array} 
        \right] \leq \negl.
\]
\end{definition}
Note that technically a complete $\eval$ protocol suffices as one can always open a polynomial by evaluating it at $\deg(f(x))+1$ points.

We additionally require a stronger extraction property that had in different variations been defined by \cite{SP:ZGKPP17} and \cite{EPRINT:MBKM19}.
\begin{definition}[PolyCommit extraction]
We say a polynomial commitment scheme, consisting of algorithms $(\setup,\commit)$ and the interactive protocol $\eval$, is extractable if there exists a rewinding \benedikt{Define more properly}extractor $\extractor$ such that for all polynomial degrees $d=\poly\in \NN$ and for any efficient adversary $\adv$:
	\[        
                \Pr\left[\begin{array}{c}\eval_{\adv,\verifier}(\crs,C,z,y)=\text{"accept"}\\
                \wedge\\
                f(x)\gets\extractor^{<\eval_{\adv,\verifier}(\crs,C,z,y)>}(\crs,C)\in \FF_p[X]\\
                \wedge\\
               \commit(f(x))\neq C\vee f(z)\neq y \vee \deg(f)\neq d
                 \end{array}  : 
                \begin{array}{l} 
                      \crs \sample \setup(\lambda, d) \\
                      \tau \sample (0,1)^\lambda\\
                      (C,z,y)\sample \adv_1(\crs,\tau)]
                      
                \end{array} 
        \right] \leq \negl.
\]
\end{definition}
\begin{lemma}[Extraction soundness]
	A polynomial commitment scheme that satisfies opening binding as well as extraction also satisfies evaluation binding.
\end{lemma}
\begin{proof}
	(SKETCH) Assume an adversary $\adv_{\eval}$ can break the evaluation binding property with non negligible probability $\gamma$. Using the extractor $\extractor$ we will construct an adversary $\adv_\open$ that will break the opening binding property with non-negligible probability. Run $\adv_\eval$ to get $(C,z,y_1,y_2)$. Now using $\ext$ we extract polynomials $f(x)$ and $g(x)$ of degree $d$ with all but negligible probability. Note that $f(z)=y_1$ and $g(z)=y_2$. This means $f(x)\neq g(x)$ but since $C=\commit(f(x))=\commit(g(x))$ we have a break of the opening binding property.
\end{proof}
\section{Protocol}
\label{sec:protocol}

\subsection{Polynomial Encoding}

	Consider the set $P_q\subset \ZZ[X]$ of integer polynomials with coefficients whose absolute value is less than $q/2$. $p(q) \in \ZZ$ for $p(x)\in P_q$ is a unique encoding of the polynomial:
\begin{itemize}
	\item Domain $P_q \subset \ZZ[X]$, Alphabet: $\ZZ$
	\item $\enc(p(x) \in X): p(q)$
	\item $\dec(y \in \ZZ): p_i=\frac{y \mod q^{i+1}-y \mod q^{i}}{q^i} \forall i \in [0,\lfloor\log_q(y)\rfloor]$\\
	$p(x)=\sum_{i=0}^{\lfloor\log_q(y)\rfloor} p_i x^i$
\end{itemize}

\begin{fact}
	The polynomial encoding scheme is uniquely decodable.
\end{fact}

Note that the encoding has limited homomorphic properties. $\enc(g(x))+\enc(h(x))=\enc(g(x)+h(x))$ if $g(x)+h(x)\in P_q$, i,.e. all its coefficients are less than $q/2$ in absolute value. This is ensured if for example the coefficients of $g$ and $h$ are less than $q/4$. Additionally $\enc(g(x))\cdot \enc(h(x))=\enc(g(x)\cdot h(x))$ if $g(x)\cdot h(x)\in P_q$.
\subsection{Polynomial Commitment}
 For now we assume that the degree $d$ is always a power of two

\begin{mdframed}[userdefinedwidth=0.8\textwidth]
\begin{minipage}{\textwidth}
	\begin{flushleft}
	\setup(\secpar,p,d):
		\begin{enumerate}[nolistsep]
			\item $ \GG \sample \ggen(\secpar)$
			\item $ g \sample \GG$
			\item $q\gets 2^k \text{s.t.} q>(d+1) \cdot 2\cdot p^{\log_2(d+1)+1} $
			\item $\pcreturn \crs=\{\secpar,p,\GG,g,q\}$
		\end{enumerate}
		$\commit(\crs,p(x)\in \FF_p[X])$
		\begin{enumerate}[nolistsep]
			\item 	$\hat{f}(x)\gets f(x) \in \ZZ[X]$
			\item $y\gets \hat{f}(q)$
			\item $\pcreturn g^y,\deg(f)$
		\end{enumerate}
		$\open(\crs,p(x)\in \FF_p[X])$
		\begin{enumerate}[nolistsep]
			\item Prover computes the integer encoding of $p$ $\hat{f}(x)\gets p(x) \in \ZZ[X]$
			\item Prover sends $\hat{p}(x),p(x)$ to verifier.
			\item Verifier checks that $\hat{f}(x) \mod p=f(x)$
			\item Verifier checks that $\deg(f(x))=d$ and that for each coefficient $f_i$, $|f_i|< q/2$
			\item If $C=g^{f(q)}$ the verifier accepts.
		\end{enumerate}
	\end{flushleft}
	
\end{minipage}
\end{mdframed}
\begin{small}
 \begin{minipage}{1.1\textwidth}
\begin{mdframed}[userdefinedwidth=1\textwidth]  \label{prot:Opening}
$\eval(\crs,C,d,z,y;f(x))$\\
Evaluation protocol $\eval$ at point $y=f(z)\in \FF_p$. Degree $d$ Commitment $C$. $f(x)=\sum_{i=0}^{d} f_i x^{i}$
\begin{enumerate}[nolistsep]
\item \pcif $d=0$:
\item \pcind[1] Prover computes $\hat{y}=\hat{f} \in \ZZ$, i.e. $\hat{f}(x)$ is a constant. 
\item \pcind[1] Prover sends $\hat{y}$ to the verifier.
\item \pcind[1] Verifier checks that $0\leq\hat{y}< (\log_2(d+1)+1)\cdot p$,  $\hat{y} \bmod p=y$ and $g^{\hat{y}}=C$ otherwise rejects
\item \pcelse: \benedikt{Doesn't quite work for $d=1$}
\item \pcind[1] Prover computes $f_0(x)=\sum_{i=0}^{d/2-1} f_i x^i\in \ZZ$ and $f_1(x)=\sum_{i=0}^{d/2-1} f_{d/2+i} x^{i}\in \ZZ$
\item \pcind[1] $y_0=f_0(z) \bmod p$, $y_1=f_1(z)\bmod p$, $C_0\gets\commit(f_0)$,$C_1\gets\commit(f_1)$
\item \pcind[1] Prover sends $y_0,y_1,C_0,C_1$ to the Verifier
\item \pcind[1] Verifier checks that $y_0+z^{d/2} y_1=y\in \FF_p$ and that $C_0C_1^{q^{d/2}}=C$\footnote{Using $\textsf{PoE}(C_1,C/C_0,q^{d/2})$}
\item \pcind[1] Verifier samples $\alpha \sample \FF_p$ and sends it to the prover
\item \pcind[1] Prover and Verifier compute $y'\gets\alpha y_0 +y_1 \bmod p$, $C'\gets C_0^{\alpha}C_1 \in \GG$. \\Prover also computs $f'(x)\gets\alpha  \cdot \hat{f}_0(x)+\hat{f}_1(x) \in \ZZ[X]$ 
\item \pcind[1] Prover and Verifier run $\eval(y',C',d/2,z,y';f'(x))$
\end{enumerate}
\end{mdframed}
\end{minipage}
\end{small}

\subsection{Fix coefficients and negative degrees}
\begin{itemize}
	\item Several ways to do this.
\end{itemize}

\section{Security}


\begin{lemma}
	The pseudo root assumption holds in the generic group model.
\end{lemma}
\begin{lemma}
	The polynomial commitment scheme satisfies the opening binding property
\end{lemma}
\begin{proof}
	Assume $\hat{f}(x)$ and $\hat{g}(x)$ are integer encodings of two distinct polynomials $f(x),p(x) \in \FF_p[X]$. Then $\hat{h}(x)=\hat{f}(x)-\hat{g}(x) \in \ZZ[X]$ is a polynomial of degree at most $d$. Since $g^{\hat{g}(q)}=g^{\hat{f}(q)}=C$ we have that $g^{\hat{h}(q)}=1$\benedikt{Change generator base name}. Note that the coefficients of $g(x)$ and $h(x)$ are all less than $q/2$ in absolute value. By triangle inequality we have that $\hat{h}(x)$ are less than $q$. $\hat{h}$ is by assumption not the zero polynomial. This implies that $\hat{h}(q)\neq 0$ is a multiple of the order of $g\in \GG$. This however can directly be used to break the pseudo root assumption (Assumption \ref{assum:fracroot}) by taking the inverse of an integer mod $h(q)$.
\end{proof}

\begin{theorem}
	The polynomial commitment scheme from Section \ref{sec:protocol} satisfies extraction under the Pseudo Root Assumption and the Order assumption. We show that we can either extract a pseudo root of $g$ or a witness \benedikt{Formally we should define witness extended emulation }
\end{theorem}
\begin{proof}
(SKETCH)
We prove the statement by showing that we can recursively either extract the encoding of an integer polynomial $f(x) \in \ZZ[X]$ of degree $d$ with bounded coefficients or a break of the pseudo root assumption for element $g$ or a non trivial element of known (random) order in $\GG$. We recurse over degree $\hat{d}$ up to the final degree $d$.
In each round the extracted witness is an integer $\hat{y}$ such that $\hat{y}=\enc(f(x))$ where the coefficients of $f(x)$ are less than $B$ in absolute value and the degree is at most $\hat{d}$ and such that $g^{\hat{y}}=C$. Also $f(z) \equiv y \mod p$.
If $\hat{d}=0$ then we can directly extract $f(x)=\hat{y}$ such that $\vert \hat{y} \vert < p^{\log_2(d+1)+1}$, $y=\hat{y}\mod p$, $f(x)=y\in \FF_p[X]$ and $g^{\hat{y}}=C$ as the witness. We proceed with $d=1$.
For $d>0$ we have $g^{\hat{y}}=C=C_0^{\alpha}C_1$. Rewinding once we get 
 $C=C_0^{\alpha}C_1=g^{\hat{y}}$ and $C'=C_0^{\alpha'}C_1=g^{\hat{y}'}$ for distinct $\alpha$ and $\alpha'$. 
 This gives us $C_0^{\alpha-\alpha'}=g^{\hat{y}-\hat{y}'}$. 
 Either $\alpha-\alpha' \not|~ \hat{y}-\hat{y}' $ which would directly break the pseudo root assumption or we can compute $D=g^{\frac{\hat{y}-\hat{y}'}{\alpha-\alpha'}}$. Either $D=C_0$ or $(D/C_0)^{\alpha-\alpha'}=1$, i.e. $D/C_0$ is an element of known order. In either of the cases the extraction succeeds and is completed. (Invoke order assumption).
 
 If $D=C_0$ then we have $\hat{y}_0=\frac{\hat{y}-\hat{y}'}{\alpha-\alpha'}$ and
 $g^{\hat{y}_0}=C_0$. In that case $C_1=g^{\hat{y}_1}=g^{\hat{y}-\alpha\hat{y}_0}$. The extractor has now successfully obtained $\hat{y}_0$ and $\hat{y}_1$
 
 If $|y|=|y'|<B$ then $|y_0|<  2 \cdot B$ and $|y_1|<|\hat{y}-\hat{y}+\alpha\hat{y}'|=|\alpha \hat{y}'|<\lambda \cdot B$. 
 We define $f(x)=\dec(y_0+x^{d} \cdot y_1)$ and with coefficients less than $2B$ in absolute value. 
 Note that $y_0=\hat{y}_0 \mod p=\dec(y_0)(z)$ and $y_1=\hat{y}_1 \mod p=\dec(y_1)(z)$. Since $y=\dec(\hat{y})(z)=\alpha \cdot \dec(\hat{y}_0)(z)+\dec(\hat{y}_1)(z)$ and $f(x)=\dec(y_0+x^{\hat{d}} \cdot y_1)$ we have that $y=y_0 +z^{\hat{d}/2}y_1 \mod p=f(z) \mod p$.
 Additionally $C=C_0^{\alpha}C_1=\commit(f(x))$ or we get a break of the adaptive root assumption.\benedikt{So this is technically a bit difficult because it's not actually a witeness}.
 
 The extractor recuses with the encoding of $f(x)$ and degree $\hat{d}'=\hat{d}\cdot 2$.
 
 Repeating this $\log_2(d+1)$ times we get a polynomial $f(x)$ of degree $d$ that has coefficients that are bounded by $(d+1) \cdot p^{\log_2(d+1)+1} <q/2$. 
 


\end{proof}

\begin{corollary}
	If $q<bla$ there exists an efficient adversary that can break the evaluation binding property of the polynomial commitment.
\end{corollary}
\benedikt{Figure out at what q we can attack the scheme. Probably needs to use negative coefficients and such.}
\section{Supersonic: A SNARK with trustless, constant size CRS}
We build \emph{supersonic} by instantiating the Sonic SNARK using our trustless setup polynomial commitment scheme.

\section{Discussion and Optimizations}
\begin{enumerate}
	\item Precompute $g^{q^i}$ to use parallelism etc.
	\item SNARK without falsifiable assumptions
	\item First polynomial commitment scheme with constant size parameters
	
\end{enumerate}
  \bibliography{references,cryptobib/abbrev0,cryptobib/crypto}

\section{Zero knowledge polynomial commitment} 
This section sketches how to make the polynomial commitment scheme zero knowledge. 

\paragraph{Commit} Let $g_1 \sample \GG$ be a random base distinct from $g$. 
The hiding polynomial commitment is $C \leftarrow g^{f(q)}g_1^r$ for $r \sample [-2^\lambda, 2^\lambda]$. 

\paragraph{Open} The opening of the entire polynomial is the same, but additionally gives the blinding factor $r$. 

\paragraph{Eval}

\begin{itemize}
\item In each recursive step we commit to polynomials $f_0$ and $f_1$ using the same hiding commitment scheme, where $f_0 + f_1 q^{d/2} = f$ as integer polynomials. 

\item Note that if $C_0 = g^{f_0(q)}g_1^{r_0}$ and $C_1 = g^{f_1(q)} g_1^{r_1}$ then $C_0 \cdot C_1^{q^{d/2}} = C \cdot g_1^{r'}$ where $r' = r_0 + q^{d/2} r_1$. The prover can give a non-interactive zk proof of this relation to the verifier using a sigma protocol. E.g., the prover provides $C_1' = C_1^{q^{d/2}}$ with a PoE, and then a zk-PoKE of $r'$ such that $g_1^{r'} = C_0 C_1' / C$. 

\item We could then recurse on $C_0^\alpha C_1$ which commits to $\alpha f_0 + f_1$ with the blinding factor $\alpha r_0 + r_1$. BUT we are not done yet, see next bullet point... 

\item The remaining problem is that the evaluation protocol opens $y_0 = f_0(z) \bmod p$ and $y_1 = f_1(z) \bmod p$, which is not zero knowledge. We need $y_0, y_1$ to be independently distributed subject to the constraint $y_0 + z^{d/2} y_1 = y \bmod p$, which the verifier checks. 

A solution is to modify $f_0$ and $f_1$ by adding constant terms $\alpha, \beta$ to each that cancel, i.e. $\alpha + z^{d/2} \beta = 0 \bmod p$, where $\alpha$ is uniformly distributed in $\ZZ_p$. This way the polynomials $f_0' = f_0 + \alpha$ and $f_1' = f_1 + \beta$ satisfy the relation $f_0'(z) + z^{d/2}f_1'(z) = f(z) \bmod p$. We end up revealing $y_0' = y_0 + \alpha \bmod p$ and $y_1' = y_1 + \beta \bmod p$, which is uniformly distributed in $\ZZ_p$ subject to $y_0' + y_1' = y \bmod p$. 

Finally, the prover needs to convince the verifier that it modified the $C_0$ and $C_1$ commitments appropriately. (It could not simply choose $f_0'$ and $f_1'$ in the first step because $f_0' + q^{d/2} f_1' \neq f$ as integer polynomials). 

However, the solution is still simple. The prover creates hiding commitments $C_\alpha$ to $\alpha$ and $C_\beta$ to $\beta$ and provides a zero-knowledge proof that $C_\alpha C_\beta ^{z^{d/2}}$ is a commitment to an integer multiple of $p$. This can be done efficiently through a combination of PoE and a PoKE. (Given $g^a$, to prove that $a = 0 \bmod p$ it suffices to provide $Q$ such that $Q^p = g^a$ and a PoKE for $Q$ base g. This can be made zero knowledge w/ the standard tricks). 

The protocol then proceeds on modified commitments $C_0' = C_0 C_\alpha$ and $C_1' = C_1 C_\beta$.

\end{itemize}

\section{Vector Commitment}


\subsection{Commitment Scheme}

In the following we denote by $(a_i)_{i=0}^{d-1} \in \mathbb{F}_p^{d}$ a vector of prime field elements. The vector commitment scheme is given by the following algorithms.
\begin{itemize}
\item $\mathsf{vcom} : \mathbb{F}_p^d \rightarrow \mathbb{G} \, , \quad (a_i)_{i=0}^{d-1} \mapsto g^{\sum_{i=0}^{d-1} a_iq^i} \enspace .$
\item $\mathsf{vopen} : \mathbb{G} \times \mathbb{Z} \rightarrow \mathbb{F}_p^d \cup \{\bot\} \, , $
\item[] $\phantom{\mathsf{vopen} :} (C, z = \sum_{i=0}^{d-1} z_iq^i) \mapsto \left\lbrace \begin{array}{ll}
(z_i \, \mathsf{mod} \, p)_{i=0}^{d-1} & \textnormal{\bf if } g^z = C \\
\quad \textnormal{where all } z_i \in \{0,\ldots,q-1\} & \\
\bot & \textnormal{\bf otherwise.}
\end{array} \right.$ 
\end{itemize}

Note: somewhat homomorphic properties: multiplication by constant, additivity. As long as coefficients don't overflow.

\subsection{Coordinate Extraction}

The following protocol enables the prover to extract a commitment to the $i$th component of the vector. Both Prover and Verifier know $i, g, C$. Only the prover knows an integer $z$ such that $g^z = C$ and corresponding to a vector $(a_j)_{j=0}^{d-1}$.
\begin{itemize}
\item Prover computes (or already knows) the $q$-ary expansion of $z$, \emph{i.e.}, $(z_j)_{j=0}^{d-1}$ such that $\sum_{j=0}^{d-1} z_j q^j = z$ and all $z_j \in \{0,\ldots, q-1\}$. He then sends to Verifier:
\begin{itemize}
\item $C_l = g^{\sum_{j=0}^{i-1} z_jq^j}, C_i = g^{z_i}, C_r = g^{\sum_{j=i+1}^{d-1} z_j q^{j-i-1}}$
\item $C_i^{q^i}, C_r^{q^{i+1}}$
\end{itemize}
\item Prover and Verifier run a proof of correct exponentiation to establish that $C_m^{q^i}, C_r^{q^{i+1}}$ were computed correctly.
\item Verifier checks that $C_l \times C_i^{q^i} \times C_r^{q^{i+1}} \stackrel{?}{=} C$ and aborts if false.
\item Prover and Verifier run a range proof to establish that the discrete logarithm of $C_l$ base $g$ is within the range $\{0, \ldots, q^i-1\}$.
\end{itemize}

Correctness, soundness, etc. (todo)

\subsection{Inner Product}
\label{section:inner_product}

The following protocol enables the prover to extract a commitment to the inner product $\mathbf{a}^\mathsf{T} \mathbf{s}$, where $\mathbf{a} = (a_i)_{i=0}^{d-1}$ is the vector to which $C$ is a commitment. The vector $\mathbf{s} \in \mathbb{F}_p^d$ is known to the verifier in the basic protocol, but later on we show how to hide this vector and simultaneously reduce the verifier's running time.
\begin{itemize}
\item Prover and Verifier flip $\mathbf{s}$ to obtain $\bar{\mathbf{s}} = (s_{d-1-i})_{i=0}^{d-1}$ and the matching integer encoding $z_{\bar{\mathbf{s}}} = \sum_{i=0}^{d-1} s_{d-1-i} q^i$.
\item Prover computes $C^{z_{\bar{\mathbf{s}}}}$ and sends this value to the verifier.
\item Prover and Verifier engage in a proof of correct exponentiation.
\item Prover and Verifier extract a commitment to coordinate $d$, which is exactly $\sum_{i=0}^{d-1} a_is_i$ modulo $p$.
\end{itemize}

Correctness, soundness, etc. (todo) Special attention for coefficient size.

Note that the Verifier must process all of $z_{\bar{\mathbf{s}}}$ in order to verify the exponentiation, which in particular is linear in $d$. However, it is possible to reduce this complexity and simultaneously hide the value of $z_{\bar{\mathbf{s}}}$. To do this, the prover must have committed to $z_{\bar{\mathbf{s}}}$ by sending $g^{z_{\bar{\mathbf{s}}}}$ (possibly with respect to a different base). At this point, a batched proof of knowledge of exponent establishes that the discrete logarithms of $C^{z_{\bar{\mathbf{s}}}}$ base $C$ and of $g^{z_{\bar{\mathbf{s}}}}$ base $g$ are equal.

\section{QAP-based SARK}

The next protocol describes an efficiently verifiable proof system for rank-one constraint satisfaction problems. Specifically, we start from a list of $m$ constraints of the form
\begin{equation} \label{equation:r1cs}
    \mathbf{a_i}^\mathsf{T} \mathbf{s} \times \mathbf{b}_i^\mathsf{T} \mathbf{s} = \mathbf{c_i}^\mathsf{T} \mathbf{s} \enspace ,
\end{equation}
where $\mathbf{s} \in \mathbb{F}_p^n$ is the secret witness and the $m$ triples $(\mathbf{a_i}, \mathbf{b_i}, \mathbf{c_i})_{i=0}^{m-1} \in \mathbb{F}_p^{3 \times m \times n}$ are the known parameters that define the constraints. Furthermore, $s_0 = 1$.

Translate this to a quadratic arithmetic program (QAP) by selecting $m$ arbitrary but different elements $\{e_0, \ldots, e_{m-1}\} \subset \mathbb{F}_p$ and defining $\mathbf{a}(x) \in \mathbb{F}^n[x]$ such that $\mathbf{a}(e_i) = \mathbf{a_i}$, and similarly for $\mathbf{b}(x)$ and $\mathbf{c}(x)$. Furthermore, set $h(x) = \prod_{i=0}^{m-1} (x-e_i)$. Then Equation~\ref{equation:r1cs} becomes
\begin{equation} \label{equation:qap_modular}
    \mathbf{a}(x)^\mathsf{T}\mathbf{s} \times \mathbf{b}(x)^\mathsf{T}\mathbf{s} \equiv \mathbf{c}(x)^\mathsf{T}\mathbf{s} \,\, \mathsf{mod} \,\, h(x) \enspace .
\end{equation}
Moreover, a prover knowledgeable of $\mathbf{s}$ can produce another polynomial $t(x)$ such that
\begin{equation} \label{equation:qap_explicit}
    \mathbf{a}(x)^\mathsf{T}\mathbf{s} \times \mathbf{b}(x)^\mathsf{T}\mathbf{s} = \mathbf{c}(x)^\mathsf{T}\mathbf{s} + t(x) \times h(x) \enspace .
\end{equation}

The proof establishes that the prover knows a vector $\mathbf{s}$ and a polynomial $h(x)$ such that Equation~\ref{equation:qap_explicit} is satisfied. Specifically:
\begin{itemize}
    \item Common input to Prover and Verifier: $A = \mathsf{vcom}(\mathbf{a}(x))$, $B = \mathsf{vcom}(\mathbf{b}(x))$, $C = \mathsf{vcom}(\mathbf{c}(x))$, $H = \mathsf{com}(h(x))$, and $D = \mathsf{vcom}((1 \, 0 \, \cdots \, 0)^\mathsf{T})$.
    \item Prover produces commitments $A_\mathbf{s} = \mathsf{com}(\mathbf{a}(x)^\mathsf{T} \mathbf{s})$, and similarly for $B_\mathbf{s}, C_\mathbf{s}$ with the inner product protocol of Section~\ref{section:inner_product}. Additionally, $D_\mathbf{s}$ is computed. All four inner product protocols are performed simultaneously, thereby establishing that the $\mathbf{s}$ used is the same in all four cases.
    \item Prover opens $D_\mathbf{s}$ to $1$, showing that $s_0$ is $1$.
    \item Prover multiplies $t(x)$ into $H$, thereby obtaining $H_t = \mathsf{com}(t(x) \times h(x))$.
    \item Prover and Verifier run a proof of knowledge of exponent.
    \item Prover multiplies $\mathbf{b}(x)^\mathsf{T} \mathbf{s}$ into $A_\mathbf{s}$, thereby obtaining $A_{\mathbf{s}B\mathbf{s}} = \mathsf{com}(\mathbf{a}(x)^\mathsf{T} \mathbf{s} \times \mathbf{b}(x)^\mathsf{T} \mathbf{s})$.
    \item Prover and Verifier run a proof of equal discrete logarithms showing that $A_{\mathbf{s}B\mathbf{s}}$ is to $A_\mathbf{s}$ as $B$ is to $g$.
    \item Verifier selects a random point, $z \xleftarrow{\$} \mathbb{F}_p$ and sends it to the prover.
    \item Prover and Verifier compute the weighted commitment $K = A_{\mathbf{s}B\mathbf{s}} \times C^{-1} \times H_T^{-1} = \mathsf{com}(\mathbf{a}(x)^\mathsf{T} \mathsf{s} \times \mathbf{b}(x)^\mathsf{T} \mathbf{s} - \mathbf{c}(x)^\mathsf{T} \mathbf{s} - h(x) \times t(x)) = \mathsf{com}(k(x))$
    \item Prover and Verifier run an evaluation proof establishing that $k(z) = 0$.
\end{itemize}

Note: we need to pay special attention to the size of the coefficients of $\mathbf{s}$ and of $t(x)$ are not too big. It is possible that the random selection of $z$ makes the prover who cheats by choosing larger coefficients overwhelmingly unlikely to succeed. Alternatively, we can devise a proof of small coefficients or something like that.

\section{Proof of Correct Flipping}

The following protocol establishes that two commitments, $c_a$ and $c_b$ represent polynomials $f_a, f_b \in \mathbb{F}_p$ (or vectors, for that matter) whose coefficients are flipped. Specifically, that $f_a = \sum_{i=0}^{d}f_i x^i$ for some coefficients $f_i$, and $f_b = \sum_{i=0}^df_ix^{d-i}$ for the same coefficients $f_i$. To see why it works, observe that $f_a(x) = x^df_b(x^{-1})$ and we can test this equation probabilistically by choosing a random $z \in \mathbb{F}_p \backslash \{0\}$ to evaluate $f_a$ and $f_b$ in. If $f_a$ is indeed the flipping of $f_b$ then the polynomial $F = f_a(x) - x^df_b(x^{-1})$ is identically zero; but otherwise it has at most $d$ zeros, and so the inequality will be exposed with overwhelming probability $(p-1-d)/(p-1)$.

Protocol:
\begin{itemize}
    \item Common knowledge: $c_a, c_b \in \mathbb{G}$.
    \item Verifier chooses $z \xleftarrow{\$} \mathbb{F}_p \backslash \{0\}$ and sends it to Prover.
    \item Prover and Verifier engage in proofs of correct evaluation producing $f_a(z)$ and $f_b(z^{-1})$, matching $c_a$ and $c_b$, respectively.
    \item Verifier checks that $f_a(z) \stackrel{?}{=} z^d f_b(z^{-1})$.
\end{itemize}

\end{document}
