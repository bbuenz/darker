\documentclass{article}
\usepackage[operators,lambda,keys,sets,primitives,adversary,asymptotics]{cryptocode}
\usepackage{notations}
\usepackage{mdframed}
\usepackage{enumitem}
\usepackage{amsmath,amsthm}
\newtheorem{definition}{Definition}
\newtheorem{theorem}{Theorem}
\newtheorem{lemma}{Lemma}

\begin{document}
\title{RSA-based Polynomial Commitment}
\maketitle
\section{Preliminaries}
The security of the scheme relies on the fractional root assumption which is a generalization of the strong RSA assumption. The assumption states that an adversary cannot compute fractional roots of random group elements. B\"unz, Boneh and Fisch \cite{accumulator} show that this assumption is satisfied in the generic group model.  
\begin{definition}[Pseudo root assumption]
The pseudo root assumption holds for $\ggen$ if for any efficient adversary $\adv$:
\[        
                \Pr\left[y^\beta = g^{\alpha} \wedge  \beta \not\vert~ \alpha   : 
                \begin{array}{l} 
                      \GG \sample \ggen(\lambda) \\ 
                      g \sample \GG \\
                      \alpha, \beta \in \ZZ,y\in \GG \sample \adv(\GG, g) 
                \end{array} 
        \right] \leq \negl.
\]
\end{definition}
We now define a security games for polynomial commitments


\begin{definition}[Evaluation Binding]
We say a polynomial commitment scheme, consisting of algorithms $(\setup,\commit)$ and the interactive protocol $\open$, is sound if for all polynomial degrees $d=\poly\in \NN$ and for any efficient adversary $\adv$:
	\[        
                \Pr\left[\begin{array}{c}\open_{\adv,\verifier}(\crs,z,y_0,C)=\open_{\adv,\verifier}(\crs,z,y_1,C)=\text{"accept"}\\
                \wedge\\
                y_0\neq y_1\end{array}  : 
                \begin{array}{l} 
                      \crs \sample \setup(\lambda, d) \\
                      (C,z,y_0,y_1)\sample \adv(\crs)]
                \end{array} 
        \right] \leq \negl.
\]
\end{definition}

\begin{definition}[PolyCommit extraction]
We say a polynomial commitment scheme, consisting of algorithms $(\setup,\commit)$ and the interactive protocol $\open$, is extractable if there exists an extractor $\extractor$ such that for all polynomial degrees $d=\poly\in \NN$ and for any efficient adversary $\adv$:
	\[        
                \Pr\left[\begin{array}{c}\open_{\adv,\verifier}(\crs,z,y,C)=\text{"accept"}\\
                \wedge\\
                p(x)\gets\extractor(\crs,\adv,C)\in \FF_p[X]\\
                \wedge\\
               \commit(p(x),d)\neq C\vee p(z)\neq y \vee \deg(p)\neq d
                 \end{array}  : 
                \begin{array}{l} 
                      \crs \sample \setup(\lambda, d) \\
                      (C,d,z,y)\sample \adv(\crs)]
                      
                \end{array} 
        \right] \leq \negl.
\]
\end{definition}
\section{Protocol}
\label{sec:protocol}
\begin{mdframed}[userdefinedwidth=0.8\textwidth]
\begin{minipage}{\textwidth}
	\begin{flushleft}
	\setup(\secpar,p,d):
		\begin{enumerate}[nolistsep]
			\item $ \GG \sample \ggen(\secpar)$
			\item $ g \sample \GG$
			\item $q\gets 2^k \text{s.t.} q> p \cdot \lceil \log_2(d+1)+1\rceil $
			\item $\pcreturn \crs=\{\secpar,p,\GG,g,q\}$
		\end{enumerate}
		$\commit(\crs,p(x)\in \FF_p[X])$
		\begin{enumerate}[nolistsep]
			\item 	$\hat{p}(x)\gets p(x) \in \ZZ[X]$
			\item $y\gets \hat{p}(q)$
			\item $\pcreturn g^y,\deg(p)$
		\end{enumerate}
		
	\end{flushleft}
\end{minipage}
\end{mdframed}
\begin{small}
 \begin{minipage}{1.1\textwidth}
\begin{mdframed}[userdefinedwidth=1\textwidth]  \label{prot:Opening}

Opening proof at point $y=p(z)\in \FF_p$ $d$. Commitment $C$. $p(x)=\sum_{i=0}^{d} c_i x^{i}$
\begin{enumerate}[nolistsep]
\item \pcif $d=0$:
\item \pcind[1] Prover computes $\hat{y}=\hat{p} \in \ZZ$, i.e. $\hat{p}(x)$ is a constant. 
\item \pcind[1] Prover sends $\hat{y}$ to the verifier.
\item \pcind[1] Verifier checks that $0\leq\hat{y}< q$,  $\hat{y} \bmod p=y$ and $g^{\hat{y}}=C$ otherwise rejects
\item \pcelse:
\item \pcind[1] Prover computes $p_0(x)=\sum_{i=0}^{d/2-1} c_i x^i\in \ZZ$ and $p_1(x)=\sum_{i=0}^{d/2-1} c_{d/2+i} x^{i}\in \ZZ$
\item \pcind[1] $y_0=p_0(z) \bmod p$, $y_1=p_1(z)\bmod p$, $C_0\gets\commit(p_0)$,$C_1\gets\commit(p_1)$
\item \pcind[1] Prover sends $y_0,y_1,C_0,C_1$ to the Verifier
\item \pcind[1] Verifier checks that $y_0+z^{d/2} y_1=y\in \FF_p$ and that $C_0C_1^{q^{d/2}}=C$\footnote{Using $\textsf{PoE}(C_1,C/C_0,q^{d/2})$}
\item \pcind[1] Verifier samples $\alpha \sample \FF_p$ and sends it to the prover
\item \pcind[1] Prover and Verifier compute $y'\gets\alpha y_0 +y_1 \bmod p$, $C'\gets C_0^{\alpha}C_1 \in \GG$. \\Prover also computs $p'(x)\gets\alpha  \cdot \hat{p}_0(x)+\hat{p}_1(x) \in \ZZ[X]$ 
\item \pcind[1] Prover and Verifier run $\textsf{PoCEv}(y',C',d/2;p'(x))$
\end{enumerate}
\end{mdframed}
\end{minipage}
\end{small}

\section{Soundness proof}

\begin{lemma}[Soundness extraction]
	Assuming the polynomial commitment has extraction it is also sound.
\end{lemma}
\begin{proof}
	Assume there is an adversary which can break the soundness property, i.e. output a tuple $(C,z,y_0,y_1)$ such that the verifier accepts with high probability both opening transcripts. Now run the extractor to extract two polynomials $p_0$ and $p_1$ such that $p_0(z)=y_0$ and $p_0(z)=y_1$. Both polynomials are of degree $d$. We write the integer representation using coefficients of size less than $q$ of $p_0$ as $\hat{p}_0$ and the serialization of $p_1$ as $\hat{p}_1$. Note tha
\end{proof}
\begin{theorem}
	The polynomial commitment scheme from Section \ref{sec:protocol} is sound under the Fractional Root Assumption. We show that we can either extract a fractional root of $g$ or a witness \footnote{Formally we should define witness extended emulation }
\end{theorem}
\begin{proof}

If $d=0$ then we can directly extract $p(x)=y\in \FF_p[X]$ as the witness. 


	We prove the statement through induction over the degree $d$. If $d=0$ and there exists an efficient adversary that can produce a tuple $(C,z,y_0,y_1)$ with non-negligble probability such that the opening verifier accepts then this implies that $g^{y_0}=g^{y_1}=C$ for $y_0\neq y_1$. This however implies that $g^{y_0-y_1}=1$. $y_0-y_1\neq 0$ is a multiple of the order of $g$ and can be used to find non trivial roots of $g$ which directly breaks the fractional root assumption.



Extraction step: $C=C_0^{\alpha}C_1=g^{y}$ and $C'=C_0^{\alpha'}C_1=g^{y'}$ for distinct $\alpha$ and $\alpha'$. This gives us $C_0^{\alpha-\alpha'}=g^{y-y'}$. Either $\alpha-\alpha'\not\vert~y-y' $ which would directly break the pseudo root assumption or $g^{\hat{y}_0}=g^{\frac{y-y'}{\alpha-\alpha'}}=C_0$. In that case $C_1=g^{\hat{y}_1}=g^{y-\alpha\frac{y-y'}{\alpha-\alpha'}}$. Let $\hat{y}_1=y-\alpha\frac{y-y'}{\alpha-\alpha'}$. Note that if $y<q$ then $\hat{y}_0\leq 2q$ and $\hat{y}_1\leq p \cdot 2q +q\leq (2p+1)q$. Note that $\hat{y}=y+r\cdot p$ and $\hat{y}'=y'+r'\cdot p$ so $\hat{y}_0 \bmod p=\frac{\hat{y}-\hat{y}'}{\alpha-\alpha'}\bmod p=\frac{y-y'}{\alpha-\alpha'}+(r-r')\cdot ({\alpha-\alpha'})^{-1}\cdot p\bmod p=\frac{y-y'}{\alpha-\alpha'}\bmod p=y_0$ and similarly for $y_1$.
The polynomial $\hat{p}(x)=\hat{y}_1\cdot x+ \hat{y}_0$
$y=y_0+z^{d/2} y_1$ and $C=C_0C_1^{q^{d/2}}$. Note that $C=g^{y_0+q^{d/2}y_1}$ 
$p(x)=y_0+x^{d/2}y_1$
$y_0+z^{d/2} y_1$


\end{proof}
  
\end{document}
