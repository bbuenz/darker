
All existing SNARK constructions can be viewed conceptually as consisting of an underlying information-theoretic statistically sound protocol that is then “cryptographically compiled” into one that achieves the desired efficiency properties (i.e. succinctness, non-interaction, etc) at the cost of \emph{computational soundness}. The information theoretic protocol is secure against unbounded provers whereas the compiled protocol is sound only against computationally bounded provers. In some cases zero-knowledge is also only achieved after compilation. This viewpoint has proved useful both as a modular method for constructing new proof systems as well as an analytical tool for classifying existing ones. 

\paragraph{CS proofs} The earliest construction of a succinct non-interactive argument system for NP, Micali’s ``CS proofs” \cite{Micali}, used random oracles and Merkle tree commitments to cryptographically compile classical PCPs via the Fiat-Shamir heuristic. In a PCP there is a verifier who has oracle access to a proof string and thus may query $q$ locations of the string in time O(q). The celebrated PCP theorem \cite{AS98, ALM+98} showed that any NP statement has a corresponding proof string of polynomial size, which the verifier only needs to check in O(1) locations in order to verify the statement with statistical soundness. In a CS proof, the first step is to build an interactive public-coin proof with succinct communication as first proposed by Kilian ~\cite{Kilian} where the prover sends the verifier a Merkle tree commitment to the PCP string, receives the verifier’s public coin queries, and provides Merkle proofs to authenticate its answers to these queries. The second step is to make this non-interactive via Fiat-Shamir. However, this construction was purely theoretical due to the concrete inefficiency of classical PCPs. 

\renewcommand{\proof}{\textcolor{blue}{[?]}} % Alan: I added this redefinition because it wouldn't compile. Not sure what is should be.
\paragraph{Short vs linear PCPs} These classical PCPs of polynomial length are called ``short PCPs''. Ishai, Kushilevitz, and Ostrovsky [IKO07] gave the first communication efficient interactive argument that did not rely on commitments to short PCPs. The underlying information theoretic object in this construction is a \emph{linear} PCP, which is an oracle computing a linear function $\proof: \mathbb{F}^m \rightarrow \mathbb{F}$, i.e. the answer to each query $\mathbf{q} \in \mathbb{F}^m$ is the inner product $\langle \proof, \mathbf{q} \rangle$. Their four-move succinct interactive argument uses linear homomorphic encryption to compile the linear PCP. The verification time in this construction is still linear and the prover time is quadratic due to the particular linear PCP instantiation based on Hadamard codes [ALM+98]. Gennaro, Gentry, Parno, and Raykova [GGPR] were the first to present a concretely practical SNARK that reduced the prover time to O(n log n) based on a more efficient instantiation of the linear PCP oracle, namely an encoding of the computation as a quadratic arithmetic program (QAP). The GGPR protocol (and followup improvements such as Pinocchio~\cite{Pinocchio}) were not initially described through the lease of linear PCPs, but were later adapted to this framework by Bitansky et. al. and Setty et. al. Bitansky et. al. generalized this construction, showing how any linear PCP of a particular kind (QAPs being one example) could be combined with linear-only encryption to get a SNARK with sublinear verification time and linear time preprocessing. The preprocessing step in these constructions requires a trusted secret setup. 

\paragraph{Linear IOPs} Another line of work … GKR based …. Can be viewed as linear IOP [IshaiCorrigan19]… In fact, linear IOPs capture all existing SNARK constructions, as they generalize both linear PCPs and short PCPs. Discuss STARK, Aurora, how it can be viewed as starting with a linear IOP where each proof oracle is a polynomial function and then turns it into a classical IOP by replacing each proof oracle with the evaluation of the polynomial at a linear number of points. This is so that they can apply weaker cryptographic compilers that don't require trusted setup (Merkle trees), but this underlying linear IOP could be compiled directly given a more advanced cryptographic compilation tool. 

\paragraph{Our results} Present our polynomial commitment and inner product argument as cryptographic compilation techniques applied to linear IOPs. Introduce terminology of algebraic linear IOPs, where queries are derived by applying bounded-degree polynomials to verifier’s coins. Subclass of algebraic linear IOPs is polynomial IOPs, where each oracle encodes a polynomial function of bounded degree and linear queries are all evaluations of the polynomial at a point. One way to see the connection to algebraic linear IOP is that each component of the query is derived via a bounded-degree monomial. 
Explain a new view of QAP as one round linear IOP instead of linear PCP and how this yields a QAP-based SNARK without trusted setup. Provide general theorem for compiling linear IOPs of two kinds: algebraic linear IOPs (theorem generalizes QAP construction), polynomial IOPs. In each give complexity of resulting preprocessing SNARK as a function of various parameters in the underlying linear IOP (number of rounds, etc). 

\subsection{Algebraic Linear IOPs} 

In this section we define \emph{algebraic linear IOPs}. Recall that in a linear PCP the verifier sends query vectors $\mathbf{q} \in \mathbb{F}^m$ and receives a response $\langle \proof, \mathbf{q} \rangle$. Bitansky et. al. \cite{Bitansky} defined a linear PCP to be \emph{algebraic} if each query vector is computed from the verifier's random coins by an arithmetic circuit of low (i.e. $poly(\lambda)$) degree. More precisely, in an algebraic degree $d$ PCP there is a query function $Q: \mathbb{F}^\mu \rightarrow \mathbb{F}^m$ such that each verifier query is of the form $\mathbf{q} \leftarrow Q(\mathbf{r})$, where $\mathbf{r}$ is uniformly sampled from $\mathbb{F}^\mu$, and the function $Q$ is computable by a vector of $\mu$-variate polynomials of degree at most $d$. 

A subclass of algebraic linear PCPs are \emph{polynomial PCPs}. A $\mu$-variate degree $d$ polynomial PCP oracle encodes a $\mu$-variate polynomial function of degree $d$, and queries to the oracle are evaluations of the polynomial. If the proof oracle is represented by the coefficient vector in $\proof \in \mathbb{F}^{d^\mu}$ then its evaluation on a point in $\mathbb{F}^\mu$ can be viewed as a linear query $\mathbf{q} \in \mathbb{F}^{d^\mu}$ with each component computed by a $\mu$-variate monomial of degree at most $d$. 

Interactive Oracle Proofs (IOPs) \cite{BCS16, RRR16} combine IPs and PCPs: in each round of an IOP the verifier sends a message $m_i$ to the prover and the prover responds with a polynomial length proof oracle $\boldsymbol{\pi_i}$, which the verifier can query via random access. The verifier can continue to query this oracle in future rounds. In other words, each $\boldsymbol{\pi_i}$ is a PCP. Boneh et. al. \cite{IshaiCorrigan} introduced linear IOPs as the IOP extension of linear PCPs, where in each round the prover's message is a linear PCP oracle. We define algebraic linear IOPs analogously to algebraic linear PCPs. To keep the syntax simple, we will only the present the definition for the case of public-coin linear IOPs. Our results will only make use of public-coin IOPs. 

\begin{definition} [Public-coin linear IOP]
\label{def:linearIOP}
Let $\mathcal{R}$ be a binary relation and $\mathbb{F}$ a finite field. A $t$-round $k$-query public-coin linear IOP for $\mathcal{R}$ over $\mathbb{F}$ with soundness error $\epsilon$ and knowledge error $\delta$ and query length $\mathbf{m} = (m_1,...,m_t)$ satisfies the following requirements: 

\begin{itemize} 
\item \underline{Protocol syntax}. There is a prover algorithm $P$, a query algorithm $Q$, and a verification algorithm $V$. For each $i$th round there is a prover state $\textsf{st}^P_i$ and a verifier state $\textsf{st}^V_i$. For any common input $x$ and $\mathcal{R}$ witness $w$, at round 0 the states are $\textsf{st}^P_0 = (x, w)$ and $\textsf{st}^V_0 = x$. 
In the $i$th round (starting at $i = 1$) the prover outputs $\proof_i \leftarrow P(\textsf{st}^P_{i-1})$ in $\mathbb{F}^{m_i}$ and the verifier outputs a uniformly sampled $\mathbf{r}_i \leftarrow_R \mathbb{F}^{\mu_i}$. 
This determines a query matrix $\mathbf{Q}_i \leftarrow Q(\textsf{st}^V_{i-1}, \mathbf{r}_i)$ consisting of $k$ column vectors in $\mathbb{F}^{m_i}$, and the verifier obtains the linear oracle response $\mathbf{a}_i \leftarrow \proof_i^\top \mathbf{Q}_i$. The updated prover state is $\textsf{st}^P_i \leftarrow (\textsf{st}^P_{i-1}, \mathbf{r}_i)$ and verifier state is $\textsf{st}^V_i \leftarrow (\textsf{st}^P_{i-1}, \mathbf{r}_i, \mathbf{a}_i)$. Finally, $V(\textsf{st}^V_t)$ returns $\textsf{Accept}$ or $\textsf{Reject}$.

\item \underline{Completeness}. For every $(x, w) \in \mathcal{R}$, following the protocol syntax above $V(\textsf{st}^V_t)$ returns $\textsf{Accept}$ with probability 1. 

\item \underline{Soundness}. If $(x, w) \notin \mathcal{R}$ for every $w$, then for every prover algorithm $P^*$ that outputs a linear function oracle $\proof^*_i: \mathbb{F}^{m_i} \rightarrow \mathbb{F}$ in the $i$th round of the protocol initialized on $x$, the probability $V(\textsf{st}^V_t)$ returns $\textsf{Accept}$ is less than $\epsilon$. 

\item \underline{Knowledge}. There exists a PPT knowledge extractor $\mathcal{E}$ such that for any prover algorithm $P^*$ and every $x$, if $V(\textsf{st}^V_t)$ outputs \textsf{Accept} in its interaction with $P^*$ with probability greater than $\delta$ then $\mathcal{E}^{P^*}(x)$ outputs $w$ such that $(x, w) \in \mathcal{R}$ in expected polynomial time. $\mathcal{E}^{P^*}(x)$ receives the input $x$ and interacts with $P^*$ via rewinding, sending partial transcripts and receiving $P^*$'s next output. (Although the verifier only accesses $P^*$'s output by querying a linear function oracle, $E$ reads the whole output). 

\end{itemize}

Furthermore, the linear IOP has \textbf{degree} $d$ if for each $i \in [t]$ and $\textsf{st}^V_{i-1}$ there are $k$ $\mu_i$-variate degree-$d$ polynomial functions $p_1,...,p_k: \mathbb{F}^{\mu_i} \rightarrow \mathbb{F}^{m_i}$ such that $Q(\textsf{st}^V_{i-1}, \mathbf{r}_i) = p_i(\mathbf{r}_i)$. The IOP is \textbf{algebraic} for a parameter $\lambda$ if $d = poly(\lambda)$. 
It is \textbf{input-oblivious} if $x$ is not included in the state passed to $Q$, and is \textbf{stateless} if the only input to $Q$ in each round is the round index $i$ and $\mathbf{r}_i$.
Finally, \textbf{polynomial IOPs} are a subclass of algebraic linear IOPs: in a $(\mu, d)$ polynomial IOP each proof oracle is a $\mu$-variate degree-$d$ polynomial and queries are evaluations of the polynomial at points in $\mathbb{F}^\mu$. 
\end{definition} 

\emph{Remark:} Connection between polynomial IOPs and point IOPs, RS encoding, STARKs etc. 

\begin{definition}[HVZK for public-coin IOPs]
Let $\textsf{View}_{\langle P(x, w), Q(x) \rangle}(V)$ denote the view of the verifier in the $t$-round $k$-query interactive protocol described in Definition~\ref{def:linearIOP} on inputs $(x,w)$ with prover algorithm $P$ and query algorithm $Q$, consisting of all public-coin challenges and oracle outputs (i.e. equivalent to the final state $\textsf{st}^V_t$). The interactive protocol has $\delta$-statistical honest verifier zero-knowledge if there exists a probabilistic polynomial time algorithm $S$ such that for every $(x, w) \in \mathcal{R}$, the distribution $S(x)$ is $\delta$-close to $\textsf{View}_{\langle P(x, w), Q(x) \rangle}(V)$ (as distributions over the randomness of $P$ and random public-coin challenges).
\end{definition}

\subsection{Compiling linear IOPs} 
We are now ready to present our main theorems. We formulate two separate theorems: the first pertains only to compiling polynomial IOPs, and the second deals with more general stateless input-oblivious algebraic IOPs. The first result is more practical because it yields interactive arguments with quasi-linear prover time. In fact, there is a concrete instantiation of the polynomial IOP (used in Sonic~\cite{Sonic}) that results in an interactive argument with both quasi-linear prover time and logarithmic communication/verification. The second result is less practical because it only guarantees polynomial prover time, but includes a much broader spectrum of concrete instantiations, including QAP-based IOPs. The prover time in a QAP-based instantiation is quadratic. These instantiations are discussied in more detail in the next section. 

\paragraph{Compilation I: Polynomial IOPs} 
Let $\Gamma = (\pro{Setup}, \pro{Commit}, \pro{Open}, \pro{Eval})$ be a $\mu$-variate polynomial commitment scheme. Given any $t$-round $(\mu, d)$-polynomial IOP for $\mathcal{R}$, construct an interactive protocol as follows: 
\begin{itemize}
\item Run $\params \leftarrow \pro{Setup}(1^\lambda)$
\item Replace each $(\mu, d)$ polynomial proof oracle $\proof: \mathbb{F}^\mu \rightarrow \mathbb{F}$ with a commitment $c_{\proof} \leftarrow \pro{Commit}(\params, \proof, \mu)$
\item Replace each verifier query $\mathbf{r}$ to proof oracle $\proof$ with a interactive execution of $\pro{Eval}(\params, c_\pi, \mathbf{z}, y, \mu, d; \proof)$, where $\proof(\mathbf{z}) = y$. 
\end{itemize}

%We begin with a lemma on using polynomial commitments (that has witness-extended emulation) to compile a public-coin polynomial IOP with negligible knowledge error into a public-coin interactive argument with witness-extended emulation. 

\begin{lemma}
If the polynomial commitment scheme $\Gamma$ has witness-extended emulation and the $t$-round polynomial IOP for $\mathcal{R}$ has negligible knowledge error, then \emph{Compilation I} generates a public-coin interactive argument for $\mathcal{R}$ that has witness-extended emulation. 
\end{lemma}

\paragraph{Compilation II: Algebraic Linear IOPs} 
Let $\Gamma = (\pro{Setup}, \pro{Commit}, \pro{Open}, \pro{Eval})$ be a $\mu+1$-variate polynomial commitment scheme [that has an inner product argument? need to define syntax...]. Given any $t$-round stateless input-oblivious $\mu$-variate degree-$d$ algebraic linear IOP for $\mathcal{R}$, construct an interactive protocol as follows: 
\begin{itemize}
\item \underline{Setup}: Run $\params \leftarrow \pro{Setup}(1^\lambda)$
\item \underline{Preprocessing}: For each $i$th round query generation function $P_i: \mathbb{F}^\mu \rightarrow \mathbb{F}^{m_i}$, where the $j$th component of $P_i(\mathbf{r})$ is $p_j(\mathbf{r})$ (for $\mu$-variate degree $d$ polynomial $p_j$), the preprocessor uses $\Gamma$ to generate a commitment $c_{P_i}$ to $(p_1,..., p_{m_i})$ as follows: 
    \begin{itemize}
    \item Encode $(p_1,...,p_{m_i})$ as a polynomial in $\mu + 1$ variables. Let $\mathbf{X} = (X_1,...,X_\mu)$ and $Z$ denote the $(\mu+1)$'st indeterminate, then define $\tilde{P}_i(\mathbf{X}, Z) := \sum_{j=1}^{m_i} p_j(\mathbf{X}) Z^j$.
    \item Compute the commitment $c_{P_i} \leftarrow \pro{Commit}(\params, \tilde{P}_i, \mu + 1)$
    \end{itemize}
\item Replace each $i$th round proof oracle $\proof_i$ with $\pro{Commit}(\proof_i^{(r)})$ where $\proof_i^{(r)}$ is the reversed vector $\proof_i$. \ben{Depends on more general syntax for polynomial commitments with the inner product argument} 

\item Replace each $i$th round verifier query $Q(\st^V_{i-1}, \mathbf{r}_i)$ with ... prover sends polynomial commitment $c_{Q_i}$ to query vector as coefficients of degree $m_i$ polynomial, and then sample random $\beta \in \mathbb{F}$, and Eval both $c_{P_i}$ at point $(\mathbf{r}_i, \beta)$ and $c_{Q_i}$ at $\beta$ to show that $Q_i(\beta) = \tilde{P}_i(\beta)$. Apply inner product argument to $c_{Q_i}$ and $c_{\proof_i}$. (When dealing with multiple queries in a round do this for each column of the query matrix). \ben{Generalize notation} 
\end{itemize}






