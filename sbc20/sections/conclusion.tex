
The compilation of a SNARK for general purpose computations generally involves several phases. In the \emph{arithmetization} phase, the computation is expressed as a satisfaction problem over some algebra. In the \emph{information-theoretic compilation}, this algebraic problem is transformed into an interactive protocol for proving satisfiability or knowledge of the witness for the algebraic claim. The last phase is \emph{cryptographic compilation}, in which information-theoretical and ideal components in the information-theoretical protocol are replaced by cryptographic components, thereby concretely realizing the abstract object at the cost of computational security. 

In the context of this three-pass compilation process, DARK protocols inhabit the last phase. Our polynomial commitment scheme based on groups of unknown order represents an alternative to similar schemes based on pairing groups or Merkle trees of Reed-Solomon codewords. Any SNARK construction that uses such schemes as black boxes can be lifted to the DARK family by replacing the polynomial commitment scheme. When the DARK protocol is instantiated with class groups, this replacement obviates the need for a trusted setup.

The polynomial commitment scheme presented in this paper highlights the need for a theoretical framework where such a tool is a native, first-order member.  The information-theoretical \emph{Polynomial IOP} formalism provides such a framework, and its capacity to capture SNARK constructions such as \textsf{Sonic} and \textsf{STARK} as special cases, indicates its conceptual soundness. The slightly more general \emph{algebraic linear IOP} formalism captures more SNARK constructions still, such as the expanding family based on GGPR's Quadratic Arithmetic Programs. It is possible to realize any algebraic linear IOP with a multivariate Polynomial IOP.

This work raises a number of important open questions and potential directions for future research.
\begin{itemize}
 \item Our polynomial commitment scheme does not support sparse polynomials. Consequently, the work of the prover when proving the correct evaluation of a bivariate polynomial, is quadratic in its degree. This limitation is particularly stringent when starting from an algebraic linear IOP. At present, quasilinear prover time can only be guaranteed for protocols that are natively univariate Polynomial IOPs. 

 \item By design, the IOP formalisms make abstraction of the cryptographic assumptions used in the layer underneath. As a result, these information theoretical formalisms fail to capture fully the expressive power of Diophantine Complexity. An interesting question left open by this work is the potential of IOPs where the proofs represent integers rather than polynomials over a finite field, and where the verifier learns only limited information about these integers from its queries.
 
 \item All known groups of unknown order have subexponential attacks. For a given level of security, when represented as a string of bits, elements from these groups are much larger than their counterparts in elliptic curve groups where no such attacks are known. An important open question is therefore the search for groups of unknown order where no such subexponential attacks are known. 
 
 \item On the topic of searching for new algebras, it is worth noting that any group of unknown order admits Shor's quantum order finding algorithm~\cite{conf/focs/Shor94} and therefore necessarily fails to provide post-quantum security. In light of the surge of interest in post-quantum cryptography, an interesting question is the search for alternative algebras that are \emph{not} groups but that \emph{can} achieve the same thing --- commit to integers while preserving homomorphic properties --- without admitting Shor's quantum attack. 

 \item Commitments to integers provide just one way to achieve a transparent polynomial commitment scheme. Another interesting open question is therefore how to instantiate Polynomial IOPs or polynomial commitment schemes with other algebras and hard problems, beyond groups of unknown order, pairing groups, and Merkle trees of Reed-Solomon codes. For instance, such an instantiation from elliptic curve groups and based on the hardness of the discrete logarithm would be well received for its better-studied hard problem, not to mention its smaller proofs. Alternatively, such an instantiation based on post-quantum hard problems that are more expressive than generic hash functions could inoculate the resulting proof systems against quantum attacks while providing smaller proofs than achievable through the use of generic hash functions.
\end{itemize}

We close with a note about the quality of cryptographic assumptions. To date, the cryptographic hardness assumptions underlying the security of all SNARK constructions are unfalsifiable. This property seems to be a by-product of the cryptographic compilation process. In pairing-based constructions the unfalsifiable Knowledge of Exponent Assumption is inherently algebraic. In contrast, the hardness assumption of groups of unknown order \emph{are} falsifiable; it is only the Fiat-Shamir heuristic for protocols with $\mathsf{polylog}$-many rounds that is unfalsifiable; a security reduction of such heuristics to standard, falsifiable assumptions is as yet unavailable.