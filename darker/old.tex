

\section{Technical Overview}
Contributions:
\begin{itemize}
	\item We reduce the evaluation time of Darker from O(d log(d)) to O(sqrt(d)*log(d))
	\item The commitment time is O(s) where s is the sparsity 
	\item Using multiexp techniques this can be computed using O(s*l/log(s*l)) group operations. 
	\item The size of the precomputed CRS is roughly O(sqrt(d))
	\item The scheme is secure under the random order assumption, basically the best assumption for groups of unknown order
	\item The interactive protocol becomes a proof not an argument
	\item We provide an implementation.
	\item (Optional) we show how to commit to polynomials in a point value representation
	\item We discuss applications to vector commitments. 
\end{itemize}

\section{Preliminaries}
\subsection{Notation}
\begin{enumerate}
	\item $\ZZ(b)=\{ z \in \ZZ | |z|<b\}$
	\item $\QQ(b)=\{\frac{n}{d} \in \QQ | |\frac{n}{\gcd(n,d)}|< b\}$
	\item $\ZZ^+=\{z \in \ZZ | z>0\}$ are the positive integers.
	\end{enumerate}
	
\subsection{Definitions}

\section{Security}

\begin{definition}[GUO Polynomial Commitment]
We describe a non-hiding succinct polynomial commitment scheme for polynomials of bounded degree over $\ZZ_p$ using a group of unknown order.
\begin{itemize}
	\item $\setup(\secparam) \rightarrow \crs:$
The setup algorithm $\setup$ on input $\secparam$ and max degree $d$ and prime $p$ samples a group of unknown order $\GG$ and a $\Generator \sample \GG$ and sets $q=2^k$ for $k \in \NN$ such that $q>p\cdot 2^{\lambda \cdot \logd}$. The algorithm outputs $\crs=(\GG,\Generator,q,p,d)$.

\item $\commit(\crs, f) \rightarrow \Commitment$: The commitment algorithm $\commit$ on input $f(X)\in \ZZ_p$ lifts $f(X)$ to the integer polynomial $\hat{f}(X)$ with coefficients bounded by $p$ and outputs $\Commitment =\hat{f}(q) \cdot \Generator$
%\item The open algorithm $\open$ outputs $f(X),\gr{W} \in \GG,\alpha \in \NN$
\item $\verify(\crs, \Commitment, f, (\hat{f}, \denom,  \gr{W}, a)) \rightarrow b$: 
The verification algorithm checks an opening of $\Commitment$ to the polynomial $f \in \ZZ_p[X]$, given the opening hint consisting of $\hat{f}\in \ZZ[X]$, $\denom \in \ZZ$, $\gr{W} \in \GG$, and $\alpha \in \ZZ$. It checks that $\alpha > 0$, $||\hat{f}||_\infty < \frac{q}{2}$, $\hat{f}(X) \equiv f(X)\cdot \denom \bmod p$, $\alpha\cdot \gr{W}=0$, $gcd(\hat{f}(q), \denom) = 1$, and $\denom \cdot \Commitment =\hat{f}(q)\cdot \Generator +\gr{W}$. If these checks pass it outputs $1$, and otherwise it outputs $0$. 
\item The evaluation protocol will be described later.
\end{itemize}
Note that an honestly generated commitment does not require an additional opening hint as $\Commitment = f(q) \cdot \Generator$. The flexibility in the $\verify$ algorithm is mainly for the security analysis of our evaluation protocol, so that the knowledge extractor is allowed to obtain a more general opening hint for the commitment. 
\end{definition}


\begin{lemma}[Binding commitment]
\label{lem:aug-com-binding}
	The commitment scheme is binding under the Random Order Assumption. 
\end{lemma}
\begin{proof}
	Suppose a polynomial time adversary is able to compute a commitment $\gr{C}$ and two valid opening hints $(\hat{f}_1, \denom_1, \gr{W}_1, \alpha_1)$ and $(\hat{f}_2, \denom_2, \gr{W}_2, \alpha_2)$ that open $\gr{C}$ to distinct values in $\ZZ_p[X]$, i.e. such that $\hat{f}_1 \cdot \denom_1^{-1} \neq \hat{f}_2 \cdot d_2^{-1} \bmod p$. 
	This implies $\hat{f}_1 \cdot \denom_2 \neq \hat{f}_2 \cdot \denom_1$. Moreover, since $||\hat{f}_1||_\infty < q/2$, $||\hat{f}_2||_\infty < q/2$, $\gcd(\hat{f}_1(q), \denom_1) = 1$, and $\gcd(\hat{f}_1(q), \denom_1) = 1$, by the contrapositive of Lemma~\ref{lem:encoding} this implies $\hat{f}_1(q) \cdot \denom_2 \neq \hat{f}_2(q) \cdot \denom_1$. Validity of the hints further implies $\alpha_1 \cdot \gr{W}_1 = 0$, $\alpha_2 \cdot \gr{W}_2 = 0$, and: 
	
	$$\denom_2 \cdot \denom_1 \cdot C = \denom_2 (\hat{f}_1(q) \cdot \gr{G} + \gr{W}_1) = \denom_1(\hat{f}_2(q) \cdot \gr{G} + \gr{W}_2)$$ 
	
	The adversary may therefore compute $\Delta = \alpha_1 \alpha_2 (\denom_2 \cdot \hat{f}_1(q)  - \denom_1 \cdot \hat{f}_2(q)) \neq 0$, which satisfies $\Delta \cdot \gr{G} = 0$. $\Delta$ is a multiple of the order of the element $\gr{G}$, which contradicts the Random Order Assumption. 
\end{proof}