%\documentclass[10pt,conference]{IEEEtran} 
\documentclass{article}
\pagestyle{plain} 
\usepackage[operators,lambda,keys,sets,primitives,adversary,asymptotics,advantage]{cryptocode}
\usepackage{notations}
%\usepackage{bm}
\usepackage{comment}
\usepackage{mdframed}
\usepackage{enumitem}
\usepackage{amsthm}
\usepackage{amsmath,amssymb}
\usepackage[utf8x]{inputenc}
\usepackage[colorinlistoftodos]{todonotes}
\usepackage{xspace}
\usepackage[hidelinks]{hyperref}
\usepackage{url}
\usepackage{cleveref}
%\usepackage{authblk}
%\usepackage{draftwatermark}
\usepackage[legalpaper, margin=1in]{geometry}
\usepackage{comment}
\usepackage{todonotes}
\theoremstyle{Definition}
\newtheorem{theorem}{Theorem}
\newtheorem{definition}{Definition}
\newtheorem{lemma}{Lemma}
\newtheorem{assumption}{Assumption}
\newtheorem{fact}{Fact}

\newif\ifcomments
\commentsfalse
\ifcomments
	\newcommand{\ben}[1]{{\textcolor{red}{[\bf Ben: #1]}}}
	\newcommand{\benedikt}[1]{{\textcolor{blue}{[\bf Benedikt: #1]}}}
	\newcommand{\alan}[1]{{\textcolor{green!50!black}{[ \bg Alan: #1]}}}
\else
	\newcommand{\ben}[1]{}
	\newcommand{\benedikt}[1]{}
	
\fi

\newcommand{\dollar}{\$}
\newcommand{\random}{\xleftarrow{\dollar}}

\title{Batch PoE}
\author{Benedikt B\"unz, Ben Fisch, Alan Szepeniek}
\date{}
\begin{document}
	
\maketitle
\section{Technical Overview}
\subsection{Binding polynomial commitment with rational encodings}
We show that the DARK commitment scheme can be extended to allow encodings of a polynomial as rationals. The core idea is that in a group of unknown order it is possible to commit to a reduced rational $a/b \in \QQ$ by computing the commitment $\gr{C}\in \GG$ such that $b \cdot \gr{C}=a \cdot \Generator$. This commitment scheme is binding under the assumption that it is hard to compute the order of $\Generator$. If $\Generator$ is a randomly sampled group element then the binding property of the commitment can be reduced to the \emph{random order assumption}. Note that under stronger assumptions such as the strong RSA assumption it is impossible to commit to any rational that is not an integer. We therefore only guarantee completeness for committing to integers (or integer polynomials) but the binding property holds for any rational opening.

We also show that the commitment remains binding under the same assumption, if the opening contains an additional element of \emph{known} order. So if instead of opening to a rational $a/b$ an adversary provides a group element $\gr{W}\in \GG$ and an integer $\alpha \in \ZZ\ \{0\}$ such that $\alpha \gr{W}=0$ then a commitment $\gr{C}$ is a binding commitment to $a/b$ if $a\cdot \Generator + \gr{W} = b\cdot \gr{C}$.  

\subsection{Batch PoE}
We show that the Pietrzak PoE is an interactive proof of knowledge for a \emph{shifted PoE Relation} in any group. That is given a Pietrzak PoE with input $(C,k,C')$ we can extract $\gr{W}\in \GG$ and $\alpha \in \ZZ/\{0\}$ such that $\alpha \gr{W}=0$ and such that $2^k C + \gr{W} =C'$. Using the new binding commitment scheme it suffices to use this knowledge property of the PoE. 
We additionally show that it is possible to combine $n$ PoEs for statements about exponents $(2,4,\dots, 2^n)$ into a single protocol with communication and verifier complexity $O(n)$ (instead of $O(n^2)$). 

\section{Protocols}

\subsection{Shifted rational commitments}
\label{sec:shiftedcommit}
\begin{mdframed}
	$\textsf{Commit}(\crs=(q\in \ZZ,\Generator \in \GG), f(X) \in \FF_p[X]):$
	\begin{itemize}[nolistsep]
		\item Set $\hat{f}(X)\in \ZZ[X]$ such that $\hat{f}(X) \bmod p =f(X)$ and $||\hat{f}||_\infty<p/2$.
		\item Output $\hat{f}(q) \cdot \Generator$
	\end{itemize}
	$\textsf{Open}(\crs=(q,\Generator),\Commitment \in \GG, f \in \F_p[X],\pi =( \hat{f} \in \ZZ[X],d \in \ZZ, \gr{W} \in \GG, \alpha \in \ZZ/\{0\}))$
	\begin{itemize}[nolistsep]
	\item Verify that $\gcd(\hat{f},d)=1$ and that $ ||\hat{f}||_\infty <q/2$
	\item Verify that $\alpha \cdot \gr{W}=0$
	\item Verify that $\hat{f}/d \equiv_p f$
		\item Verify that $\hat{f}\cdot  \Generator + \gr{W}= d \cdot \Commitment$ and $\gcd(\hat{f},d)=1$
		\item If all verifications pass output $1$ otherwise output $0$
	\end{itemize}
\end{mdframed}

\subsection{Computing all PoEs at once}
%Let the subscript $i$ denote proof elements from the $i$th round of the protocol. Let $\numrounds=\logd$ be the number of rounds in the protocol.
We present a \textsf{BatchPoE} protocol for proving that $\numrounds$ pairs of group elements $\{(\gr{C}_i, \gr{C}'_i)\}_{i=1}^k$ satisfy $q^{2^{i-1}} \cdot \gr{C}_i = \gr{C}'_i$ for all $i \in [\numrounds]$.
While this could be proved with $k$ independent PoE proofs, the protocol we present saves a factor $k$ in the communication cost over the naive solution.
%Let $\gr{D}_{i}\gets \gr{C}_{O,i}$ and $\gr{D}_{i}'\gets \gr{C}_{i}-\gr{C}_{i}$ for $i \in [1,\numrounds]$ such that $q^{2^{i-1}} \cdot \gr{D}_i=\gr{D}'_i$ for all $i\in [\numrounds]$. The prover and verifier run the following \textsf{BatchPoE} protocol that the claims hold.
\begin{mdframed}
\underline{\textsf{BatchPoE} \textbf{protocol}} \\
Prover and Verifier Input: $\{(\gr{C}_i,\gr{C}_i')\}_{i=1}^{\numrounds},q$\\
Claim: $q^{2^{i-1}}\cdot \gr{C}_i=\gr{C}'_i \quad \forall i \in \{1,...,\numrounds\}$
	\begin{enumerate}[nolistsep]
\item $\gr{X}_{\numrounds} \deq \gr{C}_{\numrounds}$, $\gr{Z}_{\numrounds} \deq \gr{C}'_{\numrounds}$
	\item $\pcwhile  i  > 1$
   \item \pcind[1]\prover computes $\gr{Y}_{i}\deq q^{2^{i-2}}\cdot \gr{X}_{i}$
\item\pcind[1]\prover sends $\gr{Y}_{i}$ to $\verifier$
	\item \pcind[1]$\verifier$ samples $\alpha \sample [-2^{\lambda-1} ,2^{\lambda-1}]$ and sends them to $\prover$
	\item \pcind[1]\prover and \verifier compute  $\gr{X}_{i-1} \deq  \gr{X}_{i}+ \alpha\cdot  \gr{Y}_{i}+\alpha^2 \cdot\gr{C}_{i-1}$
	\item $\pcind[1]$\verifier compute   $\gr{Z}_{i-1}\deq \gr{Y}_{i}+\alpha\cdot \gr{Z}_{i}+\alpha^2 \cdot \gr{C}_{i-1}^{'}$ 
	     \pccomment{Invariant: $q^{2^{i-2}} \cdot \gr{X}_{i-1} = \gr{Z}_{i-1}$ }
	\item $\pcind[1] i \deq i - 1$
\item \verifier outputs \textsf{accept} if and only if $q\cdot \gr{X}_1=\gr{Z}_1$
\end{enumerate}
\end{mdframed}
\paragraph{Challenge distribution}
Instead of using $\alpha$ and $\alpha^2$ as challenges it is also possible to use independently sampled $\alpha$ and $\beta$. The security argument is slightly more complicated as one needs to argue that the challenge matrix remains diagonalizable. The main benefit is that it decreases the verifier complexity slightly as $\alpha^2 \in \ZZ$ is about twice the size of $\alpha$ and $\beta$. 


\section{Security Proof}

.

\subsection{Proof}

\begin{fact}[Known order elements]
	\label{fact:knownorder}
	Given a two elements $\gr{W}_1,\gr{W}_2 \in \GG$ and integers 
	$\alpha_1,\alpha_2$ such that $\alpha_i \gr{W}_i=0 \; i \in \{1,2\}$
	 then for all $a,b \in \ZZ$ and $\gr{W}_3 \gets a\cdot \gr{W}_1+b\cdot \gr{W}_2$ we have that $\alpha_3 \gets \lcm(\alpha_1,\alpha_2)$ is such that $\alpha_3 \gr{W}_3=0$
\end{fact}
Let $\gr{W}_1,\gr{W}_2 \in \GG$ be such that $\alpha_1 \cdot \gr{W}_1=0$ and $\alpha_2 \cdot \gr{W}_2=0$ then for $\gr{W}'=a\gr{W}_1 + b \gr{W}_2$ we have that for $\alpha'=\lcm(\alpha_1,\alpha_2)$, $\alpha' \cdot \gr{W}'=0$
\begin{fact}
\label{fact:encoding}
Let $q \in \ZZ$ be any positive integer. For any integer $E \in \ZZ$ such that $|E|<\frac{q^{d+1}}{2}$ there exists a unique integer polynomial $f \in \ZZ[X]$ with $||f||_\infty < q/2$ such that $f(q) = E$. 
\end{fact} 

\begin{lemma}
\label{lem:encoding}
Let $q \in \Z$ be any positive integer. For any integers $z_1, z_2 \in \ZZ$ and integer polynomials $f_1, f_2 \in \ZZ[X]$ of degree at most $d$ with bounded coefficients such that $||f_1||_\infty < q/2$ and $||f_2||_\infty < q/2$, if $gcd(f_i(q), z_i) = 1$ and $z_2 \cdot f_1(q) = z_1 \cdot f_2(q)$ then 
$f_1 = f_2$ and $z_1 = z_2$. 
\end{lemma}
\begin{proof}
The rational numbers $\frac{f_1(q)}{z_1} = \frac{f_2(q)}{z_2} \in \QQ$ are in reduced rational form, hence $f_1(q) = f_2(q)$ and $z_1 = z_2$. The integers $f_1(q)$ and $f_2(q)$ have absolute value bounded by $\frac{q^{d+1}}{2}$, and thus $f_1 = f_2$ by Fact~\ref{fact:encoding}.
\end{proof}
\begin{theorem}[Binding Commitment Scheme]
	The commitment scheme presented in \cref{sec:shiftedcommit} is binding under the random order assumption. 
\end{theorem}
\begin{proof}[Proof Sketch]
	Assume you have two valid openings $(C,f,\pi=(\hat{f},d,\gr{W},\alpha) )$ and $(C,f',\pi'=(\hat{f}',d',\gr{W}',\alpha'))$ with $f \neq f'$. We will show that we can compute a multiple of the order of $\Generator$ which breaks the random order assumption.
	
	Note that by \cref{lem:encoding} we must have that $d' \hat{f} \neq d \hat{f}'$. 
	From the verification equation we have that 
	\begin{align*}
		d' \hat{f} \Generator + d' \gr{W} = d d' \cdot C\\
			d \hat{f}' \Generator + d \gr{W}' = d d' \cdot C\\
			\implies (d' \hat{f} - d \hat{f}') \Generator + \gr{W}'' = 0
	\end{align*}
	Such that there exists an efficiently computable $\alpha''$ with $\alpha'' \gr{W}=0$ by \cref{fact:knownorder}. We, therefore, have that $\alpha'' (d'\hat{f}-d\hat{f}')$ is a multiple of the order of $\Generator$ which completes the proof.
	\end{proof}


%\begin{fact}
%\label{fact:encoding}
%Let q be an integer. For any rational $z$ with numerator $n \in \ZZ$ such that $|z|<\frac{q^{d+1}}{2}$ there exists a unique degree (at most) d rational polynomial $\hat{h}(X)$ in $\QQ(q)[X]$ such that $\hat{h}(q) = z$. If $q>n$ then there exists a unique polynomial $h(X) \in \ZZ/p\ZZ$ such that $\hat{h}\equiv h \bmod p$ 
%\end{fact}

	
		
	\begin{lemma}[PoEs]
		Protocol \textsf{BatchPoE} is an interactive proof of knowledge for the following relation:
		\[ \mathcal{R_\textsf{BPoE}} = \left\{
\big\langle \{(\gr{C}_i,\gr{C}_i')\}_{i=1}^{\numrounds}, q ; \ \{(\gr{W}_i\in \GG, a_i \in \ZZ)\}_{i=1}^k  \big\rangle
: \forall i
\begin{array}{l} 
 \Commitment_i' = q^{2^{\numrounds-i}}  \cdot \Commitment_i + \gr{W}_i \\ 
a_i\cdot \gr{W}_i=0
\end{array}\right\}
\]


		%$$\mathcal{R}_{POE}=\{\langle \{(\gr{G}_i,\gr{G}_i')\}_{i=1}^{\numrounds},q ;\{(\gr{W}_i\in \GG,a_i\in \ZZ)\}_{i=1}^k \rangle: \forall i \ \gr{G}_i' = q^{2^{\numrounds-i}}  \cdot \gr{G}_i + \gr{W}_i \ \ \wedge \ \ a_i\cdot \gr{W}_i=0]_{i=1}^{\numrounds-1} \}$$
	\end{lemma}
	\begin{proof}
		We define a knowledge extractor that runs with an adversary $\adv$ who succeeds for public inputs $\{(\Commitment_i,\Commitment_i')\}_{i=1}^{\numrounds}$ and $q$ with probability $\epsilon = 1/\poly$. The extractor begins by using the tree-finding algorithm of Lemma~\ref{lem:forking} to generate a tree of accepting transcripts with the following characteristics: 
\benedikt{Update proof to use challenges alpha and beta}
\begin{itemize}
\item The tree has depth $\numrounds$ and branching factor $3$. We index nodes by $v \in [0,3^\numrounds)$.

\item The root is labeled with the publics inputs. 
\item Each node $v$ distinct from the root is labeled with a challenge $\alpha_{v}$ and a prover message $\gr{Y}_v$. \ben{Check how to label leaves} 
\item Each non-leaf node $v$ has three children each labeled with three distinct verifier challenges 
$\alpha_{v,1} \neq \alpha_{v,2} \neq \alpha_{v,3}$.

\end{itemize} 

Since the verifier challenges in the protocol are sampled from $\mathcal{X} = [-2^{\lambda -1}, 2^{\lambda -1}]$, the probability of a collision over challenges $\alpha_1, \alpha_2 \sample \mathcal{X}$ sampled uniformly and independently is $2^{-\lambda}$. Thus, by Lemma~\ref{lem:forking} the tree-finding algorithm runs for $O(\poly/\epsilon)$ steps and succeeds with probability $1 - \negl/\epsilon$ to return a tree with these characteristics.

For the root node $\textsf{rt}$, define $\gr{X}_\textsf{rt} \deq \Commitment_\numrounds$ and $\gr{Z}_\textsf{rt} \deq \Commitment'_\numrounds$. For any node $v$ on level $i$ with parent $w$ on level $i+1$, which has prover message $\gr{Y}_v$ and verifier challenge $\alpha_v$, define $\gr{X}_v \deq \gr{X}_w + \alpha_v \cdot \gr{Y}_w + \alpha^2 \cdot \Commitment_{i-1}$. Similarly, define $\gr{Z}_v \deq \gr{Y}_w + \alpha \cdot \gr{Z}_w + \alpha^2\cdot  \Commitment'_{i-1}$.

We will show that given this tree, the extractor can compute for all nodes $v \in [0, 3^\numrounds]$ group elements $\gr{W}_v$ and integers $a_v$ such that $a_v \cdot \gr{W}_v = 0$ and if $v$ is on the $i$th level then $\gr{Z}_v = q^{2^{i-1}} \cdot \gr{X}_v + \gr{W}_v$. It can also compute values $\gr{W}'_v$ and $a'_v$ such that $a'_v \cdot \gr{W}'_v = 0$ and $\Commitment'_{i-1} = \Commitment_{i-1} + \gr{W}'_v$. It will do this by starting at the leaves and working its way up the tree.
For any leaf node $v$, the fact that the transcript is accepting implies $q \cdot \gr{X}_v = \gr{Z}_v$. Now suppose that the appropriate values $\gr{W}_v$ have been computed for all children of a node $w$ on the $i$th level. Denoting the three children of nodes $w$ with indices $v_1, v_2, v_3$ and the three challenges on each by $\alpha_1, \alpha_2, \alpha_3$. This means that $\gr{X}_{v_j} = \gr{X}_w + \alpha_j \cdot \gr{Y}_w + \alpha_j^2 \cdot \Commitment_{i-1}$ and $\gr{Z}_{v_j} = \gr{Y}_w + \alpha_j \cdot \gr{Z}_w + \alpha^2_j \cdot \Commitment'_{i-1}$ for each $j \in [3]$.
Furthermore, $\gr{Z}_{v_j} = q^{2^{i-1}} \cdot \gr{X}_{v_j} + \gr{W}_{v_j}$ such that $a_{v_j} \cdot \gr{W}_{v_j} = 0$ for each $j \in [3]$. 

Let $\mathbf{W} = (\gr{W}_{v_1}, \gr{W}_{v_2}, \gr{W}_{v_3})$ and let $a \deq lcm(a_{v_1}, a_{v_2}, a_{v_3})$. We can summarize these relations in the following matrix equations, where $\mathbf{A} \in \mathbb{Z}^{3\times 3}$ is the integer matrix with rows $(1, \alpha_j, \alpha_j^2)$:

$$\mathbf{A}
\begin{bmatrix}
\gr{X}_w \\ 
\gr{Y}_w \\ 
\Commitment_{i-1}
\end{bmatrix} 
=  
\begin{bmatrix}
\gr{X}_{v_1} \\ 
\gr{X}_{v_2} \\ 
\gr{X}_{v_3}
\end{bmatrix}
\ \ \text{and} \ \ 
\mathbf{D} \cdot 
\mathbf{A} \cdot 
\begin{bmatrix}
\gr{Y}_w \\ 
\gr{Z}_w \\ 
\Commitment'_{i-1}
\end{bmatrix} 
= 
\begin{bmatrix}
\gr{Z}_{v_1} \\ 
\gr{Z}_{v_2} \\ 
\gr{Z}_{v_3}
\end{bmatrix} 
= 
q^{2^{i-1}} \cdot 
\begin{bmatrix}
\gr{X}_{v_1} \\ 
\gr{X}_{v_2} \\ 
\gr{X}_{v_3}
\end{bmatrix}
+ 
\mathbf{W}  
\ \text{and} \ \ a \cdot \mathbf{W} = 0
$$

Substituting for the column vector $(\gr{X}_{v_1}, \gr{X}_{v_2}, \gr{X}_{v_3})$ we get: 

$$  \mathbf{A} \cdot 
\begin{bmatrix}
\gr{Y}_w \\ 
\gr{Z}_w \\ 
\Commitment'_{i-1}
\end{bmatrix} 
= 
q^{2^{i-1}} \cdot \mathbf{A} \cdot  
\begin{bmatrix}
\gr{X}_{w} \\ 
\gr{Y}_{w} \\ 
\Commitment_{i-1}
\end{bmatrix}  \ \
+ 
\begin{bmatrix}
\gr{W}_{v_1} \\ 
\gr{W}_{v_2} \\ 
\gr{W}_{v_3}
\end{bmatrix}  \ \
$$

Since $\mathbf{A}$ is a Vandermonde matrix with distinct rows, $det(\mathbf{A} \cdot \mathbf{D}) \neq 0$. Therefore, by Lemma~\ref{lem:extraction}, there exists a matrix $\mathbf{P}$ such that $\mathbf{P}\cdot \mathbf{A} = \mathbf{D}$ where $\mathbf{D}$ is a diagonal matrix with non-zero entries $d_1, d_2, d_3$. This means that: 

$$  
\begin{bmatrix}
d_1 \cdot \gr{Y}_w \\ 
d_2 \cdot \gr{Z}_w \\ 
d_3 \cdot \Commitment'_{i-1}
\end{bmatrix} 
= 
q^{2^{i-1}} \cdot
\begin{bmatrix}
d_1 \cdot \gr{X}_{w} \\ 
d_2 \cdot \gr{Y}_{w} \\ 
d_3 \cdot \Commitment_{i-1}
\end{bmatrix}  \ \
+ 
\mathbf{P} \cdot 
\mathbf{W} \ \ \text{and} \ \ a \cdot \mathbf{P} \cdot \mathbf{W} = 0
$$

This implies that $d_1 d_2 \cdot \gr{Z}_w = q^{2^{i-1}} \cdot (d_2 d_1 \cdot q^{2^{i-1}} \cdot \gr{X}_w + \langle \mathbf{P}_1, \mathbf{W} \rangle) + d_1 \cdot \langle \mathbf{P}_2, \mathbf{W} \rangle$. Let $d \deq d_1 d_2$ and let $\gr{W}_w \deq \gr{Z}_w - q^{2^{i}} \cdot \gr{X}_w$. 
We have that $d \cdot \gr{W}_w = q^{2^{i-1}} \cdot \langle \mathbf{P}_1, \mathbf{W} \rangle  + d_1 \cdot \langle \mathbf{P}_2, \mathbf{W} \rangle$. Hence, $a \cdot d \cdot \gr{W}_w = 0$. Set $a_w \deq a \cdot a$. 
 
We also have the relation $d_3 \Commitment'_{i-1} = q^{2^{i-1}} \cdot \Commitment_{i-1} + \langle \mathbf{P}_3, \mathbf{W} \rangle$. Setting $\gr{W}'_w \deq \Commitment'_{i-1} - q^{2^{i-1}} \cdot \Commitment_{i-1}$ we have $d_3 \cdot \gr{W}'_w = \langle \mathbf{P}_3, \mathbf{W} \rangle$, hence $a \cdot d_3 \cdot \gr{W}'_w = 0$. Let $a'_w \deq a \cdot d_3$.  
	\end{proof}

\section{Forking lemmas}

Let $\G$ denote the abelian group (i.e., $\G$ is a $\Z$-module) that contains the space of commitments for the PCS. We first prove an elementary linear algebraic fact. 

\begin{lemma}\label{lem:triangularization}
Let $\G$ be a $\Z$-module. Let $p$ be a prime and $\F = \Z_p$. Given two vectors $\x, \y \in \G^n$ and a system of equations $\mathbf{A} \x = \y$ for a matrix $\A \in \Z^{n \times n}$ that is invertible over $\F$, 
there is an efficient algorithm to derive a diagonal matrix $\mathbf{D}$ with diagonal entries  $(d_1,...,d_n) \in \Z^n$, where $d_i \neq 0 \bmod p$ for all $i$, and a matrix $\mathbf{L}$ such that $\mathbf{D} \cdot \x = \mathbf{L} \cdot \y$. In particular, $\mathbf{L} \mathbf{A} = \mathbf{D}$.    
\end{lemma} 
\begin{proof} 
Since $det(\mathbf{A}) \neq 0 \bmod p$, by Gaussian elimination there is a triangularization matrix $\mathbf{P}$ such that $\mathbf{P} \mathbf{A} = \mathbf{T}$ is lower triangular with a diagonal of integer values that are non-zero over $\F$ (i.e., it is in reduced row-echelon form). 
Since $\mathbf{P} \mathbf{A} \cdot \x = \mathbf{T} \cdot \y= \mathbf{P}\cdot \y$, there exists an integer $d_1$ (i.e., upper left entry of $\mathbf{T}$) such that $d_1 \cdot x_1 =\langle \mathbf{P}_1, \y \rangle$, where $\mathbf{P}_1$ is the first row of $\mathbf{P}$. Denote by $\x^{(i)}$ the vector that swaps the 1st and $i$th coordinates of $\x$. Denote by $\A^{(i)}$ the matrix that results from swapping the first and $i$th columns of $\A$. We have $\A^{(i)} \x^{(i)} = \y$. 
 For each $i \in [n]$, repeat the triangularization process on $\A^{(i)}$ to get $\mathbf{P}^{(i)}$ such that $\mathbf{P}^{(i)} \cdot \A^{(i)} = \mathbf{T}^{(i)}$ is lower triangular. The upper left entry of $\mathbf{T}^{(i)}$ is an integer $d_i \neq 0 \bmod p$, and $d_i \cdot x_i = \langle \mathbf{P}^{(i)}_1, \y \rangle$.
  Return the matrix $D$ with diagonal entries $d_1,...,d_n$ and the matrix $L$ with row vectors $\mathbf{P}^{(i)}_1$ for $i \in [n]$. 
\end{proof} 

The next lemma is a direct result of this fact. 

\begin{lemma}\label{lem:extraction}
Given two vectors of commitments $\mathbf{C}, \mathbf{C^*} \in \G^n$, a system of equations $\mathbf{A} \mathbf{C} = \mathbf{C^*}$ for an integer matrix $\mathbf{A} \in \Z^{n \times n}$ that is invertible over $\F_p$, and a vector of openings of $\mathbf{C^*}$ to a vector of polynomials $\mathbf{f^*} = (f_1^*,...,f_n^*) \in (\FF)^n$, there is an efficient algorithm to derive polynomials $\mathbf{f} = (f_1,...,f_n) \in (\FF)^n$, integer vector $\mathbf{t} \in \Z^n$ such that $t_i \neq 0 \bmod p$, and openings for each $t_i \cdot \mathbf{C}_i$ to the polynomial $t_i \cdot f_i \bmod p$ such that $\mathbf{A} \cdot \mathbf{f} = \mathbf{f^*} \bmod p$. 
\end{lemma} 
\begin{proof} 
By Lemma~\ref{lem:triangularization}, there exists a diagonal matrix $\mathbf{T}$ with integer entries $t_1,...,t_n \neq 0 \bmod p$ and a matrix $\mathbf{L}$ such that $\mathbf{T} \cdot \mathbf{C} = \mathbf{L} \cdot \mathbf{C}^*$ and $\mathbf{L} \cdot \mathbf{A} = \mathbf{T}$. 
  From each linear combination of $\mathbf{C^*}$, we use $\add^*$ to derive an opening of $t_i \cdot \mathbf{C}_i$ to a polynomial $g_i = \langle \mathbf{L}_i, \mathbf{f^*} \rangle \in \FX$.
   Let $\mathbf{g} = (g_1,...,g_n)$. Finally, solve for the vector of polynomials $\mathbf{f}$ such that $\mathbf{A} \cdot \mathbf{f} = \mathbf{f^*}$ by computing $\mathbf{A}^{-1} \bmod p$. Note that $\mathbf{L} \cdot \mathbf{A} \cdot \mathbf{f} = \mathbf{T} \cdot \mathbf{f} = \mathbf{L} \cdot \mathbf{f}^*$ where $\mathbf{T}$ is a diagonal matrix with entries $t_i \neq 0 \bmod p$. Thus, $t_i f_i = g_i$, for which we have a valid opening of $t_i \cdot \mathbf{C}_i$.   
\end{proof}

\begin{lemma}[Forking Lemma]\label{lem:forking}
Let $(\prover, \verifier)$ be an $r$-round public-coin interactive proof system and $\mathcal{A}$ an adversary such that $\langle \mathcal{A}(\pp, x), \verifier(\pp, x) \rangle = 1$ with probability $\epsilon > \negl(\lambda)$ on public input $x$ and public parameters $\pp$.  Let $\{\pi_i\}_{i =1}^r$ be a set of properties $\pi_i: \mathcal{X}^2 \rightarrow \{0,1\}$ such that $\forall_{i} \ Pr[ \pi(x_1, x_2) = 1: x_1, x_2 \sample \mathcal{X}] > 1 - \negl(\lambda)$. There exists an algorithm $\mathcal{T}$ that runs in time $O(\lambda/\epsilon)$ and outputs an $(n_1,...,n_r)$-tree of accepting transcripts such that with probability $\Omega(1 - \frac{r \cdot\negl(\lambda)}{\epsilon})$ in each $i$th level of the tree every pair of sibling-node challenges $x_1, x_2$ satisfy $\pi_i(x_1, x_2) = 1$. 
\end{lemma} 

 \bibliographystyle{IEEEtran}
 \bibliography{../cryptobib/abbrev3,../cryptobib/crypto,cryptobib/additional}
\end{document}