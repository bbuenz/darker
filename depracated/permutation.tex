
\section{Protocols for Proving Permutations}

\subsection{Flip}\label{apx:reverse}

The following protocol establishes that two commitments, $c_a$ and $c_b$ represent polynomials $f_a, f_b \in \ZZ_p$ (or vectors, for that matter) whose coefficients are flipped. Specifically, that $f_a = \sum_{i=0}^{d}f_i X^i$ for some coefficients $f_i$, and $f_b = \sum_{i=0}^df_ix^{d-i}$ for the same coefficients $f_i$.
Protocol:
\begin{itemize}
    \item Common knowledge: $c_a, c_b \in \mathbb{G}$.
    \item Verifier chooses $z \sample \ZZ_p \backslash \{0\}$ and sends it to Prover.
    \item Prover and Verifier engage in proofs of correct evaluation producing $f_a(z)$ and $f_b(z^{-1})$, matching $c_a$ and $c_b$, respectively.
    \item Verifier checks that $f_a(z) \stackrel{?}{=} z^d f_b(z^{-1})$.
\end{itemize}

To see why it works, observe that $f_a(X) = X^df_b(X^{-1})$ and we can test this equation probabilistically by choosing a random $z \in \mathbb{F}_p \backslash \{0\}$ to evaluate $f_a$ and $f_b$ in. If $f_a$ is indeed the flipping of $f_b$ then the polynomial $F = f_a(X) - X^df_b(X^{-1})$ is identically zero; but otherwise it has at most $d$ zeros, and so the inequality will be exposed with overwhelming probability $(p-1-d)/(p-1)$.

\subsection{Rotation}

A similar observation gives rise to a proof of correct rotation. If $f(X) = \sum_{i=0}^d f_i X^i \in \mathbb{F}_p$ and $p(X) = \sum_{i=0}^d f_{i+r \, \mathsf{mod} \, d+1} X^i \in \mathbb{F}_p$ are polynomials consisting of the same coefficients but rotated by $r$ positions, then $p(X) = X^r f(X) \, \mathsf{mod} \, X^r - 1$ in all points. More explicitly, $p(X) = X^r f(X) + k(X) (X^r - 1)$ for some $k(X) \in \mathbb{F}_p$. The verifier can test this relation probabilistically.

\begin{itemize}
    \item Common knowledge: $c_f, c_p \in \mathbb{G}$ --- commitments to $f(X)$ and $p(X)$, respectively. Secret knowledge for the prover $f(X), p(X)$.
    \item Prover computes $k(X) = (p(X) - X^r f(X)) / X^r - 1$ and sends the commitment $c_k = g^{\hat{k}(q)}$ to it to Verifier.
    \item Verifier chooses a random point $z \sample \mathbb{F}_p$ and sends it to Prover.
    \item Prover and Verifier engage in proofs of correct evaluation for $f(z)$, $p(z)$, and $k(z)$.
    \item Verifier checks that $z^r f(z) + k(z) (z^r-1) = p(z)$.
\end{itemize}

\subsection{Generic Permutation}

The following protocol establishes that two polynomial commitments have the same coefficients but permuted according to a known permutation $\sigma : \{0,\ldots,d\} \rightarrow \{0,\ldots,d\}$. Specifically, $c_f$ is a commitment to $f(X) = \sum_{i=0}^d f_i X^i$ and $c_p$ is a commitment to $p(X) = \sum_{i=0}^d f_{\sigma(i)} X^i$. The proof makes use of the relation $p(X) = X^{\sigma(0)} f(X^{d+1}) - d(X)$ where $d(X) = \sum_{i=1}^d f_i (X^{i(d+1) + \sigma(0)} - X^{\sigma(i)})$. As $d(X)$ relies on the coefficients of $f(X)$, it is important to establish that $d(X)$ is correctly formed.

\begin{itemize}
    \item Common knowledge: $c_f, c_p, \sigma$ -- commitment to $f(X)$, commitment to $p(X)$, and permutation of coefficients.
    \item Secret knowledge for Prover: $f(X), p(X) \in \mathbb{F}_p$.
    \item Prover computes $n = \hat{f}(q^{d^2})$ and sends $c_n = g^n$ to Verifier.
    \item Prover and Verifier run an inner product protocol between $c_n$ and $(q^{di})_{i=0}^d$, and again between the result and $(q^i)_{i=0}^d$. This establishes that $n$ has the same coefficients as $c_f$ but spaced differently.
    \item Prover and Verifier run an inner product protocol between $c_n$ and $(q^{(i(d+1) + \sigma(0)} - q^{\sigma(i)})_{i=0}^d$ to compute $c_d$, the commitment to $d(X)$ that is well-formed wrt. $f(X)$.
    \item Verifier chooses a random point $z \sample \mathbb{F}_p$.
    \item Prover and Verifier engage in proofs of correct evaluation for $f(z), p(z), d(z)$.
    \item Verifier checks that $p(z) = z^{\sigma(0)} f(z^{d+1}) - d(z)$.
\end{itemize}


\subsection{Reversed Coefficients}

It is quite straightforward to show that two commitments represent polynomials with the same coefficients but in reverse order. Suppose $f(X) = \sum_{i=0}^d f_i X^i$ and $\bar{f}(X) = \sum_{i=0}^d f_{d-i}X^i$ are two polynomials with the same coefficients but in reverse order. Then the relation $\bar{f}(X) = X^d \cdot f(X^{-1})$ holds for all $X \in \mathbb{Z}_p \backslash \{0\}$. This relation can be tested probabilistically by choosing a random $z \sample \mathbb{Z}_p \backslash \{0\}$ and opening the evaluations $f(z)$ and $\bar{f}(z)$. This observation gives rise to the protocol $\pro{Reverse}(\params, \gr{C}_f, \gr{C}_{\bar{f}}; f(X)) \rightarrow b \in \{0,1\}$ which proves that $\gr{C}_f$ and $\gr{C}_{\bar{f}}$ are indeed commitments to polynomials with the same coefficients but in reverse order.

\begin{figure}[!htp]
\noindent\begin{mdframed}[userdefinedwidth=\textwidth]
\begin{minipage}{\textwidth}
	\begin{flushleft}
	$\pro{Reverse}(\params, \gr{C}_{f}, \gr{C}_{\bar{f}}; {f}(X)):$ \pccomment{$\gr{C}_f = \gr{g}^{\sum f_i q^i}$ whereas $\gr{C}_{\bar{f}} = \gr{g}^{\sum f_{d-i} q^i}$}
		\begin{enumerate}[nolistsep]
		    \item $\verifier$ samples $z \sample \mathbb{Z}_p \backslash \{0\}$ and sends $z$ to $\prover$
		    \item $\prover$ computes $y \leftarrow f(z)$ and $\bar{y} \leftarrow \bar{f}(z^{-1})$ and sends $(y, \bar{y})$ to $\verifier$
		    \item $\prover$ and $\verifier$ run $\pro{Eval}(\gr{C}_f, z, y, d; f(X))$ and $\pro{Eval}(\gr{C}_{\bar{f}}, z, \bar{y}, d; \bar{f}(X))$ \pccomment{$\bar{f}(X) = \sum_i f_{d-i} X^i$}
		    \item $\verifier$ checks that $y = z^d \bar{y}$
		    \item \pcif{}all checks pass \textbf{then} \textbf{return} 1 \textbf{else} \textbf{return} 0
		\end{enumerate}
	\end{flushleft}
\end{minipage}
\end{mdframed}
\end{figure}

\begin{lemma}
    The protocol $\pro{Reverse}$ is an argument of knowledge for the relation 
    \[
        \mathcal{R}_\mathsf{Rev}(\params) = \left\{
            \langle ( \gr{C}_f, \gr{C}_{\bar{f}} ), (\mathbf{f}, r_f, r_{\bar{f}})\rangle \ : \ \begin{array}{l}
                \mathbf{f} \in \mathbb{Z}_p^d \\
                \pro{Open}(\params, \gr{C}_f, \sum_{i=0}^d f_i X^i, r_f) = 1 \\
                \pro{Open}(\params, \gr{C}_{\bar{f}}, \sum_{i=0}^d f_{d-i} X^i, r_{\bar{f}}) = 1 
            \end{array}
        \right\} \enspace .
    \]
\end{lemma}

\begin{proof}
Full version / appendix.
\end{proof}


\section{Zero Knowledge Polynomial Commitment} \label{apx:zeroknowledge}

This section sketches how to make the polynomial commitment scheme zero knowledge. 

\paragraph{Commit} Let $g_1 \sample \GG$ be a random base distinct from $g$. 
The hiding polynomial commitment is $C \leftarrow g^{f(q)}g_1^r$ for $r \sample [-2^\lambda, 2^\lambda]$. 

\paragraph{Open} The opening of the entire polynomial is the same, but additionally gives the blinding factor $r$. 

\paragraph{Eval}

\begin{itemize}
\item In each recursive step we commit to polynomials $f_0$ and $f_1$ using the same hiding commitment scheme, where $f_0 + f_1 q^{d/2} = f$ as integer polynomials. 

\item Note that if $C_0 = g^{f_0(q)}g_1^{r_0}$ and $C_1 = g^{f_1(q)} g_1^{r_1}$ then $C_0 \cdot C_1^{q^{d/2}} = C \cdot g_1^{r'}$ where $r' = r_0 + q^{d/2} r_1$. The prover can give a non-interactive zk proof of this relation to the verifier using a sigma protocol. E.g., the prover provides $C_1' = C_1^{q^{d/2}}$ with a PoE, and then a zk-PoKE of $r'$ such that $g_1^{r'} = C_0 C_1' / C$. 

\item We could then recurse on $C_0^\alpha C_1$ which commits to $\alpha f_0 + f_1$ with the blinding factor $\alpha r_0 + r_1$. BUT we are not done yet, see next bullet point... 

\item The remaining problem is that the evaluation protocol opens $y_0 = f_0(z) \bmod p$ and $y_1 = f_1(z) \bmod p$, which is not zero knowledge. We need $y_0, y_1$ to be independently distributed subject to the constraint $y_0 + z^{d/2} y_1 = y \bmod p$, which the verifier checks. 

A solution is to modify $f_0$ and $f_1$ by adding constant terms $\alpha, \beta$ to each that cancel, i.e. $\alpha + z^{d/2} \beta = 0 \bmod p$, where $\alpha$ is uniformly distributed in $\ZZ_p$. This way the polynomials $f_0' = f_0 + \alpha$ and $f_1' = f_1 + \beta$ satisfy the relation $f_0'(z) + z^{d/2}f_1'(z) = f(z) \bmod p$. We end up revealing $y_0' = y_0 + \alpha \bmod p$ and $y_1' = y_1 + \beta \bmod p$, which is uniformly distributed in $\ZZ_p$ subject to $y_0' + y_1' = y \bmod p$. 

Finally, the prover needs to convince the verifier that it modified the $C_0$ and $C_1$ commitments appropriately. (It could not simply choose $f_0'$ and $f_1'$ in the first step because $f_0' + q^{d/2} f_1' \neq f$ as integer polynomials). 

However, the solution is still simple. The prover creates hiding commitments $C_\alpha$ to $\alpha$ and $C_\beta$ to $\beta$ and provides a zero-knowledge proof that $C_\alpha C_\beta ^{z^{d/2}}$ is a commitment to an integer multiple of $p$. This can be done efficiently through a combination of PoE and a PoKE. (Given $g^a$, to prove that $a = 0 \bmod p$ it suffices to provide $Q$ such that $Q^p = g^a$ and a PoKE for $Q$ base g. This can be made zero knowledge w/ the standard tricks). 

The protocol then proceeds on modified commitments $C_0' = C_0 C_\alpha$ and $C_1' = C_1 C_\beta$.

\end{itemize}

