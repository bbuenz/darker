\clearpage
\section*{\scalebox{1.25}{\appendixphrase}}

\section{Security Proofs}
\label{appendix:hardness}
In the preliminaries we already stated the two main hardness assumptions, the $r$-Strong RSA Assumption and the Adaptive Root Assumption.
We additionally use two more assumptions, however both of them reduce to the Strong RSA and the Adaptive Root Assumptions.

The first assumption states that computing the order for \emph{any} element is hard. It reduces to the Adaptive Root Assumption. Interestingly, it doesn't necessarily hold for all candidate groups of unknown order as we explain below. In particular it is important to exclude elements of known order such as $-1$ from the candidate unknown order group $\ZZ_n$.

\begin{assumption}[Order Assumption]
\label{assum:order}
	The Order Assumption holds for $\ggen$ if for any efficient adversary $\adv$:
\[        
                \Pr\left[\gr{w}\neq 1 \wedge \gr{w}^{\alpha}= 1: 
                \begin{array}{l} 
                      \GG \sample \ggen(\lambda) \\ 
                      (\gr{w},\alpha) \sample \adv(\GG) \\
                      \text{where } |\alpha|<2^{\poly{}}\in \ZZ\\
                      \text{and } \gr{w}\in \GG
                \end{array} 
        \right] \leq \negl \enspace .
\]
\end{assumption}
\begin{lemma}
\label{lem:ordertoadaptive}
	The Adaptive Root Assumption implies the Order Assumption.
\end{lemma}
\begin{proof}
	We show that given an adversary $\adv_{\textsf{Ord}}$ that breaks the Order Assumption we can construct with overwhelming probability $\adv_{\textsf{AR}}$ that breaks the Adaptive Root Assumption. We run $\adv_{\textsf{Ord}}$ to get a $\gr{w}\neq 1\in \GG$ and $\alpha \in \ZZ$ such that $\gr{w}^{\alpha}=1$. To construct $\adv_{\textsf{AR}}$, $\adv_{\textsf{AR},0}$ outputs $(\gr{w},\alpha)$. The challenger generates a random challenge $\ell$. If $\gcd(\ell,\alpha)=1$ then $\adv_{\textsf{AR},1}$ can compute $\beta\gets \ell^{-1} \bmod \alpha$ and output $\gr{u}\gets\gr{w}^{\beta}$. By construction $\gr{u}^{\ell}=\gr{w}$. The probability that $\gcd(\ell,\alpha)=1$ is overwhelming because $\gcd(\ell,\alpha)\neq 1 \implies \ell \not\vert \alpha$. This happens with negligible probability as $\ell$ is picked from a set of $2^\lambda$ primes and at most $\poly$ distinct primes can divide $\alpha$.
	\end{proof}
	
	
We also define the Fractional Root Assumption, which states that for random group elements $\gr{g}$ it is hard to find a tuple $(\gr{u}\in \GG,\alpha\in \ZZ,\beta\in \ZZ)$ such that $\gr{u}^{\beta}=\gr{g}^{\alpha}$. 
We say that $(\gr{u},\alpha,\beta)$ is a \emph{fractional root} of $\gr{g}$.
%unless $\frac{\alpha}{\beta}$ is an dyadic rational, \emph{i.e.}, a rational whose denominator is a power of $2$
%In RSA groups the assumption is also conjectured to hold if $\frac{\alpha}{\beta}$ is restricted to be an integer. 
Shoup\cite{CCS:CraSho99} showed that for the unknown order group of quadratic residues in $\ZZ_n$, where $n$ is the composite of two strong primes, that the Fractional Root Assumption reduces to just the Strong RSA Assumption.

\begin{assumption}[$r$-Fractional Root Assumption]
\label{assum:fracroot}
The \defn{$r$-Fractional Root Assumption} holds for $\ggen$ for any efficient adversary $\adv$:
\[        
                \Pr\left[\gr{u}^\beta = \gr{g}^{\alpha} \wedge \frac{\beta}{\gcd(\alpha,\beta)}\neq r^k,  k \in \NN   : 
                \begin{array}{l} 
                      \GG \sample \ggen(\lambda) \\ 
                      \gr{g} \sample \GG \\
                      (\alpha, \beta, \gr{u}) \sample \adv(\GG, \gr{g}) \\
                      \quad \textnormal{where} \, |\alpha|<2^{\poly}, \\
                      \quad |\beta|<2^{\poly} \in \ZZ, \\
                      \quad \textnormal{and} \, \gr{u} \in \GG 
                \end{array} 
        \right] \leq \negl \enspace .
\]
\end{assumption}
We say $(\alpha,\beta,\gr{u})$ is a non power of $r$ fractional root of $\gr{g}$.

The Fractional Root Assumption reduces to the Order Assumption (and therefore to the Adaptive Root Assumption) and the Strong RSA Assumption.
\begin{lemma}
\label{lem:strongtofractional}
	The Adaptive Root Assumption and the $r$-Strong RSA Assumption imply the $r$-Fractional Root Assumption 
	%if groups generated by $\ggen$ have order coprime with $r$ and there exists a PPT algorithm for taking $r$th roots in these groups.
\end{lemma}
\begin{proof}
	Given an adversary $\adv_{\textsf{FR}}$ that succeeds in breaking the Fractional Root Assumption for $\ggen$ we can construct either an adversary $\adv_{RSA}$ for the Strong RSA Assumption or an adversary $\adv_{\textsf{Ord}}$ that breaks the Order Assumption for $\ggen$. As shown in Lemma \ref{lem:ordertoadaptive} the Order Assumption reduces to the Adaptive Root Assumption with overwhelming probability. 
	We first generate a group of unknown order $\GG \sample \ggen(\lambda)$.
	Then we sample $\gr{g}\sample \GG$ as done in the strong \textsf{RSA} security definition.
	
	We now run the $\adv_{\textsf{FR}}$ on input $\GG$ and $\gr{g}$ to generate a tuple $(\alpha,\beta,\gr{u})$ such that $\gr{u}^{\beta}=\gr{g}^{\alpha}$. Let $\gamma=\gcd(\alpha,\beta)$ and $\alpha'=\frac{\alpha}{\gamma}\in \ZZ$ and  $\beta'=\frac{\beta}{\gamma}\in \ZZ$. Now either $\gr{g}^{\alpha'}=\gr{u}^{\beta'}$ or $\gr{g}^{\alpha'}/\gr{u}^{\beta'}$ is a non trivial element of order $\gamma$ which would directly break the Order Assumption. In that case we constructed $\adv_{\textsf{Ord}}$ that outputs $(\gr{g}^{\alpha'}/\gr{u}^{\beta'},\gamma)$.
	
	Now assume otherwise, i.e. $\gr{g}^{\alpha'}=\gr{u}^{\beta'}$. By construction $\gcd(\alpha',\beta')=1$ and we can efficiently compute integers $a,b$ such that $a \alpha'+b \beta'=1$. By assumption on $\adv_{\textsf{FR}}$ $\beta'$ is not $r^k$. Now let $\gr{w}\gets \gr{u}^{a}\gr{g}^{b}$. Note that $\gr{w}^{\alpha'\beta'}=\gr{g}^{\alpha'}$. So either $\gr{w}^{\beta'}=\gr{g}$ or $\gr{w}^{\beta'}/\gr{g}$ is a non-trivial element of order $\alpha'$. The first case breaks the Strong RSA Assumption, as we can construct $\adv_{\textsf{RSA}}$ that outputs $(\gr{w},\beta)$, and the second breaks the Order Assumption.
\end{proof}

\subsection{Binding}
\label{appendix:binding}
\def\thelemma{\ref{lem:binding}}
\begin{lemma}
	\bindinglemma
\end{lemma}
\begin{proof}
    Assume that there is an adversary that breaks the binding property of the scheme. Specifically, assume that some probabilistic polynomial time algorithm $\adv$ takes as input $\params$ and outputs $\gr{C} \in \GG, f(X) \in \ZZ(b)[X], f'(X)\in \ZZ(b)[X]$ such that with non-negligble probability $\pro{Open}(\params, \gr{C}, \tilde{f}(X), f(X)) = \pro{Open}(\params, \gr{C}, \tilde{f'}(X), f'(X)) = 1$ and $\tilde{f}(X) \neq \tilde{f'}(X)$. We proceed to show that this implies a violation of the Order Assumption~(Assumption \ref{assum:order}) and the Strong RSA Assumption~(Assumption \ref{assum:strongRSA}). The assumptions are incomparable so we show that either suffices to achieve the binding property of the commitment scheme.
    
	If $f(X)\neq f'(X)$ and $q/2>b$ then $f(q)\neq f'(q)\in \ZZ$. Since $\gr{g}^{f(q)}=\gr{g}^{f'(q)}=\gr{C}$ we have that $\gr{g}^{f(q)-f'(q)}=1$. This directly breaks the Order Assumption and we can also create an adversary $\adv_{RSA}$ that breaks the Strong RSA Assumption. To do so the $\adv_{RSA}$ picks an odd prime $\ell$ that is co-prime with $f(q)-f'(q)$ and computes $\gr{u}\gets \gr{g}^{\ell^{-1} \bmod (f(q)-f'(q))}$ as the $\ell$th root of $\gr{g}$.
\end{proof}

\subsection{Correctness}
\label{appendix:correctness}

\def\thelemma{\ref{lem:correctness}}
\begin{lemma}
	\correctnesslemma
\end{lemma}

\begin{proof}
In order to ensure correctness we must ensure that $b< q/2$ and that $|f|\leq b$. To show this we show that in each recursion step the honest prover's witness polynomial has coefficients bounded by $b$ and has degree $d$. 
We argue inductively that for each recursive call of $\pro{EvalBounded}$ the following constraints on the inputs are satisfied: The degree of $f(X)$ is bounded by $d$. $\gr{C}$ encodes the polynomial, \emph{i.e.}, $\gr{C}=\gr{g}^{f(q)}$ and $f(X)\in \ZZ(b)$. Also $f(z) = y\bmod p$.

Initially, during the execution of $\eval$, the prover maps the coefficients of a polynomial $\tilde{f}(X)\in \ZZ_p$ to an integer polynomial $f(X)$ with coefficients in $\ZZ(\frac{p-1}{2})$ and degree at most $d$ such that $\gr{C}=\gr{g}^{f(q)}$. Additionally $f(z)\bmod p=\tilde{f}(z)=y$.

 In a recursion steps where $d+1$ is odd, $f'(X)=X\cdot f(X)$ is a polynomial of degree $d+1$ such that $\gr{C}'=\gr{C}^q=\gr{g}^{q\cdot f(q)}=\gr{g}^{f'(X)}$ and the bound $b$ is unchanged as are the coefficients. Also, $f'(z)\bmod p = z \cdot f(z) \bmod p=z\cdot y\bmod p = y' \bmod p$. If $d+1$ is odd, then in the next step $d+1$ must be even.
 
 If $d+1$ is even then, $\prover$ computes $f_L(X)$ and $f_R(X)$ such that $f_L(X)+X^{\frac{d+1}{2}} f_R(X)=f(X)$. Consequently $f(z) \bmod p=f_L(z)+ z^{\frac{d+1}{2}} f_R(z)\bmod p=y_L+z^{\frac{d+1}{2}}  y_R\bmod p =y$. The \textsf{PoE} protocol has perfect correctness so {$\gr{g}^{f_L(q)+q^{\frac{d+1}{2}} f_R(X)}=\gr{C}$}.
 %\gr{C}_L\gr{C}_R^{(q^{\frac{d+1}{2}})}
 Finally $f'(X)=\alpha f_L(X) + f_R(X)\in \ZZ(\frac{p+1}{2}\cdot b)$ is a degree $d$ polynomial with coefficients bounded in absolute value by $(\frac{p+1}{2})\cdot b$. This is precisely the value of $b'$ the input to the next call of $\pro{EvalBounded}$. The value $y'$ is also correct:
$f'(z)\bmod p=\alpha f_L(z) +f_R(z) \bmod p= \alpha y_L +y_R\bmod p=y'$
 
 There are exactly $\lceil\log_2(d+1)\rceil$ recursion steps with even $d+1$. In the final recursion step we therefore have $b=\frac{p-1}{2}(\frac{p+1}{2})^{\lceil\log_2(d+1)\rceil}$ and as such the requirement that $q/2>\frac{p-1}{2}(\frac{p+1}{2})^{\lceil\log_2(d+1)\rceil}$. 
 So if $q>(p-1) (\frac{p+1}{2})^{\lceil \log_2(d+1)\rceil}$ then all verifier checks pass and the verifier outputs $1$.
\end{proof} 


\subsection{Proof of Theorem~\ref{thm:polycommitsecurity}}
\label{appendix:maintheoremproof}

\paragraph{Security of $\textsf{PoE}$ substitutions}
We first begin by showing that we can safely replace all of the $\textsf{PoE}$ evaluations with direct verification checks. Concretely, under the Adaptive Root Assumption, the $\eval$ protocol is as secure as the protocol $\eval'$ in which all $\textsf{PoE}$s are replaced by direct checks. We show that the witness-extended emulation for $\eval'$ implies the same property for $\eval$. This is useful because we will later show how to can build an extractor for $\eval'$, thereby showing that the same witness-extended emulation property extends to $\eval$.
\begin{lemma} \label{lemma:poe_security}
Let $\eval'$ be the protocol that is identical to $\eval$ but in line \ref{line:PoE} of $\pro{EvalBounded}$ $\verifier$ directly checks $\gr{C}_L\gr{C}_R^{q^{d'+1}}=\gr{C}$ instead of using a $\textsf{PoE}$. If the Adaptive Root Assumption holds for $\ggen$, and $\eval'$ has witness-extended emulation for polynomials of degree $d=\poly$, then so does $\eval$.
\end{lemma}
\begin{proof}
We show that if an extractor $E'$, as defined in Definition~\ref{def:wee}, exists for the protocol $\eval'$ then we can construct an extractor $E$ for the protocol $\eval$. Specifically, $E$ simulates $E'$ and presents it with a $\pro{Record}'(\cdots)$ oracle, while extracting the witness from its own $\pro{Record}(\cdots)$ oracle.

Whenever $E'$ queries the $\pro{Record}'$ oracle, $E$ queries its $\pro{Record}$ oracle and relays the response after dropping those portions of the transcript that correspond to the $\mathsf{PoE}$ proofs. Whenever $E'$ rewinds its prover, so does $E$ rewind its prover. When $E'$ terminates by outputting a transcript-and-witness pair $(\mathsf{tr}', f(X))$, $E$ adds $\mathsf{PoE}$s into this transcript to obtain $\mathsf{tr}$ and outputs $(\mathsf{tr}, f(X))$.

For each PPT adversary $(\adv,P^*)$, $E$ will receive a polynomial number of transcripts from its $\pro{Record}$ oracle. Any transcript $\tr$ of $\eval$ such that $\adv(\tr)=1$ and $\tr$ is accepting contains exactly $\lceil \log(d+1)\rceil$ $\textsf{PoE}s$ transcripts. 
So in total $E$ sees only a polynomial number of $\textsf{PoE}$ transcripts generated by a probabilistic polynomial-time prover and verifier. By Lemma~\ref{lem:poe} under the Adaptive Root Assumption, the probability that a polynomial time adversary can break the soundness of $\textsf{PoE}$, \emph{i.e.}, convince a verifier on an instance $(\gr{C}_R,\gr{C}/\gr{C}_{L},q^{d'+1})\not\in\mathcal{R}_{\textsf{PoE}}$, is negligible. 
Consequently, the probability that the adversary can break $\textsf{PoE}$ on \emph{any} of the polynomial number of executions of $\mathsf{PoE}$ is still negligible.

This means that with overwhelming probability all transcripts are equivalent to having the verifier directly check $(\gr{C}_R,\gr{C}/\gr{C}_{L},q^{d'+1})\in\mathcal{R}_{\textsf{PoE}}$. By assumption, the witness-candidate $f(X)$ that $E'$ outputs is a valid witness if the transcript $\mathsf{tr}'$ that $E'$ also outputs is accepting. The addition of honest $\mathsf{PoE}$ transcripts to $\mathsf{tr}'$ preserves the transcript's validity. So $\mathsf{tr}$ is an accepting transcript for $\pro{Eval}$ if and only if $\mathsf{tr}'$ is an accepting transcript for $\pro{Eval}'$. Therefore, $E'$ outputs a valid witness $f(X)$ whenever $E$ outputs a valid witness. This suffices to show that $\pro{Eval}$ has witness-extended emulation if $\pro{Eval}'$ has, and if the Adaptive Root Assumption holds for $\ggen$.
\end{proof}

\paragraph{Combining statements.} The $\eval$ protocol combines two statements into one by using a random linear combination of group elements, \emph{i.e.}, $\gr{C}'\gets \gr{C}_L^{\alpha}\gr{C}_R$. We now show that this step is sound and that given the discrete logarithm for $\gr{C}'$ the extractor can extract the discrete logarithm for $\gr{C}_L$ and $\gr{C}_R$ we also show that the we can bound the size of the discrete logarithm. We show that this statement holds in two settings. First we consider a group $\GG$ were the standard Strong RSA Assumption holds and group elements are encodings of integers. 
%Move next part to after lemma?
Then we will show that in groups in which taking square roots is easy we can extract dyadic rationals using the 2-Strong RSA Assumption.
 %For $\GG \gets \ggen(\lambda)$, and $\gr{g}\sample \GG$. Let $(\gr{C}_L\in \GG,\gr{C}_R\in \GG,\alpha\in [0,p-1],f\in [0,b];\gr{g}^{f}=\gr{C}_L^{\alpha}\gr{C}_R)$ and  $(\gr{C}_L,\gr{C}_R,\alpha'\in [0,p-1],f'\in [0,b];\gr{g}^{f'}=\gr{C}_L^{\alpha'}\gr{C}_R)$  be two transcripts for $\alpha\neq \alpha'$.
\begin{lemma}[Combining for integer witnesses]
\label{lem:intrandomcombine}
	For $\GG \gets \ggen(\lambda)$, and $\gr{g}\sample \GG$. 
	Let $(z,\gr{C}_L,\gr{C}_R,y_L,y_R,\alpha,f,y)$ and  $(z,\gr{C}_L,\gr{C}_R,y_L,y_R,\alpha',f',y')$ be two transcripts such that $\gr{g}^{f}=\gr{C}_L^{\alpha}\gr{C}_R$ and $\gr{g}^{f'}=\gr{C}_L^{\alpha'}\gr{C}_R$ for group elements $\gr{C}_L,\gr{C}_R \in \GG$, and integers $\alpha,\alpha' \in  [-\frac{p-1}{2},\frac{p-1}{2}]$, $\alpha\neq \alpha'$. Further let $f,f'\in \ZZ$ be such that $f(X)\gets\dec(f)$ and $f'(X)\gets\dec(f')$ are degree $d$ bounded polynomials with coefficients bounded by $b$, \emph{i.e.}, $f(X),f'(X)\in \ZZ(b)[X]\subset \ZZ[X]$. And finally let $y=f(z)\bmod p$ and $y'=f'(z)\bmod p$.
	 Then there exists a PPT algorithm $\mathcal{X}$ that given these transcripts computes either (1) $y_L,y_R\in \ZZ_p,f_L(X),f_R(X)\in \ZZ((p-1) \cdot b)[X]$  such that $f_L(z)=y_L\bmod p$ and $f_R(z)=y_R \bmod p$ or (2) an element in $\GG$ of known order or (3) a fractional root of $\gr{g}$.
\end{lemma}
\begin{proof}
	Using the transcripts $\mathcal{X}$ computes $\Delta_\alpha\gets\alpha-\alpha'$ and $\Delta_f\gets f-f'$ such that $\gr{C}_{L}^{\Delta_\alpha}=\gr{g}^{\Delta_f}$. 
 If $\frac{\Delta_f}{\Delta_\alpha}$ is not an integer then $\mathcal{X}$ outputs a fractional root of $\gr{g}$, that is the tuple $(\Delta_f,\Delta_\alpha,\gr{C}_{L})$.  
 If $\frac{\Delta_f}{\Delta_\alpha}$ on the other hand is an integer then $\mathcal{X}$ can compute $\gr{D}\gets\gr{g}^{\frac{\Delta_f}{\Delta_\alpha}}$. Either $\gr{D}=\gr{C}_{L}$ or $(\gr{D}/\gr{C}_{L})^{\Delta_\alpha}=1$. In the second case, $\gr{D}/\gr{C}_{L}$ is an element of known order.
   
  Otherwise $\gr{D} = \gr{C}_L$ and we have $\gr{C}_{L}=\gr{g}^{f_L}$ where $f_L=\frac{\Delta_f}{\Delta_\alpha}$ is an integer.
Additionally $\gr{C}_R=\gr{g}^{f_R}$ for $f_R\gets f-\alpha \cdot f_L$.

$\mathcal{X}$ now computes the corresponding polynomials $f_L(X)\gets \dec(f_L)$ and $f_R(X)\gets \dec(f_R)$.
Now if for all $i$, the coefficients $f_i$ and $f'_i\in [-b,b]$ and $\alpha,\alpha' \in [-\frac{p-1}{2},\frac{p-1}{2}]$ then by the triangle inequality we have that for the $i$th coefficient of $f_L(X)$, $f_{L,i}\in [-2b,2b]$. Additionally we have $f_{R,i}=\frac{f_i'\alpha-f_i \alpha'}{\Delta_\alpha}$. Using the triangle inequality again we have that $f_{R,i} \in [-(p-1) \cdot b, (p-1) \cdot b]$. For an odd prime $p$, $(p-1)\cdot p\geq 2$. The bound on $f_{R,i}$ is, therefore, greater than the bound on $f_{L,i}$. This gives us $f_L(X),f_R(X)\in \ZZ({(p-1) \cdot b})[X]$

Let $y_L=\frac{y-y'}{\Delta_\alpha} \bmod p=\frac{f(z)-f'(z)}{\Delta \alpha} \bmod p$ and $y_R= y-\alpha\frac{y-y'}{\Delta_\alpha} \bmod p$. Since $f_L(X)=\frac{f(X)-f'(X)}{\Delta \alpha}$ this shows that $y_L=f_L(z)\bmod p$ and $y_R=f_R(z)\bmod p$.
%The actual bound on f_{R,i} is b*(p-2)
\end{proof}

\def\thetheorem{\ref{thm:polycommitsecurity}}
\begin{theorem}
\maintheorem
\end{theorem}
\begin{proof}
We will prove security by showing that given a polynomial time adversary $\adv_{\eval}$ that succeeds in convincing an honest verifier in the $\eval$ protocol on any public input with non-negligible probability we can either (1) construct an adaptive root adversary $\adv_{\textsf{AR}}$, (2) extract an element of known order, and hence break the Order Assumption, (3) extract a fractional root of $\gr{g}\in \GG$ or (4) extract the polynomial $f(X)\in \ZZ[X]$ such that $f(X)$ has degree at most $d$ and the coefficients of $f(X)$ are integers bounded by $q/2$, such that $f(q)$ is a unique encoding of $f(X)$, $\gr{g}^{f(q)}=\gr{C}$, and $f(z) \bmod p=y$. The proof will use the general forking lemma (Lemma \ref{lemma:GFL}) to show that the polynomial commitment scheme has witness-extended emulation.

In particular we construct an extractor $\mathcal{X}$ that given transcripts with $2$ distinct challenges per round, \emph{i.e.}, $2^{\lceil\log_2(d+1)\rceil}<2 (d+1)$ transcripts in total, can compute either an opening to the commitment scheme, an element of known order, or a fractional root of $\gr{g}\in\GG$ as encoded in the public parameters $\params$.

Using Lemma \ref{lemma:poe_security} and under the Adaptive Root Assumption, it suffices to consider an extractor $\mathcal{X}$ that works on transcripts of $\eval'$ were all $\textsf{PoE}$s prove true statements. That is $\gr{C}_L\gr{C}_R^{q^{d'+1}}=\gr{C}$ on all transcripts.

%Now consider the case where $\gr{C}_L \gr{C}_R^{(q^{d'+1})}= \gr{C}$ for all executions. 
Given a tree of $\eval'$ transcripts, the extractor $\mathcal{X}$ recursively either extracts the encoding of an integer polynomial $f(X)\in \ZZ(b)[X]\subset \ZZ[X]$ with bounded coefficients or a break of the Order Assumption or the Fractional Root Assumption. 
In order to break the Order Assumption we instantiate the adversary $\adv_{\textsf{Ord}}$ with the description of the group $\GG$. We also instantiate the fractional root adversary $\adv_{\textsf{FR}}$ with $\GG$ and $\gr{g}$ as encoded in $\params$.

Given the tree of transcripts as specified in the general forking lemma (Lemma \ref{lemma:GFL})  with branching factor $2$ at each level, \emph{i.e.}, $2$ different challenges, we will extract a witness at each node of the tree given witnesses for both nodes' children. Each level corresponds to a separate invocation to $\pro{EvalBounded}'$. We denote the input to $\pro{Eval}'$ without subscripts, \emph{i.e.}, $\gr{C},z,y,d;f(X)$, and the input to $\pro{EvalBounded}'$ with a subscript indicating the round, \emph{e.g.}, $d_0=d$, $\gr{C}_0=\gr{C}$ and $d_{\lceil \log_2(d)\rceil }=0,\gr{C}_{\lceil \log_2(d+1)\rceil }=\gr{g}^{f}$, \emph{etc}. For the witness polynomials we use superscripts and parentheses, \emph{i.e.}, $f^{(i)}(X)$ to avoid confusion with the notation for coefficients.  
%We let $\alpha$ and $\alpha'$ denote the two distinct challenges at each node of the transcript tree. We use $'$ to denote the proof elements and witnesses corresponding to the $\alpha'$ challenge, \emph{e.g.}, $\gr{C}_i'$.

In each round the extracted witness is an integer polynomial $f^{(i)}(X)\in \ZZ[X]$ such that $\gr{g}^{f^{(i)}(q)}=\gr{C}_i$ and such that the coefficients are bounded, \emph{i.e.}, all $f^{(i)}(X)$ are in $\ZZ(b_i)[X]$. The degree of $f^{(i)}(X)$ is at most $d_i$ and $f(z) \equiv y \bmod p$. Note that for odd primes $p$ and integer $z$, $f(z)\bmod p$ is always defined.

We extract starting from the leafs of the tree, \emph{i.e.}, $d_{\lceil \log_2(d)\rceil}=0$. From the transcript we can directly extract the constant integer polynomial $f(X)=f \in \ZZ$ such that $\vert f \vert \leq(p-1) (\frac{p+1}{2})^{\lceil \log_2(d+1)\rceil}$, $y=f \mod p$, $f(X)=y\in \ZZ_p[X]$ and $\gr{g}^{f}=\gr{C}$ as the witness.

We now show how to compute the witness for $i-1$ given a witnesses for $i$. 
%%%Make into lemma?
If $d_i+1$ is odd then we have $\gr{C}_{i-1}^q=\gr{C}_i$. Since $\gr{C}_{i-1}=\gr{g}^{f^{(i-1)}(q)}$ we either have that $q$ divides $f^{(i-1)}(q)$ or since $q$ is odd we have a fractional root of $\gr{g}$. 
If this is not the case then $f^{(i-1)}(q)=f^{(i)}(q)\cdot q^{-1}$ and $f^{(i)}(X)=\dec(f^{(i)}(q))$ has a zero constant term. Additionally since $y_i=y_{i-1}\cdot z$ and $f^{(i)}(z)\equiv y_i \bmod p$ we have $f^{(i-1)}(z)\equiv y_{i-1} \bmod p$, \emph{i.e.}, $f^{(i-1)}(q)$ is a valid witness and the degree of $f^{(i-1)}(X)=\dec(f^{(i-1)}(q))$ is at most $d_{i-1}=d_i-1$. 

Now if $d_i+1$ is even then we can use Lemma~\ref{lem:intrandomcombine} to either extract a fractional root of $\gr{g}$, an element of known order in $\GG$ or the two bounded polynomials $f_{L}^{(i)}(X),f_{R}^{(i)}(X)$ of degree $\frac{d_i+1}{2}-1$ and $y_L=f_{L}^{(i)}(z)\bmod p$ as well as $y_R=f_{R}^{(i)}(z)\bmod p$. 
This yields $f^{(i)}(X)=f_{L}^{(i)}(X)+X^{\frac{d_i+1}{2}} f_{R}^{(i)}(X)$ a polynomial of degree at most $d_i$ such that $\gr{C}_i=\gr{g}^{f^{(i)}(q)}$ and such that $f^{(i)}(z) \bmod p=y_L+y_R \cdot z^{\frac{d_i+1}{2}}\bmod p=  y_i$.

Note that the application of Lemma~\ref{lem:intrandomcombine} requires that $q/2$ is greater than the magnitude of each of $f^{(i)}$'s coefficient. We will show now that this is the case. 

The check on $f$ ensures that $|f|\leq b=\frac{(p-1)}{2} (\frac{p+1}{2})^{\lceil \log_2(d+1)\rceil}$. 

Lemma \ref{lem:intrandomcombine} in each invocation guarantees that the extracted parent polynomial has coefficients at most $(p-1)$ times larger than the coefficients of the children's polynomials. Given that the transcript tree has depth $\lceil \log_2(d+1)\rceil$ we get that the final extracted polynomial $f_0(X)\in \ZZ(b)[X]$ has coefficients bounded by $b= \frac{p-1}{2}(\frac{p^2-1}{2})^{\lceil \log_2(d+1)\rceil}$.

Therefore, $q$ needs to be large enough such that $f_0(X)$ is uniquely decodable, \emph{i.e.}, $q>2\cdot b=(p-1)(\frac{p^2-1}{2})^{\lceil \log_2(d+1)\rceil}$.

We can successfully extract either a witness or a fractional root or an element of known order from any tree of valid transcripts of $\eval'$.
Under the Fractional Root Assumption and the Order Assumption, the probability that a polynomial time adversary along with a polynomial time extractor $\mathcal{X}$ can produce such a fractional root or an element of known order is negligible. $\eval'$, therefore, has witness extended emulation and under the Adaptive Root Assumption by Lemma \ref{lemma:poe_security} so does $\eval$.
Lemma \ref{lem:ordertoadaptive} and Lemma \ref{lem:strongtofractional} show that we can reduce the hardness assumptions to just the Adaptive Root Assumption and the Strong RSA Assumption.

\end{proof}



%%%%%DYADIC


\subsection{Proof of Theorem~\ref{thm:dyadicpolysecurity}}
\label{apx:dyadic}
We begin by stating and proving the combining lemma, (Lemma~\ref{lem:intrandomcombine}) for dyadic rational witnesses.
%PROOF FOR DYADIC RATIONALS
\begin{lemma}[Combining for Dyadic Rational Witnesses]
\label{lem:dyadiccombining}
Let $\tr$ and $\tr'$ be two transcripts as specified in Lemma \ref{lem:intrandomcombine} with the difference that $f,f'\in \mathbb{D}$ are dyadic rationals such that $f(X)\gets \dec(f)$ and $f'(X)\gets \dec(f')$ are degree $d$ bounded dyadic rational polynomials with coefficients' numerators bounded by $N$ and denominators bounded by $D$, i.e. $f(X),f'(X)\in\mathbb{D}(N,D)$.
Assume that there exists a PPT algorithm for taking square roots of any element in $\GG$ and that the order of $\GG$ is odd, then there exists a PPT algorithm $\mathcal{X}$ that given these transcripts computes either (1) $y_L,y_R\in \ZZ_p,f_L(X),f_R(X)\in \mathbb{D}(N\cdot (p-1),D\cdot (p-1))[X]$  such that $f_L(z)=y_L\bmod p$ and $f_R(z)=y_R \bmod p$ or (2) an element in $\GG$ of known order or (3) a non-power of $2$ fractional root of $\gr{g}$.
\end{lemma}
\begin{proof}
The proof follows a similar structure to the proof of Lemma~\ref{lem:intrandomcombine}. 
		
	Using the transcripts we get $\Delta_\alpha\gets\alpha-\alpha'$ and $\Delta_f\gets f-f'$ such that $\gr{C}_{L}^{\Delta_\alpha}=\gr{g}^{\Delta_f}$. 
 If $\frac{\Delta_f}{\Delta_\alpha}$ is not a dyadic rational then this gives us a non-power of $2$ fractional root of $\gr{g}$ root of $\gr{g}$, that is the tuple $(\Delta_f,\Delta_\alpha,\gr{C}_{L})$.  
 If $\frac{\Delta_f}{\Delta_\alpha}$ on the other hand is a dyadic rational then we can compute $\gr{D}\gets\gr{g}^{\frac{\Delta_f}{\Delta_\alpha}}$. This may requires taking a power of $2$ root. By assumption the group order is odd, so every element has a square root and there exists an efficient algorithm for taking square roots. This implies that taking higher power of $2$ roots is also efficient.
  
  Now either $\gr{D} = \gr{C}_L=\gr{g}^{f_L}$ or we can extract an element of known order. Additionally $\gr{C}_R=\gr{g}^{f_R}$ for $f_R\gets f-\alpha \cdot f_L$.

 We now compute the corresponding polynomials $f_L(X)\gets \dec(f_L)$ and $f_R(X)\gets \dec(f_R)$.
Now if the coefficients $f_i$ and $f'_i\in \mathbb{D}(N,D)$ and $\alpha,\alpha' \in [-\frac{p-1}{2},\frac{p-1}{2}]$ then by the triangle inequality we have that for the numerator of the $i$th coefficient of $f_L(X)$ is between $[-2N,2N]$. The denominator grows by at most $p-1$. The bound on the denominators is therefore $D\cdot (p-1)$. Additionally we have $f_{R,i}=\frac{f_i'\alpha-f_i \alpha'}{\Delta_\alpha}$. Using the triangle inequality again we have that the numerator of $f_{R,i} \in [-(p-1) \cdot N, (p-1) \cdot N]$. The denominator is bounded by $D\cdot (p-1)$. This gives us $f_L(X),f_R(X)\in \mathbb{D}((p-1) \cdot N,D\cdot (p-1))[X]$

Finally $y_L=f_L(z)$ and $y_R=f_R(z)$ as in Lemma~\ref{lem:intrandomcombine}. It is important that $2$ is co-prime with the odd prime $p$ such that each dyadic rational can be mapped to a field element.
\end{proof}


We now restate the theorem for the security of the protocol with dyadic rational witnesses in groups where taking square roots is easy.

\def\thetheorem{\ref{thm:dyadicpolysecurity}}
\begin{theorem}
\dyadicmaintheorem
\end{theorem}

\begin{proof}
The proof largely follows the same structure of the proof of Theorem~\ref{thm:polycommitsecurity}.
	
We will prove security by showing that we can extract a dyadic rational polynomial $f(X)\in \mathbb{Z}[X]$ such that $f(X)$ has degree at most $d$ and the coefficients of $f(X)$ are dyadic rationals such that the product of the numerator and denominator is bounded by $q/2$. This ensures that $f(q)$ is a unique encoding of $f(X)$. Additionally $\gr{g}^{f(q)}=\gr{C}$ and $f(z) \bmod p=y$. The proof will use the general forking lemma (Lemma \ref{lemma:GFL}) to show that the polynomial commitment scheme has witness-extended emulation. 

We use the same extractor $\mathcal{X}$ as in Theorem \ref{thm:polycommitsecurity} with one key distinction. 
For $d+1$ odd we invoke the extractor described by Lemma~\ref{lem:dyadiccombining} instead of Lemma~\ref{lem:intrandomcombine}. 
This means that at every tree level either bounded dyadic rational witness polynomials are extracted or an element of known order or a non-power of $2$ fractional root of $\gr{g}$. By assumption the ladder two cases happen only with negligible probability.

We, therefore, now need to compute a bound on the size of the extracted polynomial. 
The check on $f$ ensures that $f\in \ZZ$ and that $|f|\leq b=\frac{(p-1)}{2} (\frac{p+1}{2})^{\lceil \log_2(d+1)\rceil}$. We can write $f\in\mathbb{D}(\frac{(p-1)}{2} (\frac{p+1}{2})^{\lceil \log_2(d+1)\rceil},1)$. By \ref{lem:intrandomcombine} both the numerator and the denominator grow by at most a factor $p-1$ in every round.
Given that the transcript tree has depth $\lceil \log_2(d+1)\rceil$ we get that the final extracted polynomial $f_0(X)\in \mathbb{D}(N,D)[X]$ has coefficients with numerators bounded by $N= \frac{p-1}{2}(\frac{p^2-1}{2})^{\lceil \log_2(d+1)\rceil}$ and denominators bounded by $D=(p-1)^{\lceil \log_2(d+1)\rceil}$

$q$ needs to be large enough such that $f_0(X)$ is uniquely decodable, \emph{i.e.}, $q>2\cdot N\cdot b=(p-1)^{\lceil \log_2(d+1)\rceil+1}(\frac{p^2-1}{2})^{\lceil \log_2(d+1)\rceil}$.

This shows that we can successfully extract either a witness or a non-power of $2$ fractional root or an element of known order from any tree of valid transcripts.
Under the $2$-Fractional Root Assumption and the Order Assumption, the probability that a polynomial time adversary along with a polynomial time extractor $\mathcal{X}$ can produce such a a  non-power of $2$ fractional root or an element of known order is negligible. $\eval'$, therefore, has witness extended emulation and under the Adaptive Root Assumption by Lemma \ref{lemma:poe_security} so does $\eval$.
Lemma \ref{lem:ordertoadaptive} and Lemma \ref{lem:strongtofractional} show that we can reduce the hardness assumptions to just the Adaptive Root Assumption and the  $2$-Strong RSA Assumption. 
\end{proof}

\section{Hiding Polynomial Commitment with HVZK Eval} 
\label{appendix:zeroknowlege}
This section sketches how to make the polynomial commitment scheme zero knowledge. 

\paragraph{Hiding and Zero-knowledge.} For many applications it is important the polynomial commitment is hiding, and that the evaluation protocol is zero-knowledge, i.e. it reveals nothing about the polynomial other than the evaluation at the requested point.
Towards this end we construct a Pedersen commitment for groups of unknown order.
The commitment key are two random group elements $\gr{g}$ and $\gr{h}$ with unknown discrete logarithm. If computing the logarithm is hard then the commitment is binding.
 The polynomial commitment changes such that $\gr{g}^{\bar{f}(q)}$ we get $\gr{g}^{\bar{f}(q)}\gr{h}^{r}$ for random $r \sample [0, B\cdot 2^{2\lambda}]$, where $B$ is an upper bound on the order of the unknown order group $\GG$.  
The commitment is computationally hiding under a \emph{subgroup indistinguishability assumption}, as formulated by Brakerski and Goldwasser~\cite{C:BraGol10}. This assumption states that the groups generated by any two non-trivial elements are computationally indistinguishable. Under this assumption $\gr{g}^{\bar{f}(q)}\gr{h}^r$ is statistically indistinguishable from a random group element.
This basic subgroup indistinguishability assumption holds in generic groups of unknown order. 

In class groups we need to ensure that the order is guaranteed to be odd~\cite{PKC/BucHam01}. If not then Gauss's square root algorithm could be used to distinguish subgroups of even and of odd order.
% For class groups, we assume the subgroup indistinguishability assumption applies in $NQR$, the subgroup of non-quadratic residues. It is easy to convert the deterministic $H$ into a PPT hash function onto $NQR$ by iterating until hitting the first quadratic non-residue (succeeding in an expected constant number of iterations). 

We can now use the general Cramer Damgard\cite{C:CraDam98} transform to turn the evaluation scheme into an honest verifier zero-knowledge scheme.  
To do this all the prover messages are replaced by a hiding commitments and the verification checks by zero-knowledge proofs. For the polynomial commitments we use the hiding commitments described above. For the evaluations we use Pedersen commitments\cite{C:Pedersen91} in a prime order group of known order $p$. To use the Pedersen commitment we require that computing discrete logarithms is hard in these groups. 
We use the ${UO}$ subscript for the group of unknown order and $PO$ for the prime order group, i.e. $\GG_{UO}$ and $\GG_{PO}$.
Similarly $\commit_{UO}:\ZZ^2 \rightarrow\GG_{UO}=f(X),r \rightarrow \gr{g}^{f(q)}\gr{h}^{r}$ and $\commit_{PO}:\ZZ_p^2\rightarrow \GG_{PO}$ are the commitment functions.

To commit to the polynomial we compute $\gr{C}\gets \commit_{UO}(f(q);r)$ for a random blinding factor $r$. 
We also compute $\gr{Y}\gets \commit_{PO}(y;0)$ a non hiding commitment to $y$.

In every round the prover sends $\gr{C}_L\gets\commit_{UO}(f_L(q);r_L)$ and $\gr{C}_R\gets\commit_{UO}(f_R(q);r_R)$ for $r_L,r_R \sample [0,B\cdot 2^{\lambda}]$. Additionally the prover sends $\gr{Y}_R\gets\commit_{PO}(f_R(z),r_y)$ for $r_Y\sample \ZZ_p$. 
Analogously to the optimized protocol (Section~\ref{subsec:optimization}) the verifier can directly compute commitment to $y_L$: $\gr{Y}_L\gets\gr{Y}/(\gr{Y}_R^{(q^{d'+1})})\in \GG_{PO}$. Additionally we can compute $\gr{C}'\gets \gr{C}_L\gr{C}_R^{q^{d'+1}}$. We then need to show that $\gr{C}'$ and $\gr{C}$ commit to the same value. This can be done using a $\Sigma$ proof of discrete log equality that is outlined below and described in more detail in Boneh et al.\cite{C:BonBunFis19}.
 
Finally let $\gr{C}'\in \GG_{UO}$ and $\gr{Y}'\in \GG_{PO}$ be the commitments in the final round, i.e. when $d=0$. The prover now needs convince the verifier, in zero-knowledge, that $\gr{C}'$ and $\gr{Y}'$ commit to the same integer. Additionally the prover convinces the verifier that $\gr{C}$ commits to a bounded integer. Luckily, both of these can be achieved simultaneously using a single $\Sigma$ proof of discrete logarithm equality outlined below.

In the protocol $b$ is the bound on the size of $f$ and $B$ an independent upper bound on the order of $\GG_{UO}$.
 \noindent\begin{mdframed}[userdefinedwidth=\textwidth]
\begin{minipage}{\textwidth}
	\begin{flushleft}
	\pro{$\Sigma$ Proof of Equality} for Relation: $\mathcal{R}_{EQUAL}$\\
	Input: $\gr{C} \in \GG_{UO}, \gr{Y}\in \GG_{PO}, b,B\in \ZZ$, Witness: $f \in \ZZ(b),r_C \in \ZZ,r_Y \in \ZZ_p$\\
	Statement: $\gr{C}=\commit_{UO}(f;r_C)$ and $\gr{Y}=\commit_{PO}(f \bmod p;r_Y)$
	\begin{enumerate}[nolistsep]
		    \item \prover samples $k_f,k_C \sample [0,2^{2\lambda} B]$ and $k_Y \sample \ZZ_P$ and computes $\gr{R}_C\gets \commit_{UO}(k_f;k_C)$ and $\gr{R}_Y\gets \commit_{PO}(k_f \bmod p;k_C)$.
		    \item \prover sends $\gr{R}_C,\gr{Y}$ to $\verifier$
		    \item \verifier samples random $\alpha\in [0,2^\lambda]$ and sends it to $\prover$
		    \item \prover computes $s_f\gets k_f + \alpha \cdot f, s_C=k_C+ \alpha \cdot r_C \in \ZZ$, as well as $s_Y\gets k_Y+ \alpha\cdot r_Y \bmod p$. \prover sends $s_f,s_C,s_Y$ to \verifier.
		    \item \verifier checks that $s_f\in [-b,b\cdot 2^{\lambda}+2^{2\lambda}B]$, $\commit_{UO}(s_f;s_C)=\gr{R}_C C^{\alpha}$ and $\commit_{PO}(s_f \bmod p;s_Y)=\gr{R}_Y Y^{\alpha}$
		    \item \pcif{}check passes \textbf{then} \textbf{return} 1 \textbf{else} \textbf{return} 0
		\end{enumerate}
	\end{flushleft}
\end{minipage}
\end{mdframed}
 The protocol is zero-knowledge if the commitments are hiding and is an argument of knowledge for $\mathcal{R}_{EQUAL}$.
 \paragraph{Proof sketch}
 We show that the protocol is zero-knowledge by showing that we can build an efficient simulator that can create valid looking transcripts without knowing the witness. The simulator samples a challenge $\alpha$ as well as random $s_Y$, $s_C$ and $s_f$ in the appropriate domains. Then it computes $\gr{R}_C$ and $\gr{R}_Y$ directly from the verification equations and the sampled elements. Note that $\gr{R}_C$ is a random group element in both the honest and the simulated transcripts. $\gr{R}_C$ and $\gr{R}_Y$ are computationally indistinguishable from random group elements under the subgroup indistinguishability assumption as shown in \cite{AC:DamFuj02,C:BraGol10}. This shows that the simulated transcripts are computationally indistinguishable from honestly generated transcripts.
 
 To show that the protocol is extractable we receive two transcripts $\tr=(\gr{R}_C,\gr{R}_Y,\alpha,s_f,s_C,s_Y)$ and $\tr'=(\gr{R}_C,\gr{R}_Y,\alpha',s_f',s_C',s_Y')$ for $\alpha \neq \alpha'$. We compute $\commit(s_f-s_f',s_C-s_C')=\gr{C}^{\alpha-\alpha'}$. Boneh et al. \cite{C:BonBunFis19} that the probability that $s_f-s_f'$ and $s_C-s_C'$  do not divide $\alpha-\alpha'$ is negligible. We therefore compute $\commit_{UO}(\frac{ s_f-s_f'}{\alpha-\alpha'}; \frac{ s_C-s_C'}{\alpha-\alpha'})=\gr{C}$. Additionally, we can compute $\commit_{PO}( \frac{ s_f-s_f'}{\alpha-\alpha'} \bmod p; \frac{ s_y-s_y'}{\alpha-\alpha'})=\gr{Y}$. This shows that $f=\frac{ s_f-s_f'}{\alpha-\alpha'} \bmod p$ is equal to $y$ mod $p$. Additionally the bound on $s_f$ gives us an upper bound on the size of $f$. Given a sufficiently large $q$ this shows that the overall polynomial commitment scheme still has witness extended emulation.


\section{New ZK based on Benedikt's Suggestion}
Many applications, such as the construction of ZK-SNARKs, require a polynomial commitment scheme where an evaluation leaks no information about the committed polynomial beyond its value at the queried point. To provide this we show how to build a hiding polynomial commitment along with a zero-knowledge evaluation protocol.

We start by defining what it means for a polynomial commitment scheme to be \emph{hiding}:

\begin{definition}
A commitment scheme $\Gamma = (\pro{Setup}, \pro{Commit}, \pro{Open})$ is \defn{hiding} if for all probabilistic polynomial time adversaries $\adv = (\adv_0, \adv_1)$, the probability of distinguishing between commitments of different messages is negligible:
\[
	\left| 1 - 2 \cdot \mathrm{Pr}\left[
		\hat{b} = b \ \middle| \ 
		\begin{array}{l}
			\params \gets \pro{Setup}(1^\lambda) \\
			m_0, m_1, \st \gets \adv_0(\params) \\
			b \sample \{0,1\} \\
			(c; r) \gets \pro{Commit}(\params, m_b) \\
			\hat{b} \gets \adv_1(\st, c)
		\end{array}
	\right] \right| \leq \negl \enspace .
\]
\end{definition}
If the property holds for all algorithms then we say that the commitment is \emph{statistically} hiding.
\paragraph{Hiding Polynomial Commitment}
We make the polynomial commitment described in Section~\ref{sec:protocol} hiding by adding a degree $d+1$ term with a large random coefficient. Let $B\geq |\GG|$ be a publicly known upper bound on the order of $\GG$. We will choose the blinding coefficient between $0$ and $B\cdot 2^\lambda$. Formally, the hiding commitment algorithm is described as follows:
\begin{itemize}
	\item $\pro{CommitH}(f(X) \in \ZZ_p[X]) \rightarrow (\gr{C}; \hat{f}(X), d, r)$. Lift $f(X) \in \mathbb{Z}_p[X]$ to $\hat{f}(X) \in \mathbb{Z}(\frac{p-1}{2})[X]$ and select random integer $r \sample [0,B\cdot 2^\lambda)$. Compute $d \gets \deg(f(X))$ and $\gr{C} \gets (\hat{f}(q)+q^{d+1} \cdot r)\cdot \gr{g}$ and return commitment $\gr{C}$ with secret opening information $\hat{f}(X), d, r$.
\end{itemize}

Let $\gr{h} \leftarrow q^{d+1}\cdot \gr{g}$. 
To argue that $\gr{C}$ is hiding, it suffices to show that $\gr{C}$ is computationally indistinguishable from a random element of $\langle \gr{g}\rangle$, the cyclic group generated by $\gr{g}$. In a setting with trusted-setup, in which the trusted party has a trapdoor to compute the order of $\gr{g}$ (e.g. RSA groups), the trusted party can select $q$ such that $\gr{g}$ and $\gr{h}$ generate the same subgroup. In this case, $\gr{h}^r$ for $r \sample [0, B\cdot 2^\lambda)$ has statistical distance at most $2^{-\lambda}$ from uniform in $\langle \gr{g} \rangle$, so long as $B \geq |\langle \gr{g} \rangle|$. 

%The random coefficient $r$ ensures that $\gr{C}$ is nearly indistinguishable from a random group element. Assume that $\gr{g}$ and $\gr{g}^{q^{d+1}}$ generate the same group with order $n=|\langle \gr{g}\rangle|$. The element $\gr{A}\gets\gr{g}^c$ for $c\sample [0,n)$ is uniformly random from this subgroup. If $r \sample [0,B\cdot 2^\lambda)$ and $B\geq n$ then $\gr{B}\gets \gr{g}^r$ has statistical distance at most $2^{-\lambda}$ from uniform. Consequently $\gr{C}$ is statistically negligibly far from a random group element. 

Unfortunately, in a setting without a trusted setup, $\gr{h}$ might only generate a subgroup of $\langle \gr{g} \rangle$. The commitment then becomes computationally hiding under a \emph{Subgroup Indistinguishability Assumption}\footnote{Brakersi and Goldwasser define subgroup indistinguishability assumptions in a related but slightly different way.}~\cite{C:BraGol10}: our precise assumption is that no efficient adversary can distinguish a random element of $\langle \gr{h} \rangle$ from $\langle \gr{g} \rangle$ for any non-trivial $\gr{h} \in \langle \gr{g} \rangle$. %If an adversary can distinguish $\gr{C}$ from a random element with probability more than $2^{-\lambda}$ then he must be able to distinguish the subgroup generated by $\gr{g}^{q^{d+1}}$ from the subgroup generated by $\gr{g}$. 
For simplicity, Theorem~\ref{thm:hiding} assumes that $\gr{g}$ and $\gr{h}$ generate the same group. 

If the condition $\langle \gr{g} \rangle = {q^{d+1}} \cdot \langle \gr{g}\rangle$ cannot be guaranteed then the security proof must be modified to show that an adversary that can efficiently distinguish commitments with non-negligible probability must be able to distinguish between random elements sampled from $\langle \gr{g} \rangle$ and random elements sampled from $\langle q^{d+1}\cdot \gr{g} \rangle$. The informal proof sketch is as follows. 
Suppose there exists a non-uniform distinguisher $D_{f,g}$ that can distinguish freshly generated commitments to $f$ and $g$ with non-negligible probability, then the non-uniform adversary $\mathcal{A}_{f,g}$ may be constructed as follows: upon receipt of $\gr{g}'$ sampled either from $\langle \gr{g} \rangle$ or from $\langle  q^{d+1}\cdot \gr{g}\rangle$, it sends $f(q)\cdot \gr{g}+ \gr{g}'$ and $g(q)\cdot \gr{g}+ \gr{g}'$ to the distinguisher $D_{f,g}$. 
In the case that $\gr{g}'$ was sampled from $\langle  q^{d+1}\cdot \gr{g} \rangle$ this is a statistical simulation of a pair of commitments to $f$ and $g$ respectively, hence the distinguisher should succeed with non-negligible probability. In the case that $\gr{g}$ was sampled from $\langle \gr{g} \rangle$, the pair is actually statistically indistinguishable, and thus the distinguisher must fail. Thus, $\mathcal{A}_{f,g}$ is able to distinguish from which group $\gr{g}'$ was sampled with non-negligible probability, contradicting the subgroup indistinguishability assumption. 

\begin{theorem}\label{thm:hiding}
The commitment scheme $\Gamma = (\pro{Setup}, \pro{CommitH}, \pro{Open})$ is statistically hiding if $B \gg |\mathbb{G}|$ and if $\langle \gr{g} \rangle = \langle q^{d+1}\cdot \gr{g} \rangle$; and it is binding if the commitment described in Section~\ref{subsec:concretepoly} is binding.
\end{theorem}

\begin{proof}
The hiding commitment is a commitment to a degree $d+1$ polynomial. It therefore directly inherits the binding property from the non-hiding scheme.

To show hiding, we use the fact that the uniform distributions $[0,b]$ and $[a,a+b]$ have statistical distance $\frac{a}{b}$, \emph{i.e.}, the probability that any algorithm can distinguish the distributions from a single sample is less than $\frac{a}{b}$. Similarly $\gr{C}\gets ({f(q)+r\cdot q^{d+1}}) \cdot \gr{g}$ for $r\sample[0,B\cdot 2^{\lambda})$ has statistical distance at most $2^{-\lambda}$ from a uniform element generated by $\gr{g}$ if $B\geq |\langle \gr{g}\rangle|$. This means that two polynomial commitments can be distinguished by any algorithm with probability at most $2^{-\lambda+1}$.
\end{proof}


\paragraph{Zero-Knowledge Evaluation Protocol}

We now build a zero-knowledge evaluation protocol, which is an \eval protocol for a hiding polynomial commitment. The zero-knowledge protocol shows that the prover must know a degree $d$ polynomial $f(X)$ with bounded coefficients such that $f(z)\bmod p = y$ but does not leak any other information about $f$. Formally, we will show that the interactive $\pro{ZK-Eval}$ argument is honest verifier zero-knowledge according to Definition~\ref{def:hvzk} by constructing an efficient simulator $\mathcal{S}$ that can generate a distribution of transcripts that is indistinguishable from honestly generated transcripts.

The idea for the $\pro{ZK-Eval}$ protocol is a simple blinding of the polynomial borrowed from Zero-Knowledge Sumcheck~\cite{EPRINT:ChiForSpo17} and Bulletproofs~\cite{EC:BCCGP16,SP:BBBPWM18}. Let $f(X)$ be the committed polynomial, using the hiding commitment scheme. The prover wants to convince the verifier that $f(z)\bmod p=y$. To do this the prover commits to a degree $d$ polynomial $r(X)$ with random coefficients. The prover also reveals $y'\gets r(z)\bmod p$. The verifier then sends a random challenge $c$ and the prover and verifier can compute a commitment to $s(X)\gets r(X)+c\cdot f(X)$. The random polynomial $r(X)$ ensures that $s(X)$ is distributed statistically close to a random polynomial. The prover could just reveal $s(X)$ and the verifier can check that $s(z)\bmod p=y'+c \cdot y\bmod p$. Instead of sending $s(X)$ in the clear, the prover can additionally just send the commitment randomness to provide the verifier with a non-hiding commitment to $s(X)$. The prover and verifier can then use the standard $\eval$ protocol to efficiently evaluate $s$ at $z$.
 \noindent\begin{mdframed}[userdefinedwidth=\textwidth]
\begin{minipage}{\textwidth}
	\begin{flushleft}
	\pro{ZK-Eval}$(\params, \gr{C} \in \GG, z \in \ZZ_p, y \in \ZZ_p, d \in \NN; f(X) \in \ZZ(b)[X], r \in \ZZ):$\\
		%Statement: $\langle(\gr{C},z,y,d),(f(X) \in \ZZ(\frac{p-1}{2})[X],r)\rangle \in \mathcal{R}_\eval(\params)$\\
	%Input: $\params,\gr{C} \in \GG,z,y \in \ZZ_p, b\in \ZZ,d$, Witness: $f(X) \ZZ(b),\alpha \in \ZZ(2^\lambda)$\\
	\begin{enumerate}[nolistsep]
		    \item \prover samples a random degree $d$ $k(X) \sample \ZZ(p\cdot 2^{2 \lambda})[X]$ and $r_k \sample [0,B\cdot 2^\lambda)$ and computes $\gr{R} \gets (k(q) + r_k \cdot q^{d+1})+\gr{g}$ and $y_k \gets k(z) \bmod p$
		    \item \prover sends $\gr{R}$ and $y_k$ to $\verifier$
		    \item \verifier samples random $c\sample [0,2^\lambda)$ and sends it to $\prover$
		    \item \prover computes $s(X)\gets k(X) + c \cdot f(X)$, as well as $r_s \gets r_k + c\cdot r$.
		    \item \prover sends $r_s$ to \verifier
		    \item \prover and \verifier compute $\gr{C}_s\gets \gr{R} + c\cdot \gr{C} +(-q^{d+1}\cdot r_s)\cdot \gr{g}$ and $y_s\gets y_k+c \cdot y \bmod p$ \pccomment{$\gr{C}_s=s(q)\cdot \gr{g}$}
		    \item \prover and \verifier run $\pro{EvalB}(\crs,\gr{C}_s,z,y_s,d,\frac{p^2}{4}\cdot 2^{\lambda+1};s(X))$ \pccomment{$s(z)\bmod p=y_s$}		   		\end{enumerate}
	\end{flushleft}
\end{minipage}
\end{mdframed}

\benedikt{update/wrong}

\begin{theorem}
	Let $\CSZ,\EBL$ and $\CorrectnessBound$ be defined as in \cref{thm:darkisdarkss} . Let  $\log q \geq 4(\lambda + 1 + \CSZ[\mu+1]) + \EBL[\mu+1] + \mathsf{CB}_{p^2\cdot 2^{\lambda},\mu+1,\lambda} + 1$.
Under the adaptive root assumption for $\ggen$, the \textsf{JoinedEval} protocol has witness-extended-emulation (\Cref{def:wee}) for the relation $\mathcal{R_\textsf{JE}}$.

Let  $\eval$ have perfect completeness and witness extended emulation for $q> b\cdot \boldsymbol{\varsigma}_{p, d}$. Assuming that $\pro{commitH}$ is statistically hiding and both the order assumption and the strong RSA assumption hold for $\ggen$ the protocol $\pro{EvalZK}$ has perfect completeness, witness extended emulation and $\delta$-statistical honest-verifier zero-knowledge for $q>2^\lambda\cdot (p-1)(\frac{p^2-1}{2})^{\lceil \log_2(d+1)\rceil+1}<p^{2\log_2(d+1)+4}$ and $\delta \leq d \cdot 2^{-\lambda}$.
\end{theorem}

\begin{proof}
A simple application of Theorem~\ref{thm:algebraicIOPcompiler} and Lemma~\ref{lem:intrandomcombine} shows that the protocol maintains witness extended emulation. The extractor extracts $s(X)$ and $s'(X)$ from the $\eval$ protocol for challenges $c$ and $c'$. We can directly use Lemma~\ref{lem:intrandomcombine} to extract the witness $f(X)$ or a break of an assumption from these two transcripts. The bound on $q$ grows by a factor of less than $p^2$ (or $p^3$ under the $2$-Strong RSA assumption).   

To show zero-knowledge, we build the simulator $\mathcal{S}$ as follows. Start with a polynomial $s(X) \sample \mathbb{Z}(\frac{p^2-1}{4})[X]$ with uniform random coefficients, and a blinding factor $r_s \sample [0,B\cdot 2^{2\cdot\lambda+1})$. The simulator $\mathcal{S}$ then chooses a random challenge $c \sample [0,2^\lambda)$ and computes $\gr{R} = (s(q) + r_s \cdot q^{d+1})\gr{g} -c\cdot \gr{C}$. The simulator then performs the rest of the $\eval$ protocol honestly using $s(X)$ as the witness. 

The randomizer $r_s$ is distributed identically to the honest $r_s$. Given the hiding property of the commitment scheme, $\gr{R}$ is statistically indistinguishable from any other commitment. Finally the simulated and the honest $s(X)$ have statistical distance at most $2^{-\lambda}$ from a random polynomial. The coefficients of $c\cdot f(x)$ are in $\ZZ(\frac{p^2}{4})$. The coefficients of the blinding polynomial $s(X)$ are sampled from a range that is larger by a factor $2^{\lambda}$. So the distribution of coefficients of $s(X) = k(X) + c \cdot f(X)$ is at a statistical distance at most $2^{-\lambda}$ away from the uniform distribution over $\ZZ(\frac{p^2}{4})$. Since the distributions of simulated and real coefficients are both uniform but merely over different sets, the statistical distance between the simulated $s(X)$ and the real $s(X)$ is at most $d \cdot 2^{-\lambda}$. The evaluation with $\pro{EvalB}$ cannot leak more than $s(X)$ itself. The views of the simulated and real transcripts are, therefore, $\delta$-close with $\delta \leq d \cdot 2^{-\lambda}$. Consequently, the protocol has $\delta$-statistically honest verifier zero-knowledge.
\end{proof}


%%%DEPRECATED SECTIONS BELOW%%%%%%

\if 0 
\section{Deprecated witness extended emulation for OpenIndex queries} 

We can show that if $\pro{Eval}$ has witness-extended emulation, then so does $\pro{OpenIndex}$ as an interactive argument for the relation: 

\[ 
\mathcal{R_\textsf{Index}}(\params) = \left\{
\langle (c, a, i, d), (f(X), r) \rangle
: 
\begin{array}{l} 
f \in R[X] \ \text{has degree at most} \ d \ \text{with} \ f_i = a \\ 
 \text{and} \ \pro{Open}(\params, c, f(X), r) = 1 \\
\end{array}
\right\}
\] 


\begin{lemma} 
If $\pro{Eval}$ is an interactive argument for $\mathcal{R_\textsf{Eval}}(\params)$ with witness-extended emulation then $\pro{OpenIndex}$ is an interactive argument for $\mathcal{R_\textsf{Index}}(\params)$ with witness-extended emulation. 
\end{lemma}

\begin{proof}[Proof sketch]
Let $E_{\eval}$ be an emulator for $\pro{Eval}$. We construct an emulator $E$ for $\pro{OpenIndex}$ that makes calls to $E_{\eval}$. At a high level, $E$ will invoke $E_{\eval}$ in order to extract witness polynomials $f_R(X)$, $f_L(X)$, $f(X)$ of appropriate degrees from the respective successful executions of $\pro{Eval}$ and also piece together a consistent emulation transcript. We then argue that if $f(X) - a X^i \neq X^{i+1} f_R(X) + f_L(X)$, then the evaluation binding of the polynomial commitment $\eval$ is broken. $E$ could rewind the protocol to the second step after receiving commitments $c, c_R, c_L$ and rerun on a fresh challenge $\beta' \leftarrow_R \FF$, doing this until it gets another accepting transcript (in expected polynomial time) with succesful openings of $f(\beta') = y'$, $f_L(\beta') = y'_L$, and $f_R(\beta') = y'_R$ such that $y' = y_L' + \beta'^{i+1} y'_R - a \beta^i \bmod p$. Yet, $f(\beta') - a X^i \neq X^{i+1} f_R(\beta') + f_L(\beta')$ except with probability $\poly / |\FF|$ (union bound over the $\poly$ rewindings), which implies that one of the claimed evaluations was incorrect. The full proof is in the appendix. 

\ben{TODO move to appendix}

Given adversarial prover $P^*$ and transcript oracle $\pro{Record}(P^*, \params, (c, a, i, d), \st)$, $E$ does the following: 

\begin{itemize} 
\item Run $\pro{Record}$ from the start until the first $\pro{Eval}$. $E$ obtains a transcript containing $c, c_R, c_L, y, y_R, y_L$ and $\beta$. 

\item Invoke $E_{\eval}$ by simulating its transcript oracle $\pro{Record}(P^*, \params, c, \beta, y, d, \st)$. ($E$ simulates this transcript oracle for $E_{\eval}$ by consulting its own transcript oracle, which is running $\eval$ as a subprotocol). $E_{\eval}$ returns an output $(\tr_1, f^*)$, and w.l.o.g. interpret $f^*$ as giving a canonical encoding of an integer polynomial $f^*(X)$. Do the same for the subprotocol $\eval$ executions on $(c_R, \beta, y_R, d - i - 1)$ and $(c_L, \beta, y_L, i-1)$ respectively, which returns $(\tr_2, f_L^*)$ and $(\tr_3, f_R^*)$. 

\item Piece together the transcript obtained from running $\pro{Record}$ in the first step (i.e. up until the first $\eval$ together with the transcripts $\tr_1, \tr_2, \tr_3$ into a complete transcript $\tr^*$. This $\tr$ is an ``accepting transcript" if and only if the verifier in the last step would output $1$. 

\item Output $(\tr^*, f^*)$.  
\end{itemize}

We now argue that for any PPT $\mathcal{A}$ and $\params \leftarrow \pro{Setup}(1^\lambda)$, if the experiment sampling $(X, \st) \leftarrow \mathcal{A}(\params)$ and $\tr \leftarrow \pro{Record}(P^*, \params, x, \st)$, where $x$ is a tuple of the form $(c, a, i, d)$, produces an ``accepting transcript" $\tr$ with probability $\delta$, then the modified experiment that samples $(\tr^*, f^*) \leftarrow E^{\textsf{Record}(P^*, \params, x, \st)}(\params, x)$ returns an $f^*$ such that $((c, a, i, d), f^*) \in \mathcal{R_\textsf{Index}}(\params)$ with probability $\delta - \negl$. 

If $\delta$ is negligible the claim holds trivially. For now on assume that $\delta$ is non-negligible. 
The fact that $\tr$ is an accepting transcript with probability $\delta$ implies that each $\eval$ subprotocol generates accepting transcripts with probability at least $\delta$. Therefore, by the hypothesis that $\eval$ has witness-extended emulation, if each of the subprotocol transcripts $\tr_1$, $\tr_2$, $\tr_3$ are accepting then with probability $\delta - \negl$: subprotocol witnesses $f^*, f_L^*$ and $f_R^*$ are all valid $\mathcal{R_\textsf{Eval}}(\params)$ witnesses for $(c, \beta, y, d)$, $(c_L, \beta, y_L, i-1)$, and $(c_R, \beta, y_R, d - i - 1)$ respectively. 

We have already shown that the extracted polynomial $f^*(X)$ is valid for $c$ with probability $\delta - \negl$. As a final step, suppose towards contradiction that $f_L^*(X) + X^i f_R^*(X) \neq f^*(X) - a X^i$. Since $\delta$ is non-negligible, $E$ can rewind the transcript oracle to Step 2 (i.e. after receiving the commitments $c, c_L, c_R$), restarting the protocol from this point on a fresh verifier challenge $\beta' \leftarrow_R \FF$. It does this until (in expected polynomial time) it finds a $\beta'$ that produces an accepting transcript, which includes $y', y_L', y_R'$ such that $y' = y_L' + \beta'^{i+1} y'_R - a \beta'^i \bmod p$.

The probability that $f_L^*(\beta') + \beta'^i f_R^*(\beta') = f^*(\beta') - a \beta'^i$ is less than $\poly / |\FF|$ (the $\poly$ numerator comes from union bound over number of rewindings). In this case, either $y' \neq f^*(\beta')$ or $y_L' \neq f^*_L(\beta')$ or $y_R' \neq f^*_R(\beta')$. Yet, $\eval$ passes on all three, which contradicts the evaluation binding property of $\eval$. Hence, we conclude that except with $\negl$ probability $f_L^*(X) + X^i f_R^*(X) \neq f^*(X) - a X^i$, and thus $f^*_i = a$. 

\end{proof}
\fi 


\if 0 %%Deprecated discussion of vector commitments 
\subsection{Vector Commitments}

A commitment to a polynomial is a commitment to a list of coefficients, and the ability to extract any indicated coefficient from a polynomial commitment effectively upgrades the scheme to a vector commitment scheme. While it is possible to extract an indicated coefficient using only a polynomial commitment scheme (see Section~\ref{section:generic_coefficient_extraction} for a demonstration of this fact), it is possible to achieve this task much more efficiently by exploiting the homomorphic properties of our commitment scheme. For the following description we will identify polynomials $f(X)$ with their coefficient vectors $\mathbf{f}$ and vice versa, and we will switch between notations whenever it is convenient.

We achieve this task in two steps. Protocol $\pro{ExtractCoefficient}(\params, \gr{C}, i, \gr{C}_a; \mathbf{f}) \rightarrow b \in \{0,1\}$ validates that a given commitment really is a commitment to an indicated coefficient of a vector commitment. Next, Protocol $\pro{OpenIndex}$ uses this as a subprotocol to realize the syntax defined in the preliminaries (Section~\ref{subsection:openindex}). 

\begin{figure}[!htp]
\noindent\begin{mdframed}[userdefinedwidth=\textwidth]
\begin{minipage}{\textwidth}
	\begin{flushleft}
	$\pro{ExtractCoefficient}(\params, \gr{C}, i, \gr{C}_a; \mathbf{f}):$ \pccomment{$\mathbf{f} = (f_0, \ldots, f_d)^\mathsf{T} \in \ZZ^{d+1}$ and $a = f_i$}
		\begin{enumerate}[nolistsep]
		    \item \prover computes $f_L(X) \gets \sum_{j=0}^{i-1} f_j X^j$ and $f_R \gets \sum_{j=i+1}^d f_j X^{j-i-1}$
		    \item \prover computes $\gr{C}_L \gets \gr{g}^{f_L(q)}$, and $\gr{C}_R \gets \gr{g}^{f_R(q)}$
		    \item \prover computes $\gr{C}_a' \gets \gr{C}_a^{q^i}$ and $\gr{C}_R' \gets \gr{C}_R^{q^{i+1}}$
		    \item \prover sends $\gr{C}_L, \gr{C}_a', \gr{C}_R$ to \verifier
		    \item \verifier computes $\gr{C}_R' \gets \gr{C} \gr{C}_L^{-1} {\gr{C}_a'}^{-1}$
		    \item \prover and \verifier run $\pro{PoE}(\gr{C}_a, \gr{C}_a', q^i)$ and $\pro{PoE}(\gr{C}_R, \gr{C}_R', q^{i+1})$
		    \item \prover and \verifier run $\pro{EvalBounded}(\params, \gr{C}_L, z, f_L(z), i-1, b; f_L(X))$ for an arbitrary $z$ and any $b$ such that $\max_j f_j \leq b \ll q$
		    \item \pcif{}all checks pass \textbf{then} \textbf{return} 1 \textbf{else} \textbf{return} 0
		\end{enumerate}
	\end{flushleft}
\end{minipage}
\end{mdframed}
\end{figure}

\textit{Note.} Instead of line 7, \prover and \verifier might as well run any range proof that establishes that $\gr{C}_L$ is a commitment to an integer smaller than $q^i$ in absolute value.

\begin{lemma}
    Protocol $\pro{ExtractCoefficient}$ has witness-extended emulation for the relation
    \[
        \mathcal{R}_\mathsf{EC}(\params) = \left\{
            \langle(\gr{C}, i, \gr{C}_a), (\mathbf{f}, r_f, r_a)\rangle \ : \ \begin{array}{l}
                 \mathbf{f} = (f_0, \ldots, f_d)^\mathsf{T} \in \mathbb{Z}_p^{d+1} \\
                 \pro{Open}(\params, \gr{C}, \mathbf{f}, r_f) = 1 \\
                 \pro{Open}(\params, \gr{C}_a, f_i, r_a) = 1
            \end{array}
        \right \} \enspace .
    \]
\end{lemma}
\begin{proof}
Full version/appendix.
\end{proof}

\begin{figure}[!htp]
\noindent\begin{mdframed}[userdefinedwidth=\textwidth]
\begin{minipage}{\textwidth}
	\begin{flushleft}
	$\pro{OpenIndex}(\params, \gr{C}, a, i; \mathbf{f}):$ \pccomment{$\mathbf{f} \in \ZZ_p^{d+1}$}
		\begin{enumerate}[nolistsep]
		    \item \prover computes $\gr{C}_a \gets \gr{c}^{f_i}$ and sends it to \verifier
		    \item \prover and \verifier run $\pro{ExtractCoefficient}(\params, \gr{C}, i, \gr{C}_a; \mathbf{f})$
		    \item \prover and \verifier run $\pro{Open}(\params, \gr{C}_a, a; f_i)$
		    \item \pcif{}all checks pass \textbf{then} \textbf{return} 1 \textbf{else} \textbf{return} 0
		\end{enumerate}
	\end{flushleft}
\end{minipage}
\end{mdframed}
\end{figure}

\begin{lemma}
    Protocol $\pro{OpenIndex}$ has witness-extended emulation for the relation
    \[
        \mathcal{R}_\mathsf{Index}(\params) = \left\{
            \langle(\gr{C}, a, i, d), (\mathbf{f}, r_f)\rangle \ : \ \begin{array}{l}
                \mathbf{f} = (f_0, \ldots, f_d)^\mathsf{T} \in \mathbb{Z}_p^{d+1} \\
                f_i = a \\
                \pro{Open}(\params, \gr{C}, \mathbf{f}, r_f) = 1
            \end{array}
        \right\} \enspace .
    \]
\end{lemma}

\begin{proof}
Full version/appendix.
\end{proof}
\fi 

\if 0 %%Deprecated discussion of inner products 
\subsection{Inner Products}

The polynomial commitment scheme has a multiplicative homomorphism. Specifically, let $f(X), g(X) \in \ZZ_p[X]$ and let $\gr{C}$ be a commitment to $f(X)$. Then, provided that $q$ is large enough to prevent overflow, $\gr{C}^{g(q)}$ is a commitment to $f(X) \times g(X)$. This feature is particularly useful in the context of vector commitments where the goal is to extract not an indicated coefficient but a linear combination of all coefficients. To see how this might work, consider the coefficient vectors $\mathbf{f} = (f_0, \ldots, f_{d})$ and $\mathbf{g} = (g_0, \ldots, g_d)$. Then $\gr{C}$ is simultaneously a vector commitment to $\mathbf{f}$, and raising this commitment to %the integer encoding of the reciprocal of $g(X)$
the power $\sum_{i=0}^d g_{d-i} q^i$
gives a commitment to a new vector whose middle coefficient contains the inner product $\langle \mathbf{f}, \mathbf{g} \rangle$. To see this, consider the logarithm of $\gr{C}^{\sum\limits_{i=0}^{d} g_{d-i} q^i}$ base $\gr{g}$:
\begin{flalign*}
\left( \sum\limits_{i=0}^d f_i q^i \right) \left( \sum\limits_{i=0}^d g_{d-i} q^i \right) = \sum\limits_{i=0}^d \sum\limits_{j=0}^d f_i g_j q^{d-j+i} = q^d \sum_{i=0}^d f_i g_i \ + o(q^{d+1}) + \omega(q^{d-1}) \enspace .
\end{flalign*}

We use this property to realize protocols for extracting inner products. A minor issue is that the right hand vector commitment must represent the coefficients in reversed order. To circumvent this obstacle we denote by $\mathbf{\tilde{g}}$ the vector $\mathbf{g}$ but with its coefficients reversed. For applications where this issue cannot be solved with notational cleverness, Appendix \textbf{[todo]} presents a protocol to establish that two vector commitments represent the same coefficients but in reversed order. 

\begin{figure}[!htp]
\noindent\begin{mdframed}[userdefinedwidth=\textwidth]
\begin{minipage}{\textwidth}
	\begin{flushleft}
	$\pro{InnerProduct}(\params, \gr{C}_\mathbf{f}, \gr{C}_\mathbf{\tilde{g}}, a; \mathbf{f}, \mathbf{\tilde{g}}):$ \pccomment{$\mathbf{f}, \mathbf{\tilde{g}} \in \ZZ^{d+1}$ and $\gr{C}_\mathbf{\tilde{g}} = \gr{g}^{\sum_{i=0}^d g_{d-i} q^i}$}
		\begin{enumerate}[nolistsep]
		    \item \prover computes $\gr{C}_h \gets \gr{C}_\mathbf{f}^{\sum_{i=0}^{d} g_{d-i} q^i}$ and sends it to \verifier
		    \item \verifier samples $z \sample \mathbb{Z}_p$ and sends it to \prover
		    \item \prover computes $h(X) \gets \left(\sum_{i=0}^d f_i X^i\right) \left(\sum_{i=0}^d g_{d-i} X^i\right)$
		    \item \prover computes $y_f \gets f(z)$, $y_g \gets g(z)$, and $y_h \gets h(z)$ and sends $(y_f, y_g, y_h)$ to \verifier
		    \item \verifier checks that $y_h = y_f \times y_g$
		    \item \verifier samples $\beta, \gamma \sample \mathbb{Z}_p$ and sends $(\beta, \gamma)$ to \prover
		    \item \prover computes $\gr{C}' \gets (\gr{C}_\mathbf{f}^\beta \gr{C}_\mathbf{g}^\gamma)^{q^{d-1}}$ and sends $\gr{C}'$ to \verifier
		    \item \prover and \verifier run $\pro{PoE}(\gr{C}_\mathbf{f}^\beta \gr{C}_\mathbf{g}^\gamma, \gr{C}', q^{d-1})$
		    \item \prover and \verifier run $\pro{Eval}(\gr{C}'\gr{C}_h^{-1}, z, {\beta{} z^{d-1} y_f + \gamma{} z^{d-1} y_g - y_h} , {2d-1} ;$ ${\beta{} X^{d-1} f(X) + \gamma{} X^{d-1} g(X) - h(X)})$
		    \item \prover computes $\gr{C}_a \leftarrow \gr{g}^{h_d}$
		    \item \prover and \verifier run $\pro{ExtractCoefficient}(\params, \gr{C}_h, d, \gr{C}_a; h(X))$
		    \item \prover and \verifier run $\pro{Open}(\params, \gr{C}_a, a, h_d)$
		    \item \pcif{}all checks pass \textbf{then} \textbf{return} 1 \textbf{else} \textbf{return} 0
		\end{enumerate}
	\end{flushleft}
\end{minipage}
\end{mdframed}
\end{figure}

\begin{lemma}
    The protocol $\pro{InnerProduct}$ has witness-extended emulation for the relation
\[\mathcal{R}_\mathsf{IP}(\params) = \left\{
    \langle(\gr{C}_\mathbf{f}, \gr{C}_\mathbf{\tilde{g}}, a), (\mathbf{f}, \mathbf{\tilde{g}}, r_f, r_g)\rangle \ : \     \begin{array}{l}
            \mathbf{f}, \mathbf{\tilde{g}} \in \mathbb{Z}_p^{d+1} \\
            \langle \mathbf{f}, \mathbf{g} \rangle = a \\
            \pro{Open}(\params, \gr{C}_\mathbf{f}, \mathbf{f}, r_f) = 1 \\
            \pro{Open}(\params, \gr{C}_\mathbf{\tilde{g}}, \mathbf{\tilde{g}}, r_g) = 1
        \end{array}
    \right\} \enspace .
\]
\end{lemma}

\begin{proof}
In the full version of the paper.
\end{proof}
\fi 

\if 0 %%Deprecated coefficient extraction and inner product 
\subsection{Coefficient Extraction} \label{section:generic_coefficient_extraction}

Given a polynomial commitment scheme as a black box, one can generate protocol to open the $i$th coefficient of a committed polynomial. Specifically, protocol $\pro{OpenIndex}(\params, c, a, i, d; f(X)) \rightarrow b \in \{0,1\}$ verifies that $a$ is the $i$th coefficient of $f(X)$, which is the polynomial of degree at most $d$ that $c$ commits to.

\begin{figure}[!htp]
\noindent\begin{mdframed}[userdefinedwidth=\textwidth]
\begin{minipage}{\textwidth}
	\begin{flushleft}
	$\pro{OpenIndex}(\params, c, a, i, d; f(X)):$ \pccomment{$f_i = a$}
		\begin{enumerate}[nolistsep]
		    \item $\prover$ computes $f_L(X) \gets \sum_{j=0}^{i-1} f_jX^j$ and $f_R(X) \gets \sum_{j=i+1}^d f_j X^{j-i-1}$
		    \item \prover computes $c_L \gets \pro{Commit}(\params, f_L(X))$ and $c_R \gets \pro{Commit}(\params, f_R(X))$ and sends these commitments to \verifier
		    \item \verifier samples $\beta \sample \mathbb{F}$ and sends $\beta$ to \prover
		    \item \prover evaluates $y_L \gets f_L(\beta)$ and $y_R \gets f_R(\beta)$ and $y \gets f(\beta)$ and sends $(y, y_L, y_R)$ to \verifier
		    \item \verifier checks that $y = y_L + \beta^{i+1} y_R - \beta^i a$
		    \item \prover and \verifier run $\eval(\params, c_L, \beta, y_L, i-i; L(X))$ and $\eval(\params, c_R, \beta, y_R, d-i; f_R(X))$ and $\eval(\params, c, \beta, y, d; f(X))$
		    \item \pcif{}all checks pass \textbf{then} \textbf{return} 1 \textbf{else} \textbf{return} 0
		\end{enumerate}
	\end{flushleft}
\end{minipage}
\end{mdframed}
\end{figure}

We can show that if $\pro{Eval}$ has witness-extended emulation, then so does $\pro{OpenIndex}$ as an interactive argument for the relation
\[ 
\mathcal{R_\textsf{Index}}(\params) = \left\{
\langle (c, a, i, d), (f(X), r) \rangle
: 
\begin{array}{l} 
f \in R[X] \ \text{has degree at most} \ d \ \text{with} \ f_i = a \\ 
 \text{and} \ \pro{Open}(\params, c, f(X), r) = 1 \\
\end{array}
\right\} \enspace .
\] 

\begin{lemma} 
If $\pro{Eval}$ is an interactive argument for $\mathcal{R_\textsf{Eval}}(\params)$ with witness-extended emulation then $\pro{OpenIndex}$ is an interactive argument for $\mathcal{R_\textsf{Index}}(\params)$ with witness-extended emulation. 
\end{lemma}

\begin{proof}[Proof sketch]
Let $E_{\eval}$ be an emulator for $\pro{Eval}$. We construct an emulator $E$ for $\pro{OpenIndex}$ that makes calls to $E_{\eval}$. At a high level, $E$ will invoke $E_{\eval}$ in order to extract witness polynomials $f_R(X)$, $f_L(X)$, $f(X)$ of appropriate degrees from the respective successful executions of $\pro{Eval}$ and also piece together a consistent emulation transcript. We then argue that if $f(X) - a X^i \neq X^{i+1} f_R(X) + f_L(X)$, then the evaluation binding of the polynomial commitment $\eval$ is broken. $E$ could rewind the protocol to the second step after receiving commitments $c, c_R, c_L$ and rerun on a fresh challenge $\beta' \leftarrow_R \mathbb{F}$, doing this until it gets another accepting transcript (in expected polynomial time) with succesful openings of $f(\beta') = y'$, $f_L(\beta') = y'_L$, and $f_R(\beta') = y'_R$ such that $y' = y_L' + \beta'^{i+1} y'_R - a \beta^i \bmod p$. Yet, $f(\beta') - a X^i \neq X^{i+1} f_R(\beta') + f_L(\beta')$ except with probability $\poly / |\mathbb{F}|$ (union bound over the $\poly$ rewindings), which implies that one of the claimed evaluations was incorrect. The full proof is in the appendix. 

\ben{TODO move to appendix}

Given adversarial prover $P^*$ and transcript oracle $\pro{Record}(P^*, \params, (c, a, i, d), \st)$, $E$ does the following: 

\begin{itemize} 
\item Run $\pro{Record}$ from the start until the first $\pro{Eval}$. $E$ obtains a transcript containing $c, c_R, c_L, y, y_R, y_L$ and $\beta$. 

\item Invoke $E_{\eval}$ by simulating its transcript oracle $\pro{Record}(P^*, \params, c, \beta, y, d, \st)$. ($E$ simulates this transcript oracle for $E_{\eval}$ by consulting its own transcript oracle, which is running $\eval$ as a subprotocol). $E_{\eval}$ returns an output $(\tr_1, f^*)$, and w.l.o.g. interpret $f^*$ as giving a canonical encoding of an integer polynomial $f^*(X)$. Do the same for the subprotocol $\eval$ executions on $(c_R, \beta, y_R, d - i - 1)$ and $(c_L, \beta, y_L, i-1)$ respectively, which returns $(\tr_2, f_L^*)$ and $(\tr_3, f_R^*)$. 

\item Piece together the transcript obtained from running $\pro{Record}$ in the first step (i.e. up until the first $\eval$ together with the transcripts $\tr_1, \tr_2, \tr_3$ into a complete transcript $\tr^*$. This $\tr$ is an ``accepting transcript" if and only if the verifier in the last step would output $1$. 

\item Output $(\tr^*, f^*)$.  
\end{itemize}

We now argue that for any PPT $\mathcal{A}$ and $\params \leftarrow \pro{Setup}(1^\lambda)$, if the experiment sampling $(X, \st) \leftarrow \mathcal{A}(\params)$ and $\tr \leftarrow \pro{Record}(P^*, \params, x, \st)$, where $x$ is a tuple of the form $(c, a, i, d)$, produces an ``accepting transcript" $\tr$ with probability $\delta$, then the modified experiment that samples $(\tr^*, f^*) \leftarrow E^{\textsf{Record}(P^*, \params, x, \st)}(\params, x)$ returns an $f^*$ such that $((c, a, i, d), f^*) \in \mathcal{R_\textsf{Index}}(\params)$ with probability $\delta - \negl$. 

If $\delta$ is negligible the claim holds trivially. For now on assume that $\delta$ is non-negligible. 
The fact that $\tr$ is an accepting transcript with probability $\delta$ implies that each $\eval$ subprotocol generates accepting transcripts with probability at least $\delta$. Therefore, by the hypothesis that $\eval$ has witness-extended emulation, if each of the subprotocol transcripts $\tr_1$, $\tr_2$, $\tr_3$ are accepting then with probability $\delta - \negl$: subprotocol witnesses $f^*, f_L^*$ and $f_R^*$ are all valid $\mathcal{R_\textsf{Eval}}(\params)$ witnesses for $(c, \beta, y, d)$, $(c_L, \beta, y_L, i-1)$, and $(c_R, \beta, y_R, d - i - 1)$ respectively. 

We have already shown that the extracted polynomial $f^*(X)$ is valid for $c$ with probability $\delta - \negl$. As a final step, suppose towards contradiction that $f_L^*(X) + X^i f_R^*(X) \neq f^*(X) - a X^i$. Since $\delta$ is non-negligible, $E$ can rewind the transcript oracle to Step 2 (i.e. after receiving the commitments $c, c_L, c_R$), restarting the protocol from this point on a fresh verifier challenge $\beta' \leftarrow_R \mathbb{F}$. It does this until (in expected polynomial time) it finds a $\beta'$ that produces an accepting transcript, which includes $y', y_L', y_R'$ such that $y' = y_L' + \beta'^{i+1} y'_R - a \beta'^i \bmod p$.

The probability that $f_L^*(\beta') + \beta'^i f_R^*(\beta') = f^*(\beta') - a \beta'^i$ is less than $\poly / |\mathbb{F}|$ (the $\poly$ numerator comes from union bound over number of rewindings). In this case, either $y' \neq f^*(\beta')$ or $y_L' \neq f^*_L(\beta')$ or $y_R' \neq f^*_R(\beta')$. Yet, $\eval$ passes on all three, which contradicts the evaluation binding property of $\eval$. Hence, we conclude that except with $\negl$ probability $f_L^*(X) + X^i f_R^*(X) \neq f^*(X) - a X^i$, and thus $f^*_i = a$. 

\end{proof}

\subsection{Inner Product}

Using the $\pro{Reverse}$ and $\pro{OpenIndex}$ protocols as subprotocols, it is straightforward to construct a protocol to prove that a given scalar is the inner product of the coefficient vectors of two committed polynomiails. To see this, observe that if $\tilde{g}(X)$ is the polynomial with the same coefficients as $g(X)$ but in reverse order, and if both $\tilde{g}(X)$ and $f(X)$ have degree $d$, then the coefficient of the monomial $X^d$ in the polynomial $f(X) \cdot \tilde{g}(X)$ is exactly the inner product between $\mathbf{f}$ and $\mathbf{g}$. This gives rise to the protocol $\pro{InnerProduct}(\params, \gr{C}_{f}, \gr{C}_{\tilde{g}}, a; {f}(X), {\tilde{g}}(X)) \rightarrow b \in \{0,1\}$, which proves that $a = \langle \mathbf{f}, \mathbf{g} \rangle$. Note that this protocol assumes that $\gr{C}_{\tilde{g}}$ is a commitment to the polynomial $\tilde{g}(X)$, which has the same coefficients as $g(X)$ but in reverse order. Protocol $\pro{Reverse}$ can be used to establish that two commitments $\gr{C}_g$ and $\gr{C}_{\tilde{g}}$ represent two polynomials with the same coefficients but in reversed order.

\begin{figure}[!htp]
\noindent\begin{mdframed}[userdefinedwidth=\textwidth]
\begin{minipage}{\textwidth}
	\begin{flushleft}
	$\pro{InnerProduct}(\params, \gr{C}_{{f}}, \gr{C}_{\tilde{g}}, a; f(X), \tilde{g}(X)):$ 
		\begin{enumerate}[nolistsep]
		    \item $\prover$ computes $h(X) \gets f(X) \cdot \tilde{g}(X)$ and $(\gr{C}_h; r_h) \gets \pro{Commit}(\params, h(X))$
		    \item \prover sends $\gr{C}_h$ to \verifier
		    \item \verifier samples $z \sample \mathbb{Z}_p$ and sends $z$ to \prover
		    \item \prover computes $y_f \gets f(z)$ and $y_{\tilde{g}} \gets \tilde{g}(z)$ and $y_h \gets h(z)$ and sends $(y_f, y_{\tilde{g}}, y_h)$ to \verifier
		    \item \prover and \verifier run $\pro{Eval}(\params, \gr{C}_f, z, y_f, d; f(X))$ and $\pro{Eval}(\params, \gr{C}_{\tilde{g}}, z, y_{\tilde{g}}, d; \tilde{g}(X))$ and $\pro{Eval}(\params, \gr{C}_h, z, y_f, d; h(X))$
		    \item \verifier checks that $y_f \cdot y_{\tilde{g}} = y_h$
		    \item \prover and \verifier run $\pro{OpenIndex}(\params, \gr{C}_h, a, d; h(X))$
		    \item \pcif{}all checks pass \textbf{then} \textbf{return} 1 \textbf{else} \textbf{return} 0
		\end{enumerate}
	\end{flushleft}
\end{minipage}
\end{mdframed}
\end{figure}

\begin{lemma}
    Protocol $\pro{InnerProduct}$ is an argument of knowledge for the relation
\[\mathcal{R}_\mathsf{IP}(\params) = \left\{
    \langle(\gr{C}_{f}, \gr{C}_{\tilde{g}}, a), ({f}(X), \tilde{g}, r_f, r_{\tilde{g}})\rangle \ : \     \begin{array}{l}
            \mathbf{f}, \mathbf{\tilde{g}} \in \mathbb{Z}_p^{d+1} \\
            \langle \mathbf{f}, \mathbf{g} \rangle = a \\
            \pro{Open}(\params, \gr{C}_{f}, {f}(X), r_f) = 1 \\
            \pro{Open}(\params, \gr{C}_{\tilde{g}}, {\tilde{g}}(X), r_{\tilde{g}}) = 1
        \end{array}
    \right\} \enspace .
\]
\end{lemma}

\begin{proof}
Full version / appendix.
\end{proof}
\fi 

\section{Proof of Theorem~\ref{thm:IOPcompiler} (Polynomial IOP Compilation)}\label{sec:IOPcompilerproof}
The fact that the compilation preserves HVZK is straightforward. We prove this part first and then move on to proving witness-extended emulation. 

\paragraph{HVZK} Let $S_\eval$ denote the HVZK simulator for $\eval$ and $S_\pro{IOP}$ denote the HVZK simulator for the original polynomial IOP. We construct an HVZK simulator $S$ for the compiled interactive argument as follows. 
$S$ begins by running $S_\pro{IOP}$ on the input $x$, which produces a series of query/response pairs to arbitrarily labeled oracles that are ``sent" from the IOP prover to the verifier. $S$ simulates the view of the honest verifier in the compiled interactive proof by replacing each distinctly labeled oracle with a fresh $\Gamma$ commitment to $0$, \emph{i.e.}, the zero polynomial over $\FF_p$. By the hiding property of $\Gamma$ this has negligible distance $\delta_0$ from the commitment sent in the real protocol.
(It places this commitment at the location in the transcript where the commitment to this oracle would be sent in the compiled protocol).
 For each query/response pair $(z, y)$ to an oracle, $S$ runs $S_\eval$ to simulate the view of an honest-verifier in the $\eval$ protocol opening a hiding polynomial commitment to the value $y$ at the point $z$. Let $P$ denote an upper bound on the total number of oracles sent and $Q$ denote an upper bound on the total number of queries to IOP oracles. 
If the simulation of $S_\pro{IOP}$ has statistical distance $\delta_1$ from the real IOP verifier's view, and each simulated $\eval$ subprotocol has statistical distance $\delta_2$ to the real $\eval$ verifier's view, then the output of $S$ has statistical distance at most $P \delta_0 + \delta_1 + Q \delta_2$ from $\textsf{View}_{\langle P(x, w), V(x) \rangle}$. For $P, Q < \poly$ and $\delta_0, \delta_1, \delta_2 < \negl$ this statistical distance is negligible in $\lambda$. 

\paragraph{Witness-extended emulation (knowledge)}

\begin{proof}
Without loss of generality, assume the original IOP makes at least one query to each oracle sent. An oracle which is never queried can be omitted from the IOP.

We denote by $\verifier$ the IP verifier for the compiled IP, and $\verifier_{O}$ the verifier for the original IOP. 
Given a record oracle $\pro{Record}(P^*, \params, x, \st)$ for an IP prover $P^*$ that produces accepting transcripts with non-negligible probability, we build an emulator $E$ for the compiled IP. $E$ begins by constructing an IOP adversary $P'_{O}$, which succeeds also with non-negligible probability on input $x$. Every successful interaction of $P'_{O}$ with $\verifier_{O}$ on input $x$ corresponds to a successful transcript of $P^*$ with $V$ on $x$. In showing how $E$ builds $P'_{O}$ we also show how $E$ can obtain this corresponding transcript. $E$ will make use of the emulator $E_\eval$ for the commitment scheme $\Gamma$. %We will describe how this is done in the later parts of the proof. Accepting that this is true. 

Finally, $E$ can use the IOP knowledge extractor $E_{\pro{IOP}}^{P'_O}(x)$ in order to output a witness for $x$ along with the corresponding transcript. 

\paragraph{Constructing $P'_O$ (IOP adversary)}
$P'_O$ runs as follows on initial state $\st_0$ and input $x$. It internally simulates the interaction of $P^*$ and $V$, using the record oracle $\pro{Record}(P^*, \params, x, \st)$. It begins by running this for the first round on state $\st_0$. For every message that $P^*$ sends in this first round, $P'_O$ continues simulation until there is an $\eval$ on this commitment. (There is guaranteed to be at least one $\eval$ on each commitment, independent of the randomness). Therefore, denoting by $E_\eval$ the extractor for the $\eval$ subprotocol between $P^*$ and $\verifier$ on a given commitment and evaluation point, the record oracle can be used to simulate $E_\eval$'s record oracle.% for each $m$ evaluated at some point determined by $V$'s challenge.

For each message $m$ that $P^*$ sends to $V$ at the beginning of the first round, $P'_O$ interprets $m$ as a commitment, and attempts to extract from it a polynomial by 
running the PPT emulator $E_\eval$, simulating its record oracle as just described. \textbf{If it fails in any extraction attempt it aborts.} 

If $P'_O$ succeeds in all these extractions, then it uses these extracted polynomials as its first round proof oracles that it gives to $\verifier_O$. Upon receiving the first public-coin challenge from the IOP verifier, $P'$ uses the query function to derive the corresponding queries to each of these proof oracles. Before answering, it rewinds $P^*$ and $\verifier$ back to the point immediately after $P$ sent its first messages, and now substitutes random challenge from $\verifier_O$ in order to simulate $P^*$ and $V$ on these same queries. It checks that $P^*$'s answers are consistent with the answers it can compute on its own from the extracted polynomials. \textbf{If any answers are inconsistent, $P'_O$ aborts}. Otherwise, it sends the answers to $\verifier_O$. 

At the end of this first round (assuming $P'$ has not yet aborted), $P'_O$ has stored an updated state $\st'$ for $P^*$ based on this simulation. It proceeds to the next round and repeats the same process, using the record oracle $\pro{Record}(P^*, \params, x, \st')$. Finally, if $P'$ makes it through all rounds without aborting, then it has a final state $\st_V$ for $\verifier_O$ based on its internal simulation of $P^*$ and $V$ up through the end of the last round. Finally, $\verifier_O(\st_V)$ outputs $\pro{Accept}$ or $\pro{Reject}$. %(Observe that $\verifier_O$ accepts if and only if $\verifier$ would accept in the simulated transcript with $P^*$ because they run the same decision algorithm on the final state of query/response pairs). % $1$ on $\st_V$. (This is due to the fact that the verifier in the compiled IOP runs the same final decision algorithm as the IOP verifier). 

\paragraph{Analysis of $P'_O$ success probability} 
We claim that if $\pro{Record}(P^*, \params, x, \st_0)$ outputs an accepting transcript $\tr$ with non-negligible probability, then $P'_O$ succeeds with non-negligible probability. 

Observe that for any accepting $\tr$ between $P^*$ and $V$, if $P'_O$ happens to follow the same exact sequence of query/responses without ever aborting then it succeeds because $\verifier_O$ and $\verifier$ run the same decision algorithm on the final state of query/response pairs. Thus, it remains only to take a closer look at what events cause $P'_O$ to abort, and bound the fraction of accepting $\tr$ for which this occurs. 

As indicated in bold above, there are two kinds of events that cause $P'_O$ to abort: 
\begin{itemize}
\item It fails to extract from a ``commitment" message $m$ sent by $P^*$
\item After successfully extracting a polynomial $f$ from a commitment, $P^*$ answer queries to $f$ in a way that is inconsistent with $f$. 
\end{itemize}

The second type of event contradicts the evaluation binding property of $\Gamma$, therefore it occurs with negligible probability. 

To analyze the first type of event, let us define ``bad commitments" for a parameter $D$. We define this as a property of a message $m$ (purportedly a commitment) sent in a transcript state $\st$.

\paragraph{Bounding probability of commitment extraction failure} 
The pair $(m, \st)$ is a ``bad commitment" if there is less than a $1/D$ probability that extending the transcript between $P^*$ and $\verifier$, starting from state $\st$, will contain a successful execution of $\eval$ on $m$. This probability is over the randomness of the public-coins of $\verifier$ in the extended transcript. %have a succesful execution on (over the randomness of the public-coins) that $\pro{Record}(P^*, \params, x, \st)$ contains a successful execution of $\eval$ on $m$ on the queries defined by $\st$ and the next public coin challenge, where $\st$ is determined by running $\tr$ up until the point $m$ appears. 


Let $A(\tr)$ denote the event that a transcript $\tr$ sampled from $\pro{Record}(P^*, \params, x, \st_0)$ is accepting. Let $B(\tr)$ denote the event that $\tr$ contains a ``bad commitment" (i.e. some message $m$ sent in state $\st$ such that $\pro{Bad}(m, \st) = 1$). The conditional probability of event $A(\tr)$ conditioned on event $B(\tr)$ is less than $1/D$. To see this, fix $(m, \st)$ with $\pro{Bad}(m, \st) = 1$ and consider ``sampling" a random $\tr$ that contains $m$ at state $\st$. This is done by first choosing randomly from all partial transcripts that result in $(m, \st)$ via brute force, and then running the transcript normally from state $\st$ on random public-coins. No matter how $(m, \st)$ is chosen, the probability that this process produces an accepting transcript is by definition less than $1/D$. (The second part of the transcript following $(m, \st)$ contains at least one execution of $\eval$ on $m$ by hypothesis, and by the definition of $B(m, \st) = 1$ this execution is accepting with probability less than $1/D$).

Assume that $P(A(\tr)) \geq 1/\poly$. Applying Bayes' law, %letting $A(\tr)$ denote the event that $\tr$ is accepting and $P(A(\tr)) > 1/\poly$ the a-priori probability of this event, 
\[ P[B(\tr) | A(\tr)) \leq \frac{ P[A(\tr) | B(\tr)] }{ P(A(\tr)) } \leq \poly / D \enspace . \]
In other words, at least a $1 - \poly/D$ fraction of accepting transcripts do not contain ``bad commitments". %By a union bound, in a length $L$ transcript void of bad commitments, the transcript does not contain any failed $\eval$ with probability at least $L/B$. 
Furthermore, so long as a commitment $m$ is not ``bad", we can invoke the witness-emulation property of $\eval$ to say that the PPT $E_\Gamma$ emulator extracts a witness polynomial from each $m$ with overwhelming probability.


Setting $D = 2 \poly$ we get that on at least a $1/2$ fraction of accepting transcripts, $P'_O$s simulation also succeeds (i.e. successfully extracts from each prover commitment message) with probability at least $1/2$. This means that $P'_O$ has a non-negligible success probability conditioned on the event that $\tr$ is an accepting transcript. 

In conclusion, if $\tr$ is accepting with non-negligible probability, then there is a non-negligible probability that $P'_O$ succeeds. 
\end{proof}

\ifappendix

\section{Optimizations for Polynomial Commitment Scheme}
\label{subsec:optimization}
We present several ideas for optimizing the performance of the $\pro{Eval}$ protocol.

\paragraph{Precomputation.} The prover has to compute powers of $\gr{g}$ as large as $q^d$. While this can be done in linear time, this expense can be shifted to a preprocessing phase in which all elements $\gr{g}^{q^i}, i \in \{1, \ldots, d_{\it max}\}$ are computed. Since for coefficient $|f_i|\leq -\frac{p-1}{2}$ this allows the computation of $\gr{g}^{f(q)}$ in $O(\lambda d)$ group operations as opposed to $O(\lambda d \log(d))$.
In addition to reducing the prover's workload, this optimization enables parallelizing it. The computation of the $\textsf{PoE}$ proofs can simiarly be parallelized. The prover can express each $Q$ as a power of $\gr{g}$ which enables pre-computation of powers of $\gr{g}$ and parallelism as described by Boneh~\emph{et al.}~\cite{C:BonBunFis19}.
%The elements $\gr{g}^{q^i}$ can themselves be accompanied by non-interactive $\mathsf{PoE}$s to establish their correct computation.

The pre-computation also enables the use of multi-scalar multiplication techniques~\cite{pippenger1980evaluation}. Boneh~\emph{et al.}~\cite{C:BonBunFis19} and Wesolowski~\cite{EC:Wesolowski19} showed how to use these techniques to reduce the complexity of the $\textsf{PoE}$ prover. The largest $\textsf{PoE}$ exponent $q^{\frac{d+1}{2}}$ has $O(\lambda d \log(d))$ bits. Multi-scalar multiplication can therefore reduce the prover work to $O(\lambda d)$ instead of $O(\lambda d \log(d))$.

%\paragraph{Early termination.} The protocol specifies the recursion ends when $d=0$, but the communication cost might be reduced if it terminates earlier. This reduction holds when the size of the fewer group elements $\gr{C}_L$ and $\gr{C}_R$ outweigh the size of the larger polynomial $f(X)$ instead of the constant $f$.

%\paragraph{Fiat-Shamir.} All the challenges of the verifier are public coin and as a result the protocol can be made non-interactive in the random oracle model with the Fiat-Shamir heuristic~\cite{C:FiaSha86}. This technique replaces each message of the verifier with the hash of all previous protocol messages, lifted to the appropriate domain. For the \textsf{PoE}s, it is beneficial to reuse the same $\ell$ across all \textsf{PoE}s and to compute this prime as the hash of the entire transcript after (dropping the $\ell$s and) replacing every instance of $\gr{Q}$ by its matching $\gr{C}_R^{q^{d'+1}}$ counterpart. This optimization requires that $\ell$ be transmitted as part of the proof so that the verifier can infer the $\gr{C}_R^{q^{d'+1}}$ and $\gr{C}_L$, and only after this inference can the verifier check that $\ell$ was computed correctly. The concrete benefit of this optimization is the reduced work for the verifier: previously he had to perform $\lceil\log(d+1)\rceil$ exponentiations of $q \bmod \ell$ to the power $d'+1$, whereas now he can do this task once and record the intermediate results.

\paragraph{Two group elements per round.} In each round the verifier has a value $\gr{C}$ and receives $\gr{C}_L$ and $\gr{C}_R$ such that $\gr{C}_L+q^{d'+1}\cdot \gr{C}_R=\gr{C}$. This is redundant. It suffices that the verifier sends $\gr{C}_R$. The verifier could now compute $\gr{C}_L\gets \gr{C} -q^{d'+1} \gr{C}_R$, but this is expensive as it involves an scalar multiplication by $q^d$. Instead, the verifier infers $q^{d'+1}\cdot \gr{C}_R$ from the \textsf{PoE}: the prover's message is $\gr{Q}$ and the verifier can directly compute $q^{d'+1}\cdot \gr{C}_R\gets \ell \cdot \gr{Q}+r\cdot \gr{C}_R$ for a challenge $\ell$ and $r\gets q^{d'+1} \bmod \ell$. From this the verifier infers $\gr{C}_L \gets \gr{C}-q^{d'+1} \cdot \gr{C}_R$. The security of $\textsf{PoE}$ does not require that $q^{d'+1}\cdot \gr{C}_R$ be sent before the challenge $\ell$ as it is uniquely defined by $\gr{C}_R$ and $q^{d'+1}$.
The same optimization can be applied to the non-interactive variant of the protocol. 

Similarly the verifier can infer $y_L$ as $y_L\gets y-z^{d'+1} y_R$. This reduces the communication to two group elements per round and 1 field element. Additionally the prover sends $f$ which has roughly the size of $\log(d+1)$ field elements, which increases the total communication to roughly $2\log(d)$ elements in $\GG$ and $2\log(d)$ elements in $\ZZ_p$. 

%When the $\mathsf{PoE}$s are made non-interactive, the prover can get away with producing only two group elements instead of three. With a naïve application of the Fiat-Shamir heuristic, the $\mathsf{PoE}$ proof consists of $(\gr{C}_R, \gr{C}_R^\star, \gr{Q})$ where $\gr{Q}$ is determined by $\ell$, which in turn is determined by hashing all previous protocol messages: $\ell \gets \mathsf{H}(\cdot \Vert \gr{C}_R \Vert \gr{C}_R^\star)$. The optimization sends $(\gr{C}_R, \gr{Q}, \ell)$ instead. The verifier can infer $\gr{C}_R^\star = \gr{C}_R^{(q^{d'+1} \bmod \ell)}$ and then test $\mathsf{H}(\cdots \Vert \gr{C}_R \Vert \gr{C}_R^\star) \stackrel{?}{=} \ell$. This optimization is particularly compatible with the previous batching of $\mathsf{PoE}$s optimization, because while there is a unique $\gr{Q}$ for each round, there need only be one $\ell$ for the entire $\eval$ protocol.

\paragraph{Evaluation at multiple points}
The protocol and the security proof extend naturally to the evaluation in a vector of points $\boldsymbol{z}$ resulting in a vector of values $\boldsymbol{y}$, where both are members of $\mathbb{Z}_p^k$. The prover still sends $\gr{C}_L\in \GG$ and $\gr{C}_R\in \GG$ in each round and additionally $\boldsymbol{y}_L,\boldsymbol{y}_R \in \ZZ^k_p$. In the final round the prover only sends a single integer $f$ such that $\gr{g}^{f}=\gr{C}$ and $f \bmod p=y$.

This is significantly more efficient than independent executions of the protocol as the encoding of group elements is usually much larger than the encoding of elements in $\ZZ_p$. Using the optimization above, the marginal cost with respect to $k$ of the protocol is a single element in $\ZZ_p$. If $\lambda=\lceil\log_2(p)\rceil$ is $120$, then this means evaluating the polynomial at an additional point increases the proof size by only $15\log(d+1)$ bytes.

\paragraph{Joining $\mathsf{Eval}$s.} 
In many applications such as compiling polynomial IOPs to SNARKs (see Section~\ref{sec:polyiop}) multiple polynomial commitments need to be evaluated at the same point $z$. 
This can be done efficiently by taking a random linear combination of the polynomials and evaluating that combination at $z$. The prover simply sends the evaluations of the individual polynomials and then a single evaluation proof for the combined polynomials. The communication cost for evaluating $m$ polynomials at $1$ point is still linear in $m$ but only because the evaluation of each polynomial at the point is being sent. The size of the eval proof, however, is independent of $m$. 
Taking a random linear combination does increase the bound on $q$ slightly, as shown in Theorem~\ref{thm:joined} which is presented below.

\[
\mathcal{R_\textsf{JE}}(\params) = \left\lbrace
\langle (\gr{C}_1,\gr{C}_2, z, y_1,y_2,d), (f_1(X), f_2(X)) \rangle
: \\
\begin{array}{l} 
\gr{C}_1, \gr{C}_2 \in \GG \\
z, y_1, y_2 \in \mathbb{Z}_p \\
f_1(X), f_2(X) \in \ZZ(b) \\
(\gr{C}_1,z,y_1,d) \in \mathcal{R_\textsf{Eval}}(\params) \\
(\gr{C}_2,z,y_2,d) \in \mathcal{R_\textsf{Eval}}(\params)
\end{array}
\right\rbrace
\]





\begin{mdframed}
	$\pro{JoinedEval}(\crs, \gr{C}_1, \gr{C}_2, z, y_1, y_2, d; f_1(X),f_2(X)) :$ \pccomment{$f_1(X), f_2(X) \in \ZZ(\frac{p-1}{2})[X]$} \\
%	Statement: $f_1(z)=y_1\bmod p \wedge f_2(z)=y_2\bmod p \wedge \gr{g}^{f_1(q)}=\gr{C}_1$ and $\gr{g}^{f_2(q)}=\gr{C}_2$
Statement: $(\crs,\gr{C}_1,\gr{C}_2,z,y_1,y_2,b,d)\in \mathcal{R}_{\pro{JE}}$
			\begin{enumerate}[nolistsep]
			\item $b=||f_1,f_2||_\infty$
        \item \verifier samples $\alpha \sample [0,2^\lambda)$ and sends it to \prover
			\item \prover and \verifier compute $\gr{C}'\gets \alpha \cdot \gr{C}_1+\gr{C}_2$ and $y'\gets \alpha \cdot y_1 +y_2 \bmod p$
			\item \prover computes $f'(X)\gets \alpha f_1(X) +f_2(X)$
			\item \prover and \verifier run $\pro{EvalB}(\params,\gr{C}',z,y',d,b\cdot 2^{\lambda};f'(X))$
		    \end{enumerate}
\end{mdframed}

\newcommand{\theoremjoined}{
The protocol $\pro{JoinedEval}$ is an interactive argument for the relation $\mathcal{R}_{\pro{JE}}$ and has perfect completeness and witness extended emulation if the Strong RSA and Order Assumption hold for $\ggen$ and if $q>(p-1)(\frac{p^2-1}{2})^{\lceil \log_2(d+1)\rceil+1}$ (e.g., $q > p^{2\log_2(d+1)+3}$). }
\benedikt{Theorem needs to be updated. Protocol as well}
\begin{theorem}
\label{thm:joined}
\theoremjoined
\end{theorem}
\begin{proof}
	Security directly follows from \cref{thm:mvariate} as $C_1,C_2$ is a binding virtual commitment to the bivariate polynomial $f_1+ Y \cdot f_2$. That is, $C=C_1+q^d \cdot C_2$ can be computed from $C_1,C_2$ thus if $C$ is a binding commitment then so is $(C_1,C_2)$. Further \pro{JoinedEval} is identical to an invocation of \pro{MultiEval} on input $(\alpha,z)$
\end{proof}
The proof is presented in Appendix~\ref{appendix:joined}.

We can additionally combine this optimization with the previous optimization of evaluating a single polynomial at different points. This allows us to evaluate $m$ polynomials at $k$ points with very little overhead. 
The prover groups the polynomials by evaluation points and first takes linear combinations of the polynomials with the same evaluation point and computes $y_1$ to $y_k$ using the same linear combinations. Then it takes another combination of the joined polynomials. In each round of the $\eval$ protocol the prover sends $y_{L,1}$ through $y_{L,k}$, i.e. one field element per evaluation point and computes $y_{R,1}$ through $y_{R,k}$. In the final step the prover sends $f$ and the verifier can check whether the final $y$ values are all equal to $f\bmod p$.
 This enables an $\eval$ proof of $m$, degree $d$ polynomials at $k$ points using only $2\log_2(d+1)$ group elements and $(1+k)\log_2(d+1)$ field elements.\benedikt{Adapt}
 
\paragraph{Evaluating the polynomial over multiple fields}
The polynomial commitment scheme is highly flexible. For example it does not specify a prime field $\ZZ_p$ or a degree $d$ in the setup. It instead commits to an integer polynomial with bounded coefficients. That integer polynomial can be evaluated modulo arbitrary primes which are exponential in the security parameter $\lambda$ as the soundness error is proportional to its inverse.
Note that $q$ also needs large enough such that the scheme is secure for the given prime $p$ and degree $d$ (see Theorem \ref{thm:polycommitsecurity}). The second condition, however, can be relaxed. A careful analysis shows that the challenges $\alpha$ just need to be sampled from an exponential space, \emph{e.g.}, $[0,2^{\lambda})$. So as long as $q>p \cdot 2^{\lambda\cdot 2\lceil \log_2(d+1)\rceil}$ for RSA groups or  $q>p \cdot 2^{\lambda \cdot 3\lceil \log_2(d+1)\rceil}$ for class groups one can evaluate degree $d$ polynomial with coefficients bounded by $2^\lambda$ over any prime field.

Additionally, the proof elements $\gr{C}_L$, $\gr{C}_R \in \GG$ are independent of the field over which the polynomial is evaluated. This means that it is possible to evaluate a committed polynomial $f(X) \in \ZZ(b)$ over two separate fields $\ZZ_{p}$ and $\ZZ_{p'}$ in parallel using only $2\log(d+1)$ group elements. 

%This property can be used to efficiently evaluate the polynomial modulo a large integer $m$ by choosing multiple $\lambda$ bit primes $p_1,\dots p_k$ such that $\prod_{i=1}^k p_i\geq m$ and using the Chinese Remainder Theorem to simulate the evaluation modulo $m$.



\section{Multivariate Commitment Scheme}
\label{sec:multivariate}


We can extend our polynomial commitment scheme to multivariate polynomials. The idea is simply to use higher degrees of $q$ to encode the next indeterminate. The protocol is linear in the number of variables and logarithmic in the total degree of the polynomial. For simplicity we only present a protocol for $\mu$-variate polynomials where the degree in each variable is $d$. The protocol extends naturally to different degrees per variable.

\paragraph{Encoding}
Let $q_i=q^{(d+1)^i}$ then $\hat{f}(q_1,\dots,q_\mu)\in \ZZ$ is an encoding of the multivariate polynomial $f(X_1,\dots,X_\mu)$ with maximum degree $d$. We use $\dec_{Multi}(f(q),\mu,d)$ to denote the decoding of an $\mu$-variate polynomial with degree exactly $d$ in each variable. The decoding algorithm simply uses the univariate decoding algorithm described in Section \ref{sec:encoding} to decode a univariate polynomial $\hat{h}(X)$ of degree $(d+1)^{\mu}-1$.
Then it associates each monomial of the univariate polynomial with a degree vector $(d_1,\dots,d_\mu)$ of the multivariate polynomial. The coefficient of the $i$th monomial becomes the coefficient of the $(d1,\dots,d_\mu)$-monomial, where $(d_1,\dots,d_\mu)$ is the base-$(d+1)$ decomposition of $i$. 
\paragraph{Protocols}
 Using this encoding we can naturally derive the multivariate commitment scheme and $\eval$ protocol. The $\eval$ protocol computes the univariate polynomials $f(q_1,\dots,q_{\mu-1},X_\mu)$ and then uses the univariate eval protocol to reduce the claim from a claim about an $\mu$-variate polynomial to one about an $(\mu-1)$-variate one. At the final step the prover opens the now constant polynomial and the verifier can check the claim. For example, the protocol would reduce a bivariate (say $X$ and $Y$) cubic polynomial to a univariate one (in $Y$) in two rounds of interaction and then reduce the degree of $Y$ using another two rounds.
 
 \begin{mdframed}[userdefinedwidth=\textwidth]
\begin{minipage}{\textwidth}
	\begin{flushleft}
	$\pro{MultiSetup}(1^\secpar):$
		\begin{enumerate}[nolistsep]
			\item $ \GG \sample \ggen(\secpar)$
			\item $ \gr{g} \sample \GG$
			%\item $q \gets 2^k$ such that $q > (d+1) \cdot 2\cdot p^{\log_2(d+1)+1} $
			%\item Pick a prime $p\in \NN$ such that $\lceil\log_2(p)\rceil=\lambda$.
			%\item Pick a sufficiently large and odd $q\in \NN$ \pccomment{$q=O_\lambda(p^{\mu \cdot \log(d)})$}
			\item $\pcreturn \params = (\secpar,\GG,\gr{g})$
		\end{enumerate}
	$\pro{MultiCommit}(\params;f(X_1,\dots,X_\mu) \in \ZZ(\frac{p-1}{2})[X_1, \ldots, X_\mu]\subset \ZZ[X_1, \ldots, X_\mu]):$ 		\begin{enumerate}[nolistsep]
			\item $d\gets \deg(f)$\pccomment{For simplicity assume $f(X_1,\dots,X_n)$ has degree $d$ in each variable}
			\item $q_i\gets q^{(d+1)^{i-1}}$ for each $i\in [\mu]$
			\item $\gr{C} \gets \gr{g}^{f(q_1,\dots,q_\mu)}$
			\item $\pcreturn (\gr{C};f(X_1,\dots,X_\mu))$
		\end{enumerate}
			\end{flushleft}
\end{minipage}
\end{mdframed}
 
 \begin{mdframed}
\begin{minipage}{\textwidth}
			$\pro{MultiEval}(\params, \gr{C}\in \GG, \boldsymbol{z}\in \ZZ^\mu_p,y \in \ZZ_p, d,\mu,b \in \NN; f(X_1,\dots,X_\mu)\in \ZZ(b)[X_1, \ldots, X_\mu]) :$
			\begin{enumerate}[nolistsep]
			\item \pcif{$\mu=1$} 
			\item \pcind[1] \prover and \verifier run $\pro{EvalBounded}(\params,\gr{C},z_1,y,d,b,x;f(X_1))$ 
			\item \pcelse
			\item \pcind[1] Let $\hat{f}(X_\mu)\gets f(q_1,\dots,q_{\mu-1},X_\mu)$
			\item \pcind[1] Let $\crs_\mu \gets \{\lambda,\GG,\gr{g},p,q_\mu\}$
			\item \pcind[1] \prover and \verifier run the univariate $\pro{EvalBounded}(\params_\mu,\gr{C},z_\mu,y,d,q_\mu;\hat{f}(X))$
			\item \pcind[2] \textbf{except:} when $d=0$, $f$ is not sent; instead the protocol returns its input at this point, \emph{i.e.}, $(\gr{C}',y',b')$ along with the prover's witness $f'(X_1,\dots,X_{\mu-1})=\dec_{Multi}(f,\mu-1,d)$ (Lines~\ref{line:basestart}-\ref{line:baseend} of $\pro{EvalBounded}$). 
			\item \pcind[1]$\boldsymbol{z}'\gets (z_1,\dots,z_{\mu-1})\in \ZZ_p^{\mu-1}$
			\item \pcind[1]\prover and \verifier run $\pro{MultiEval}(\crs,C',\boldsymbol{z}',y',d,\mu-1,b';f')$
		    \end{enumerate}
      \end{minipage}
\end{mdframed}
We only proof security under the strong RSA assumption. The security proof, however, directly extends to groups where taking square roots is easy under the $2$-Strong-RSA Assumption. In that case $q>p^{3\mu \log_2(d+1)+1}$ suffices.
\begin{theorem}[Multivariate Eval]
	The polynomial commitment scheme for multi-variate polynomials consisting of protocols $(\pro{MultiSetup},\pro{MultiCommit},\pro{MultiEval})$ has perfect correctness and witness extended emulation if the Adaptive Root Assumption and the Strong RSA Assumption hold for $\ggen$ for $\mu$-variate polynomials of degree $d$ and if $d^\mu=\poly$ if $q> p^{2 \mu \log_2(d+1)+1}$.
\end{theorem}
\begin{proof}
	Perfect correctness follows from the correctness of the univariate commitment scheme and the fact that the coefficients of the witness polynomial in the honest execution are less than $\frac{p-1}{2}p^{\mu \lceil\log(d+1)\rceil}<q/2$.
	
	To show witness extended emulation we use the forking lemma (Lemma \ref{lemma:GFL}) and build a polynomial time extractor algorithm $\mathcal{X}_{\pro{MultiEval}}$ that given a binary tree of transcripts of depth $\mu \cdot\lceil\log(d+1)\rceil$, extracts a witness. Each node corresponds to a different challenge $\alpha$ as described in the forking lemma. The tree consists of at most $(d+1)^{\mu}=\poly$ transcripts. 
	Lemma~\ref{lem:poe} states that the probability that an adversary can create any accepting transcript for which the $\textsf{PoE}$ can't be replaced by a direct check is negligible under the Adaptive Root Assumption.
We can therefore invoke the lemma to replace all \textsf{PoE} executions with direct verification checks that $\gr{C}_L\gr{C}_R^{q^{d'+1}}=\gr{C}$. 
%The lemma focuses on the univariate \pro{Eval} protocol but works identically for the multivariate protocol. 

In constructing $\mathcal{X}_{\pro{MultiEval}}$ we use the extractor $\mathcal{X}_{\pro{Eval'}}$ described in the proof of Theorem~\ref{thm:polycommitsecurity}. $\mathcal{X}_{\pro{Eval'}}$ computes, given a tree of transcripts for $\pro{Eval'}$ a valid witness of $\pro{Eval'}$ or a fractional root of $\gr{g}$ or an element of known order in $\GG$. We construct $\mathcal{X}_{\pro{MultiEval}}$ recursively invoking $\mathcal{X}_{\pro{Eval'}}$ once per degree $\mu$. The probability that a polynomial time adversary and a polynomial time extractor $\mathcal{X}_{\pro{Eval'}}$ can produce a fractional root or an element of known order in $\GG$ is negligible under the strong-RSA and the the adaptive root assumptions. From hence on we will consider the case where neither of these events happen.

We use the superscript $(i)$ to denote the inputs to $\pro{MultiEval}$ where $\mu=i$. 
If $\mu=1$ then the extractor $\mathcal{X}_{\pro{Eval'}}$ directly extracts $f^{(1)}(X)\in \ZZ(b)$, a univariate degree $d$ polynomial with coefficients bounded by $b=\frac{p-1}{2}p^{2 \lceil\log_2(d+1)\rceil}$ and such that $f(z)=y \bmod p$. Note that $q/2>b$ so the extraction succeeds.

For $\mu>1$, let's assume that $f^{(\mu-1)}(X_1,\dots,X_{\mu-1})\in \ZZ(b)$ is an extracted $\mu-1$ variate polynomial with degree $d$ in each variable such that $f^{(\mu-1)}(z_1,\dots,z_{\mu-1}) \bmod p=y'$.
Let $f'\gets \enc_{Multi}(f^{(\mu-1)}(X_1,\dots,X_{\mu-1})\in \ZZ$ be the encoding of $f^{(\mu-1)}(X_1,\dots,X_n)$, such that $\gr{C}^{(\mu-1)}\gets \gr{g}^{f'}$. Note that $f'$ is equivalent to an encoding of a univariate degree $(d+1)^{\mu-1}$ polynomial with the same coefficients as the multivariate polynomial. Let $g^{(\mu-1)}(X)=\dec(f')\in \ZZ(b)[X]$ be that polynomial. 
Using $g^{(\mu-1)}(X)$ as the witness the extractor $\mathcal{X}_{\pro{Eval'}}$ extracts a univariate degree $(d+1)^{\mu}$ polynomial $g^{(\mu)}(X)$ with coefficients in $\ZZ(b \cdot p^{\lceil\log(d+1)}\rceil)$. 
Let $f''\gets g^{(\mu)}(q)$ be the encoding of $g^{(\mu)}$ such that $\gr{C}^{(\mu)}=\gr{g}^{f''}$. Note that using the multivariate decoding algorithm $f''$ also encodes a $\mu$-variate degree $d$ polynomial, i.e. $f^{(\mu)}(X_1,\dots,X_\mu)\gets \dec_{Multi}(f'',\mu,d)$. The $X^i$th coefficient of $g^{\mu}(X)$ is the coefficient for the base monomial defined by the base-$(d+1)$ decomposition of $i$, i.e. $\prod_{j=1}^\mu  X_j^{\lfloor i/(d+1)^{j-1}\rfloor \bmod d+1 }$ in $f(X_1,\dots,X_\mu)$. Note that the extraction additionally guarantees that the polynomial evaluation is correct, i.e. $f(z_1,\dots,z_\mu)\bmod p=y$.

The final extracted polynomial has coefficients in $\ZZ(\frac{p-1}{2}p^{2\mu \lceil\log_2(d+1)\rceil})$. Since $q>p^{2\mu\lceil\log_2(d+1)\rceil+1}$ both the univariate and the multivariate decoding succeed and the extractor extracts a valid $\mu$-variate degree $d$ witness polynomial.
\end{proof}

\fi

%\ifappendix
%\section{Other Instantiations of Polynomial IOPs} \label{appendix:other_polynomial_iops}

%
\subsubsection{Spartan}
\textsf{Spartan}~\cite{Spartan} transforms an arbitrary circuit satisfaction problem into a Polynomial IOP based on an arithmetization technique developed by Blumberg~\emph{et al.} \cite{EPRINT:BTVW14}, which improved on the classical techniques of Babai, Fortnow, and Lund~\cite{BFL}. Specifically, satisfiability of a 2-fan-in arithmetic circuit on $n$ gates can be transformed into the expression: 
\begin{equation}\label{eqn:hypercubesum}
\sum_{x, y, z \in \{0,1\}^{\log n}} G(x, y, z) = 0
\end{equation} 
for a multilinear polynomial $G$ on $3 \log n$ variables over $\FF$. 
Furthermore, $G$ decomposes into the form: 
$$G(x,y,z) = A(x,y,z) F(x) + B(x, y, z) F(y) + C(x, y, z) F(y) F(z)$$
where $A, B, C,$ and $F$ are all multilinear poylnomials. The polynomials $A, B, C$ are derived from the arithmetic circuit defining the relation $\mathcal{R}$ and are input-independent. $F$ is degree $1$ with $\log n$ variables and is derived from a particular $(x, w) \in \mathcal{R}$. For uniform circuits, the verifier can evaluate $A, B, C$ locally in $O(\log n )$ time. The LFKN sum-check protocol~\cite{FOCS:LFKN90} is applied in order to prove Expression~\ref{eqn:hypercubesum} in a $3\log n$ round Polynomial IOP, where the prover's oracle consist of $Z$ and the low-degree polynomials sent in the sumcheck. Since the extra low-degree polynomials are constant size they can be read entirely by the verifier in constant time rather than via oracle access, and hence we ignore them in the total oracle count. The main result in Spartan can be summarized in our framework as follows: 

\begin{theorem}[Setty19]
There exists a $3 \log n$ round Polynomial IOP for any NP relation $\mathcal{R}$ computed by a \textbf{uniform} circuit with arithmetic complexity $n$, which makes three queries to a $\log n$-variate degree 1 polynomial oracle.  
\end{theorem}

Applying our multivariate compiler to the \textsf{Spartan} Polynomial IOP we obtain an $O(\log n)$-round public-coin interactive argument of knowledge for uniform circuits of size $n$. In our multivariate scheme (Section~\ref{sec:multivariate}), the $\log n$-variate degree 1 polynomial is tranformed into a univariate polynomial of degree $n$. With only three queries overall, the communication is just $6 \log n$ group elements and $6 \log n$ field elements. 

\subsubsection{QAPs} 

QAPs can be expressed as linear PCPs~\cite{TCC:BCIOP13,C:BCGTV13}. We review here how to express QAPs as a one round public-coin $(1, n)$ algebraic IOP. (This captures the satisfiability of any circuit with multiplicative complexity $n$, which is first translated to a system of quadratic equations over degree $n$ polynomials). Each linear query is computed by a vector of degree $n$ univariate polynomials evaluated at a random point chosen by the public-coin verifier. '

For illustration, will use the language \emph{satisfiability of rank-1 quadratic equations} over $\FF$ described by Ben-Sasson~\emph{et al.}~\cite{C:BCGTV13}. An instance of this language is defined by length $m+1$ polynomial vectors $A(X)$, $B(X)$, $C(X)$ such that the $i$th components $A_i(X)$, $B_i(X)$, $C_i(X)$ are all degree $n-1$ polynomials over $\FF_p[X]$ for $i \in [0,m]$, and $A_m = B_m = C_m$ is the degree $n$ polynomial $Z(X)$ that vanishes on a specified set of $n$ points in $\FF_p$. There is a length $m-1$ witness vector $\mathbf{w}$ whose first $\ell$ components are equal to the instance $\mathbf{x} \in \FF^\ell$, and a degree $n$ ``quotient" polynomial $H(X)$, such that the following constraint equation is satisfied: 
\begin{equation} \label{eqn:R1CS} 
\begin{split}
[(1, \mathbf{w}, \delta_1)^\top A(X)][(1, \mathbf{w}, \delta_2)^\top B(X)] 
- (1, \mathbf{w}, \delta_3)^\top C(X) = H(X)Z(X) \\ 
\ and \ (1,\mathbf{w})^\top (1,X,...,X^{\ell}, \mathbf{0}^{m- \ell -1}) = (1,\mathbf{x})^\top (1, X,...,X^{\ell})
\end{split} 
\end{equation} 
%\alan{What do $\delta_0, \delta_1, \delta_2$ do? Also, I'm not sure the dimensions work out.}

The deltas (i.e. $\delta_1, \delta_2, \delta_3$) are used as randomization terms for HVZK. 

\paragraph{QAP algebraic linear PCP} Equation~\ref{eqn:R1CS} is turned into a set of linear queries by evaluating the polynomials at a random point in $\FF$. Satisfaction of the equation evaluated at a random point implies satisfaction of the polynomial equation with error at most $2n / |\FF|$ by Schwartz-Zippel. Translated to an algebraic IOP, the prover sends a proof oracle $\proofO_w$ containing the vector $(1, \mathbf{w}, \delta_1, \delta_2, \delta_3)$ as well as a proof oracle $\proofO_h$ containing the coefficient vector of $H(X)$. A common proof oracle $\proofO_z$ is jointly established containing the coefficient vector of $Z(X)$. 

The verifier chooses a random point $\alpha \in \FF$ and makes four queries to $\proofO_w$, computed by the polynomial vectors $A(X), B(X), C(X)$ and $D(X) = (1, X,...,X^\ell, \mathbf{0}^{m- \ell -1})$. The verifier makes one query each to $\proofO_h$ and $\proofO_z$, which is the evaluation of $H(\alpha)$ and $Z(\alpha)$ respectively. The verifier obtains query responses $y_a, y_b, y_c, y_d, y_h, y_z$ and checks that $y_a \cdot y_b - y_c = y_h y_z$ and $y_d = \langle (1, \mathbf{x}), D(\alpha) \rangle$. 

\paragraph{Compiling QAP to public-coin argument} 

Following the compilation in Theorem~\ref{thm:algebraicIOPcompiler} (Section~\ref{sec:algebraicIOP}), the R1CS algebraic linear PCP can be transformed into a $2$-round Polynomial IOP. For simplicity, assume $m+3 < n$, where $m-1$ is the length of the witness and $n$ is the multiplicative complexity of the circuit. The preprocessing establishes three bivariate degree $n$ polynomials (\emph{i.e.}, encoding $A(X), B(X), C(X)$) and two univariate degree $n$ polynomials (\emph{i.e.}, encoding $Z(X)$ and $D(X)$). In the 2-round online phase the prover sends a degree $n$ univariate oracle for the witness vector $(1, \mathbf{w}, \delta_1, \delta_2, \delta_3)$, a degree $n$ univariate oracle for $H(X)$, four degree $n$ univariate oracles encoding linear PCP queries, four degree $2n$ univariate oracles encoding polynomial products, and eight degree $2n$ univariate oracles for opening inner products. The total number of polynomial oracle evaluation queries is $3$ bivariate degree $n$, $8$ univariate degree $2n$, and $7$ univariate degree $n$.

\begin{theorem}[QAP Polynomial IOP]
There exists a $2$-round Polynomial IOP with preprocessing for any NP relation $\mathcal{R}$ (with multiplicative complexity $n$) that makes $7$ queries to univariate degree $n$ oracles, $8$ queries to univariate degree $2n$ oracles, and $3$ queries to bivariate degree $n$ oracles.  
\end{theorem}
 
While theoretically intriguing, compiling the QAP-based IOP with our polynomial commitments of Section~\ref{sec:protocol} is less practical than compiling the \textsf{Sonic} IOP. While the R1CS Polynomial IOP has only $15$ univariate queries (compared to \pro{Sonic}'s $27$ queries to polynomials of approximately the same degree), the $3$ bivariate polynomial oracles take quadratic time to preprocess and open. Unfortunately, our polynomial commitment scheme does not take advantage of the sparsity of these bivariate polynomials. Furthermore, ignoring prover time complexity, the size of the bivariate $\eval$ proofs are twice as large as univariate $\eval$ proofs so the number of queries is effectively equivalent to $21$ univariate degree $n$ queries. 
%\fi


