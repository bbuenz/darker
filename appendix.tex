\clearpage
\section*{\scalebox{1.25}{\appendixphrase}}

\section{Security Proofs}
\label{appendix:hardness}
In the preliminaries we already stated the two main hardness assumptions, the $r$-Strong RSA Assumption and the Adaptive Root Assumption.
We additionally use two more assumptions, however both of them reduce to the Strong RSA and the Adaptive Root Assumptions.

The first assumption states that computing the order for \emph{any} element is hard. It reduces to the Adaptive Root Assumption. Interestingly, it doesn't necessarily hold for all candidate groups of unknown order as we explain below. In particular it is important to exclude elements of known order such as $-1$ from the candidate unknown order group $\ZZ_n$.

\begin{assumption}[Order Assumption]
\label{assum:order}
	The Order Assumption holds for $\ggen$ if for any efficient adversary $\adv$:
\[        
                \Pr\left[\gr{w}\neq 1 \wedge \gr{w}^{\alpha}= 1: 
                \begin{array}{l} 
                      \GG \sample \ggen(\lambda) \\ 
                      (\gr{w},\alpha) \sample \adv(\GG) \\
                      \text{where } |\alpha|<2^{\poly{}}\in \ZZ\\
                      \text{and } \gr{w}\in \GG
                \end{array} 
        \right] \leq \negl \enspace .
\]
\end{assumption}
\begin{lemma}
\label{lem:ordertoadaptive}
	The Adaptive Root Assumption implies the Order Assumption.
\end{lemma}
\begin{proof}
	We show that given an adversary $\adv_{\textsf{Ord}}$ that breaks the Order Assumption we can construct with overwhelming probability $\adv_{\textsf{AR}}$ that breaks the Adaptive Root Assumption. We run $\adv_{\textsf{Ord}}$ to get a $\gr{w}\neq 1\in \GG$ and $\alpha \in \ZZ$ such that $\gr{w}^{\alpha}=1$. To construct $\adv_{\textsf{AR}}$, $\adv_{\textsf{AR},0}$ outputs $(\gr{w},\alpha)$. The challenger generates a random challenge $\ell$. If $\gcd(\ell,\alpha)=1$ then $\adv_{\textsf{AR},1}$ can compute $\beta\gets \ell^{-1} \bmod \alpha$ and output $\gr{u}\gets\gr{w}^{\beta}$. By construction $\gr{u}^{\ell}=\gr{w}$. The probability that $\gcd(\ell,\alpha)=1$ is overwhelming because $\gcd(\ell,\alpha)\neq 1 \implies \ell \vert \alpha$. This happens with negligible probability as $\ell$ is picked from a set of $2^\lambda$ primes and at most $\poly$ distinct primes can divide $\alpha$.
	\end{proof}
	
	
We also define the Fractional Root Assumption, which states that for random group elements $\gr{g}$ it is hard to find a tuple $(\gr{u}\in \GG,\alpha\in \ZZ,\beta\in \ZZ)$ such that $\gr{u}^{\beta}=\gr{g}^{\alpha}$. 
We say that $(\gr{u},\alpha,\beta)$ is a \emph{fractional root} of $\gr{g}$.
%unless $\frac{\alpha}{\beta}$ is an dyadic rational, \emph{i.e.}, a rational whose denominator is a power of $2$
%In RSA groups the assumption is also conjectured to hold if $\frac{\alpha}{\beta}$ is restricted to be an integer. 
Shoup\cite{CCS:CraSho99} showed that for the unknown order group of quadratic residues in $\ZZ_n$, where $n$ is the composite of two strong primes, that the Fractional Root Assumption reduces to just the Strong RSA Assumption.

\begin{assumption}[$r$-Fractional Root Assumption]
\label{assum:fracroot}
The \defn{$r$-Fractional Root Assumption} holds for $\ggen$ for any efficient adversary $\adv$:
\[        
                \Pr\left[\gr{u}^\beta = \gr{g}^{\alpha} \wedge \frac{\beta}{\gcd(\alpha,\beta)}\neq r^k,  k \in \NN   : 
                \begin{array}{l} 
                      \GG \sample \ggen(\lambda) \\ 
                      \gr{g} \sample \GG \\
                      (\alpha, \beta, \gr{u}) \sample \adv(\GG, \gr{g}) \\
                      \quad \textnormal{where} \, |\alpha|<2^{\poly}, \\
                      \quad |\beta|<2^{\poly} \in \ZZ, \\
                      \quad \textnormal{and} \, \gr{u} \in \GG 
                \end{array} 
        \right] \leq \negl \enspace .
\]
\end{assumption}
We say $(\alpha,\beta,\gr{u})$ is a non power of $r$ fractional root of $\gr{g}$.

The Fractional Root Assumption reduces to the Order Assumption (and therefore to the Adaptive Root Assumption) and the Strong RSA Assumption.
\begin{lemma}
\label{lem:strongtofractional}
	The Adaptive Root Assumption and the $r$-Strong RSA Assumption imply the $r$-Fractional Root Assumption 
	%if groups generated by $\ggen$ have order coprime with $r$ and there exists a PPT algorithm for taking $r$th roots in these groups.
\end{lemma}
\begin{proof}
	Given an adversary $\adv_{\textsf{FR}}$ that succeeds in breaking the Fractional Root Assumption for $\ggen$ we can construct either an adversary $\adv_{RSA}$ for the Strong RSA Assumption or an adversary $\adv_{\textsf{Ord}}$ that breaks the Order Assumption for $\ggen$. As shown in Lemma \ref{lem:ordertoadaptive} the Order Assumption reduces to the Adaptive Root Assumption with overwhelming probability. 
	We first generate a group of unknown order $\GG \sample \ggen(\lambda)$.
	Then we sample $\gr{g}\sample \GG$ as done in the strong \textsf{RSA} security definition.
	
	We now run the $\adv_{\textsf{FR}}$ on input $\GG$ and $\gr{g}$ to generate a tuple $(\alpha,\beta,\gr{u})$ such that $\gr{u}^{\beta}=\gr{g}^{\alpha}$. Let $\gamma=\gcd(\alpha,\beta)$ and $\alpha'=\frac{\alpha}{\gamma}\in \ZZ$ and  $\beta'=\frac{\beta}{\gamma}\in \ZZ$. Now either $\gr{g}^{\alpha'}=\gr{u}^{\beta'}$ or $\gr{g}^{\alpha'}/\gr{u}^{\beta'}$ is a non trivial element of order $\gamma$ which would directly break the Order Assumption. In that case we constructed $\adv_{\textsf{Ord}}$ that outputs $(\gr{g}^{\alpha'}/\gr{u}^{\beta'},\gamma)$.
	
	Now assume otherwise, i.e. $\gr{g}^{\alpha'}=\gr{u}^{\beta'}$. By construction $\gcd(\alpha',\beta')=1$ and we can efficiently compute integers $a,b$ such that $a \alpha'+b \beta'=1$. By assumption on $\adv_{\textsf{FR}}$ $\beta'$ is not $r^k$. Now let $\gr{w}\gets \gr{u}^{a}\gr{g}^{b}$. Note that $\gr{w}^{\alpha'\beta'}=\gr{g}^{\alpha'}$. So either $\gr{w}^{\beta'}=\gr{g}$ or $\gr{w}^{\beta'}/\gr{g}$ is a non-trivial element of order $\alpha'$. The first case breaks the Strong RSA Assumption, as we can construct $\adv_{\textsf{RSA}}$ that outputs $(\gr{w},\beta')$, and the second breaks the Order Assumption.
\end{proof}

\subsection{Binding}
\label{appendix:binding}
%\def\thelemma{\ref{lem:binding}}
\newtheorem*{lemmabinding}{Lemma \ref{lem:binding}}
\begin{lemmabinding}
	\bindinglemma
\end{lemmabinding}
\begin{proof}
    Assume that there is an adversary that breaks the binding property of the scheme. Specifically, assume that some probabilistic polynomial time algorithm $\adv$ takes as input $\params$ and outputs $\gr{C} \in \GG, f(X) \in \ZZ(b)[X], f'(X)\in \ZZ(b)[X]$ such that with non-negligble probability $\pro{Open}(\params, \gr{C}, \tilde{f}(X), f(X)) = \pro{Open}(\params, \gr{C}, \tilde{f'}(X), f'(X)) = 1$ and $\tilde{f}(X) \neq \tilde{f'}(X)$. We proceed to show that this implies a violation of the Order Assumption~(Assumption \ref{assum:order}) and the Strong RSA Assumption~(Assumption \ref{assum:strongRSA}). The assumptions are incomparable so we show that either suffices to achieve the binding property of the commitment scheme.
    
	If $f(X)\neq f'(X)$ and $q/2>b$ then $f(q)\neq f'(q)\in \ZZ$. Since $\gr{g}^{f(q)}=\gr{g}^{f'(q)}=\gr{C}$ we have that $\gr{g}^{f(q)-f'(q)}=1$. This directly breaks the Order Assumption and we can also create an adversary $\adv_{RSA}$ that breaks the Strong RSA Assumption. To do so the $\adv_{RSA}$ picks an odd prime $\ell$ that is co-prime with $f(q)-f'(q)$ and computes $\gr{u}\gets \gr{g}^{\ell^{-1} \bmod (f(q)-f'(q))}$ as the $\ell$th root of $\gr{g}$.
\end{proof}

\subsection{Correctness}
\label{appendix:correctness}

%\def\thelemma{\ref{lem:correctness}}
\newtheorem*{lemmacorrectness}{Lemma \ref{lem:correctness}}
\begin{lemmacorrectness}
	\correctnesslemma
\end{lemmacorrectness}

\begin{proof}
In order to ensure correctness we must ensure that $b< q/2$ and that $|f|\leq b$. To show this we show that in each recursion step the honest prover's witness polynomial has coefficients bounded by $b$ and has degree $d$. 
We argue inductively that for each recursive call of $\pro{EvalBounded}$ the following constraints on the inputs are satisfied: The degree of $f(X)$ is bounded by $d$. $\gr{C}$ encodes the polynomial, \emph{i.e.}, $\gr{C}=\gr{g}^{f(q)}$ and $f(X)\in \ZZ(b)$. Also $f(z) = y\bmod p$.

Initially, during the execution of $\eval$, the prover maps the coefficients of a polynomial $\tilde{f}(X)\in \ZZ_p$ to an integer polynomial $f(X)$ with coefficients in $\ZZ(\frac{p-1}{2})$ and degree at most $d$ such that $\gr{C}=\gr{g}^{f(q)}$. Additionally $f(z)\bmod p=\tilde{f}(z)=y$.

 In a recursion steps where $d+1$ is odd, $f'(X)=X\cdot f(X)$ is a polynomial of degree $d+1$ such that $\gr{C}'=\gr{C}^q=\gr{g}^{q\cdot f(q)}=\gr{g}^{f'(X)}$ and the bound $b$ is unchanged as are the coefficients. Also, $f'(z)\bmod p = z \cdot f(z) \bmod p=z\cdot y\bmod p = y' \bmod p$. If $d+1$ is odd, then in the next step $d+1$ must be even.
 
 If $d+1$ is even then, $\prover$ computes $f_L(X)$ and $f_R(X)$ such that $f_L(X)+X^{\frac{d+1}{2}} f_R(X)=f(X)$. Consequently $f(z) \bmod p=f_L(z)+ z^{\frac{d+1}{2}} f_R(z)\bmod p=y_L+z^{\frac{d+1}{2}}  y_R\bmod p =y$. The \textsf{PoE} protocol has perfect correctness so {$\gr{g}^{f_L(q)+q^{\frac{d+1}{2}} f_R(X)}=\gr{C}$}.
 %\gr{C}_L\gr{C}_R^{(q^{\frac{d+1}{2}})}
 Finally $f'(X)=\alpha f_L(X) + f_R(X)\in \ZZ(\frac{p+1}{2}\cdot b)$ is a degree $d$ polynomial with coefficients bounded in absolute value by $(\frac{p+1}{2})\cdot b$. This is precisely the value of $b'$ the input to the next call of $\pro{EvalBounded}$. The value $y'$ is also correct:
$f'(z)\bmod p=\alpha f_L(z) +f_R(z) \bmod p= \alpha y_L +y_R\bmod p=y'$
 
 There are exactly $\lceil\log_2(d+1)\rceil$ recursion steps with even $d+1$. In the final recursion step we therefore have $b=\frac{p-1}{2}(\frac{p+1}{2})^{\lceil\log_2(d+1)\rceil}$ and as such the requirement that $q/2>\frac{p-1}{2}(\frac{p+1}{2})^{\lceil\log_2(d+1)\rceil}$. 
 So if $q>p^{\lceil\log_2(d+1)\rceil+1}\geq (p-1) (\frac{p+1}{2})^{\lceil \log_2(d+1)\rceil}$ then all verifier checks pass and the verifier outputs $1$.
\end{proof} 


\subsection{Proof of Theorem~\ref{thm:polycommitsecurity}}
\label{appendix:maintheoremproof}

\paragraph{Security of $\textsf{PoE}$ substitutions}
We first begin by showing that we can safely replace all of the $\textsf{PoE}$ evaluations with direct verification checks. Concretely, under the Adaptive Root Assumption, the $\eval$ protocol is as secure as the protocol $\eval'$ in which all $\textsf{PoE}$s are replaced by direct checks. We show that the witness-extended emulation for $\eval'$ implies the same property for $\eval$. This is useful because we will later show how to can build an extractor for $\eval'$, thereby showing that the same witness-extended emulation property extends to $\eval$.

\begin{lemma} \label{lemma:poe_security}
Let $\eval'$ be the protocol that is identical to $\eval$ but in line \ref{line:PoE} of $\pro{EvalBounded}$ $\verifier$ directly checks $\gr{C}_L\gr{C}_R^{q^{d'+1}}=\gr{C}$ instead of using a $\textsf{PoE}$. If the Adaptive Root Assumption holds for $\ggen$, and $\eval'$ has witness-extended emulation for polynomials of degree $d=\poly$, then so does $\eval$.
\end{lemma}

\begin{proof}
We show that if an extractor $E'$, as defined in Definition~\ref{def:wee}, exists for the protocol $\eval'$ then we can construct an extractor $E$ for the protocol $\eval$. Specifically, $E$ simulates $E'$ and presents it with a $\pro{Record}'(\cdots)$ oracle, while extracting the witness from its own $\pro{Record}(\cdots)$ oracle.

Whenever $E'$ queries the $\pro{Record}'$ oracle, $E$ queries its $\pro{Record}$ oracle and relays the response after dropping those portions of the transcript that correspond to the $\mathsf{PoE}$ proofs. Whenever $E'$ rewinds its prover, so does $E$ rewind its prover. When $E'$ terminates by outputting a transcript-and-witness pair $(\mathsf{tr}', f(X))$, $E$ adds $\mathsf{PoE}$s into this transcript to obtain $\mathsf{tr}$ and outputs $(\mathsf{tr}, f(X))$.

For each PPT adversary $(\adv,P^*)$, $E$ will receive a polynomial number of transcripts from its $\pro{Record}$ oracle. Any transcript $\tr$ of $\eval$ such that $\adv(\tr)=1$ and $\tr$ is accepting contains exactly $\lceil \log(d+1)\rceil$ $\textsf{PoE}s$ transcripts. 
So in total $E$ sees only a polynomial number of $\textsf{PoE}$ transcripts generated by a probabilistic polynomial-time prover and verifier. By Lemma~\ref{lem:poe} under the Adaptive Root Assumption, the probability that a polynomial time adversary can break the soundness of $\textsf{PoE}$, \emph{i.e.}, convince a verifier on an instance $(\gr{C}_R,\gr{C}/\gr{C}_{L},q^{d'+1})\not\in\mathcal{R}_{\textsf{PoE}}$, is negligible. 
Consequently, the probability that the adversary can break $\textsf{PoE}$ on \emph{any} of the polynomial number of executions of $\mathsf{PoE}$ is still negligible.

This means that with overwhelming probability all transcripts are equivalent to having the verifier directly check $(\gr{C}_R,\gr{C}/\gr{C}_{L},q^{d'+1})\in\mathcal{R}_{\textsf{PoE}}$. By assumption, the witness-candidate $f(X)$ that $E'$ outputs is a valid witness if the transcript $\mathsf{tr}'$ that $E'$ also outputs is accepting. The addition of honest $\mathsf{PoE}$ transcripts to $\mathsf{tr}'$ preserves the transcript's validity. So $\mathsf{tr}$ is an accepting transcript for $\pro{Eval}$ if and only if $\mathsf{tr}'$ is an accepting transcript for $\pro{Eval}'$. Therefore, $E'$ outputs a valid witness $f(X)$ whenever $E$ outputs a valid witness. This suffices to show that $\pro{Eval}$ has witness-extended emulation if $\pro{Eval}'$ has, and if the Adaptive Root Assumption holds for $\ggen$.
\end{proof}

\paragraph{Combining statements.} The $\eval$ protocol combines two statements into one by using a random linear combination of group elements, \emph{i.e.}, $\gr{C}'\gets \gr{C}_L^{\alpha}\gr{C}_R$. We now show that this step is sound and that given the discrete logarithm for $\gr{C}'$ the extractor can extract the discrete logarithm for $\gr{C}_L$ and $\gr{C}_R$ we also show that the we can bound the size of the discrete logarithm. We show that this statement holds in two settings. First we consider a group $\GG$ were the standard Strong RSA Assumption holds and group elements are encodings of integers. 
%Move next part to after lemma?
Then we will show that in groups in which taking square roots is easy we can extract dyadic rationals using the 2-Strong RSA Assumption.
 %For $\GG \gets \ggen(\lambda)$, and $\gr{g}\sample \GG$. Let $(\gr{C}_L\in \GG,\gr{C}_R\in \GG,\alpha\in [0,p-1],f\in [0,b];\gr{g}^{f}=\gr{C}_L^{\alpha}\gr{C}_R)$ and  $(\gr{C}_L,\gr{C}_R,\alpha'\in [0,p-1],f'\in [0,b];\gr{g}^{f'}=\gr{C}_L^{\alpha'}\gr{C}_R)$  be two transcripts for $\alpha\neq \alpha'$.
\begin{lemma}[Combining for integer witnesses]
\label{lem:intrandomcombine}
	For $\GG \gets \ggen(\lambda)$, and $\gr{g}\sample \GG$. 
	Let $(z,\gr{C}_L,\gr{C}_R,y_L,y_R,\alpha,f,y)$ and  $(z,\gr{C}_L,\gr{C}_R,y_L,y_R,\alpha',f',y')$ be two transcripts such that $\gr{g}^{f}=\gr{C}_L^{\alpha}\gr{C}_R$ and $\gr{g}^{f'}=\gr{C}_L^{\alpha'}\gr{C}_R$ for group elements $\gr{C}_L,\gr{C}_R \in \GG$, and integers $\alpha,\alpha' \in  [-\frac{p-1}{2},\frac{p-1}{2}]$, $\alpha\neq \alpha'$. Further let $f,f'\in \ZZ$ be such that $f(X)\gets\dec(f)$ and $f'(X)\gets\dec(f')$ are degree $d$ bounded polynomials with coefficients bounded by $b$, \emph{i.e.}, $f(X),f'(X)\in \ZZ(b)[X]\subset \ZZ[X]$. And finally let $y=f(z)\bmod p$ and $y'=f'(z)\bmod p$.
	 Then there exists a PPT algorithm $\mathcal{X}$ that given these transcripts computes either (1) $y_L,y_R\in \ZZ_p,f_L(X),f_R(X)\in \ZZ((p-1) \cdot b)[X]$  such that $f_L(z)=y_L\bmod p$ and $f_R(z)=y_R \bmod p$ or (2) an element in $\GG$ of known order or (3) a fractional root of $\gr{g}$.
\end{lemma}
\begin{proof}
	Using the transcripts $\mathcal{X}$ computes $\Delta_\alpha\gets\alpha-\alpha'$ and $\Delta_f\gets f-f'$ such that $\gr{C}_{L}^{\Delta_\alpha}=\gr{g}^{\Delta_f}$. 
 If $\frac{\Delta_f}{\Delta_\alpha}$ is not an integer then $\mathcal{X}$ outputs a fractional root of $\gr{g}$, that is the tuple $(\Delta_f,\Delta_\alpha,\gr{C}_{L})$.  
 If $\frac{\Delta_f}{\Delta_\alpha}$ on the other hand is an integer then $\mathcal{X}$ can compute $\gr{D}\gets\gr{g}^{\frac{\Delta_f}{\Delta_\alpha}}$. Either $\gr{D}=\gr{C}_{L}$ or $(\gr{D}/\gr{C}_{L})^{\Delta_\alpha}=1$. In the second case, $\gr{D}/\gr{C}_{L}$ is an element of known order.
   
  Otherwise $\gr{D} = \gr{C}_L$ and we have $\gr{C}_{L}=\gr{g}^{f_L}$ where $f_L=\frac{\Delta_f}{\Delta_\alpha}$ is an integer.
Additionally $\gr{C}_R=\gr{g}^{f_R}$ for $f_R\gets f-\alpha \cdot f_L$.

$\mathcal{X}$ now computes the corresponding polynomials $f_L(X)\gets \dec(f_L)$ and $f_R(X)\gets \dec(f_R)$.
Now if for all $i$, the coefficients $f_i$ and $f'_i\in [-b,b]$ and $\alpha,\alpha' \in [-\frac{p-1}{2},\frac{p-1}{2}]$ then by the triangle inequality we have that for the $i$th coefficient of $f_L(X)$, $f_{L,i}\in [-2b,2b]$. Additionally we have $f_{R,i}=\frac{f_i'\alpha-f_i \alpha'}{\Delta_\alpha}$. Using the triangle inequality again we have that $f_{R,i} \in [-(p-1) \cdot b, (p-1) \cdot b]$. For an odd prime $p$, $(p-1)\cdot p\geq 2$. The bound on $f_{R,i}$ is, therefore, greater than the bound on $f_{L,i}$. This gives us $f_L(X),f_R(X)\in \ZZ({(p-1) \cdot b})[X]$

Let $y_L=\frac{y-y'}{\Delta_\alpha} \bmod p=\frac{f(z)-f'(z)}{\Delta \alpha} \bmod p$ and $y_R= y-\alpha\frac{y-y'}{\Delta_\alpha} \bmod p$. Since $f_L(X)=\frac{f(X)-f'(X)}{\Delta \alpha}$ this shows that $y_L=f_L(z)\bmod p$ and $y_R=f_R(z)\bmod p$.
%The actual bound on f_{R,i} is b*(p-2)
\end{proof}

\def\thetheorem{\ref{thm:polycommitsecurity}}
\begin{theorem}
\maintheorem
\end{theorem}
\begin{proof}
We will prove security by showing that given a polynomial time adversary $\adv_{\eval}$ that succeeds in convincing an honest verifier in the $\eval$ protocol on any public input with non-negligible probability we can either (1) construct an adaptive root adversary $\adv_{\textsf{AR}}$, (2) extract an element of known order, and hence break the Order Assumption, (3) extract a fractional root of $\gr{g}\in \GG$ or (4) extract the polynomial $f(X)\in \ZZ[X]$ such that $f(X)$ has degree at most $d$ and the coefficients of $f(X)$ are integers bounded by $q/2$, such that $f(q)$ is a unique encoding of $f(X)$, $\gr{g}^{f(q)}=\gr{C}$, and $f(z) \bmod p=y$. The proof will use the general forking lemma (Lemma \ref{lemma:GFL}) to show that the polynomial commitment scheme has witness-extended emulation.

In particular we construct an extractor $\mathcal{X}$ that given transcripts with $2$ distinct challenges per round, \emph{i.e.}, $2^{\lceil\log_2(d+1)\rceil}<2 (d+1)$ transcripts in total, can compute either an opening to the commitment scheme, an element of known order, or a fractional root of $\gr{g}\in\GG$ as encoded in the public parameters $\params$.

Using Lemma \ref{lemma:poe_security} and under the Adaptive Root Assumption, it suffices to consider an extractor $\mathcal{X}$ that works on transcripts of $\eval'$ were all $\textsf{PoE}$s prove true statements. That is $\gr{C}_L\gr{C}_R^{q^{d'+1}}=\gr{C}$ on all transcripts.

%Now consider the case where $\gr{C}_L \gr{C}_R^{(q^{d'+1})}= \gr{C}$ for all executions. 
Given a tree of $\eval'$ transcripts, the extractor $\mathcal{X}$ recursively either extracts the encoding of an integer polynomial $f(X)\in \ZZ(b)[X]\subset \ZZ[X]$ with bounded coefficients or a break of the Order Assumption or the Fractional Root Assumption. 
In order to break the Order Assumption we instantiate the adversary $\adv_{\textsf{Ord}}$ with the description of the group $\GG$. We also instantiate the fractional root adversary $\adv_{\textsf{FR}}$ with $\GG$ and $\gr{g}$ as encoded in $\params$.

Given the tree of transcripts as specified in the general forking lemma (Lemma \ref{lemma:GFL})  with branching factor $2$ at each level, \emph{i.e.}, $2$ different challenges, we will extract a witness at each node of the tree given witnesses for both nodes' children. Each level corresponds to a separate invocation to $\pro{EvalBounded}'$. We denote the input to $\pro{Eval}'$ without subscripts, \emph{i.e.}, $\gr{C},z,y,d;f(X)$, and the input to $\pro{EvalBounded}'$ with a subscript indicating the round, \emph{e.g.}, $d_0=d$, $\gr{C}_0=\gr{C}$ and $d_{\lceil \log_2(d)\rceil }=0,\gr{C}_{\lceil \log_2(d+1)\rceil }=\gr{g}^{f}$, \emph{etc}. For the witness polynomials we use superscripts and parentheses, \emph{i.e.}, $f^{(i)}(X)$ to avoid confusion with the notation for coefficients.  
%We let $\alpha$ and $\alpha'$ denote the two distinct challenges at each node of the transcript tree. We use $'$ to denote the proof elements and witnesses corresponding to the $\alpha'$ challenge, \emph{e.g.}, $\gr{C}_i'$.

In each round the extracted witness is an integer polynomial $f^{(i)}(X)\in \ZZ[X]$ such that $\gr{g}^{f^{(i)}(q)}=\gr{C}_i$ and such that the coefficients are bounded, \emph{i.e.}, all $f^{(i)}(X)$ are in $\ZZ(b_i)[X]$. The degree of $f^{(i)}(X)$ is at most $d_i$ and $f(z) \equiv y \bmod p$. Note that for odd primes $p$ and integer $z$, $f(z)\bmod p$ is always defined.

We extract starting from the leafs of the tree, \emph{i.e.}, $d_{\lceil \log_2(d)\rceil}=0$. From the transcript we can directly extract the constant integer polynomial $f(X)=f \in \ZZ$ such that $\vert f \vert \leq(p-1) (\frac{p+1}{2})^{\lceil \log_2(d+1)\rceil}$, $y=f \mod p$, $f(X)=y\in \ZZ_p[X]$ and $\gr{g}^{f}=\gr{C}$ as the witness.

We now show how to compute the witness for $i-1$ given a witnesses for $i$. 
%%%Make into lemma?
If $d_i+1$ is odd then we have $\gr{C}_{i-1}^q=\gr{C}_i$. Since $\gr{C}_{i-1}=\gr{g}^{f^{(i-1)}(q)}$ we either have that $q$ divides $f^{(i-1)}(q)$ or since $q$ is odd we have a fractional root of $\gr{g}$. 
If this is not the case then $f^{(i-1)}(q)=f^{(i)}(q)\cdot q^{-1}$ and $f^{(i)}(X)=\dec(f^{(i)}(q))$ has a zero constant term. Additionally since $y_i=y_{i-1}\cdot z$ and $f^{(i)}(z)\equiv y_i \bmod p$ we have $f^{(i-1)}(z)\equiv y_{i-1} \bmod p$, \emph{i.e.}, $f^{(i-1)}(q)$ is a valid witness and the degree of $f^{(i-1)}(X)=\dec(f^{(i-1)}(q))$ is at most $d_{i-1}=d_i-1$. 

Now if $d_i+1$ is even then we can use Lemma~\ref{lem:intrandomcombine} to either extract a fractional root of $\gr{g}$, an element of known order in $\GG$ or the two bounded polynomials $f_{L}^{(i)}(X),f_{R}^{(i)}(X)$ of degree $\frac{d_i+1}{2}-1$ and $y_L=f_{L}^{(i)}(z)\bmod p$ as well as $y_R=f_{R}^{(i)}(z)\bmod p$. 
This yields $f^{(i)}(X)=f_{L}^{(i)}(X)+X^{\frac{d_i+1}{2}} f_{R}^{(i)}(X)$ a polynomial of degree at most $d_i$ such that $\gr{C}_i=\gr{g}^{f^{(i)}(q)}$ and such that $f^{(i)}(z) \bmod p=y_L+y_R \cdot z^{\frac{d_i+1}{2}}\bmod p=  y_i$.

Note that the application of Lemma~\ref{lem:intrandomcombine} requires that $q/2$ is greater than the magnitude of each of $f^{(i)}$'s coefficient. We will show now that this is the case. 

The check on $f$ ensures that $|f|\leq b=\frac{(p-1)}{2} (\frac{p+1}{2})^{\lceil \log_2(d+1)\rceil}$. 

Lemma \ref{lem:intrandomcombine} in each invocation guarantees that the extracted parent polynomial has coefficients at most $(p-1)$ times larger than the coefficients of the children's polynomials. Given that the transcript tree has depth $\lceil \log_2(d+1)\rceil$ we get that the final extracted polynomial $f_0(X)\in \ZZ(b)[X]$ has coefficients bounded by $b= \frac{p-1}{2}(\frac{p^2-1}{2})^{\lceil \log_2(d+1)\rceil}$.

Therefore, $q$ needs to be large enough such that $f_0(X)$ is uniquely decodable, \emph{i.e.}, $q>2\cdot b=(p-1)(\frac{p^2-1}{2})^{\lceil \log_2(d+1)\rceil}$.

We can successfully extract either a witness or a fractional root or an element of known order from any tree of valid transcripts of $\eval'$.
Under the Fractional Root Assumption and the Order Assumption, the probability that a polynomial time adversary along with a polynomial time extractor $\mathcal{X}$ can produce such a fractional root or an element of known order is negligible. $\eval'$, therefore, has witness extended emulation and under the Adaptive Root Assumption by Lemma \ref{lemma:poe_security} so does $\eval$.
Lemma \ref{lem:ordertoadaptive} and Lemma \ref{lem:strongtofractional} show that we can reduce the hardness assumptions to just the Adaptive Root Assumption and the Strong RSA Assumption. \qed

\end{proof}



%%%%%DYADIC


\subsection{Proof of Theorem~\ref{thm:dyadicpolysecurity}}
\label{apx:dyadic}
We begin by stating and proving the combining lemma, (Lemma~\ref{lem:intrandomcombine}) for dyadic rational witnesses.
%PROOF FOR DYADIC RATIONALS
\begin{lemma}[Combining for Dyadic Rational Witnesses]
\label{lem:dyadiccombining}
Let $\tr$ and $\tr'$ be two transcripts as specified in Lemma \ref{lem:intrandomcombine} with the difference that $f,f'\in \mathbb{D}$ are dyadic rationals such that $f(X)\gets \dec(f)$ and $f'(X)\gets \dec(f')$ are degree $d$ bounded dyadic rational polynomials with coefficients' numerators bounded by $N$ and denominators bounded by $D$, i.e. $f(X),f'(X)\in\mathbb{D}(N,D)$.
Assume that there exists a PPT algorithm for taking square roots of any element in $\GG$ and that the order of $\GG$ is odd, then there exists a PPT algorithm $\mathcal{X}$ that given these transcripts computes either (1) $y_L,y_R\in \ZZ_p,f_L(X),f_R(X)\in \mathbb{D}(N\cdot (p-1),D\cdot (p-1))[X]$  such that $f_L(z)=y_L\bmod p$ and $f_R(z)=y_R \bmod p$ or (2) an element in $\GG$ of known order or (3) a non-power of $2$ fractional root of $\gr{g}$.
\end{lemma}
\begin{proof}
The proof follows a similar structure to the proof of Lemma~\ref{lem:intrandomcombine}. 
		
	Using the transcripts we get $\Delta_\alpha\gets\alpha-\alpha'$ and $\Delta_f\gets f-f'$ such that $\gr{C}_{L}^{\Delta_\alpha}=\gr{g}^{\Delta_f}$. 
 If $\frac{\Delta_f}{\Delta_\alpha}$ is not a dyadic rational then this gives us a non-power of $2$ fractional root of $\gr{g}$ root of $\gr{g}$, that is the tuple $(\Delta_f,\Delta_\alpha,\gr{C}_{L})$.  
 If $\frac{\Delta_f}{\Delta_\alpha}$ on the other hand is a dyadic rational then we can compute $\gr{D}\gets\gr{g}^{\frac{\Delta_f}{\Delta_\alpha}}$. This may requires taking a power of $2$ root. By assumption the group order is odd, so every element has a square root and there exists an efficient algorithm for taking square roots. This implies that taking higher power of $2$ roots is also efficient.
  
  Now either $\gr{D} = \gr{C}_L=\gr{g}^{f_L}$ or we can extract an element of known order. Additionally $\gr{C}_R=\gr{g}^{f_R}$ for $f_R\gets f-\alpha \cdot f_L$.

 We now compute the corresponding polynomials $f_L(X)\gets \dec(f_L)$ and $f_R(X)\gets \dec(f_R)$.
Now if the coefficients $f_i$ and $f'_i\in \mathbb{D}(N,D)$ and $\alpha,\alpha' \in [-\frac{p-1}{2},\frac{p-1}{2}]$ then by the triangle inequality we have that for the numerator of the $i$th coefficient of $f_L(X)$ is between $[-2N,2N]$. The denominator grows by at most $p-1$. The bound on the denominators is therefore $D\cdot (p-1)$. Additionally we have $f_{R,i}=\frac{f_i'\alpha-f_i \alpha'}{\Delta_\alpha}$. Using the triangle inequality again we have that the numerator of $f_{R,i} \in [-(p-1) \cdot N, (p-1) \cdot N]$. The denominator is bounded by $D\cdot (p-1)$. This gives us $f_L(X),f_R(X)\in \mathbb{D}((p-1) \cdot N,D\cdot (p-1))[X]$

Finally $y_L=f_L(z)$ and $y_R=f_R(z)$ as in Lemma~\ref{lem:intrandomcombine}. It is important that $2$ is co-prime with the odd prime $p$ such that each dyadic rational can be mapped to a field element.
\end{proof}


We now restate the theorem for the security of the protocol with dyadic rational witnesses in groups where taking square roots is easy.

\def\thetheorem{\ref{thm:dyadicpolysecurity}}
\begin{theorem}
\dyadicmaintheorem
\end{theorem}

\begin{proof}
The proof largely follows the same structure of the proof of Theorem~\ref{thm:polycommitsecurity}.
	
We will prove security by showing that we can extract a dyadic rational polynomial $f(X)\in \mathbb{Z}[X]$ such that $f(X)$ has degree at most $d$ and the coefficients of $f(X)$ are dyadic rationals such that the product of the numerator and denominator is bounded by $q/2$. This ensures that $f(q)$ is a unique encoding of $f(X)$. Additionally $\gr{g}^{f(q)}=\gr{C}$ and $f(z) \bmod p=y$. The proof will use the general forking lemma (Lemma \ref{lemma:GFL}) to show that the polynomial commitment scheme has witness-extended emulation. 

We use the same extractor $\mathcal{X}$ as in Theorem \ref{thm:polycommitsecurity} with one key distinction. 
For $d+1$ odd we invoke the extractor described by Lemma~\ref{lem:dyadiccombining} instead of Lemma~\ref{lem:intrandomcombine}. 
This means that at every tree level either bounded dyadic rational witness polynomials are extracted or an element of known order or a non-power of $2$ fractional root of $\gr{g}$. By assumption the ladder two cases happen only with negligible probability.

We, therefore, now need to compute a bound on the size of the extracted polynomial. 
The check on $f$ ensures that $f\in \ZZ$ and that $|f|\leq b=\frac{p-1}{2} (\frac{p+1}{2})^{\lceil \log_2(d+1)\rceil}$. We can write $f\in\mathbb{D}(\frac{p-1}{2} (\frac{p+1}{2})^{\lceil \log_2(d+1)\rceil},1)$. By \ref{lem:intrandomcombine} both the numerator and the denominator grow by at most a factor $p-1$ in every round.
Given that the transcript tree has depth $\lceil \log_2(d+1)\rceil$ we get that the final extracted polynomial $f_0(X)\in \mathbb{D}(N,D)[X]$ has coefficients with numerators bounded by $N= \frac{p-1}{2}(\frac{p^2-1}{2})^{\lceil \log_2(d+1)\rceil}$ and denominators bounded by $D=(p-1)^{\lceil \log_2(d+1)\rceil}$

$q$ needs to be large enough such that $f_0(X)$ is uniquely decodable, \emph{i.e.}, $q>2\cdot N\cdot b=(p-1)^{\lceil \log_2(d+1)\rceil+1}(\frac{p^2-1}{2})^{\lceil \log_2(d+1)\rceil}$. In a simpler form $q>p^{3\lceil\log_2(d+1)\rceil+1}$ suffices.

This shows that we can successfully extract either a witness or a non-power of $2$ fractional root or an element of known order from any tree of valid transcripts.
Under the $2$-Fractional Root Assumption and the Order Assumption, the probability that a polynomial time adversary along with a polynomial time extractor $\mathcal{X}$ can produce such a a  non-power of $2$ fractional root or an element of known order is negligible. $\eval'$, therefore, has witness extended emulation and under the Adaptive Root Assumption by Lemma \ref{lemma:poe_security} so does $\eval$.
Lemma \ref{lem:ordertoadaptive} and Lemma \ref{lem:strongtofractional} show that we can reduce the hardness assumptions to just the Adaptive Root Assumption and the  $2$-Strong RSA Assumption. 
\end{proof}

%\section{Hiding Polynomial Commitment with HVZK Eval} 
%\label{appendix:zeroknowlege}
%This section sketches how to make the polynomial commitment scheme zero knowledge. 

\paragraph{Hiding and Zero-knowledge.} Turning the polynomial commitment into a hiding scheme is easy by replacing $g^{\bar{f}(q)}$ with a Pedersen commitment to $\bar{f}(q)$ over $\GG$. The setup parameters must contain two independent generators $g, h$ for which the discrete logarithm between $g$ and $h$ is unknown. (Previously, in the non-hiding scheme, any generator would suffice). A commitment to $f \in \FF_p$ is then $g^{\bar{f}(q)} h^r$ for random $r \sample [0, 2^\lambda)$. If a trusted (or MPC) setup is acceptable then $g, h$ can be chosen such that $\langle g \rangle = \langle h \rangle$ generate the same subgroup (e.g. set $h = g^\alpha$ for large random $\alpha$). In this case the commitment is statistically hiding~\cite{AC:DamFuj02}. 

The setup can be made publicly verifiable using a hash function onto the group $\GG$, i.e. a collision-resistant function $H: \{0,1\}^* \rightarrow \GG$ that behaves as a random oracle. In this case the commitment is computationally hiding under a \emph{subgroup indistinguishability assumption}, as formulated by Brakerski and Goldwasser~\cite{C:BraGol10}. A subgroup $\GG' \subseteq \GG$ is computationally indistinguishable from $\GG$ if $g' \sample \GG'$ is computationally indistinguishable from $g \sample \GG$. A sufficient assumption that would imply Pedersen commitments with the random oracle setup are computationally hiding is that for $h \sample \GG$, the subgroup $\langle h \rangle$ is indistinguishable from $\GG$. This basic subgroup indistinguishability assumption holds in generic groups of unknown order. 

In class groups we need to be careful applying this subgroup indistinguishability assumption because Gauss's square root algorithm may be used to test quadratic residuosity efficiently, i.e. it is possible to distinguish the subgroup of quadratic residues from non-quadratic residues. However, the class group can be constructed such that the order is guaranteed to be odd~\cite{PKC/BucHam01}, in which case every element has a $2^n$ root for any $n \geq 0$.% For class groups, we assume the subgroup indistinguishability assumption applies in $NQR$, the subgroup of non-quadratic residues. It is easy to convert the deterministic $H$ into a PPT hash function onto $NQR$ by iterating until hitting the first quadratic non-residue (succeeding in an expected constant number of iterations). 

It is also possible to transform the interactive $\eval$ scheme into one with HVZK. This transformation impacts performance as it requires sending several extra group elements per level of recursion resulting in a multiplicative increase in proof size. It also requires a stronger security assumption due to a reliance on a \emph{proof of knowledge of exponent} (PoKE) by Boneh et. al.~\cite{C:BonBunFis19}. The PoKE protocol has been proven secure in the generic group model, but does not reduce to any concrete falsifiable assumption.

\paragraph{Commit} Let $g_1 \sample \GG$ be a random base distinct from $g$. 
The hiding polynomial commitment is $C \leftarrow g^{f(q)}g_1^r$ for $r \sample [-2^\lambda, 2^\lambda]$. 

\paragraph{Open} The opening of the entire polynomial is the same, but additionally gives the blinding factor $r$. 

\paragraph{Eval}

\begin{itemize}
\item In each recursive step we commit to polynomials $f_0$ and $f_1$ using the same hiding commitment scheme, where $f_0 + f_1 q^{d/2} = f$ as integer polynomials. 

\item Note that if $C_0 = g^{f_0(q)}g_1^{r_0}$ and $C_1 = g^{f_1(q)} g_1^{r_1}$ then $C_0 \cdot C_1^{q^{d/2}} = C \cdot g_1^{r'}$ where $r' = r_0 + q^{d/2} r_1$. The prover can give a non-interactive zk proof of this relation to the verifier using a sigma protocol. E.g., the prover provides $C_1' = C_1^{q^{d/2}}$ with a PoE, and then a zk-PoKE of $r'$ such that $g_1^{r'} = C_0 C_1' / C$. 

\item We could then recurse on $C_0^\alpha C_1$ which commits to $\alpha f_0 + f_1$ with the blinding factor $\alpha r_0 + r_1$. BUT we are not done yet, see next bullet point... 

\item The remaining problem is that the evaluation protocol opens $y_0 = f_0(z) \bmod p$ and $y_1 = f_1(z) \bmod p$, which is not zero knowledge. We need $y_0, y_1$ to be independently distributed subject to the constraint $y_0 + z^{d/2} y_1 = y \bmod p$, which the verifier checks. 

A solution is to modify $f_0$ and $f_1$ by adding constant terms $\alpha, \beta$ to each that cancel, i.e. $\alpha + z^{d/2} \beta = 0 \bmod p$, where $\alpha$ is uniformly distributed in $\ZZ_p$. This way the polynomials $f_0' = f_0 + \alpha$ and $f_1' = f_1 + \beta$ satisfy the relation $f_0'(z) + z^{d/2}f_1'(z) = f(z) \bmod p$. We end up revealing $y_0' = y_0 + \alpha \bmod p$ and $y_1' = y_1 + \beta \bmod p$, which is uniformly distributed in $\ZZ_p$ subject to $y_0' + y_1' = y \bmod p$. 

Finally, the prover needs to convince the verifier that it modified the $C_0$ and $C_1$ commitments appropriately. (It could not simply choose $f_0'$ and $f_1'$ in the first step because $f_0' + q^{d/2} f_1' \neq f$ as integer polynomials). 

However, the solution is still simple. The prover creates hiding commitments $C_\alpha$ to $\alpha$ and $C_\beta$ to $\beta$ and provides a zero-knowledge proof that $C_\alpha C_\beta ^{z^{d/2}}$ is a commitment to an integer multiple of $p$. This can be done efficiently through a combination of PoE and a PoKE. (Given $g^a$, to prove that $a = 0 \bmod p$ it suffices to provide $Q$ such that $Q^p = g^a$ and a PoKE for $Q$ base g. This can be made zero knowledge w/ the standard tricks). 

The protocol then proceeds on modified commitments $C_0' = C_0 C_\alpha$ and $C_1' = C_1 C_\beta$.

\end{itemize}



\subsection{Proof of Theorem~\ref{thm:joined}} \label{appendix:joined}

\def\thetheorem{\ref{thm:joined}}
\begin{theorem}
\theoremjoined
\end{theorem}

\begin{proof}
    Correctness is immediate and witness extended emulation requires a single application of Lemma~\ref{lem:intrandomcombine} which leads to an updated bound on $q$. In general $q$ needs to be $\frac{p^2-1}{2}$ larger for any random linear combination that is taken with $\alpha \in [-\frac{p-1}{2},\frac{p-1}{2}]$. For class groups, Lemma~\ref{lem:dyadiccombining} is used and the lower bound on $q$ grows by a factor of $(p-1)^2\frac{p+1}{2}\leq p^3$.
\end{proof}



\section{Proof of Theorem~\ref{thm:IOPcompiler} (Polynomial IOP Compilation)}\label{sec:IOPcompilerproof}


\def\thetheorem{\ref{thm:IOPcompiler}}
\begin{theorem}
\theoremIOPcompiler
\end{theorem}

The fact that the compilation preserves HVZK is straightforward. We prove this part first and then move on to proving witness-extended emulation. 

\paragraph{HVZK}
\begin{proof} Let $S_\eval$ denote the HVZK simulator for $\eval$ and $S_\pro{IOP}$ denote the HVZK simulator for the original polynomial IOP. We construct an HVZK simulator $S$ for the compiled interactive argument as follows. 
$S$ begins by running $S_\pro{IOP}$ on the input $x$, which produces a series of query/response pairs to arbitrarily labeled oracles that are ``sent" from the IOP prover to the verifier. $S$ simulates the view of the honest verifier in the compiled interactive proof by replacing each distinctly labeled oracle with a fresh $\Gamma$ commitment to $0$, \emph{i.e.}, the zero polynomial over $\FF_p$. By the hiding property of $\Gamma$ this has negligible distance $\delta_0$ from the commitment sent in the real protocol.
(It places this commitment at the location in the transcript where the commitment to this oracle would be sent in the compiled protocol).
 For each query/response pair $(z, y)$ to an oracle, $S$ runs $S_\eval$ to simulate the view of an honest-verifier in the $\eval$ protocol opening a hiding polynomial commitment to the value $y$ at the point $z$. Let $P$ denote an upper bound on the total number of oracles sent and $Q$ denote an upper bound on the total number of queries to IOP oracles. 
If the simulation of $S_\pro{IOP}$ has statistical distance $\delta_1$ from the real IOP verifier's view, and each simulated $\eval$ subprotocol has statistical distance $\delta_2$ to the real $\eval$ verifier's view, then the output of $S$ has statistical distance at most $P \delta_0 + \delta_1 + Q \delta_2$ from $\textsf{View}_{\langle P(x, w), V(x) \rangle}$. For $P, Q < \poly$ and $\delta_0, \delta_1, \delta_2 < \negl$ this statistical distance is negligible in $\lambda$. 
\end{proof}

\paragraph{Witness-extended emulation (knowledge)}

\begin{proof}
Without loss of generality, assume the original IOP makes at least one query to each oracle sent. An oracle which is never queried can be omitted from the IOP.

We denote by $\verifier$ the IP verifier for the compiled IP, and $\verifier_{O}$ the verifier for the original IOP. 
Given a record oracle $\pro{Record}(P^*, \params, x, \st)$ for an IP prover $P^*$ that produces accepting transcripts with non-negligible probability, we build an emulator $E$ for the compiled IP. $E$ begins by constructing an IOP adversary $P'_{O}$, which succeeds also with non-negligible probability on input $x$. Every successful interaction of $P'_{O}$ with $\verifier_{O}$ on input $x$ corresponds to a successful transcript of $P^*$ with $V$ on $x$. In showing how $E$ builds $P'_{O}$ we also show how $E$ can obtain this corresponding transcript. $E$ will make use of the emulator $E_\eval$ for the commitment scheme $\Gamma$. %We will describe how this is done in the later parts of the proof. Accepting that this is true. 

Finally, $E$ can use the IOP knowledge extractor $E_{\pro{IOP}}^{P'_O}(x)$ in order to output a witness for $x$ along with the corresponding transcript. 

\paragraph{Constructing $P'_O$ (IOP adversary)}
$P'_O$ runs as follows on initial state $\st_0$ and input $x$. It internally simulates the interaction of $P^*$ and $V$, using the record oracle $\pro{Record}(P^*, \params, x, \st)$. It begins by running this for the first round on state $\st_0$. For every message that $P^*$ sends in this first round, $P'_O$ continues simulation until there is an $\eval$ on this commitment. (There is guaranteed to be at least one $\eval$ on each commitment, independent of the randomness). Therefore, denoting by $E_\eval$ the extractor for the $\eval$ subprotocol between $P^*$ and $\verifier$ on a given commitment and evaluation point, the record oracle can be used to simulate $E_\eval$'s record oracle.% for each $m$ evaluated at some point determined by $V$'s challenge.

For each message $m$ that $P^*$ sends to $V$ at the beginning of the first round, $P'_O$ interprets $m$ as a commitment, and attempts to extract from it a polynomial by 
running the PPT emulator $E_\eval$, simulating its record oracle as just described. \textbf{If it fails in any extraction attempt it aborts.} 

If $P'_O$ succeeds in all these extractions, then it uses these extracted polynomials as its first round proof oracles that it gives to $\verifier_O$. Upon receiving the first public-coin challenge from the IOP verifier, $P'$ uses the query function to derive the corresponding queries to each of these proof oracles. Before answering, it rewinds $P^*$ and $\verifier$ back to the point immediately after $P$ sent its first messages, and now substitutes random challenge from $\verifier_O$ in order to simulate $P^*$ and $V$ on these same queries. It checks that $P^*$'s answers are consistent with the answers it can compute on its own from the extracted polynomials. \textbf{If any answers are inconsistent, $P'_O$ aborts}. Otherwise, it sends the answers to $\verifier_O$. 

At the end of this first round (assuming $P'$ has not yet aborted), $P'_O$ has stored an updated state $\st'$ for $P^*$ based on this simulation. It proceeds to the next round and repeats the same process, using the record oracle $\pro{Record}(P^*, \params, x, \st')$. Finally, if $P'$ makes it through all rounds without aborting, then it has a final state $\st_V$ for $\verifier_O$ based on its internal simulation of $P^*$ and $V$ up through the end of the last round. Finally, $\verifier_O(\st_V)$ outputs $\pro{Accept}$ or $\pro{Reject}$. %(Observe that $\verifier_O$ accepts if and only if $\verifier$ would accept in the simulated transcript with $P^*$ because they run the same decision algorithm on the final state of query/response pairs). % $1$ on $\st_V$. (This is due to the fact that the verifier in the compiled IOP runs the same final decision algorithm as the IOP verifier). 

\paragraph{Analysis of $P'_O$ success probability} 
We claim that if $\pro{Record}(P^*, \params, x, \st_0)$ outputs an accepting transcript $\tr$ with non-negligible probability, then $P'_O$ succeeds with non-negligible probability. 

Observe that for any accepting $\tr$ between $P^*$ and $V$, if $P'_O$ happens to follow the same exact sequence of query/responses without ever aborting then it succeeds because $\verifier_O$ and $\verifier$ run the same decision algorithm on the final state of query/response pairs. Thus, it remains only to take a closer look at what events cause $P'_O$ to abort, and bound the fraction of accepting $\tr$ for which this occurs. 

As indicated in bold above, there are two kinds of events that cause $P'_O$ to abort: 
\begin{itemize}
\item It fails to extract from a ``commitment" message $m$ sent by $P^*$
\item After successfully extracting a polynomial $f$ from a commitment, $P^*$ answer queries to $f$ in a way that is inconsistent with $f$. 
\end{itemize}

The second type of event contradicts the evaluation binding property of $\Gamma$, therefore it occurs with negligible probability. 

To analyze the first type of event, let us define ``bad commitments" for a parameter $D$. We define this as a property of a message $m$ (purportedly a commitment) sent in a transcript state $\st$.

\paragraph{Bounding probability of commitment extraction failure} 
The pair $(m, \st)$ is a ``bad commitment" if there is less than a $1/D$ probability that extending the transcript between $P^*$ and $\verifier$, starting from state $\st$, will contain a successful execution of $\eval$ on $m$. This probability is over the randomness of the public-coins of $\verifier$ in the extended transcript. %have a succesful execution on (over the randomness of the public-coins) that $\pro{Record}(P^*, \params, x, \st)$ contains a successful execution of $\eval$ on $m$ on the queries defined by $\st$ and the next public coin challenge, where $\st$ is determined by running $\tr$ up until the point $m$ appears. 


Let $A(\tr)$ denote the event that a transcript $\tr$ sampled from $\pro{Record}(P^*, \params, x, \st_0)$ is accepting. Let $B(\tr)$ denote the event that $\tr$ contains a ``bad commitment" (i.e. some message $m$ sent in state $\st$ such that $\pro{Bad}(m, \st) = 1$). The conditional probability of event $A(\tr)$ conditioned on event $B(\tr)$ is less than $1/D$. To see this, fix $(m, \st)$ with $\pro{Bad}(m, \st) = 1$ and consider ``sampling" a random $\tr$ that contains $m$ at state $\st$. This is done by first choosing randomly from all partial transcripts that result in $(m, \st)$ via brute force, and then running the transcript normally from state $\st$ on random public-coins. No matter how $(m, \st)$ is chosen, the probability that this process produces an accepting transcript is by definition less than $1/D$. (The second part of the transcript following $(m, \st)$ contains at least one execution of $\eval$ on $m$ by hypothesis, and by the definition of $B(m, \st) = 1$ this execution is accepting with probability less than $1/D$).

Assume that $P(A(\tr)) \geq 1/\poly$. Applying Bayes' law, %letting $A(\tr)$ denote the event that $\tr$ is accepting and $P(A(\tr)) > 1/\poly$ the a-priori probability of this event, 
\[ P[B(\tr) | A(\tr)) \leq \frac{ P[A(\tr) | B(\tr)] }{ P(A(\tr)) } \leq \poly / D \enspace . \]
In other words, at least a $1 - \poly/D$ fraction of accepting transcripts do not contain ``bad commitments". %By a union bound, in a length $L$ transcript void of bad commitments, the transcript does not contain any failed $\eval$ with probability at least $L/B$. 
Furthermore, so long as a commitment $m$ is not ``bad", we can invoke the witness-emulation property of $\eval$ to say that the PPT $E_\Gamma$ emulator extracts a witness polynomial from each $m$ with overwhelming probability.


Setting $D = 2 \poly$ we get that on at least a $1/2$ fraction of accepting transcripts, $P'_O$s simulation also succeeds (i.e. successfully extracts from each prover commitment message) with probability at least $1/2$. This means that $P'_O$ has a non-negligible success probability conditioned on the event that $\tr$ is an accepting transcript. 

In conclusion, if $\tr$ is accepting with non-negligible probability, then there is a non-negligible probability that $P'_O$ succeeds. 
\end{proof}

\ifappendix

\section{Optimizations for Polynomial Commitment Scheme}
\label{subsec:optimization}
We present several ideas for optimizing the performance of the $\pro{Eval}$ protocol.

\paragraph{Precomputation.} The prover has to compute powers of $\gr{g}$ as large as $q^d$. While this can be done in linear time, this expense can be shifted to a preprocessing phase in which all elements $\gr{g}^{q^i}, i \in \{1, \ldots, d_{\it max}\}$ are computed. Since for coefficient $|f_i|\leq -\frac{p-1}{2}$ this allows the computation of $\gr{g}^{f(q)}$ in $O(\lambda d)$ group operations as opposed to $O(\lambda \log(d) d)$.
In addition to reducing the prover's workload, this optimization enables parallelizing it. The computation of the $\textsf{PoE}$ proofs can simiarly be parallelized. The prover can express each $Q$ as a power of $\gr{g}$ which enables pre-computation of powers of $\gr{g}$ and parallelism as described by Boneh~\emph{et al.}~\cite{C:BonBunFis19}.
%The elements $\gr{g}^{q^i}$ can themselves be accompanied by non-interactive $\mathsf{PoE}$s to establish their correct computation.

The pre-computation also enables the use of multi-exponentiation techniques~\cite{pippenger1980evaluation}. Boneh~\emph{et al.}~\cite{C:BonBunFis19} and Wesolowski~\cite{EC:Wesolowski19} showed how to use these techniques to reduce the complexity of the $\textsf{PoE}$ prover. The largest $\textsf{PoE}$ exponent $q^{\frac{d+1}{2}}$ has $O(\lambda d \log(d))$ bits. Multi-exponentiation can therefore reduce the prover work to $O(\lambda d)$ instead of $O(\lambda d \log(d))$.

%\paragraph{Early termination.} The protocol specifies the recursion ends when $d=0$, but the communication cost might be reduced if it terminates earlier. This reduction holds when the size of the fewer group elements $\gr{C}_L$ and $\gr{C}_R$ outweigh the size of the larger polynomial $f(X)$ instead of the constant $f$.

%\paragraph{Fiat-Shamir.} All the challenges of the verifier are public coin and as a result the protocol can be made non-interactive in the random oracle model with the Fiat-Shamir heuristic~\cite{C:FiaSha86}. This technique replaces each message of the verifier with the hash of all previous protocol messages, lifted to the appropriate domain. For the \textsf{PoE}s, it is beneficial to reuse the same $\ell$ across all \textsf{PoE}s and to compute this prime as the hash of the entire transcript after (dropping the $\ell$s and) replacing every instance of $\gr{Q}$ by its matching $\gr{C}_R^{q^{d'+1}}$ counterpart. This optimization requires that $\ell$ be transmitted as part of the proof so that the verifier can infer the $\gr{C}_R^{q^{d'+1}}$ and $\gr{C}_L$, and only after this inference can the verifier check that $\ell$ was computed correctly. The concrete benefit of this optimization is the reduced work for the verifier: previously he had to perform $\lceil\log(d+1)\rceil$ exponentiations of $q \bmod \ell$ to the power $d'+1$, whereas now he can do this task once and record the intermediate results.

\paragraph{Two group elements per round.} In each round the verifier has a value $\gr{C}$ and receives $\gr{C}_L$ and $\gr{C}_R$ such that $\gr{C}_L\gr{C}_R^{q^{d'+1}}=\gr{C}$. This is redundant. It suffices that the verifier sends $\gr{C}_R$. The verifier could now compute $\gr{C}_L\gets \gr{C} \cdot \gr{C}_R^{-q^{d'+1}}$ in principle, but this is expensive in practice. Instead, the verifier can infer $\gr{C}_R^{(q^{d'+1})}$ from the \textsf{PoE}: the prover's message is $\gr{Q}$ and from this value and from $\ell \in \ZZ$ and $r \in \ZZ$ the verifier computes $\gr{C}_R^{q^{d'+1}}\gets \gr{Q}^{\ell} \gr{C}_R^{r}$ as well as $\gr{C}_L \gets \gr{C}/\gr{C}_R^{q^{d'+1}}$. The security of $\textsf{PoE}$ does not require that $\gr{C}_R^{q^{d'+1}}$ be sent before the challenge $\ell$ as it is uniquely defined by $\gr{C}_R$ and $q^{d'+1}$.
The same optimization can be applied to the non-interactive variant of the protocol. 

Similarly the verifier can infer $y_L$ as $y_L\gets y-z^{d'+1} y_R$. This reduces the communication to two group elements per round and 1 field element. Additionally the prover sends $f$ which has roughly the size of $\log(d+1)$ field elements, which increases the total communication to roughly $2\log(d)$ elements in $\GG$ and $2\log(d)$ elements in $\ZZ_p$. 

%When the $\mathsf{PoE}$s are made non-interactive, the prover can get away with producing only two group elements instead of three. With a naïve application of the Fiat-Shamir heuristic, the $\mathsf{PoE}$ proof consists of $(\gr{C}_R, \gr{C}_R^\star, \gr{Q})$ where $\gr{Q}$ is determined by $\ell$, which in turn is determined by hashing all previous protocol messages: $\ell \gets \mathsf{H}(\cdot \Vert \gr{C}_R \Vert \gr{C}_R^\star)$. The optimization sends $(\gr{C}_R, \gr{Q}, \ell)$ instead. The verifier can infer $\gr{C}_R^\star = \gr{C}_R^{(q^{d'+1} \bmod \ell)}$ and then test $\mathsf{H}(\cdots \Vert \gr{C}_R \Vert \gr{C}_R^\star) \stackrel{?}{=} \ell$. This optimization is particularly compatible with the previous batching of $\mathsf{PoE}$s optimization, because while there is a unique $\gr{Q}$ for each round, there need only be one $\ell$ for the entire $\eval$ protocol.

\paragraph{Evaluation at multiple points}
The protocol and the security proof extend naturally to the evaluation in a vector of points $\boldsymbol{z}$ resulting in a vector of values $\boldsymbol{y}$, where both are members of $\mathbb{Z}_p^k$. The prover still sends $\gr{C}_L\in \GG$ and $\gr{C}_R\in \GG$ in each round and additionally $\boldsymbol{y}_L,\boldsymbol{y}_R \in \ZZ^k_p$. In the final round the prover only sends a single integer $f$ such that $\gr{g}^{f}=\gr{C}$ and $f \bmod p=y$.

This is significantly more efficient than independent executions of the protocol as the encoding of group elements is usually much larger than the encoding of elements in $\ZZ_p$. Using the optimization above, the marginal cost with respect to $k$ of the protocol is a single element in $\ZZ_p$. If $\lambda=\lceil\log_2(p)\rceil$ is $120$, then this means evaluating the polynomial at an additional point increases the proof size by only $15\log(d+1)$ bytes.

\paragraph{Joining $\mathsf{Eval}$s.} 
In many applications such as compiling polynomial IOPs to SNARKs (see Section~\ref{sec:polyiop}) multiple polynomial commitments need to be evaluated at the same point $z$. 
This can be done efficiently by taking a random linear combination of the polynomials and evaluating that combination at $z$. The prover simply sends the evaluations of the individual polynomials and then a single evaluation proof for the combined polynomials. The communication cost for evaluating $m$ polynomials at $1$ point is still linear in $m$ but only because the evaluation of each polynomial at the point is being sent. The size of the eval proof, however, is independent of $m$. 
Taking a random linear combination does increase the bound on $q$ slightly, as shown in Theorem~\ref{thm:joined} which is presented below.

\[
\mathcal{R_\textsf{JE}}(\params) = \left\lbrace
\langle (\gr{C}_1,\gr{C}_2, z, y_1,y_2,d), (f_1(X), f_2(X)) \rangle
: \\
\begin{array}{l} 
\gr{C}_1, \gr{C}_2 \in \GG \\
z, y_1, y_2 \in \mathbb{Z}_p \\
f_1(X), f_2(X) \in \ZZ(b) \\
(\gr{C}_1,z,y_1,d) \in \mathcal{R_\textsf{Eval}}(\params) \\
(\gr{C}_2,z,y_2,d) \in \mathcal{R_\textsf{Eval}}(\params)
\end{array}
\right\rbrace
\]





\begin{mdframed}
	$\pro{JoinedEval}(\crs, \gr{C}_1, \gr{C}_2, z, y_1, y_2, d; f_1(X),f_2(X)) :$ \pccomment{$f_1(X), f_2(X) \in \ZZ(\frac{p-1}{2})[X]$} \\
%	Statement: $f_1(z)=y_1\bmod p \wedge f_2(z)=y_2\bmod p \wedge \gr{g}^{f_1(q)}=\gr{C}_1$ and $\gr{g}^{f_2(q)}=\gr{C}_2$
Statement: $(\crs,\gr{C}_1,\gr{C}_2,z,y_1,y_2,b,d)\in \mathcal{R}_{\pro{JE}}$
			\begin{enumerate}[nolistsep]
        \item \verifier samples $\alpha \sample [-\frac{p-1}{2},\frac{p-1}{2}]$ and sends it to \prover
			\item \prover and \verifier compute $\gr{C}'\gets \gr{C}_1^{\alpha}\gr{C}_2$ and $y'\gets \alpha \cdot y_1 +y_2 \bmod p$
			\item \prover computes $f'(X)\gets \alpha f_1(X) +f_2(X)$
			\item \prover and \verifier run $\pro{EvalBounded}(\params,\gr{C}',z,y',d,\frac{p^2-1}{4};f(X))$
		    \end{enumerate}
\end{mdframed}

\begin{theorem}
\label{thm:joined}
The protocol $\pro{JoinedEval}$ is an interactive argument for the relation $\mathcal{R}_{\pro{JE}}$ and has perfect completeness and witness extended emulation if the $r$-Strong RSA and Order Assumption hold for $\ggen$ and if $q>(p-1)(\frac{p^2-1}{2})^{\lceil \log_2(d+1)\rceil+1}<p^{2\log(d+1)+3}$
\end{theorem}
\begin{proof}
    Correctness is immediate and witness extended emulation requires a single application of Lemma~\ref{lem:intrandomcombine} which leads to an updated bound on $q$. In general $q$ needs to be $\frac{p^2-1}{2}$ larger for any random linear combination that is taken with $\alpha \in [-\frac{p-1}{2},\frac{p-1}{2}]$. For class groups, Lemma~\ref{lem:dyadiccombining} is used and the lower bound on $q$ grows by a factor of $(p-1)^2\frac{p+1}{2}\leq p^3$.
\end{proof}

We can additionally combine this optimization with the previous optimization of evaluating a single polynomial at different points. This allows us to evaluate $m$ polynomials at $k$ points with very little overhead. 
The prover groups the polynomials by evaluation points and first takes linear combinations of the polynomials with the same evaluation point and computes $y_1$ to $y_k$ using the same linear combinations. Then it takes another combination of the joined polynomials. In each round of the $\eval$ protocol the prover sends $y_{L,1}$ through $y_{L,k}$, i.e. one field element per evaluation point and computes $y_{R,1}$ through $y_{R,k}$. In the final step the prover sends $f$ and the verifier can check whether the final $y$ values are all equal to $f\bmod p$.
 This enables an $\eval$ proof of $m$, degree $d$ polynomials at $k$ points using only $2\log_2(d+1)$ group elements and $(1+k)\log_2(d+1)$ field elements.
 
%to a non-falsiable  but we do advise against this option. The reason for this advice is two-fold. First, in a SNARK compilation process where the commitment scheme is used as a primitive, zero-knowledge is typically added at the end anyway and it suffices to have a primitive that does not provide zero-knowledge on it own.
%Second, the natural way to add zero-knowledge is via a Pedersen-like commitment function $f(X) \mapsto \gr{g}^{f(q)}\gr{h}^r$, where $\gr{g}, \gr{h}$ are both random group elements and where $r \sample [0; 2^\lambda]$ is a randomizer with enough entropy.
%The scheme's homomorphism translates naturally to $\eval$ protocol and the only change the protocol needs is in the last step where $\gr{g}^f\gr{h}^{r'}$ must be opened to $f \bmod p$ and \emph{should not leak any other information}.
%The natural tool to achieve this is a zk-$\mathsf{PoKE}$ (zero-knowledge proof-of-knowledge of exponent)~\cite[\S A.4]{C:BonBunFis19} but unfortunately it is only provably secure in the generic group model.
\paragraph{Evaluating the polynomial over multiple fields}
The polynomial commitment scheme is highly flexible. For example it does not specify a prime field $\ZZ_p$ or a degree $d$ in the setup. It instead commits to an integer polynomial with bounded coefficients. That integer polynomial can be evaluated modulo arbitrary primes which are exponential in the security parameter $\lambda$ as the soundness error is proportional to its inverse.
Note that $q$ also needs large enough such that the scheme is secure for the given prime $p$ and degree $d$ (see Theorem \ref{thm:polycommitsecurity}). The second condition, however, can be relaxed. A careful analysis shows that the challenges $\alpha$ just need to be sampled from an exponential space, e.g. $[-2^{\lambda},2^{\lambda}]$. So as long as $q>2^{\lambda 2\lceil \log_2(d+1)\rceil+1}$ for RSA groups or  $q>2^{\lambda 3\lceil \log_2(d+1)\rceil+1}$ for Class groups one can evaluate degree $d$ polynomial with coefficients bounded by $2^\lambda$ over any prime field.

Additionally the proof elements $\gr{C}_L$, $\gr{C}_R \in \GG$ are independent from the field over which the polynomial is evaluated. This means that it is possible to evaluate a committed polynomial $f(X) \in \ZZ(b)$ over two separate fields $\ZZ_{p}$ and $\ZZ_{p'}$ in parallel using only $2\log(d+1)$ group elements. 

%This property can be used to efficiently evaluate the polynomial modulo a large integer $m$ by choosing multiple $\lambda$ bit primes $p_1,\dots p_k$ such that $\prod_{i=1}^k p_i\geq m$ and using the Chinese Remainder Theorem to simulate the evaluation modulo $m$.



\section{Multivariate Commitment Scheme}
\label{sec:multivariate}


We can extend our polynomial commitment scheme to multivariate polynomials. The idea is simply to use higher degrees of $q$ to encode the next indeterminate. The protocol is linear in the number of variables and logarithmic in the total degree of the polynomial. For simplicity we only present a protocol for $\mu$-variate polynomials where the degree in each variable is $d$. The protocol extends naturally to different degrees per variable.

\paragraph{Encoding}
Let $q_i=q^{(d+1)^i}$ then $\hat{f}(q_1,\dots,q_\mu)\in \ZZ$ is an encoding of the multivariate polynomial $f(X_1,\dots,X_\mu)$ with maximum degree $d$. We use $\dec_{Multi}(f(q),\mu,d)$ to denote the decoding of an $\mu$-variate polynomial with degree exactly $d$ in each variable. The decoding algorithm simply uses the univariate decoding algorithm described in Section \ref{sec:encoding} to decode a univariate polynomial $\hat{h}(X)$ of degree $(d+1)^{\mu}-1$.
Then it associates each monomial of the univariate polynomial with a degree vector $(d_1,\dots,d_\mu)$ of the multivariate polynomial. The coefficient of the $i$th monomial becomes the coefficient of the $(d_1,\dots,d_\mu)$-monomial, where $(d_1,\dots,d_\mu)$ is the base-$(d+1)$ decomposition of $i$. 
\paragraph{Protocols}
 Using this encoding we can naturally derive the multivariate commitment scheme and $\eval$ protocol. The $\eval$ protocol computes the univariate polynomials $f(q_1,\dots,q_{\mu-1},X_\mu)$ and then uses the univariate eval protocol to reduce the claim from a claim about an $\mu$-variate polynomial to one about an $(\mu-1)$-variate one. At the final step the prover opens the now constant polynomial and the verifier can check the claim. For example, the protocol would reduce a bivariate (say $X$ and $Y$) cubic polynomial to a univariate one (in $Y$) in two rounds of interaction and then reduce the degree of $Y$ using another two rounds.
 
 \begin{mdframed}[userdefinedwidth=\textwidth]
\begin{minipage}{\textwidth}
	\begin{flushleft}
	$\pro{MultiSetup}(1^\secpar):$
		\begin{enumerate}[nolistsep]
			\item $ \GG \sample \ggen(\secpar)$
			\item $ \gr{g} \sample \GG$
			%\item $q \gets 2^k$ such that $q > (d+1) \cdot 2\cdot p^{\log_2(d+1)+1} $
			%\item Pick a prime $p\in \NN$ such that $\lceil\log_2(p)\rceil=\lambda$.
			%\item Pick a sufficiently large and odd $q\in \NN$ \pccomment{$q=O_\lambda(p^{\mu \cdot \log(d)})$}
			\item $\pcreturn \params = (\secpar,\GG,\gr{g})$
		\end{enumerate}
	$\pro{MultiCommit}(\params;f(X_1,\dots,X_\mu) \in \ZZ(\frac{p-1}{2})[X_1, \ldots, X_\mu]\subset \ZZ[X_1, \ldots, X_\mu]):$ 		\begin{enumerate}[nolistsep]
			\item $d\gets \deg(f)$\pccomment{For simplicity assume $f(X_1,\dots,X_n)$ has degree $d$ in each variable}
			\item $q_i\gets q^{(d+1)^{i-1}}$ for each $i\in [\mu]$
			\item $\gr{C} \gets \gr{g}^{f(q_1,\dots,q_\mu)}$
			\item $\pcreturn (\gr{C};f(X_1,\dots,X_\mu))$
		\end{enumerate}
			\end{flushleft}
\end{minipage}
\end{mdframed}
 \begin{mdframed}
\begin{minipage}{\textwidth}
			$\pro{MultiEval}(\params, \gr{C}\in \GG, \boldsymbol{z}\in \ZZ^\mu_p,y \in \ZZ_p, d,\mu,b \in \NN; f \in \ZZ_p[\vec{X}= X_1,\ldots,X_\mu], \hat{f}\in \ZZ(b)[\vec{X}]) :$
			\begin{enumerate}[nolistsep]
			\item \pcif{$\mu=1$} 
			\item \pcind[1] \prover and \verifier run $\pro{EvalB}(\params,\gr{C},z_1,y,d,||f||_\infty,z_1;f,\hat{f})$ 
			\item \pcelse
			\item \pcind[1] Let $\hat{f}(X_\mu)\gets \hat{f}(q_1,\dots,q_{\mu-1},X_\mu)$
			\item \pcind[1] Let $f(X_\mu)\gets f(z_1,\dots,z_{\mu-1},X_\mu)$

			\item \pcind[1] Let $\crs_\mu \gets \{\lambda,\GG,\gr{g},p,q_\mu\}$
			\item \pcind[1] \prover and \verifier run the univariate $\pro{EvalB}(\params_\mu,\gr{C},z_\mu,y,d,q_\mu;f(X),\hat{f}(X))$
			\item \pcind[2] \textbf{except:} when $d=0$, $f$ is not sent; instead the protocol returns its input at this point, \emph{i.e.}, $(\gr{C}',y',b')$ along with the prover's witness $(f',\hat{f}') \in \ZZ_p[X_1,\dots,X_{\mu-1}],\ZZ[X_1,\dots,X_{\mu-1}]$ (Lines~\ref{line:basestart}-\ref{line:baseend} of $\pro{EvalB}$). 
			\item \pcind[1]$\boldsymbol{z}'\gets (z_1,\dots,z_{\mu-1})\in \ZZ_p^{\mu-1}$
			\item \pcind[1]\prover and \verifier run $\pro{MultiEval}(\crs,C',\boldsymbol{z}',y',d,\mu-1,b';f')$
		    \end{enumerate}
      \end{minipage}
\end{mdframed}
 \benedikt{Fix proof}


\begin{theorem}[Multivariate Eval]
\label{thm:mvariate}
	The polynomial commitment scheme for multi-variate polynomials consisting of protocols $(\pro{MultiSetup},\pro{MultiCommit},\pro{MultiEval})$ has perfect correctness and witness extended emulation for $\mu$-variate polynomials of degree $d$ if the Adaptive Root Assumption holds for $\ggen$  and if $(d+1)^\mu=\poly$ if $\log q \geq 4(\lambda + 1 + \CSZ[\mu \cdot (d+1)]) + \EBL[\mu \cdot (d+1)] + \CorrectnessBound[\mu \cdot (d+1)] + 1$ where $\mathsf{CB},\mathsf{CSZ}$, and $\mathsf{EBL}$ are defined as in   \Cref{thm:darkisdarkss}
\end{theorem}
\quickanddirty{
\begin{proof}
Correctness directly follows from the correctness of the univariate commitment scheme. 

We can invoke \cref{lemma:poe_security} to replace all invocations of \textsf{PoE} with direct verifications. This is secure under the adaptive root assumption. 
Further we have that each invocation of $\textsf{EvalB}$ is $(d+1, \frac{3\mu}{2^{\lambda}},\com,\phi=(\phi_a,\phi_b))$-almost-special-sound by \cref{thm:darkisdarkss}.
\end{proof}



\begin{proof}
	Perfect correctness follows from the correctness of the univariate commitment scheme and the fact that the coefficients of the witness polynomial in the honest execution are less than $\frac{p-1}{2}p^{\mu \lceil\log(d+1)\rceil}<q/2$.
	
	To show witness extended emulation we show that 
	use the forking lemma (Lemma \ref{lemma:GFL})\benedikt{replace with new FL} and build a polynomial time extractor algorithm $\mathcal{X}_{\pro{MultiEval}}$ that given a binary tree of transcripts of depth $\mu \cdot\lceil\log(d+1)\rceil$, extracts a witness. Each node corresponds to a different challenge $\alpha$ as described in the forking lemma. The tree consists of at most $(d+1)^{\mu}=\poly$ transcripts. 
	Lemma~\ref{lem:poe} states that the probability that an adversary can create any accepting transcript for which the $\textsf{PoE}$ can't be replaced by a direct check is negligible under the Adaptive Root Assumption.
We can therefore invoke the lemma to replace all \textsf{PoE} executions with direct verification checks that $\gr{C}_L\gr{C}_R^{q^{d'+1}}=\gr{C}$. 
%The lemma focuses on the univariate \pro{Eval} protocol but works identically for the multivariate protocol. 

In constructing $\mathcal{X}_{\pro{MultiEval}}$ we use the extractor $\mathcal{X}_{\pro{Eval'}}$ described in the proof of Theorem~\ref{thm:darkisdarkss}. $\mathcal{X}_{\pro{Eval'}}$ computes, given a tree of transcripts for $\pro{Eval'}$ a valid witness of $\pro{Eval'}$ or a fractional root of $\gr{g}$ or an element of known order in $\GG$. We construct $\mathcal{X}_{\pro{MultiEval}}$ recursively invoking $\mathcal{X}_{\pro{Eval'}}$ once per degree $\mu$. The probability that a polynomial time adversary and a polynomial time extractor $\mathcal{X}_{\pro{Eval'}}$ can produce a fractional root or an element of known order in $\GG$ is negligible under the strong-RSA and the the adaptive root assumptions. From hence on we will consider the case where neither of these events happen.

We use the superscript $(i)$ to denote the inputs to $\pro{MultiEval}$ where $\mu=i$. 
If $\mu=1$ then the extractor $\mathcal{X}_{\pro{Eval'}}$ directly extracts $f^{(1)}(X)\in \ZZ(b)$, a univariate degree $d$ polynomial with coefficients bounded by $b=\frac{p-1}{2}p^{2 \lceil\log_2(d+1)\rceil}$ and such that $f(z)=y \bmod p$. Note that $q/2>b$ so the extraction succeeds.

For $\mu>1$, let's assume that $f^{(\mu-1)}(X_1,\dots,X_{\mu-1})\in \ZZ(b)$ is an extracted $\mu-1$ variate polynomial with degree $d$ in each variable such that $f^{(\mu-1)}(z_1,\dots,z_{\mu-1}) \bmod p=y'$.
Let $f'\gets \enc_{Multi}(f^{(\mu-1)}(X_1,\dots,X_{\mu-1})\in \ZZ$ be the encoding of $f^{(\mu-1)}(X_1,\dots,X_n)$, such that $\gr{C}^{(\mu-1)}\gets \gr{g}^{f'}$. Note that $f'$ is equivalent to an encoding of a univariate degree $(d+1)^{\mu-1}$ polynomial with the same coefficients as the multivariate polynomial. Let $g^{(\mu-1)}(X)=\dec(f')\in \ZZ(b)[X]$ be that polynomial. 
Using $g^{(\mu-1)}(X)$ as the witness the extractor $\mathcal{X}_{\pro{Eval'}}$ extracts a univariate degree $(d+1)^{\mu}$ polynomial $g^{(\mu)}(X)$ with coefficients in $\ZZ(b \cdot p^{\lceil\log(d+1)}\rceil)$. 
Let $f''\gets g^{(\mu)}(q)$ be the encoding of $g^{(\mu)}$ such that $\gr{C}^{(\mu)}=\gr{g}^{f''}$. Note that using the multivariate decoding algorithm $f''$ also encodes a $\mu$-variate degree $d$ polynomial, i.e. $f^{(\mu)}(X_1,\dots,X_\mu)\gets \dec_{Multi}(f'',\mu,d)$. The coefficient on $X_i$ in $g^{(\mu)}(X)$ is the coefficient of the monomial in $f^{(\mu)}$ with degree vector defined by the base-$(d+1)$ decomposition of $i$, i.e. $\prod_{j=1}^\mu  X_j^{\lfloor i/(d+1)^{j-1}\rfloor \bmod d+1 }$. Note that the extraction additionally guarantees that the polynomial evaluation is correct, i.e. $f(z_1,\dots,z_\mu)\bmod p=y$.

The final extracted polynomial has coefficients in $\ZZ(\frac{p-1}{2}p^{2\mu \lceil\log_2(d+1)\rceil})$. Since $q>p^{2\mu\lceil\log_2(d+1)\rceil+1}$ both the univariate and the multivariate decoding succeed and the extractor extracts a valid $\mu$-variate degree $d$ witness polynomial.
\end{proof}
}

\fi

%\ifappendix
%\section{Other Instantiations of Polynomial IOPs} \label{appendix:other_polynomial_iops}

%
\subsubsection{Spartan}
\textsf{Spartan}~\cite{Spartan} transforms an arbitrary circuit satisfaction problem into a Polynomial IOP based on an arithmetization technique developed by Blumberg~\emph{et al.} \cite{EPRINT:BTVW14}, which improved on the classical techniques of Babai, Fortnow, and Lund~\cite{BFL}. Specifically, satisfiability of a 2-fan-in arithmetic circuit on $n$ gates can be transformed into the expression: 
\begin{equation}\label{eqn:hypercubesum}
\sum_{x, y, z \in \{0,1\}^{\log n}} G(x, y, z) = 0
\end{equation} 
for a multilinear polynomial $G$ on $3 \log n$ variables over $\FF$. 
Furthermore, $G$ decomposes into the form: 
$$G(x,y,z) = A(x,y,z) F(x) + B(x, y, z) F(y) + C(x, y, z) F(y) F(z)$$
where $A, B, C,$ and $F$ are all multilinear polynomials. The polynomials $A, B, C$ are derived from the arithmetic circuit defining the relation $\mathcal{R}$ and are input-independent. $F$ is degree $1$ with $\log n$ variables and is derived from a particular $(x, w) \in \mathcal{R}$. The classical LFKN sum-check protocol %~\cite{FOCS:LFKN90} 
is applied in order to prove Expression~\ref{eqn:hypercubesum} in a $3\log n$ round Polynomial IOP, where the prover's oracle consist of $Z$ and the low-degree polynomials sent in the sumcheck. Since the extra low-degree polynomials are constant size they can be read entirely by the verifier in constant time rather than via oracle access, and hence we ignore them in the total oracle count. The verifier must also evaluate $A, B, C$ locally, which come from the multi-linear extension of the circuit. This can be done in $O(\log n )$ time for certain circuits with a succinct representation. The main result in Spartan can be summarized in our framework as follows: 

\begin{theorem}[\cite{Spartan}]
There exists a $3 \log n$ round Polynomial IOP for any NP relation $\mathcal{R}$ computed by any circuit with arithmetic complexity $n$, which makes three queries to a $\log n$-variate degree 1 polynomial oracle. %The verifier's time complexity is proportional 
\end{theorem}

Applying our multivariate compiler to the \textsf{Spartan} Polynomial IOP we obtain an $O(\log n)$-round public-coin interactive argument of knowledge for circuits size $n$, where the verifier's work is dependent on the succinctness of the circuit representation (\emph{i.e.}, the complexity of evaluating the multilinear extension of the circuit). In our multivariate scheme (Section~\ref{sec:multivariate}), the $\log n$-variate degree 1 polynomial is tranformed into a univariate polynomial of degree $n$. With only three queries overall, the communication is just $6 \log n$ group elements and $6 \log n$ field elements. 

\subsubsection{Quadratic Arithmetic Programs} 

Quadratic Arithmetic Programs (QAPs) can be expressed as linear PCPs~\cite{TCC:BCIOP13,C:BCGTV13}. We review here how to express QAPs as a one round public-coin $(1, n)$ algebraic IOP. (This captures the satisfiability of any circuit with multiplicative complexity $n$, which is first translated to a system of quadratic equations over degree $n$ polynomials.) Each linear query is computed by a vector of degree $n$ univariate polynomials evaluated at a random point chosen by the public-coin verifier.

% (Alan:) What's described here is QAP. R1CS is described in terms of vectors and sometimes matrices but never polynomials and divisibility.
%For illustration, will use the language \emph{satisfiability of rank-1 quadratic equations} over $\FF$ as described by Ben-Sasson~\emph{et al.}~\cite[\S E.1]{C:BCGTV13}. An instance of this language 
For illustration, we will use the description of the QAP language due to Ben-Sasson~\emph{et al.}~\cite[\S E.1]{C:BCGTV13}.
This language is defined by length $m+1$ polynomial vectors $A(X)$, $B(X)$, $C(X) \in (\mathbb{F}[X])^{m+1}$ such that the $i$th components $A_i(X)$, $B_i(X)$, $C_i(X)$ are all degree-$(n-1)$ polynomials over $\FF_p[X]$ for $i \in [0,m-1]$, and $A_m = B_m = C_m$ is the degree-$n$ polynomial $Z(X)$ that vanishes on a specified set of $n$ distinct points in $\FF_p$.
There is a length-$(m-1)$ witness vector $\mathbf{w}$ whose first $\ell$ components are equal to the instance $\mathbf{x} \in \FF^\ell$, and a degree-$n$ ``quotient" polynomial $H(X)$, such that the following constraint equation is satisfied: 
\begin{equation} \label{eqn:R1CS} 
\begin{split}
[(1, \mathbf{w}^\top, \delta_1) A(X)] \cdot [(1, \mathbf{w}^\top, \delta_2) B(X)] 
- (1, \mathbf{w}^\top, \delta_3) C(X) = H(X) \cdot Z(X) \\ 
\ \mathrm{and} \ (1,\mathbf{w}^\top) (1,X,...,X^{\ell}, \mathbf{0}^{m- \ell -1}) = (1,\mathbf{x}^\top) (1, X,...,X^{\ell})
\end{split} 
\end{equation} 
%\alan{What do $\delta_0, \delta_1, \delta_2$ do? Also, I'm not sure the dimensions work out.}

The deltas $\delta_1, \delta_2, \delta_3 \in \FF$ are used as randomizers for zero-knowledge. 

\paragraph{QAP algebraic linear PCP} Equation~\ref{eqn:R1CS} is turned into a set of linear queries by evaluating the polynomials at a random point in $\FF$. Satisfaction in this random point implies satisfaction of the polynomial equation with error at most $2n / |\FF|$ by the Schwartz-Zippel lemma. Translated to an algebraic IOP, the prover sends a proof oracle $\proofO_w$ containing the vector $(1, \mathbf{w}, \delta_1, \delta_2, \delta_3)$ as well as a proof oracle $\proofO_h$ containing the coefficient vector of $H(X)$. A common proof oracle $\proofO_z$ is jointly established containing the coefficient vector of $Z(X)$. 

Let $\alpha \in \FF$ be a random point. The verifier makes four queries to $\proofO_w$, computed by the polynomial vectors $A(X), B(X), C(X)$ and $D(X) = (1, X,...,X^\ell, \mathbf{0}^{m- \ell -1})^\top$, evaluated in $\alpha$. The verifier makes one query each to $\proofO_h$ and $\proofO_z$, which is the evaluation of $H(\alpha)$ and $Z(\alpha)$ respectively. The verifier obtains query responses $y_a, y_b, y_c, y_d, y_h, y_z$ and checks that $y_a \cdot y_b - y_c = y_h y_z$ and $y_d = \langle (1, \mathbf{w}^\top), D(\alpha) \rangle$. 

\paragraph{QAP polynomial IOP} 

Following the compilation in Theorem~\ref{thm:algebraicIOPcompiler} (Section~\ref{sec:algebraicIOP}), the QAP algebraic linear PCP can be transformed into a $2$-round Polynomial IOP. For simplicity, assume $m+3 < n$, where $m-1$ is the length of the witness and $n$ is the multiplicative complexity of the circuit. The preprocessing establishes three bivariate degree-$n$ polynomials (\emph{i.e.}, encoding $A(X), B(X), C(X)$) and two univariate degree-$n$ polynomials (\emph{i.e.}, encoding $Z(X)$ and $D(X)$). In the 2-round online phase the prover sends a degree-$n$ univariate oracle for the witness vector $(1, \mathbf{w}, \delta_1, \delta_2, \delta_3)$, a degree-$n$ univariate oracle for $H(X)$, four degree-$n$ univariate oracles encoding linear PCP queries, four degree-$2n$ univariate oracles encoding polynomial products, and eight degree-$2n$ univariate oracles for opening inner products. The total number of polynomial oracle evaluation queries is $3$ bivariate degree-$n$, $8$ univariate degree-$2n$, and $7$ univariate degree-$n$. 

\begin{theorem}[QAP Polynomial IOP]
There exists a $2$-round Polynomial IOP with preprocessing for any NP relation $\mathcal{R}$ (with multiplicative complexity $n$) that makes $7$ queries to univariate degree-$n$ oracles, $8$ queries to univariate degree-$2n$ oracles, and $3$ queries to bivariate degree-$n$ oracles.  
\end{theorem}
 
While theoretically intriguing, compiling the QAP-based IOP with our polynomial commitments of Section~\ref{sec:protocol} is less practical than compiling the \textsf{Sonic} IOP. While the QAP Polynomial IOP has only $15$ univariate queries (compared to \pro{Sonic}'s $39$ queries to polynomials of twice the degree), the $3$ bivariate polynomial oracles take quadratic time to preprocess and open. Unfortunately, our polynomial commitment scheme does not take advantage of the sparsity of these bivariate polynomials. Furthermore, ignoring prover time complexity, the size of the bivariate $\eval$ proofs are twice as large as univariate $\eval$ proofs.% so the number of queries is effectively equivalent to $21$ univariate degree-$n$ queries. 
%\fi


