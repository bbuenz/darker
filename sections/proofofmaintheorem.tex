\begin{proof}
We will prove security by showing that given a polynomial time adversary $\adv_{\eval}$ that succeeds in convincing an honest verifier in the $\eval$ protocol on any public input with non-negligible probability we can either (1) construct an adaptive root adversary $\adv_{\textsf{AR}}$, (2) extract an element of known order, and hence break the Order Assumption, (3) extract a fractional root of $\gr{g}\in \GG$ or (4) extract the polynomial $f(X)\in \ZZ[X]$ such that $f(X)$ has degree at most $d$ and the coefficients of $f(X)$ are integers bounded by $q/2$, such that $f(q)$ is a unique encoding of $f(X)$, $\gr{g}^{f(q)}=\gr{C}$, and $f(z) \bmod p=y$. The proof will use the general forking lemma (Lemma \ref{lemma:GFL}) to show that the polynomial commitment scheme has witness-extended emulation.

In particular we construct an extractor $\mathcal{X}$ that given transcripts with $2$ distinct challenges per round, \emph{i.e.}, $2^{\lceil\log_2(d+1)\rceil}<2 (d+1)$ transcripts in total, can compute either an opening to the commitment scheme, an element of known order, or a fractional root of $\gr{g}\in\GG$ as encoded in the public parameters $\params$.

Using Lemma \ref{lemma:poe_security} and under the Adaptive Root Assumption, it suffices to consider an extractor $\mathcal{X}$ that works on transcripts of $\eval'$ were all $\textsf{PoE}$s prove true statements. That is $\gr{C}_L\gr{C}_R^{q^{d'+1}}=\gr{C}$ on all transcripts.

%Now consider the case where $\gr{C}_L \gr{C}_R^{(q^{d'+1})}= \gr{C}$ for all executions. 
Given a tree of $\eval'$ transcripts, the extractor $\mathcal{X}$ recursively either extracts the encoding of an integer polynomial $f(X)\in \ZZ(b)[X]\subset \ZZ[X]$ with bounded coefficients or a break of the Order Assumption or the Fractional Root Assumption. 
In order to break the Order Assumption we instantiate the adversary $\adv_{\textsf{Ord}}$ with the description of the group $\GG$. We also instantiate the fractional root adversary $\adv_{\textsf{FR}}$ with $\GG$ and $\gr{g}$ as encoded in $\params$.

Given the tree of transcripts as specified in the general forking lemma (Lemma \ref{lemma:GFL})  with branching factor $2$ at each level, \emph{i.e.}, $2$ different challenges, we will extract a witness at each node of the tree given witnesses for both nodes' children. Each level corresponds to a separate invocation to $\pro{EvalBounded}'$. We denote the input to $\pro{Eval}'$ without subscripts, \emph{i.e.}, $\gr{C},z,y,d;f(X)$, and the input to $\pro{EvalBounded}'$ with a subscript indicating the round, \emph{e.g.}, $d_0=d$, $\gr{C}_0=\gr{C}$ and $d_{\lceil \log_2(d)\rceil }=0,\gr{C}_{\lceil \log_2(d+1)\rceil }=\gr{g}^{f}$, \emph{etc}. For the witness polynomials we use superscripts and parentheses, \emph{i.e.}, $f^{(i)}(X)$ to avoid confusion with the notation for coefficients.  
%We let $\alpha$ and $\alpha'$ denote the two distinct challenges at each node of the transcript tree. We use $'$ to denote the proof elements and witnesses corresponding to the $\alpha'$ challenge, \emph{e.g.}, $\gr{C}_i'$.

In each round the extracted witness is an integer polynomial $f^{(i)}(X)\in \ZZ[X]$ such that $\gr{g}^{f^{(i)}(q)}=\gr{C}_i$ and such that the coefficients are bounded, \emph{i.e.}, all $f^{(i)}(X)$ are in $\ZZ(b_i)[X]$. The degree of $f^{(i)}(X)$ is at most $d_i$ and $f(z) \equiv y \bmod p$. Note that for odd primes $p$ and integer $z$, $f(z)\bmod p$ is always defined.

We extract starting from the leafs of the tree, \emph{i.e.}, $d_{\lceil \log_2(d)\rceil}=0$. From the transcript we can directly extract the constant integer polynomial $f(X)=f \in \ZZ$ such that $\vert f \vert \leq(p-1) (\frac{p+1}{2})^{\lceil \log_2(d+1)\rceil}$, $y=f \mod p$, $f(X)=y\in \ZZ_p[X]$ and $\gr{g}^{f}=\gr{C}$ as the witness.

We now show how to compute the witness for $i-1$ given a witnesses for $i$. 
%%%Make into lemma?
If $d_i+1$ is odd then we have $\gr{C}_{i-1}^q=\gr{C}_i$. Since $\gr{C}_{i-1}=\gr{g}^{f^{(i-1)}(q)}$ we either have that $q$ divides $f^{(i-1)}(q)$ or since $q$ is odd we have a fractional root of $\gr{g}$. 
If this is not the case then $f^{(i-1)}(q)=f^{(i)}(q)\cdot q^{-1}$ and $f^{(i)}(X)=\dec(f^{(i)}(q))$ has a zero constant term. Additionally since $y_i=y_{i-1}\cdot z$ and $f^{(i)}(z)\equiv y_i \bmod p$ we have $f^{(i-1)}(z)\equiv y_{i-1} \bmod p$, \emph{i.e.}, $f^{(i-1)}(q)$ is a valid witness and the degree of $f^{(i-1)}(X)=\dec(f^{(i-1)}(q))$ is at most $d_{i-1}=d_i-1$. 

Now if $d_i+1$ is even then we can use Lemma~\ref{lem:intrandomcombine} to either extract a fractional root of $\gr{g}$, an element of known order in $\GG$ or the two bounded polynomials $f_{L}^{(i)}(X),f_{R}^{(i)}(X)$ of degree $\frac{d_i+1}{2}-1$ and $y_L=f_{L}^{(i)}(z)\bmod p$ as well as $y_R=f_{R}^{(i)}(z)\bmod p$. 
This yields $f^{(i)}(X)=f_{L}^{(i)}(X)+X^{\frac{d_i+1}{2}} f_{R}^{(i)}(X)$ a polynomial of degree at most $d_i$ such that $\gr{C}_i=\gr{g}^{f^{(i)}(q)}$ and such that $f^{(i)}(z) \bmod p=y_L+y_R \cdot z^{\frac{d_i+1}{2}}\bmod p=  y_i$.

Note that the application of Lemma~\ref{lem:intrandomcombine} requires that $q/2$ is greater than the magnitude of each of $f^{(i)}$'s coefficient. We will show now that this is the case. 

The check on $f$ ensures that $|f|\leq b=\frac{(p-1)}{2} (\frac{p+1}{2})^{\lceil \log_2(d+1)\rceil}$. 

Lemma \ref{lem:intrandomcombine} in each invocation guarantees that the extracted parent polynomial has coefficients at most $(p-1)$ times larger than the coefficients of the children's polynomials. Given that the transcript tree has depth $\lceil \log_2(d+1)\rceil$ we get that the final extracted polynomial $f_0(X)\in \ZZ(b)[X]$ has coefficients bounded by $b= \frac{p-1}{2}(\frac{p^2-1}{2})^{\lceil \log_2(d+1)\rceil}$.

Therefore, $q$ needs to be large enough such that $f_0(X)$ is uniquely decodable, \emph{i.e.}, $q>2\cdot b=(p-1)(\frac{p^2-1}{2})^{\lceil \log_2(d+1)\rceil}$.

We can successfully extract either a witness or a fractional root or an element of known order from any tree of valid transcripts of $\eval'$.
Under the Fractional Root Assumption and the Order Assumption, the probability that a polynomial time adversary along with a polynomial time extractor $\mathcal{X}$ can produce such a fractional root or an element of known order is negligible. $\eval'$, therefore, has witness extended emulation and under the Adaptive Root Assumption by Lemma \ref{lemma:poe_security} so does $\eval$.
Lemma \ref{lem:ordertoadaptive} and Lemma \ref{lem:strongtofractional} show that we can reduce the hardness assumptions to just the Adaptive Root Assumption and the Strong RSA Assumption.

\end{proof}

