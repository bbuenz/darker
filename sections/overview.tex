%The key technical contribution is a polynomial commitment from groups of unknown order. A polynomial commitment is a short, ideally constant size, commitment to a polynomial. The commitment enables the prover to give a verifier an evaluation of the polynomial at a point along with a (possibly interactive) evaluation proof that convinces the verifier that the evaluation is correct. This protocol can be dropped into recent SNARK constructions\cite{Sonic,Plonk,Spartan,Libra} to achieve SNARKs without trusted setup.
This technical overview provides an informal description of our key technical contribution, a polynomial commitment scheme with logarithmic evaluation proofs and verification time.
The commitment relies on four separate tools described below.
\paragraph{Integer encoding of polynomials}
Given a univariate polynomial $f(X)\in \ZZ_p[X]$ the prover first encodes the polynomial as an integer. Interpreting the coefficients of $f(X)$ as integers in\footnote{The choice to represent the coefficients by integers in $[0,p)$ optimizes for clarity, but later on we will in fact choose a balanced set of representatives, \emph{i.e.}, $[-\frac{p-1}{2}; \frac{p-1}{2}]$.} $[0, p)$, define $\hat{f}(X)$ to be the \emph{integer} polynomial with these coefficients. The prover computes $\hat{f}(q)\in \ZZ$ for some large integer $q\geq p$. This an injective map from polynomials with bounded coefficients to integers and is also decodable: the coefficients of $f(q)$ can be recovered from the base-$q$ decomposition of $\hat{f}(q)$.
\begin{comment}
For example assume that $f(X)=2X^3+3X^2+4X+1 \in \ZZ_5[X]$ and $q=10$. Then the integer $f(q)=2341$ encodes the polynomial $f(X)$ because its coefficients appear in the $q$-ary expansion of $f(q)$.
\end{comment}
Note that this encoding is also additively homomorphic, assuming that $q$ is sufficiently large. 
\begin{comment}
For example, let $g(X)=4X^3+1X^2+3$ such that $g(10)=4103$. Then $f(10)+g(10)=6444=(g+f)(10)$. 
\end{comment}
The more homomorphic operations we want to permit, the larger $q$ needs to be.
The encoding additionally permits multiplication by polynomials (for example, $f(q)\cdot q^k$ is equal to the encoding of $f(X)\cdot X^k$). 
\begin{comment}
Or in our example $100 \cdot f(10)=234100$ which is the encoding of $2\cdot X^5+3\cdot X^4+4\cdot X^3+X^2$.
\end{comment}

\paragraph{Succint integer commitments}
The integer $x \leftarrow \hat{f}(q)$ encoding a degree $d$ polynomial $f(q)$ is between $q^d$ and $q^{d+1}$, i.e. of size $d \log_2 q$ bits. The prover commits to $x$ using a \emph{succinct} integer commitment scheme that is additively homomorphic. Specifically, we use exponentiation in a group $\GG$ of unknown order: the commitment is the single group element $\gr{g}^x$ for a base element $\gr{g} \in \GG$ specified in the setup. (Note that if the order $n$ of $\GG$ is known then this is not an integer commitment; $\gr{g}^x$ could be opened to any integer $x' \equiv x \bmod n$). 

\begin{comment} 
The binary representation of this integer consists of $d\cdot \log_2(q)$ bits, which is about as large as the description of the polynomial itself. We therefore need a succinct cryptographic commitment\footnote{For now, we consider binding-only commitments which do not hide the committed value.} of the integer that preserves the homomorphic properties of the polynomial encoding. For this purpose we use exponentiation in a group of unknown order: $\ZZ \rightarrow \mathbb{G}, x \mapsto \gr{C} = \gr{g}^x$ for some random but fixed group element $\gr{g}$. As the order is unknown in these groups, the prover cannot reduce $x\in\ZZ$ and cannot learn a different integer discrete logarithm between $\gr{g}$ and $\gr{C}$. 
The commitment is succinct as the size of group elements in $\GG$ such as $\gr{g}^x$ is just determined by a security parameter.
This commitment function is also homomorphic, \emph{i.e.}, $\gr{g}^x\cdot \gr{g}^y=\gr{g}^{x+y}$, and thus preserves the homomorphic properties of the integer encoding of polynomials.
\end{comment}

\paragraph{Evaluation protocol}
The evaluation protocol is an interactive argument to convince a verifier that $\gr{C}$ is an integer commitment to $\hat{f}(q)$ such that $f(z) = y$ at a provided point $z \in \FF_p$. The protocol must be \emph{evaluation binding}: it should be infeasible for the prover to succeed in arguing that $f(z) = y$ and $f(z) = y'$ for $y \neq y'$. The protocol should also be an \emph{argument of knowledge}, which informally means that any prover who succeeds at any point $x$ must ``know" the coefficients of the committed $f$. 

As a warmup, we first describe how a prover can efficiently convince a verifier that $\gr{C}$ is a commitment to an integer polynomial of degree $d$ with bounded coefficients. The protocol uses a recursive strategy. 
In each step we split $f(X)$ into two degree $d'=\frac{d+1}{2}-1$ polynomials $f_L(X)$ and $f_R(X)$. 
The left half $f_L(X)$ contains the first $d'+1$ coefficients of $f(X)$ and the right half $f_R(X)$ the second, such that $f(X)=f_L(X)+X^{d'+1}f_R(X)$. The prover now commits to $f_L$ and $f_R$ by computing $\gr{C}_L\gets \gr{g}^{f_L(q)}$ and $\gr{C}_R\gets \gr{g}^{f_R(q)}$.
\begin{comment}
In our running example, $f_L(X)=4X+1$ and $f_R(X)=2X+3$. 
\end{comment} 
The verifier checks the consistency of these commitments by testing $\gr{C}_L\gr{C}_R^{q^{d'+1}}=\gr{C}$. The verifier then samples random  $\alpha \in \ZZ_p$ and computes $\gr{C}'\gets \gr{C}_L^{\alpha}\gr{C}_R$, which is an integer commitment to $\alpha f_L(q) + f_R(q)$. The prover and verifier recurse on the statement that $\gr{C}'$ is a commitment to a degree $d'$ polynomial, thus halving the ``size'' of the statement. %by taking a random linear combination between $\gr{C}_L$ and $\gr{C}_R$. The verifier generates and sends a random challenge $\alpha$ and both the prover and verifier compute $\gr{C}'\gets \gr{C}_L^{\alpha}\gr{C}_R=\gr{g}^{(\alpha f_L+f_R)(q)}$. The protocol now recurses on $\gr{C}'$ with the statement that it commits to a degree $d'$ polynomial. 
After $\log_2(d+1)$ rounds, the commitment $\gr{C}'$ exchanged between prover and verifier is to a polynomial of degree $0$, \emph{i.e.}, a scalar $c \in \ZZ_p$. So $\gr{C}'$ is of the form $\gr{g}^{\hat{c}}$ where $\hat{c} \equiv c \bmod p$. 
The prover sends $\hat{c}$ to the verifier directly. 
The verifier checks that $\gr{g}^{\hat{c}} = \gr{C}'$ and also that $\hat{c} < q$.\footnote{In the full scheme, the verifier actually checks that $\hat{c} < B$ for a bound $B \leq q$ that depends on the field size $p$ and the polynomial's maximum degree $d$} 
%This demonstrates that $\gr{C}'$ $f$ is a commitment to a degree $0$ polynomial and additionally, through an extraction argument, that the original $C$ committed to an integer encoding of a degree $d$ polynomial.

Extending this protocol to also show that $f(z) = y$ at a provided point $z$ is simple. 
In each round, the prover additionally sends $y_L=f_L(z)\bmod p$ and $y_R=f_R(z)\bmod p$. The verifier checks consistency with the claim, \emph{i.e.}, that $y_L+z^{d'+1}y_R=y$, and also computes $y' \leftarrow \alpha y_L+y_R\bmod p$ to proceed to the next round. (The recursive claim is that $\gr{C}'$ commits to $f'$ such that $f'(z) = y' \bmod p$.) In the final round of recursion, the value of the constant polynomial at $z$ is the constant itself. So in addition to testing $\gr{C} = \gr{g}^{\hat{c}}$, the verifier checks that $\hat{c} \equiv y \bmod p$ and he can do this because he is given $\hat{c}$. 
%The same extraction argument shows that the evaluation claim is true, i.e. that for the extracted degree $d$ polynomial $f(X)$, $f(z)\bmod p=y$. 

\paragraph{Outsourcing exponentiation for efficiency}
The evaluation protocol requires communicating only $2$ group elements and $2$ field elements per round. However, the verifier needs to check that $\gr{C}_L\gr{C}_R^{(q^{d'+1})}=\gr{C}$, and naïvely performing the exponentiation requires $O(d \cdot \log q)$ work. To reduce this workload, we employ a recent technique for proofs of exponentiation (\textsf{PoE}) in groups of unknown order due to Wesolowski~\cite{EC:Wesolowski19} in which the verifier performs only $O(\log q + \log d)$ work. This outsourcing reduces the total verifier time to a logarithmic quantity in $d$. %While the entire protocol is interactive it is also public coin, and so with the Fiat-Shamir heuristic we can turn it into a non-interactive evaluation argument.

