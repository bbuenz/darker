%The key technical contribution is a polynomial commitment from groups of unknown order. A polynomial commitment is a short, ideally constant size, commitment to a polynomial. The commitment enables the prover to give a verifier an evaluation of the polynomial at a point along with a (possibly interactive) evaluation proof that convinces the verifier that the evaluation is correct. This protocol can be dropped into recent SNARK constructions\cite{Sonic,Plonk,Spartan,Libra} to achieve SNARKs without trusted setup.
We now informally describe our key technical contribution, a polynomial commitment scheme with logarithmic evaluation proofs and verification time.
The commitment is very simple and relies on four separate tools which are described below.
\paragraph{Integer encoding of polynomials}
Given a univariate polynomial $f(X)\in \ZZ_p[X]$ the prover first encodes the polynomial as an integer. To do this we interpret the coefficients of $f(X)$ as integers between\footnote{The choice to represent the coefficients by integers between $0$ and $p$ optimizes for clarity, but later on we will in fact choose a balanced set of representatives, \emph{i.e.}, $[-\frac{p-1}{2}; \frac{p-1}{2}]$.} $0$ and $p-1$ and compute $f(q)\in \ZZ$ for some large integer $q\geq p$. This is an injective map from polynomials with bounded coefficients to integers and can be used as a unique encoding.
For example assume that $f(X)=2X^3+3X^2+4X+1 \in \ZZ_5[X]$ and $q=10$. Then the integer $f(q)=2341$ encodes the polynomial $f(X)$ because its coefficients appear in the $q$-ary expansion of $f(q)$. This encoding is additively homomorphic, assuming that $q$ is sufficiently large. For example, let $g(X)=4X^3+1X^2+3$ such that $g(10)=4103$. Then $f(10)+g(10)=6444=(g+f)(10)$. The more homomorphic operations we want to permit, the larger $q$ needs to be.
The encoding additionally permits multiplication by polynomials and especially monomials. We can compute $f(q)\cdot q^k$ which is equal to the encoding of $f(X)\cdot X^k$, or in our example $100 \cdot f(10)=234100$ which is the encoding of $2\cdot X^5+3\cdot X^4+4\cdot X^3+X^2$.
\paragraph{Integer commitments}
Note that the integer encoding the polynomial is between $q^d$ and $q^{d+1}$, where $d$ is the degree of the polynomial. The binary representation of this integer consists of $d\cdot \log_2(q)$ bits, which is about as large as the description of the polynomial itself. We therefore need a succinct cryptographic commitment\footnote{For now, we consider binding-only commitments which do not hide the committed value.} of the integer that preserves the homomorphic properties of the polynomial encoding. For this purpose we use exponentiation in a group of unknown order: $\ZZ \rightarrow \mathbb{G}, x \mapsto \gr{C} = \gr{g}^x$ for some random but fixed group element $\gr{g}$. As the order is unknown in these groups, the prover cannot reduce $x\in\ZZ$ and cannot learn a different integer discrete logarithm between $\gr{g}$ and $\gr{C}$. 
The commitment is succinct as the size of group elements in $\GG$ such as $\gr{g}^x$ is just determined by a security parameter.
This commitment function is also homomorphic, \emph{i.e.}, $\gr{g}^x\cdot \gr{g}^y=\gr{g}^{x+y}$, and thus preserves the homomorphic properties of the integer encoding of polynomials.
\paragraph{Evaluation protocol}
We first describe how a prover can efficiently convince a verifier that $\gr{C}$ is a commitment to an integer polynomial of degree $d$ with bounded coefficients. The protocol uses a recursive strategy. 
In each step we split $f(X)$ into two degree $d'=\frac{d+1}{2}-1$ polynomials $f_L(X)$ and $f_R(X)$. 
The left half $f_L(X)$ contains the first $d'+1$ coefficients of $f(X)$ and the right half $f_R(X)$ the second, such that $f(X)=f_L(X)+X^{d'+1}f_R(X)$. The prover now commits to $f_L$ and $f_R$ by computing $\gr{C}_L\gets \gr{g}^{f_L(q)}$ and $\gr{C}_R\gets \gr{g}^{f_R(q)}$. 
In our running example, $f_L(X)=4X+1$ and $f_R(X)=2X+3$. Note that the verifier can check the consistency of these commitments by testing $\gr{G}_L\gr{C}_R^{q^{d'+1}}=\gr{C}$. Now we reduce the problem to a single, smaller statement by taking a random linear combination between $\gr{C}_L$ and $\gr{C}_R$. The verifier generates and sends a random challenge $\alpha$ and both the prover and verifier compute $\gr{C}'\gets \gr{C}_L^{\alpha}\gr{C}_R=\gr{g}^{(\alpha f_L+f_R)(q)}$. The protocol now recurses on $\gr{C}'$ with the statement that it commits to a degree $d'$ polynomial. 
After $\log_2(d+1)$ rounds $\gr{C}'$ is a commitment to a constant $f$ which the prover can send to the verifier. 
The verifier checks that $\gr{g}^f=\gr{C}'$ and also that $f < q$\footnote{In the protocol we actually check that $f<b$ for a constant $b\leq q$}.
This ensures that a) $f$ is decodable to a degree $0$ polynomial and additionally, through an extraction argument, that the original $C$ committed to an integer encoding of a degree $d$ polynomial.

This Diophantine Argument of Knowledge (DARK) can be extended to also prove correct evaluation. Suppose the prover wants to show that $f(z)=y\bmod p$ for $z,y\in \ZZ_p$, and both $z$ and $y$ are known to the verifier. In each round the prover sends $y_L=f_L(z)\bmod p$ and $y_R=f_R(z)\bmod p$. The verifier checks consistency by testing $y_L+z^{d'+1}y_R=y$ and computes $y'=\alpha y_L+y_R\bmod p=f'(z)\bmod p$ to proceed to the next round. In the final step the verifier checks that the opened constant-polynomial $f$ is equal to the final value of $y$ modulo $p$. 
The same extraction argument shows that the evaluation claim is true, i.e. that for the extracted degree $d$ polynomial $f(X)$, $f(z)\bmod p=y$. 
\paragraph{Outsourcing exponentiation}
The communication of the evaluation protocol requires only the communication of $2$ group elements and $2$ field elements per round. However, the verifier needs to check that $\gr{C}_L\gr{C}_R^{(q^{d'+1})}=\gr{C}$, which requires raising a group element to the large exponent $q^{d'+1}$. This operation takes time linear in $d$.
Fortunately, Pietrzak~\cite{ITCS:Pietrzak19b} and Wesolowski~\cite{EC:Wesolowski19} recently gave two beautiful constructions for proofs of exponentiation (\textsf{PoE}) in groups of unknown order that can be used to outsource this expensive computation. This outsourcing reduces the total verifier time to be logarithmic in $d$. While the entire protocol is interactive it is also public coin, and so with the Fiat-Shamir heuristic we can turn it into a non-interactive evaluation argument.

