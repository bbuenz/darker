For many applications it is important that the evaluation protocol offer zero-knowledge in addition to knowledge soundness. We show here how to adapt the scheme to achieve this feature.

We start by defining a computational \emph{hiding} property for commitment schemes:

\begin{definition}
A commitment scheme $\Gamma = (\pro{Setup}, \pro{Commit}, \pro{Open})$ is \defn{hiding} if for all probabilistic polynomial time adversaries $\adv = (\adv_0, \adv_1)$, the probability of distinguishing between commitments of different is negligble:
\[
	\left| 1 - 2 \cdot \mathrm{Pr}\left[
		\hat{b} = b \ \middle| \ 
		\begin{array}{l}
			\params \gets \pro{Setup}(1^\lambda) \\
			m_0, m_1, \st \gets \adv_0(\params) \\
			b \sample \{0,1\} \\
			(c; r) \gets \pro{Commit}(\params, m_b) \\
			\hat{b} \gets \adv_1(\st, c)
		\end{array}
	\right] \right| \leq \negl \enspace .
\]
\end{definition}

\paragraph{Hiding Polynomial Commitment}
We make the polynomial commitment described in Section~\ref{sec:protocol} hiding by adding a large random coefficient at degree $d+1$. Let $B\geq |\GG|$ be a publicly known upper bound on the order of $|\GG|$. We will choose the blinding coefficient between $0$ and $B\cdot 2^\lambda$. 

Formally, the hiding commitment algorithm is described as follows:
\begin{itemize}
	\item $\pro{CommitH}(f(X) \in \ZZ_p[X]) \rightarrow (\gr{C}; \hat{f}(X), d, r)$. Lift $f(X) \in \mathbb{Z}_p[X]$ to $\tilde{f}(X) \in \mathbb{Z}(\frac{p-1}{2})[X]$ and select random integer $r \sample [0,B\cdot 2^\lambda)$. Compute $d \gets \deg(f(X))$ and $\gr{C} \gets \gr{g}^{\tilde{f}(q)+q^{d+1} \cdot r}$ and return commitment $\gr{C}$ with secret opening information $\tilde{f}(X), d, r$.
\end{itemize}

The random coefficient $r$ ensures that $\gr{C}$ is nearly indistinguishable from a random group element. Let's assume that $\gr{g}$ and $\gr{g}^{q^{d+1}}$ generate the same group, e.g. because $\gr{g}$ generates a prime order group. Let $n=|\langle \gr{g}\rangle|$ be the order of that subgroup generate $\gr{g}$. $\gr{A}\gets\gr{g}^c$ for $c\sample [0,n)$ is a random subgroup element. If $r \sample [0,B\cdot 2^\lambda)$ and $B\geq n$ then $\gr{B}\gets \gr{g}^r$ has statistical distance at most $2^{-\lambda}$ from $\gr{B}$. Consequently $\gr{C}$ is statistically negligibly far from a random group element. 
 Unfortunately, without a trusted setup we can only guarantee that $\gr{g}^{q^{d+1}}$ spans a subgroup of $\langle \gr{g} \rangle$. The commitment then becomes computationally hiding under the \emph{Subgroup Indistinguishability Assumption}~\cite{C:BraGol10}. The assumption states that no efficient adversary can distinguish subgroups. If an adversary can distinguish $\gr{C}$ from a random element with probability more than $2^{-\lambda}$ then he must be able to distinguish the subgroup generated by $\gr{g}^{q^{d+1}}$ from the subgroup generated by $\gr{g}$. For simplicity we will only proof security for the case that $\gr{g}$ and $\gr{g}^{q^{d+1}}$ span the same group.
\begin{theorem}
The commitment scheme $\Gamma = (\pro{Setup}, \pro{CommitH}, \pro{Open})$ is statistically hiding if $B \gg |\mathbb{G}|$ and if $\langle \gr{g} \rangle = \langle \gr{g}^{q^{d+1}} \rangle$.
\end{theorem}

\begin{proof}
We use the fact that the uniform distributions $[0,b]$ and $[a,a+b]$ have statistical distance $\frac{a}{b}$, i.e. the probability that any algorithm can distinguish the distributions from a single sample is less than $\frac{a}{b}$. Similarly $\gr{C}\gets\gr{g}^{f(q)+rq^{d+1}}$ for $r\sample[0,B\cdot 2^{\lambda}$ has statistical distance at most $2^{-\lambda}$ from a uniform element generated by $\gr{g}$ if $B\geq |\langle \gr{g}\rangle|$. This means that two polynomial commitments will be at most $2^{-\lambda}$ distinguishable from random which means that 
\end{proof}



The next question is how to adapt the \eval protocol to provide zero-knowledge. Recall the definition of honest-verifier zero-knowledge (HVZK):

\begin{definition}[HVZK for interactive arguments]
Let $\textsf{View}_{\langle \prover(x, w), \verifier(x) \rangle}$ denote the view (list of sent and received messages) of the verifier in an interactive protocol described in Definition~\ref{def:argument} on common input $x$ and prover witness input $w$. The interactive protocol has $\delta$-statistical honest verifier zero-knowledge if there exists a probabilistic polynomial time algorithm $\mathcal{S}$ such that for every $(x, w) \in \mathcal{R}$, the distribution $\mathcal{S}(x)$ is $\delta$-close to $\textsf{View}_{\langle \prover(x, w), \verifier(x) \rangle}$ (as distributions over the randomness of $\prover$ and $\verifier$).
\end{definition}


Now using the hiding commitment scheme we can begin to define a zero-knowledge evaluation scheme
 \noindent\begin{mdframed}[userdefinedwidth=\textwidth]
\begin{minipage}{\textwidth}
	\begin{flushleft}
	\pro{ZK-Eval}\\
	Input: $\crs,\gr{C} \in \GG,z,y \in \ZZ_p b\in \ZZ,d$, Witness: $f(X) \ZZ(b),\alpha \in \ZZ(2^\lambda)$\\
	Statement: $\gr{C}=\commit(f(X);\alpha),f(X) \in \ZZ(b\cdot 2^\lambda),\deg(f(X))=d, f(z)=y \bmod p$
	\begin{enumerate}[nolistsep]
		    \item \prover samples a random degree $d$ $k(X) \sample \ZZ(b\cdot 2^\lambda)[X]$ and $\beta \sample \ZZ(2^\lambda)$ and computes $\gr{R}\gets \commit(k(X);\beta)$ and $y'\gets k(z) \bmod p$
		    \item \prover sends $\gr{R}$ and $y'$ to $\verifier$
		    \item \verifier samples random $c\in [0,2^\lambda]$ and sends it to $\prover$
		    \item \prover computes $s(X)\gets k(X) + c \cdot f(X)$, as well as $\gamma\gets \alpha+ c\cdot \gamma$. 
		    \item \prover sends $\gamma$ and to \verifier.
		    \item \prover and \verifier compute $\gr{C}'\gets \gr{R}\gr{C}^c/g^{q^{d+1}\cdot \gamma}$ and $y''\gets y'+c \cdot y \bmod p$
		    \item \prover and \verifier run $\eval(\crs,C',z,y'',d,b\cdot 2^{\lambda++1};s(X))$.
		   		\end{enumerate}
	\end{flushleft}
\end{minipage}
\end{mdframed}

The strategy to provide an evaluation proof with zero-knowledge consists of several steps. Assume that the verifier already holds a commitment $\gr{C}_f$ and the prover holds a matching polynomial $f(X)$ of degree $d$. Furthermore, both the prover and the verifier already agree on the evaluation point $z$ and the claimed value of $f(X)$ in that point, $y = f(z)$. Let $\pro{EvalZK}$ denote the following protocol.
\begin{itemize}
\item The prover creates another hiding commitment $\gr{C}_g$ to a random polynomial $g(X) \sample \mathbb{Z}(P)[X]$ of degree $d$ with coefficients drawn from a somewhat larger space than $\mathbb{Z}(p)$.
\item The prover sends the hiding commitment $\gr{C}_g$ to the verifier along with the claimed evaluation $y_g = g(z)$ and obtains a challenge $\alpha \sample \mathbb{Z}(p)$ from the verifier.
\item The prover and verifier use $\alpha$ to compute a random linear combination polynomial $h(X) = \alpha \cdot f(X) + g(X)$, a commitment to this polynomial $\gr{C}_h = \gr{C}_f^\alpha \cdot \gr{C}_g$, and an evaluation of this polynomial in $z$: $y_h = \alpha \cdot y_f + y_g \bmod p$.
\item Observe that the new commitment can be described as $\gr{C}_h = \gr{g}^{h(q)} \cdot \gr{g}^{q^{d+1} \cdot (\alpha \cdot r_f + r_g)}$ where $r_f$ and $r_g$ was the randomness used in the commitments to $f(X)$ and $g(X)$, respectively. The prover splits this commitment along that factorization and sends $\gr{C} = \gr{g}^{h(q)}$ and $\gr{C}^\star = \gr{g}^{q^{d+1} \cdot (\alpha \cdot r_f + r_g)}$ to the verifier. The verifier checks that $\gr{C} \cdot \gr{C}^\star = \gr{C}_h$.
\item The prover sends $\gr{g}^{\alpha \cdot r_f + r_g}$ and then the prover and verifier run a \textsf{PoE} to establish that $\gr{C}^\star$ really is this expression raised to the power $q^{d+1}$.
\item At this point the verifier is in possession of a non-hiding commitment $\gr{C}$ to the polynomial $h(X) = \alpha f(X) + g(X)$ of degree $d$, along with its value $y_h$ in $z$, and the prover is in possession of the polynomial $h(X)$. So they run the original \eval protocol taking into account a larger bound for the coefficients.
\end{itemize}

The intuition for zero-knowledge is that the large coefficients of $g(X)$ statistically hide the smaller coefficients of $\alpha \cdot f(X)$. As a result, the prover could in principle open $\gr{C}$ directly without leaking more than a negligible amount of information on $f(X)$. The operations performed on $f(X)$ and on $\gr{C}$ from that point onwards can only reduce the amount of information leaked, not increase it.

To make this argument formal, we need to establish that for any given $\alpha \in \mathbb{Z}(p)$, the distribution $\mathcal{D}$ of $g + \alpha \cdot f$ for $f \sample \mathbb{Z}(p)$ and $g \sample \mathbb{Z}(P)$ is statistically close to the uniform distribution $\mathcal{U}$ over $\mathbb{Z}(P)$. Observe that (if $\alpha \neq 0$) the support of $\mathcal{D}$ has cardinality $P + p - 1$. Partition the support of $\mathcal{D}$ into $\mathbb{Z}(P)$ and $\mathcal{S} = \mathsf{Supp}(\mathcal{D}) \backslash \mathbb{Z}(P)$ and observe that by the law of total probability, $\mathrm{Pr}[c \in \mathbb{Z}(P)] + \mathrm{Pr}[c \in \mathcal{S}] = 1$. Moreover, the event that $c = g + \alpha \cdot f \in \mathcal{S}$ can only happen if $g$ is already close to $\mathcal{S}$. There are at most $\alpha \cdot p$ such locations. This lets us bound the probability that $c \in \mathcal{S}$ as $\mathrm{Pr}[c \in \mathcal{S}] \leq \frac{\alpha \cdot p}{P}$. There are as many events where $g$ is close to $\mathcal{S}$ but $c \in \mathbb{Z}(P)$. The remaining probability mass is uniformly distributed across $\mathbb{Z}(P-\alpha \cdot p) = \{-\frac{P-\alpha\cdot p}{2}, \ldots, \frac{P - \alpha \cdot p}{2}\}$. In summary, the statistical distance between $\mathcal{D}$ and $\mathcal{U}$ is bounded by
\begin{flalign*}
\mathrm{SD}(\mathcal{D}, \mathcal{U}) &\leq \frac{1}{2} \left( (P-\alpha p) \cdot \left| \frac{1 - \frac{2 \alpha p}{P}}{P - \alpha p} - \frac{1}{P} \right| + 2 \alpha p \cdot \frac{\alpha p}{P}\right) \\
 & = \frac{1}{2} \left( (P - \alpha p) \cdot \frac{1}{P} \left| \frac{P - 2 \alpha p}{P - \alpha p} - 1 \right| + 2 \alpha p \cdot \frac{\alpha p}{P} \right) \\
 & = \frac{1}{2} \left( (P - \alpha p) \cdot \frac{1}{P} \cdot \frac{2 \alpha p}{P - \alpha p} + 2 \cdot \frac{\alpha^2 p^2}{P} \right) \enspace .
\end{flalign*}
This statistical distance is negligible when $P \gg p^4$. While the negligible statistical distance holds only for a single polynomial coefficient, it translates to the entire polynomial provided that there are at most a polynomial number of coefficients.

We now have the tools we need to prove HVZK.

\begin{theorem}
The protocol $\pro{EvalZK}$ has statistical honest-verifier zero-knowledge.
\end{theorem}

\begin{proof}
We build the HVZK simulator $\mathcal{S}$ as follows. Choose a polynomial $h(X) \sample \mathbb{Z}(P)$ with coefficients uniformly random from $\mathbb{Z}(P)$, and compute $y_h \gets h(z) \bmod p$. The polynomial fixes the non-binding commitment $\gr{C}$. Choose a randomizer $R \sample \mathbb{Z}(B)$ and set $\gr{C}^\star \gets \gr{g}^{q^{d+1} \cdot R}$. At this point the simulator can run the \textsf{PoE} and \eval protocols as the honest prover would.

Next, sample $\alpha \sample \mathbb{Z}(p)$ and compute $\gr{C}_g \gets \gr{C}_h \gr{C}_f^{-\alpha}$ and $y_g \gets y_h - \alpha \cdot y_f$. This provides the simulator $\mathcal{S}$ with all the elements for an accepting transcript.

To see why the resulting transcript is indistinguishable from the verifier's view of an authentic execution, observe that the only elements that have a different distribution are $\gr{C}_g$, $\gr{C}_h$ and $\gr{C}$. Since $\gr{C}_g$ and $\gr{C}_h$ are hiding commitments, they are indistinguishable. That leaves $\gr{C}$, which is a deterministic, non-hiding commitment to $h(X)$. However, the point is that $h(X)$ hides $f(X)$ statistically even if $h(X)$ were released in the clear. In other words, the distinguisher can learn no more information about $f(X)$ from $\gr{C} = \gr{g}^{h(q)}$ than if he were to receive $h(X)$ instead, and that is a negligible amount already.
\end{proof}