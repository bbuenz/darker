For many applications, such as the construction of ZK-SNARKs, require a polynomial commitment which given an evaluation leaks no additional information about the committed polynomial. To provide this we show how to build a hiding polynomial commitment along with a zero-knowledge evaluation protocol.

We start by defining a computational \emph{hiding} property for commitment schemes:

\begin{definition}
A commitment scheme $\Gamma = (\pro{Setup}, \pro{Commit}, \pro{Open})$ is \defn{hiding} if for all probabilistic polynomial time adversaries $\adv = (\adv_0, \adv_1)$, the probability of distinguishing between commitments of different is negligble:
\[
	\left| 1 - 2 \cdot \mathrm{Pr}\left[
		\hat{b} = b \ \middle| \ 
		\begin{array}{l}
			\params \gets \pro{Setup}(1^\lambda) \\
			m_0, m_1, \st \gets \adv_0(\params) \\
			b \sample \{0,1\} \\
			(c; r) \gets \pro{Commit}(\params, m_b) \\
			\hat{b} \gets \adv_1(\st, c)
		\end{array}
	\right] \right| \leq \negl \enspace .
\]
\end{definition}
If the property holds for all algorithms then we say that the commitment is \emph{statistically} hiding.
\paragraph{Hiding Polynomial Commitment}
We make the polynomial commitment described in Section~\ref{sec:protocol} hiding by adding a large random coefficient at degree $d+1$. Let $B\geq |\GG|$ be a publicly known upper bound on the order of $|\GG|$. We will choose the blinding coefficient between $0$ and $B\cdot 2^\lambda$. 

Formally, the hiding commitment algorithm is described as follows:
\begin{itemize}
	\item $\pro{CommitH}(f(X) \in \ZZ_p[X]) \rightarrow (\gr{C}; \hat{f}(X), d, r)$. Lift $f(X) \in \mathbb{Z}_p[X]$ to $\tilde{f}(X) \in \mathbb{Z}(\frac{p-1}{2})[X]$ and select random integer $r \sample [0,B\cdot 2^\lambda)$. Compute $d \gets \deg(f(X))$ and $\gr{C} \gets \gr{g}^{\tilde{f}(q)+q^{d+1} \cdot r}$ and return commitment $\gr{C}$ with secret opening information $\tilde{f}(X), d, r$.
\end{itemize}

The random coefficient $r$ ensures that $\gr{C}$ is nearly indistinguishable from a random group element. Let's assume that $\gr{g}$ and $\gr{g}^{q^{d+1}}$ generate the same group, e.g. because $\gr{g}$ generates a prime order group. Let $n=|\langle \gr{g}\rangle|$ be the order of that subgroup generate $\gr{g}$. $\gr{A}\gets\gr{g}^c$ for $c\sample [0,n)$ is a random subgroup element. If $r \sample [0,B\cdot 2^\lambda)$ and $B\geq n$ then $\gr{B}\gets \gr{g}^r$ has statistical distance at most $2^{-\lambda}$ from $\gr{B}$. Consequently $\gr{C}$ is statistically negligibly far from a random group element. 
 Unfortunately, without a trusted setup we can only guarantee that $\gr{g}^{q^{d+1}}$ spans a subgroup of $\langle \gr{g} \rangle$. The commitment then becomes computationally hiding under the \emph{Subgroup Indistinguishability Assumption}~\cite{C:BraGol10}. The assumption states that no efficient adversary can distinguish subgroups. If an adversary can distinguish $\gr{C}$ from a random element with probability more than $2^{-\lambda}$ then he must be able to distinguish the subgroup generated by $\gr{g}^{q^{d+1}}$ from the subgroup generated by $\gr{g}$. For simplicity we will only proof security for the case that $\gr{g}$ and $\gr{g}^{q^{d+1}}$ span the same group.
\begin{theorem}
The commitment scheme $\Gamma = (\pro{Setup}, \pro{CommitH}, \pro{Open})$ is statistically hiding if $B \gg |\mathbb{G}|$ and if $\langle \gr{g} \rangle = \langle \gr{g}^{q^{d+1}} \rangle$ and computationally hiding if the commitment described in Section~\ref{subsec:concretepoly} is binding.
\end{theorem}

\begin{proof}
The hiding commitment is a commitment to a degree $d+1$ polynomial. It therefore directly inherits the binding property from the not hiding scheme.

To show hiding, we use the fact that the uniform distributions $[0,b]$ and $[a,a+b]$ have statistical distance $\frac{a}{b}$, i.e. the probability that any algorithm can distinguish the distributions from a single sample is less than $\frac{a}{b}$. Similarly $\gr{C}\gets\gr{g}^{f(q)+rq^{d+1}}$ for $r\sample[0,B\cdot 2^{\lambda})$ has statistical distance at most $2^{-\lambda}$ from a uniform element generated by $\gr{g}$ if $B\geq |\langle \gr{g}\rangle|$. This means that two polynomial commitments can be distinguished by any algorithm with probability at most $2^{-\lambda+1}$.
\end{proof}
We note that if the condition $\langle \gr{g} \rangle = \langle \gr{g}^{q^{d+1}} \rangle$ can't be guaranteed then the commitment is still computationally hiding under the subgroup indistinguishability assumption~\cite{C:BraGol10}. The security proof then shows that an efficient adversary that can distinguish commitments with non-negligble propability must be able to distinguish elements in the subgroup generated by $\gr{g}$ and the subgroup generated by $\gr{g}^{q^{d+1}}$.

\paragraph{Zero-Knowledge Eval}
We know build a zero-knowledge eval protocol which is an \eval protocol for a hiding polynomial commitment. The zero-knowledge eval protocol informally reveals no information about the committed polynomial other than that the \eval statement holds. The zero-knowledge protocol shows that the prover must know a degree $d$ polynomial $f(X)$ with bounded coefficients such that $f(z)\bmod p = y$ but does not leak any other information about $f$. Formally we will show that the interactive $\pro{ZK-Eval}$ argument is honest verifier zero-knowledge according to Definition~\ref{def:hvzk} by constructing an efficient simulator $\mathcal{S}$ that can generate a distribution of transcripts that is indistinguishable from honestly generated transcripts.
\begin{comment}
	\begin{definition}[HVZK for interactive arguments]
Let $\textsf{View}_{\langle \prover(x, w), \verifier(x) \rangle}$ denote the view (list of sent and received messages) of the verifier in an interactive protocol described in Definition~\ref{def:argument} on common input $x$ and prover witness input $w$. The interactive protocol has $\delta$-statistical honest verifier zero-knowledge if there exists a probabilistic polynomial time algorithm $\mathcal{S}$ such that for every $(x, w) \in \mathcal{R}$, the distribution $\mathcal{S}(x)$ is $\delta$-close to $\textsf{View}_{\langle \prover(x, w), \verifier(x) \rangle}$ (as distributions over the randomness of $\prover$ and $\verifier$).
\end{definition}
\end{comment}


The idea for the zk-eval protocol is a simple blinding of the polynomial and is borrowed from Zero-Knowledge \cite{EPRINT:ChiForSpo17} and Bulletproofs\cite{EC:BCCGP16,SP:BBBPWM18}. Let $f(X)$ be the committed polynomial, using the hiding commitment scheme. The prover wants to convince the verifier that $f(z)\bmod p=y$. To do this the prover commits to a degree $d$ polynomial $r(X)$ with random coefficients. The prover also reveals $y'\gets r(z)\bmod p$. The verifier then sends a random challenge $c$ and the prover and verifier can compute a commitment to $s(X)\gets r(X)+c\cdot f(X)$. $r(X)$ ensures that $s(X)$ is distributed statistically close to a random polynomial. The prover could just reveal $s(X)$ and the verifier can check that $s(z)\bmod p=y'+c \cdot y\bmod p$. Instead of sending $s(X)$ in the clear the prover can additionally just send the commitment randomness to provide the verifier with a non-hiding commitment to $s(X)$. The prover and verifier can then use the standard $\eval$ protocol to efficiently evaluated $s$ at $z$.
 \noindent\begin{mdframed}[userdefinedwidth=\textwidth]
\begin{minipage}{\textwidth}
	\begin{flushleft}
	\pro{ZK-Eval}\\
	Input: $\crs,\gr{C} \in \GG,z,y \in \ZZ_p, b\in \ZZ,d$, Witness: $f(X) \ZZ(b),\alpha \in \ZZ(2^\lambda)$\\
	Statement: $\gr{C}=\commit(f(X);\alpha),f(X) \in \ZZ(b\cdot 2^\lambda),\deg(f(X))=d, f(z)=y \bmod p$
	\begin{enumerate}[nolistsep]
		    \item \prover samples a random degree $d$ $k(X) \sample \ZZ(b\cdot 2^\lambda)[X]$ and $\beta \sample \ZZ(2^\lambda)$ and computes $\gr{R}\gets \commit(k(X);\beta)$ and $y'\gets k(z) \bmod p$
		    \item \prover sends $\gr{R}$ and $y'$ to $\verifier$
		    \item \verifier samples random $c\in [-\frac{p-1}{2},\frac{p-1}{2}]$ and sends it to $\prover$
		    \item \prover computes $s(X)\gets k(X) + c \cdot f(X)$, as well as $\gamma\gets \alpha+ c\cdot \gamma$. 
		    \item \prover sends $\gamma$ and to \verifier.
		    \item \prover and \verifier compute $\gr{C}'\gets \gr{R}\gr{C}^c/g^{q^{d+1}\cdot \gamma}$ and $y''\gets y'+c \cdot y \bmod p$
		    \item \prover and \verifier run $\eval(\crs,C',z,y'',d,b\cdot 2^{\lambda+1};s(X))$.
		   		\end{enumerate}
	\end{flushleft}
\end{minipage}
\end{mdframed}



\begin{theorem}
Let $\eval$ have perfect completeness and witness extended emulation for $q> b\cdot \boldsymbol{\varsigma}_{p, d}$. Assuming that $\pro{commitH}$ is statistically hiding and both the order assumption and the strong RSA assumption hold for $\ggen$ the protocol $\pro{EvalZK}$ has perfect completeness, witness extended emulation and statistical honest-verifier zero-knowledge for $q>\boldsymbol{\varsigma}_{p, d} \cdot p^2$.
\end{theorem}

\begin{proof}
A simple application of Theorem~\ref{thm:algebraicIOPcompiler} and Lemma~\ref{lem:intrandomcombine} shows that the protocol maintains witness extended emulation.   

We build the HVZK simulator $\mathcal{S}$ as follows. Choose a polynomial $h(X) \sample \mathbb{Z}(P)$ with coefficients uniformly random from $\mathbb{Z}(P)$, and compute $y_h \gets h(z) \bmod p$. The polynomial fixes the non-binding commitment $\gr{C}$. Choose a randomizer $R \sample \mathbb{Z}(B)$ and set $\gr{C}^\star \gets \gr{g}^{q^{d+1} \cdot R}$. At this point the simulator can run the \textsf{PoE} and \eval protocols as the honest prover would.

Next, sample $\alpha \sample \mathbb{Z}(p)$ and compute $\gr{C}_g \gets \gr{C}_h \gr{C}_f^{-\alpha}$ and $y_g \gets y_h - \alpha \cdot y_f$. This provides the simulator $\mathcal{S}$ with all the elements for an accepting transcript.

To see why the resulting transcript is indistinguishable from the verifier's view of an authentic execution, observe that the only elements that have a different distribution are $\gr{C}_g$, $\gr{C}_h$ and $\gr{C}$. Since $\gr{C}_g$ and $\gr{C}_h$ are hiding commitments, they are indistinguishable. That leaves $\gr{C}$, which is a deterministic, non-hiding commitment to $h(X)$. However, the point is that $h(X)$ hides $f(X)$ statistically even if $h(X)$ were released in the clear. In other words, the distinguisher can learn no more information about $f(X)$ from $\gr{C} = \gr{g}^{h(q)}$ than if he were to receive $h(X)$ instead, and that is a negligible amount already.
\end{proof}