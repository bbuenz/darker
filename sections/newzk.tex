For many applications, such as the construction of ZK-SNARKs, require a polynomial commitment which given an evaluation leaks no additional information about the committed polynomial. To provide this we show how to build a hiding polynomial commitment along with a zero-knowledge evaluation protocol.

We start by defining a computational \emph{hiding} property for commitment schemes:

\begin{definition}
A commitment scheme $\Gamma = (\pro{Setup}, \pro{Commit}, \pro{Open})$ is \defn{hiding} if for all probabilistic polynomial time adversaries $\adv = (\adv_0, \adv_1)$, the probability of distinguishing between commitments of different is negligble:
\[
	\left| 1 - 2 \cdot \mathrm{Pr}\left[
		\hat{b} = b \ \middle| \ 
		\begin{array}{l}
			\params \gets \pro{Setup}(1^\lambda) \\
			m_0, m_1, \st \gets \adv_0(\params) \\
			b \sample \{0,1\} \\
			(c; r) \gets \pro{Commit}(\params, m_b) \\
			\hat{b} \gets \adv_1(\st, c)
		\end{array}
	\right] \right| \leq \negl \enspace .
\]
\end{definition}
If the property holds for all algorithms then we say that the commitment is \emph{statistically} hiding.
\paragraph{Hiding Polynomial Commitment}
We make the polynomial commitment described in Section~\ref{sec:protocol} hiding by adding a large random coefficient at degree $d+1$. Let $B\geq |\GG|$ be a publicly known upper bound on the order of $|\GG|$. We will choose the blinding coefficient between $0$ and $B\cdot 2^\lambda$. 

Formally, the hiding commitment algorithm is described as follows:
\begin{itemize}
	\item $\pro{CommitH}(f(X) \in \ZZ_p[X]) \rightarrow (\gr{C}; \hat{f}(X), d, r)$. Lift $f(X) \in \mathbb{Z}_p[X]$ to $\tilde{f}(X) \in \mathbb{Z}(\frac{p-1}{2})[X]$ and select random integer $r \sample [0,B\cdot 2^\lambda)$. Compute $d \gets \deg(f(X))$ and $\gr{C} \gets \gr{g}^{\tilde{f}(q)+q^{d+1} \cdot r}$ and return commitment $\gr{C}$ with secret opening information $\tilde{f}(X), d, r$.
\end{itemize}

The random coefficient $r$ ensures that $\gr{C}$ is nearly indistinguishable from a random group element. Let's assume that $\gr{g}$ and $\gr{g}^{q^{d+1}}$ generate the same group, e.g. because $\gr{g}$ generates a prime order group. Let $n=|\langle \gr{g}\rangle|$ be the order of that subgroup generate $\gr{g}$. $\gr{A}\gets\gr{g}^c$ for $c\sample [0,n)$ is a random subgroup element. If $r \sample [0,B\cdot 2^\lambda)$ and $B\geq n$ then $\gr{B}\gets \gr{g}^r$ has statistical distance at most $2^{-\lambda}$ from $\gr{B}$. Consequently $\gr{C}$ is statistically negligibly far from a random group element. 
 Unfortunately, without a trusted setup we can only guarantee that $\gr{g}^{q^{d+1}}$ spans a subgroup of $\langle \gr{g} \rangle$. The commitment then becomes computationally hiding under the \emph{Subgroup Indistinguishability Assumption}~\cite{C:BraGol10}. The assumption states that no efficient adversary can distinguish subgroups. If an adversary can distinguish $\gr{C}$ from a random element with probability more than $2^{-\lambda}$ then he must be able to distinguish the subgroup generated by $\gr{g}^{q^{d+1}}$ from the subgroup generated by $\gr{g}$. For simplicity we will only proof security for the case that $\gr{g}$ and $\gr{g}^{q^{d+1}}$ span the same group.
\begin{theorem}
The commitment scheme $\Gamma = (\pro{Setup}, \pro{CommitH}, \pro{Open})$ is statistically hiding if $B \gg |\mathbb{G}|$ and if $\langle \gr{g} \rangle = \langle \gr{g}^{q^{d+1}} \rangle$ and computationally hiding if the commitment described in Section~\ref{subsec:concretepoly} is binding.
\end{theorem}

\begin{proof}
The hiding commitment is a commitment to a degree $d+1$ polynomial. It therefore directly inherits the binding property from the not hiding scheme.

To show hiding, we use the fact that the uniform distributions $[0,b]$ and $[a,a+b]$ have statistical distance $\frac{a}{b}$, i.e. the probability that any algorithm can distinguish the distributions from a single sample is less than $\frac{a}{b}$. Similarly $\gr{C}\gets\gr{g}^{f(q)+rq^{d+1}}$ for $r\sample[0,B\cdot 2^{\lambda})$ has statistical distance at most $2^{-\lambda}$ from a uniform element generated by $\gr{g}$ if $B\geq |\langle \gr{g}\rangle|$. This means that two polynomial commitments can be distinguished by any algorithm with probability at most $2^{-\lambda+1}$.
\end{proof}
We note that if the condition $\langle \gr{g} \rangle = \langle \gr{g}^{q^{d+1}} \rangle$ can't be guaranteed then the commitment is still computationally hiding under the subgroup indistinguishability assumption~\cite{C:BraGol10}. The security proof then shows that an efficient adversary that can distinguish commitments with non-negligble propability must be able to distinguish elements in the subgroup generated by $\gr{g}$ and the subgroup generated by $\gr{g}^{q^{d+1}}$.

\paragraph{Zero-Knowledge Eval}
We know build a zero-knowledge eval protocol which is an \eval protocol for a hiding polynomial commitment. The zero-knowledge eval protocol informally reveals no information about the committed polynomial other than that the \eval statement holds. The zero-knowledge protocol shows that the prover must know a degree $d$ polynomial $f(X)$ with bounded coefficients such that $f(z)\bmod p = y$ but does not leak any other information about $f$. Formally we will show that the interactive $\pro{ZK-Eval}$ argument is honest verifier zero-knowledge according to Definition~\ref{def:hvzk} by constructing an efficient simulator $\mathcal{S}$ that can generate a distribution of transcripts that is indistinguishable from honestly generated transcripts.
\begin{comment}
	\begin{definition}[HVZK for interactive arguments]
Let $\textsf{View}_{\langle \prover(x, w), \verifier(x) \rangle}$ denote the view (list of sent and received messages) of the verifier in an interactive protocol described in Definition~\ref{def:argument} on common input $x$ and prover witness input $w$. The interactive protocol has $\delta$-statistical honest verifier zero-knowledge if there exists a probabilistic polynomial time algorithm $\mathcal{S}$ such that for every $(x, w) \in \mathcal{R}$, the distribution $\mathcal{S}(x)$ is $\delta$-close to $\textsf{View}_{\langle \prover(x, w), \verifier(x) \rangle}$ (as distributions over the randomness of $\prover$ and $\verifier$).
\end{definition}
\end{comment}


The idea for the zk-eval protocol is a simple blinding of the polynomial and is borrowed from Zero-Knowledge \cite{EPRINT:ChiForSpo17} and Bulletproofs\cite{EC:BCCGP16,SP:BBBPWM18}. Let $f(X)$ be the committed polynomial, using the hiding commitment scheme. The prover wants to convince the verifier that $f(z)\bmod p=y$. To do this the prover commits to a degree $d$ polynomial $r(X)$ with random coefficients. The prover also reveals $y'\gets r(z)\bmod p$. The verifier then sends a random challenge $c$ and the prover and verifier can compute a commitment to $s(X)\gets r(X)+c\cdot f(X)$. $r(X)$ ensures that $s(X)$ is distributed statistically close to a random polynomial. The prover could just reveal $s(X)$ and the verifier can check that $s(z)\bmod p=y'+c \cdot y\bmod p$. Instead of sending $s(X)$ in the clear the prover can additionally just send the commitment randomness to provide the verifier with a non-hiding commitment to $s(X)$. The prover and verifier can then use the standard $\eval$ protocol to efficiently evaluated $s$ at $z$.
 \noindent\begin{mdframed}[userdefinedwidth=\textwidth]
\begin{minipage}{\textwidth}
	\begin{flushleft}
	\pro{ZK-Eval}$(crs,\gr{C} \in \GG,z,y \in \ZZ_p,d\in \NN;f(X)\in \ZZ(b)[X],r \in \ZZ):$\\
		Statement: $\langle(\gr{C},z,y,d),(f(X) \in \ZZ(\frac{p-1}{2})[X],r)\rangle \in \mathcal{R}_\eval(pp)$\\
	%Input: $\crs,\gr{C} \in \GG,z,y \in \ZZ_p, b\in \ZZ,d$, Witness: $f(X) \ZZ(b),\alpha \in \ZZ(2^\lambda)$\\
	\begin{enumerate}[nolistsep]
		    \item \prover samples a random degree $d$ $k(X) \sample \ZZ(\frac{p^2}{4}\cdot 2^\lambda)[X]$ and $r_k \sample [0,B\cdot 2^\lambda)$ and computes $\gr{R}\gets \commit(k(X);r_k)$ and $y_k\gets k(z) \bmod p$
		    \item \prover sends $\gr{R}$ and $y_k$ to $\verifier$
		    \item \verifier samples random $c\sample [-\frac{p-1}{2},\frac{p-1}{2}]$ and sends it to $\prover$
		    \item \prover computes $s(X)\gets k(X) + c \cdot f(X)$, as well as $r_s\gets r_k+ c\cdot r$. 
		    \item \prover sends $\gamma$ and to \verifier.
		    \item \prover and \verifier compute $\gr{C}'\gets \gr{R}\gr{C}^c/g^{q^{d+1}\cdot r_s}$ and $y_s\gets y_k+c \cdot y \bmod p$\pccomment{$\gr{C}'=\gr{g}^{s(q)}$}
		    \item \prover and \verifier run $\pro{EvalBounded}(\crs,\gr{C}',z,y_s,d,\frac{p^2}{4}\cdot 2^{\lambda+1};s(X))$.\pccomment{$s(z)\bmod p=y_s$}
		   		\end{enumerate}
	\end{flushleft}
\end{minipage}
\end{mdframed}



\begin{theorem}
Let $\eval$ have perfect completeness and witness extended emulation for $q> b\cdot \boldsymbol{\varsigma}_{p, d}$. Assuming that $\pro{commitH}$ is statistically hiding and both the order assumption and the strong RSA assumption hold for $\ggen$ the protocol $\pro{EvalZK}$ has perfect completeness, witness extended emulation and statistical honest-verifier zero-knowledge for $q>2^\lambda\cdot (p-1)(\frac{p^2-1}{2})^{\lceil \log_2(d+1)\rceil+1}<p^{2\log_2(d+1)+4}$.
\end{theorem}

\begin{proof}
A simple application of Theorem~\ref{thm:algebraicIOPcompiler} and Lemma~\ref{lem:intrandomcombine} shows that the protocol maintains witness extended emulation. The extractor extracts $s(X)$ and $s'(X)$ from the $\eval$ protocol for challenges $c$ and $c'$. We can directly use Lemma~\ref{lem:intrandomcombine} to extract the witness $f(X)$ or a break of an assumption from these two transcripts. The bound on $q$ grows by a factor of less than $p^2$ ($p^3$ under the $2$-Strong RSA assumption).   

To show zero-knowledge we build the simulator $\mathcal{S}$ as follows. Choose a polynomial $s(X) \sample \mathbb{Z}(\frac{p^2-1}{4})[X]$ with uniform random coefficients, and a blinding factor $r_s\sample [0,B\cdot 2^{2\cdot\lambda+1})$. $\mathcal{S}$ then chooses a random challenge $c\sample (-p/2,p/2)$ and computes $\gr{R}=\commit(s(X),r_2)\gr{C}^{-c}$. The simulator then performs the rest of the $\eval$ protocol honestly using $s(X)$ as the witness. 

$r_s$ is distributed identically to the honest $r_s$. Given the hiding property of the commitment scheme, $\gr{R}$ is statistically indistinguishable from any other commitment. Finally the simulated and the honest $s(X)$ have statistical distance at most $2^{-\lambda}$ from a random polynomial. The coefficients of $c\cdot f(x)$ are in $\ZZ(\frac{p^2}{4})$. The blinding $s(X)$ are sampled from a $2^{2\lambda}$ larger range. This means that the coefficients of $s(X)=k(X)+c\cdot f(X)$ are at most $2^{-\lambda}$ from uniform. This implies that the simulated $s(X)$ and the real $s(X)$ are statistically indistinguishable. Given that the simulator can simulate $s(X)$ it is clear that the evaluation of $\pro{EvalBounded}$ cannot leak more than $s$ itself. The views of the simulated and real transcripts are, therefore, statistically indistinguishable and the protocol has honest verifier zero-knowledge.
\end{proof}