We present the preliminaries on the computational assumptions in groups of unknown orders and our definitions. The preliminaries on proof systems are found in the new security proof in Appendix \ref{appendix:prelimns}.
\subsection{Assumptions}
The cryptographic compilers make extensive use of groups of unknown order, \emph{i.e.}, groups for which the order cannot be computed efficiently.
Concretely, we require groups for which two specific hardness assumptions hold.
The binding property of the polynomial commitment and the evaluation protocol, rely on the most basic assumption in groups of unknown order. The assumption states that it is hard to compute the order of random group elements.
This assumption is implied by the famous RSA Assumption~\cite{RivShaAdl78} which states that it is hard to take \emph{random} roots (technically scalar divisions) of \emph{random} elements. 
Secondly, our proofs of exponentiation which are used to make the verifier efficient, rely on the much newer Adaptive Root Assumption~\cite{EC:Wesolowski19} which is the dual of the Strong RSA Assumption and states that it is hard to take \emph{random} roots of \emph{arbitrary} group elements. The assumption, is also used to show that the commitment scheme is \emph{computationally unique}, that is given a message an adversary can only output a single valid commitment to the message.
Both of these assumptions hold in generic groups of unknown order~\cite{EC:DamKop02,C:BonBunFis19}. %That is to say, there are no efficient algorithms that only have black-box access to the group but are able to break these assumptions. 
%It is an open research problem to show whether one of these assumption implies the other.


\begin{assumption}[Random Order Assumption]
\label{assum:randomorder}
	The random order assumption states that an efficient adversary cannot compute a multiple of the order of a given random group element. Specifically, it holds for $\ggen$ if for any probabilistic polynomial time adversary $\adv$:
	\[
    \Pr\left[a\cdot \Generator =0:
    \begin{array}{l}
         \GG, N \leftarrow \ggen(\lambda)  \\
         \Generator, \sample \GG\\
         a \in \ZZ \leftarrow \adv(\mathbb{G},N, \Generator) \\
    \end{array}\right] \leq \negl \enspace .
    \]
\end{assumption}



\begin{assumption}[RSA assumption, \cite{RivShaAdl78,EC:CouPetPoi17}]
\label{assum:rsa}
	The RSA assumption states that an efficient adversary cannot compute a random root (co-prime with the order of the group) for a given random group element. Specifically, it holds for $\ggen$ if for any probabilistic polynomial time adversary $\adv$:
	\[
    \Pr\left[
    \begin{array}{l}
    \ell \cdot \gr{u} = \Generator \,     \end{array} :
    \begin{array}{l}
         \GG, N \leftarrow \ggen(\lambda)  \\
         \Generator \sample \GG, \ell \sample [N]  \\
         \gr{u} \in \mathbb{G} \leftarrow \adv(\mathbb{G}, \Generator) \\
    \end{array}\right] \leq \negl \enspace .
\]

\end{assumption}



%We note that our definition of the Strong RSA Assumption additionally require that $\ell$ be an odd prime~\cite{EC:BarPfi97}. Our definition is a stronger assumption which states that all roots are difficult.
\begin{assumption}[Adaptive Root Assumption]
\label{assum:adaptiveroot}
The \emph{Adaptive Root Assumption} holds for $\ggen$ if 
there is no efficient adversary $(\adv_0,\adv_1)$ that succeeds 
in the following task.
First, $\adv_0$ outputs an element $\gr{w} \in \GG$ and some $\state$.
Then, a random prime $\ell$ in $\primes$ is chosen
and $\adv_1(\ell,\state)$ outputs $\gr{w}^{1/\ell} \in \GG$.
For all efficient $(\adv_0,\adv_1)$:
\begin{small}
\[           
                \Pr\left[\ell \cdot \gr{u} = \gr{w} \neq 1 \ : \ 
                \begin{array}{l}
                      \GG \sample \ggen(\lambda) \\ 
                      (\gr{w},\state) \sample \adv_0(\GG) \\
                      \ell \sample \primes \\ 
                      \gr{u} \gets \adv_1(\ell,w, \state)
                \end{array} 
        \right] \leq \negl.
\]
\end{small}
\end{assumption}
\begin{lemma}
\label{lem:roa-to-rsa}
	The RSA Assumption for $\ggen$ implies the Random Order Assumption
	\end{lemma}
\begin{proof}
	Given an efficient adversary $\adv_{\textsf{Order}}$ for the random order assumption that succeeds with non-negligible probability $\epsilon$ we will construct an efficient adversary $\adv_{\textsf{RSA}}$ for the RSA assumption. On input $\GG,\Generator,\ell$ to $\adv_{\textsf{RSA}}$ we will forward $\GG,\Generator$ to $\adv_{\textsf{Order}}$. $\adv_{\textsf{Order}}$ outputs $a$ such that $a\cdot \Generator=0$ with non-negligible probability $\epsilon$. 
	$\adv_{\textsf{RSA}}$ computes $a'\gets \frac{a}{\gcd(a,\ell^k)}$ for $k=\lceil\log_\ell(a)\rceil$. The probability that $\ell$ is not co-prime to the order of $\GG$ is bounded by $\frac{\log_2{|\GG|}}{|\primes|}$ which is negligible in $\lambda$. Otherwise $\ell$ is co-prime with the order of $\Generator$ and $a$ is a multiple of the order of $\Generator$ we have that $a'$ is still a multiple of the order of $\Generator$. Now $\adv_{\textsf{RSA}}$ computes $w\gets \ell^{-1} \bmod a'$ and outputs $\gr{U} \gets w\cdot \Generator$. Now we have $\ell \cdot \gr{U}=\Generator$ so $\adv_{\textsf{RSA}}$ succeeds with probability $\epsilon-\negl $.
\end{proof}
\begin{lemma}
\label{lem:roa-to-ar}
	The Adaptive Root Assumption for $\ggen$ implies the Random Order Assumption
	\end{lemma}
	\begin{proof}
	Given an efficient adversary $\adv_{\textsf{Order}}$ for the random order assumption we will construct an efficient adversary $\adv_{\textsf{AR}}=(\adv_0,\adv_1)$ for the Adaptive Root assumption. On input $\GG$ to $\adv_{0}$ we will sample a random group element $\Generator$ from $\GG$ and forward it to $\adv_{\textsf{Order}}$. $\adv_{\textsf{Order}}$ outputs $a$ such that $a\cdot \Generator=0$ with non-negligible probability $\epsilon$. And we set the output of $\adv_0$ to be $(\Generator,a)$. The adaptive root game then samples a random prime $\ell$.
	$\adv_1$ on input $(a,g,\ell)$ computes $a'\gets \frac{a}{\gcd(a,\ell^k)}$ for $k=\lceil\log_\ell(a)\rceil$. Note that since $\ell$ is co-prime to the order of $\GG$ and thus also the order of $\Generator$ and $a$ is a multiple of the order of $\Generator$ we have that $a'$ is still a multiple of the order of $\Generator$. If we don't abort then we compute $\ell^{-1} \bmod a$ and $\gr{u}=\ell^{-1}\cdot \Generator$.
	Finally $A_1$ outputs $\gr{u}$, which by construction is such that $\ell\cdot \gr{u}=\gr{g}$. 
	\end{proof}
	


\paragraph{Groups of unknown order.}
We consider two candidate groups of unknown order. Both have their own upsides and downsides.

\textit{RSA Group.} In the multiplicative group $\ZZ_n^*$ of integers modulo a product $n=p\cdot q$ of large primes $p$ and $q$, computing the order of the group is as hard as factoring $n$. The Adaptive Root Assumption does not hold for $\ZZ_n^*$ because $-1 \in \ZZ_n^*$ can be easily computed and has order two. This can be resolved though by working instead in the quotient group $\ZZ_n^* / \{ x \, | \, x^2 = 1 \} \cong \mathrm{QR}_n$. %Additionally if $n$ is the product of strong primes, \emph{i.e.}, $\frac{p-1}{2}$ and $\frac{q-1}{2}$ are primes, then $\mathrm{QR}_n$ does not contain elements of low order\cite{JC:FisSch00,C:HofKil09}. In this group, the Order Assumption and the Strong RSA Assumption are equivalent to the hardness of factoring $n$.\alan{citation needed}
The downside of using an RSA group, or more precisely, the group of quadratic residues modulo an RSA modulus, is that this modulus cannot be generated in a publicly verifiable way without exposing the order, and thus requires a trusted setup.

%\textit{UFO Group.} An alternative to RSA groups that still uses similar arithmetic is the multiplicative group of integers $\ZZ_n^*$ modulo a large random modulus $n$, called an \emph{UnFactorizable Object (UFO)}~\cite{conf/icics/Sander99}. This modulus is chosen to be so large so that with overwhelming probability its factorization contains two primes large enough to guarantee the targeted security level. As a result, elements of this group are hundreds of thousands of bits in size for reasonable security levels, compared to just thousands of bits for RSA group elements. The upside though is that the unfactorizable object $n$ can be sampled with public randomness, and therefore requires no trusted setup.

\textit{Class Group.} The class group of an imaginary quadratic order is defined as the quotient group of fractional ideals by principal ideals of an order of a number field $\mathbb{Q}(\sqrt{\Delta})$, with ideal multiplication. A class group $\mathcal{C}\ell(\Delta)$ is fully defined by its discriminant $\Delta$, which needs to satisfy only public constraints such as $\Delta \equiv 1 \bmod 4$ and $-\Delta$ must be prime. As a result, $\Delta$ can be generated from public coins, thus obviating the need for a trusted setup. A group element can be represented by two integers strictly smaller (in absolute value) than $-\Delta$, which in turn is on the same order of magnitude as RSA group elements for a similar security level.  We refer the reader to Buchmann and Hamdy's survey~\cite{PKC/BucHam01} and Straka's accessible blog post~\cite{web/Stra19} for more details.

Working in $\mathcal{C}\ell(\Delta)$ does present an important difficulty: there is an efficient algorithm due to Gauss to compute square roots of arbitrary elements~\cite{jtn/BosSte96}, and by repetition, arbitrary power of two roots.
Despite this, the random order and the adaptive root assumption still hold in class groups. Computing power of two roots does not directly enable one to compute the order of a random element or compute random (large prime) roots of a chosen element. The new security proof only relies on the random order assumption for extraction and the adaptive root assumption for the PoEs. It, therefore, holds even for adversaries that can compute square (or other small) roots of elements.


\subsection{Commitment Schemes}

In defining the syntax of the various protocols, we use the following convention with respect to public values (known to both the prover and the verifier) and secret ones (known only to the prover). In any list of arguments or returned tuple $(a, b, c; d, e)$ those variables listed before the semicolon are public, and those variables listed after it are secret. When there is no secret information, the semicolon is omitted.

\begin{definition}[Commitment scheme]
\label{def:commitment}
A commitment scheme $\Gamma$ is a tuple $\Gamma = (\pro{Setup}, \pro{Commit},$ $\pro{Open})$ of PPT algorithms where:
\begin{itemize}
    \item $\pro{Setup}(1^\lambda) \rightarrow \params$ generates public parameters $\params$;
    \item $\pro{Commit}(\params; x) \rightarrow (C; r)$ takes a secret message $x$ and outputs a public commitment $C$ and (optionally) a secret opening hint $r$ (which might or might not be the randomness used in the computation).
    \item $\pro{Open}(\params, C, x, r) \rightarrow b \in \{0, 1\}$ verifies the opening of commitment $C$ to the message $x$ provided with the opening hint $r$. 
\end{itemize}

A commitment scheme $\Gamma$ is \defn{binding} if for all PPT adversaries $\adv$:
\begin{small}
\[
    \Pr\left[
        b_0 = b_1 \neq 0 \, \wedge \, x_0 \neq x_1 \ : \
        \begin{array}{l}
             \params \gets \pro{Setup}(1^\lambda) \\
             (C, x_0, x_1, r_0, r_1) \gets \adv(\params) \\
             b_0 \gets \pro{Open}(\params, C, x_0, r_0) \\
             b_1 \gets \pro{Open}(\params, C, x_1, r_1) \\
        \end{array}
    \right] \leq \negl \enspace 
\]
\end{small}
%\ben{We don't use the hiding property so why present?} \alan{For the ZK Eval protocol, maybe. Not sure yet.}
\begin{comment}
A commitment scheme $\Gamma$ is \defn{hiding} if for all probabilistic polynomial time adversaries $\adv=(\adv_0,\adv_1)$,
\[
    \left|
        1 - 2\Pr\left[
            \hat{b} = b \ : \
        \begin{array}{l}
             \params \gets \pro{Setup}(1^\lambda) \\
             (\state, x_0, x_1) \gets \adv_0(\params) \\
             b \sample \{0,1\} \\
             (\gr{C}; *) \gets \pro{Commit}(\params; x_b) \\
             \hat{b} \gets \adv_1(\state, \gr{C})
        \end{array}
        \right]
    \right| \leq \negl \enspace .
\]
\end{comment}
\end{definition}

We now extend the syntax to polynomial commitment schemes. The following definition generalizes that of Kate~\emph{et. al.}~\cite{AC:KatZavGol10} to allow interactive evaluation proofs. It also stipulates that the polynomial's degree be an argument to the protocol, contrary to Kate~\emph{et al.} where the degree is known and fixed.


\begin{definition} (Polynomial commitment) 
A polynomial commitment scheme is a tuple of protocols $\Gamma = (\pro{Setup}, \pro{Commit}, \pro{Open}, \pro{Eval})$ where $(\pro{Setup},$ $\pro{Commit}, \pro{Open})$ is a binding commitment scheme for a message space $R[X]$ of polynomials over some ring $R$, and
\begin{itemize}
    \item $\pro{Eval}(\params, C, z, y, d, \mu; f) \rightarrow b \in \{0, 1\}$ is an interactive public-coin protocol between a PPT prover $\prover$ and verifier $\verifier$. Both $\prover$ and $\verifier$ have as input a commitment $C$, points $z, y \in R $, and a degree $d$. The prover additionally knows the opening of $C$ to a secret polynomial $f(X) \in R[X]$ with $\deg(f(X)) \leq d$. The protocol convinces the verifier that $f(z) = y$. {In a multivariate extension of polynomial commitments, the input $\mu > 1$ indicates the number of variables in the committed polynomial} and $z \in R^\mu$.
\end{itemize}

A polynomial commitment scheme is \defn{correct} if an honest committer can successfully convince the verifier of any evaluation. 
Specifically, if the prover is honest, then for all polynomials $f(X) \in R[X]$ and all points $z \in R$,
\begin{small}
\[
    \Pr\left[b = 1 \ : \ \begin{array}{l}
        \params \gets \setup(1^\lambda) \\
        (C; r) \gets \pro{Commit}(\params, f(X)) \\
        y \gets f(z) \\
        d \gets \deg(f(X)) \\
        b \gets \pro{Eval}(\params, c, z, y, d; f(X), r) \\
    \end{array} \right] = 1 \enspace .
\]
\end{small}

A polynomial commitment scheme is \defn{evaluation binding} if no efficient adversary can convince the verifier that the committed polynomial $f(X)$ evaluates to different values $y_0 \neq y_1 \in R$ in the same point $z \in R$. However, our applications require a stronger property called \emph{knowledge soundness}. 
\end{definition}

\paragraph{Knowledge soundness} %In our application of polynomial commitments to the construction of arguments of knowledge, we also require the polynomial commitment to satisfy a \emph{knowledge} property. Informally, w
Any successful prover in the $\eval$ protocol must \emph{know} a polynomial $f(X)$ such that $f(z) = y$ and $C$ is a commitment to $f(X)$. More formally, since $\eval$ is a public-coin interactive argument we define this knowledge property as a special case of witness-extended emulation (Definition~\ref{def:wee}). 

Define the following NP relation given $\params \leftarrow \pro{Setup}(1^\lambda)$: 
\begin{small}
\[ 
\reval(\params) = \left\{
\langle (C, z, y, d), (f(X), r) \rangle
: 
\begin{array}{l} 
f \in R[X] \ \text{and} \ \deg(f(X)) \leq d \ \text{and} \ f(z) = y \\ 
 \text{and} \ \pro{Open}(\params, C, f(X), r) = 1 \\
\end{array}
\right\}
\] 
\end{small}
The correctness definition above implies that if $\Gamma = (\pro{Setup}, \pro{Commit}, \pro{Open}, \pro{Eval})$ is \emph{correct} then $\eval$ is a correct interactive argument for $\mathcal{R_\textsf{Eval}}(\params)$, with overwhelming probability over the randomness of $\pro{Setup}$. We say that $\Gamma$ has \textbf{witness-extended emulation} if $\eval$ has witness-extended emulation as an interactive argument for $\mathcal{R_\textsf{Eval}}(\params)$. 

It is easy to see that witness-extended emulation implies evaluation binding when the $\pro{Setup}, \pro{Commit},$ and $\pro{Open}$ part of $\Gamma$ form a binding commitment scheme. If the adversary succeeds in $\eval$ on both $(C, z, y_0, d_0)$ and $(C, z, y_1, d_1)$ for $y_0 \neq y_1$ or $d_0 \neq d_1$ then the emulator obtains two distinct witnesses $f(X) \neq f'(X)$ such that $C$ is a valid commitment to both. This would contradict the binding property of the commitment scheme. 

%\paragraph{Opening individual coefficients} The coefficients of a committed polynomial can be revealed and checked all at once using $\pro{Open}$, however, in some cases it is useful to reveal an individual coefficient more efficiently (\emph{i.e.}, with sublinear communication). 
%There is a generic one-round protocol for this that uses $\pro{Eval}$ as a black-box, and inherits the efficiency properties of $\pro{Eval}$. We present this in Section~\ref{sec:opencoefficient}. 

%\paragraph{Inner product argument} Another helpful feature for polynomial commitment schemes is an inner product argument that shows for commitments $(c_1, c_2)$ to degree $d$ polynomials $(f_1, f_2)$ the inner product of their coefficient vectors $a = \langle f_1, f_2 \rangle$. There is a one-round protocol for the inner product of committed \emph{univariate} polynomials that makes black-box calls to $\pro{Eval}$. We present this in Section~\ref{sec:innerproduct}. 

\subsection{Proofs of Exponentiation}
Wesolowski \cite{EC:Wesolowski19} introduced a simple yet powerful proof of correct exponentiation (``$\mathsf{PoE}$'') in groups of unknown order. A prover can efficiently convince a verifier that a large scalar multiplication in such a group was done correctly. For instance, the prover wishes to convince the verifier that $\gr{w} = \gr{u}^x$ for known group elements $\gr{u}, \gr{w} \in \mathbb{G}$ and exponent $x \in \mathbb{Z}$, and the verifier wants to verify this with much less work than performing the scalar multiplication. To do this, the verifier samples a large enough prime $\ell$ at random and the prover provides him with $\gr{Q} \gets \gr{u}^q$ where $q = \lfloor \frac{x}{\ell} \rfloor$. The verifier then simply computes the remainder $r \gets (x \mod \ell)$ and checks that $\gr{Q}^\ell\gr{u}^r = \gr{w}$. The protocol is an argument for the relation $\mathcal{R}_\mathsf{PoE} = \left\{ \langle(\gr{u}, \gr{w}, x), \varnothing\rangle \ : \ \gr{u}^x = \gr{w} \right\}$. 
The proof verification uses just $O(\lambda)$ group operations. When $x$ is $x=q^d$ the verifier can compute $r\gets x \bmod \ell$ using just $\log(d)$ $\ell$-bit multiplications.
\noindent\begin{mdframed}[userdefinedwidth=\textwidth]
\begin{minipage}{\textwidth}
	\begin{flushleft}
	$\pro{PoE}(\gr{u}, \gr{w}, x):$
	\begin{enumerate}[nolistsep]
		    \item \verifier samples $\ell \sample \primes$ and sends $\ell$ to \prover
		    \item \prover computes quotient $q$ and remainder $r$ such that $x = q\ell + r$ and $r \in \{0, \ldots, \ell-1\}$
		    \item \prover computes $\gr{Q} \gets q\cdot \gr{u}$ and sends it to \verifier
		    \item \verifier computes $r \gets (x \mod \ell)$ and checks that $\ell \cdot \gr{Q}+ r\cdot \gr{u} = \gr{w}$
		    \item \pcif{}check passes \textbf{then} \textbf{return} 1 \textbf{else} \textbf{return} 0
		\end{enumerate}
	\end{flushleft}
\end{minipage}
\end{mdframed}
%Wesolowski showed that an adversary that succeeds in the $\textsf{PoE}$ protocol for statements not in $\mathcal{R}_{\textsf{PoE}}$ can compute adaptive roots in the group $\GG$.

\begin{lemma}[\textsf{PoE} soundness~\cite{EC:Wesolowski19}]
\label{lem:poe}
\textsf{PoE} is an argument system for  relation $\mathcal{R}_\textsf{PoE}$ with negligible soundness error,
assuming the Adaptive Root Assumption (Assumption \ref{assum:adaptiveroot}) holds for~$\ggen$.
\end{lemma}
\begin{lemma}[\textsf{PoE} random oracle  soundness~\cite{EC:Wesolowski19}]
\label{lem:poero}
The Fiat-Shamir transform of \textsf{PoE}, replacing the verifier message $\ell$ with $\ell\gets \hash(u,w,x)$ and $\hash$ is an argument system for relation $\mathcal{R}_\textsf{PoE}$ with negligible soundness error,
assuming that $\hash$ is modeled as a random oracle and that the Adaptive Root Assumption (Assumption \ref{assum:adaptiveroot}) holds for~$\ggen$.
\end{lemma}

