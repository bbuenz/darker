
We can extend our polynomial commitment scheme to multivariate polynomials. The idea is simply to use higher degrees of $q$ to encode the next indeterminate. The protocol is linear in the number of variables and logarithmic in the total degree of the polynomial. For simplicity we only present a protocol for $\mu$-variate polynomials where the degree in each variable is $d$. The protocol extends naturally to different degrees per variable.

\paragraph{Encoding}
Let $q_i=q^{(d+1)^i}$ then $\hat{f}(q_1,\dots,q_\mu)\in \ZZ$ is an encoding of the multivariate polynomial $f(X_1,\dots,X_\mu)$ with maximum degree $d$. We use $\dec_{Multi}(f(q),\mu,d)$ to denote the decoding of an $\mu$-variate polynomial with degree exactly $d$ in each variable. The decoding algorithm simply uses the univariate decoding algorithm described in Section \ref{sec:encoding} to decode a univariate polynomial $\hat{h}(X)$ of degree $(d+1)^{\mu}-1$.
Then it associates each monomial of the univariate polynomial with a degree vector $(d_1,\dots,d_\mu)$ of the multivariate polynomial. The coefficient of the $i$th monomial becomes the coefficient of the $(d_1,\dots,d_\mu)$-monomial, where $(d_1,\dots,d_\mu)$ is the base-$(d+1)$ decomposition of $i$. 
\paragraph{Protocols}
 Using this encoding we can naturally derive the multivariate commitment scheme and $\eval$ protocol. The $\eval$ protocol computes the univariate polynomials $f(q_1,\dots,q_{\mu-1},X_\mu)$ and then uses the univariate eval protocol to reduce the claim from a claim about an $\mu$-variate polynomial to one about an $(\mu-1)$-variate one. At the final step the prover opens the now constant polynomial and the verifier can check the claim. For example, the protocol would reduce a bivariate (say $X$ and $Y$) cubic polynomial to a univariate one (in $Y$) in two rounds of interaction and then reduce the degree of $Y$ using another two rounds.
 
 \begin{mdframed}[userdefinedwidth=\textwidth]
\begin{minipage}{\textwidth}
	\begin{flushleft}
	$\pro{MultiSetup}(1^\secpar):$
		\begin{enumerate}[nolistsep]
			\item $ \GG \sample \ggen(\secpar)$
			\item $ \gr{g} \sample \GG$
			%\item $q \gets 2^k$ such that $q > (d+1) \cdot 2\cdot p^{\log_2(d+1)+1} $
			%\item Pick a prime $p\in \NN$ such that $\lceil\log_2(p)\rceil=\lambda$.
			%\item Pick a sufficiently large and odd $q\in \NN$ \pccomment{$q=O_\lambda(p^{\mu \cdot \log(d)})$}
			\item $\pcreturn \params = (\secpar,\GG,\gr{g})$
		\end{enumerate}
	$\pro{MultiCommit}(\params;f(X_1,\dots,X_\mu) \in \ZZ(\frac{p-1}{2})[X_1, \ldots, X_\mu]\subset \ZZ[X_1, \ldots, X_\mu]):$ 		\begin{enumerate}[nolistsep]
			\item $d\gets \deg(f)$\pccomment{For simplicity assume $f(X_1,\dots,X_n)$ has degree $d$ in each variable}
			\item $q_i\gets q^{(d+1)^{i-1}}$ for each $i\in [\mu]$
			\item $\gr{C} \gets \gr{g}^{f(q_1,\dots,q_\mu)}$
			\item $\pcreturn (\gr{C};f(X_1,\dots,X_\mu))$
		\end{enumerate}
			\end{flushleft}
\end{minipage}
\end{mdframed}
 
 \begin{mdframed}
\begin{minipage}{\textwidth}
			$\pro{MultiEval}(\params, \gr{C}\in \GG, \boldsymbol{z}\in \ZZ^\mu_p,y \in \ZZ_p, d,\mu,b \in \NN; f(X_1,\dots,X_\mu)\in \ZZ(b)[X_1, \ldots, X_\mu]) :$
			\begin{enumerate}[nolistsep]
			\item \pcif{$\mu=1$} 
			\item \pcind[1] \prover and \verifier run $\pro{EvalBounded}(\params,\gr{C},z_1,y,d,b,x;f(X_1))$ 
			\item \pcelse
			\item \pcind[1] Let $\hat{f}(X_\mu)\gets f(q_1,\dots,q_{\mu-1},X_\mu)$
			\item \pcind[1] Let $\crs_\mu \gets \{\lambda,\GG,\gr{g},p,q_\mu\}$
			\item \pcind[1] \prover and \verifier run the univariate $\pro{EvalBounded}(\params_\mu,\gr{C},z_\mu,y,d,q_\mu;\hat{f}(X))$
			\item \pcind[2] \textbf{except:} when $d=0$, $f$ is not sent; instead the protocol returns its input at this point, \emph{i.e.}, $(\gr{C}',y',b')$ along with the prover's witness $f'(X_1,\dots,X_{\mu-1})=\dec_{Multi}(f,\mu-1,d)$ (Lines~\ref{line:basestart}-\ref{line:baseend} of $\pro{EvalBounded}$). 
			\item \pcind[1]$\boldsymbol{z}'\gets (z_1,\dots,z_{\mu-1})\in \ZZ_p^{\mu-1}$
			\item \pcind[1]\prover and \verifier run $\pro{MultiEval}(\crs,C',\boldsymbol{z}',y',d,\mu-1,b';f')$
		    \end{enumerate}
      \end{minipage}
\end{mdframed}
We only prove security under the strong RSA assumption. The security proof, however, directly extends to groups where taking square roots is easy under the $2$-Strong-RSA Assumption. In that case $q>p^{3\mu \log_2(d+1)+1}$ suffices.
\begin{theorem}[Multivariate Eval]
	The polynomial commitment scheme for multi-variate polynomials consisting of protocols $(\pro{MultiSetup},\pro{MultiCommit},\pro{MultiEval})$ has perfect correctness and witness extended emulation if the Adaptive Root Assumption and the Strong RSA Assumption hold for $\ggen$ for $\mu$-variate polynomials of degree $d$ and if $d^\mu=\poly$ if $q> p^{2 \mu \log_2(d+1)+1}$.
\end{theorem}
\begin{proof}
	Perfect correctness follows from the correctness of the univariate commitment scheme and the fact that the coefficients of the witness polynomial in the honest execution are less than $\frac{p-1}{2}p^{\mu \lceil\log(d+1)\rceil}<q/2$.
	
	To show witness extended emulation we use the forking lemma (Lemma \ref{lemma:GFL}) and build a polynomial time extractor algorithm $\mathcal{X}_{\pro{MultiEval}}$ that given a binary tree of transcripts of depth $\mu \cdot\lceil\log(d+1)\rceil$, extracts a witness. Each node corresponds to a different challenge $\alpha$ as described in the forking lemma. The tree consists of at most $(d+1)^{\mu}=\poly$ transcripts. 
	Lemma~\ref{lem:poe} states that the probability that an adversary can create any accepting transcript for which the $\textsf{PoE}$ can't be replaced by a direct check is negligible under the Adaptive Root Assumption.
We can therefore invoke the lemma to replace all \textsf{PoE} executions with direct verification checks that $\gr{C}_L\gr{C}_R^{q^{d'+1}}=\gr{C}$. 
%The lemma focuses on the univariate \pro{Eval} protocol but works identically for the multivariate protocol. 

In constructing $\mathcal{X}_{\pro{MultiEval}}$ we use the extractor $\mathcal{X}_{\pro{Eval'}}$ described in the proof of Theorem~\ref{thm:polycommitsecurity}. $\mathcal{X}_{\pro{Eval'}}$ computes, given a tree of transcripts for $\pro{Eval'}$ a valid witness of $\pro{Eval'}$ or a fractional root of $\gr{g}$ or an element of known order in $\GG$. We construct $\mathcal{X}_{\pro{MultiEval}}$ recursively invoking $\mathcal{X}_{\pro{Eval'}}$ once per degree $\mu$. The probability that a polynomial time adversary and a polynomial time extractor $\mathcal{X}_{\pro{Eval'}}$ can produce a fractional root or an element of known order in $\GG$ is negligible under the strong-RSA and the the adaptive root assumptions. From hence on we will consider the case where neither of these events happen.

We use the superscript $(i)$ to denote the inputs to $\pro{MultiEval}$ where $\mu=i$. 
If $\mu=1$ then the extractor $\mathcal{X}_{\pro{Eval'}}$ directly extracts $f^{(1)}(X)\in \ZZ(b)$, a univariate degree $d$ polynomial with coefficients bounded by $b=\frac{p-1}{2}p^{2 \lceil\log_2(d+1)\rceil}$ and such that $f(z)=y \bmod p$. Note that $q/2>b$ so the extraction succeeds.

For $\mu>1$, let's assume that $f^{(\mu-1)}(X_1,\dots,X_{\mu-1})\in \ZZ(b)$ is an extracted $\mu-1$ variate polynomial with degree $d$ in each variable such that $f^{(\mu-1)}(z_1,\dots,z_{\mu-1}) \bmod p=y'$.
Let $f'\gets \enc_{Multi}(f^{(\mu-1)}(X_1,\dots,X_{\mu-1})\in \ZZ$ be the encoding of $f^{(\mu-1)}(X_1,\dots,X_n)$, such that $\gr{C}^{(\mu-1)}\gets \gr{g}^{f'}$. Note that $f'$ is equivalent to an encoding of a univariate degree $(d+1)^{\mu-1}$ polynomial with the same coefficients as the multivariate polynomial. Let $g^{(\mu-1)}(X)=\dec(f')\in \ZZ(b)[X]$ be that polynomial. 
Using $g^{(\mu-1)}(X)$ as the witness the extractor $\mathcal{X}_{\pro{Eval'}}$ extracts a univariate degree $(d+1)^{\mu}$ polynomial $g^{(\mu)}(X)$ with coefficients in $\ZZ(b \cdot p^{\lceil\log(d+1)}\rceil)$. 
Let $f''\gets g^{(\mu)}(q)$ be the encoding of $g^{(\mu)}$ such that $\gr{C}^{(\mu)}=\gr{g}^{f''}$. Note that using the multivariate decoding algorithm $f''$ also encodes a $\mu$-variate degree $d$ polynomial, i.e. $f^{(\mu)}(X_1,\dots,X_\mu)\gets \dec_{Multi}(f'',\mu,d)$. The coefficient on $X_i$ in $g^{(\mu)}(X)$ is the coefficient of the monomial in $f^{(\mu)}$ with degree vector defined by the base-$(d+1)$ decomposition of $i$, i.e. $\prod_{j=1}^\mu  X_j^{\lfloor i/(d+1)^{j-1}\rfloor \bmod d+1 }$. Note that the extraction additionally guarantees that the polynomial evaluation is correct, i.e. $f(z_1,\dots,z_\mu)\bmod p=y$.

The final extracted polynomial has coefficients in $\ZZ(\frac{p-1}{2}p^{2\mu \lceil\log_2(d+1)\rceil})$. Since $q>p^{2\mu\lceil\log_2(d+1)\rceil+1}$ both the univariate and the multivariate decoding succeed and the extractor extracts a valid $\mu$-variate degree $d$ witness polynomial.
\end{proof}
