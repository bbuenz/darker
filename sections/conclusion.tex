
The compilation of a SNARK for general purpose computations generally involves several phases. In the \emph{arithmetization} phase, the computation is expressed as a satisfaction problem over some algebra. In the \emph{information-theoretic compilation}, this algebraic problem is transformed into an interactive protocol for proving satisfiability or knowledge of the witness for the algebraic claim. The last phase is \emph{cryptographic compilation}, in which information-theoretical and ideal components in the information-theoretical protocol are replaced by cryptographic components, thereby concretely realizing the abstract object at the cost of computational security. 

In the context of this three-pass compilation process, DARK protocols inhabit the last phase. Our polynomial commitment scheme based on groups of unknown order represents an alternative to similar schemes based on pairing groups or Merkle trees of Reed-Solomon codewords. Any SNARK construction that uses these latter schemes as black boxes can be lifted to the DARK family by replacing the polynomial commitment scheme. When the DARK protocol is instantiated with class groups, this replacement obviates the need for a trusted setup.

The polynomial commitment scheme presented by this paper highlights the need for a theoretical framework where such a tool is a native first-order member.  The information-theoretical Polynomial IOP formalism provides exactly such a framework, and its capacity to capture SNARK constructions such as \textsf{Sonic} and \textsf{STARK}, indicates its conceptual soundness. The slightly more general algebraic linear IOP formalism captures more SNARK constructions still, such as the expanding family based on GGPR's Quadratic Arithmetic Programs. While all Polynomial IOPs can be instantiated with DARK protocols, when it is first compiled from an algebraic linear IOP, it is inherently multivariate and as a result a quasilinear prover time can no longer be guaranteed. 

By design, the IOP formalisms make abstraction of the cryptographic assumptions used in the layer underneath. As a result, these information theoretical formalisms fail to capture fully the expressive power of Diophantine Complexity. An interesting question left open by this work is the potential of IOPs where the proofs represent integers rather than polynomials over a finite field. An associated question is the search for alternative algebras that naturally commit to integers. This question includes searching for alternative groups of unknown order. However, any such group admits Shor's quantum order finding algorithm~\cite{conf/focs/Shor94} and must therefore necessarily fail to provide post-quantum security. In light of the surge of interest in post-quantum cryptography, the more relevant question is the search for alternative algebras that can achieve the same thing --- commit to integers while preserving homomorphic properties --- without admitting Shor's quantum attack.

Commitments to integers provide just one way to achieve a transparent polynomial commitment scheme. Another interesting open question is therefore how to instantiate Polynomial IOPs or polynomial commitment schemes with other algebras and hard problems, beyond groups of unknown order and Merkle trees of Reed-Solomon codes. For instance, such an instantiation from elliptic curve groups and based on the hardness of the discrete logarithm would be well received for its better studied hard problem, and perhaps for its smaller proof sizes as well. Alternatively, such an instantiation based on post-quantum hard problems that are more expressive than generic hash functions could inoculate the resulting proof systems against quantum attacks while providing smaller proofs than achievable through the use of generic hash functions.

The cryptographic hardness assumptions underlying the security of SNARK constructions are rarely falsifiable, and this seems to be a by-product of the cryptographic compilation process. In pairing based constructions the unfalsifiable hardness assumption is inherently algebraic. In contrast, the hardness assumption of groups of unknown order \emph{are} falsifiable; it is only the Fiat-Shamir heuristic for $\mathsf{polylog}$-many round protocols that is unfalsifiable because it is only provably secure in the random oracle model; a security reduction of such heuristics to standard assumptions is as yet unavailable.