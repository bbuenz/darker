
\subsubsection{Spartan}
\textsf{Spartan}~\cite{Spartan} transforms an arbitrary circuit satisfaction problem into a Polynomial IOP based on an arithmetization technique developed by Blumberg~\emph{et al.} \cite{EPRINT:BTVW14}, which improved on the classical techniques of Babai, Fortnow, and Lund~\cite{BFL}. Specifically, satisfiability of a 2-fan-in arithmetic circuit on $n$ gates can be transformed into the expression: 
\begin{equation}\label{eqn:hypercubesum}
\sum_{x, y, z \in \{0,1\}^{\log n}} G(x, y, z) = 0
\end{equation} 
for a multilinear polynomial $G$ on $3 \log n$ variables over $\FF$. 
Furthermore, $G$ decomposes into the form: 
$$G(x,y,z) = A(x,y,z) F(x) + B(x, y, z) F(y) + C(x, y, z) F(y) F(z)$$
where $A, B, C,$ and $F$ are all multilinear poylnomials. The polynomials $A, B, C$ are derived from the arithmetic circuit defining the relation $\mathcal{R}$ and are input-independent. $F$ is degree $1$ with $\log n$ variables and is derived from a particular $(x, w) \in \mathcal{R}$. For uniform circuits, the verifier can evaluate $A, B, C$ locally in $O(\log n )$ time. The LFKN sum-check protocol~\cite{FOCS:LFKN90} is applied in order to prove Expression~\ref{eqn:hypercubesum} in a $3\log n$ round Polynomial IOP, where the prover's oracle consist of $Z$ and the low-degree polynomials sent in the sumcheck. Since the extra low-degree polynomials are constant size they can be read entirely by the verifier in constant time rather than via oracle access, and hence we ignore them in the total oracle count. The main result in Spartan can be summarized in our framework as follows: 

\begin{theorem}[Setty19]
There exists a $3 \log n$ round Polynomial IOP for any NP relation $\mathcal{R}$ computed by a \textbf{uniform} circuit with arithmetic complexity $n$, which makes three queries to a $\log n$-variate degree 1 polynomial oracle.  
\end{theorem}

Applying our multivariate compiler to the \textsf{Spartan} Polynomial IOP we obtain an $O(\log n)$-round public-coin interactive argument of knowledge for uniform circuits of size $n$. In our multivariate scheme (Section~\ref{sec:multivariate}), the $\log n$-variate degree 1 polynomial is tranformed into a univariate polynomial of degree $n$. With only three queries overall, the communication is just $6 \log n$ group elements and $6 \log n$ field elements. 

\subsubsection{QAPs} 

QAPs can be expressed as linear PCPs~\cite{TCC:BCIOP13,C:BCGTV13}. We review here how to express QAPs as a one round public-coin $(1, n)$ algebraic IOP. (This captures the satisfiability of any circuit with multiplicative complexity $n$, which is first translated to a system of quadratic equations over degree $n$ polynomials). Each linear query is computed by a vector of degree $n$ univariate polynomials evaluated at a random point chosen by the public-coin verifier. '

For illustration, will use the language \emph{satisfiability of rank-1 quadratic equations} over $\FF$ described by Ben-Sasson~\emph{et al.}~\cite{C:BCGTV13}. An instance of this language is defined by length $m+1$ polynomial vectors $A(X)$, $B(X)$, $C(X)$ such that the $i$th components $A_i(X)$, $B_i(X)$, $C_i(X)$ are all degree $n-1$ polynomials over $\FF_p[X]$ for $i \in [0,m]$, and $A_m = B_m = C_m$ is the degree $n$ polynomial $Z(X)$ that vanishes on a specified set of $n$ points in $\FF_p$. There is a length $m-1$ witness vector $\mathbf{w}$ whose first $\ell$ components are equal to the instance $\mathbf{x} \in \FF^\ell$, and a degree $n$ ``quotient" polynomial $H(X)$, such that the following constraint equation is satisfied: 
\begin{equation} \label{eqn:R1CS} 
\begin{split}
[(1, \mathbf{w}, \delta_1)^\top A(X)][(1, \mathbf{w}, \delta_2)^\top B(X)] 
- (1, \mathbf{w}, \delta_3)^\top C(X) = H(X)Z(X) \\ 
\ and \ (1,\mathbf{w})^\top (1,X,...,X^{\ell}, \mathbf{0}^{m- \ell -1}) = (1,\mathbf{x})^\top (1, X,...,X^{\ell})
\end{split} 
\end{equation} 
%\alan{What do $\delta_0, \delta_1, \delta_2$ do? Also, I'm not sure the dimensions work out.}

The deltas (i.e. $\delta_1, \delta_2, \delta_3$) are used as randomization terms for HVZK. 

\paragraph{QAP algebraic linear PCP} Equation~\ref{eqn:R1CS} is turned into a set of linear queries by evaluating the polynomials at a random point in $\FF$. Satisfaction of the equation evaluated at a random point implies satisfaction of the polynomial equation with error at most $2n / |\FF|$ by Schwartz-Zippel. Translated to an algebraic IOP, the prover sends a proof oracle $\proofO_w$ containing the vector $(1, \mathbf{w}, \delta_1, \delta_2, \delta_3)$ as well as a proof oracle $\proofO_h$ containing the coefficient vector of $H(X)$. A common proof oracle $\proofO_z$ is jointly established containing the coefficient vector of $Z(X)$. 

The verifier chooses a random point $\alpha \in \FF$ and makes four queries to $\proofO_w$, computed by the polynomial vectors $A(X), B(X), C(X)$ and $D(X) = (1, X,...,X^\ell, \mathbf{0}^{m- \ell -1})$. The verifier makes one query each to $\proofO_h$ and $\proofO_z$, which is the evaluation of $H(\alpha)$ and $Z(\alpha)$ respectively. The verifier obtains query responses $y_a, y_b, y_c, y_d, y_h, y_z$ and checks that $y_a \cdot y_b - y_c = y_h y_z$ and $y_d = \langle (1, \mathbf{x}), D(\alpha) \rangle$. 

\paragraph{Compiling QAP to public-coin argument} 

Following the compilation in Theorem~\ref{thm:algebraicIOPcompiler} (Section~\ref{sec:algebraicIOP}), the R1CS algebraic linear PCP can be transformed into a $2$-round Polynomial IOP. For simplicity, assume $m+3 < n$, where $m-1$ is the length of the witness and $n$ is the multiplicative complexity of the circuit. The preprocessing establishes three bivariate degree $n$ polynomials (\emph{i.e.}, encoding $A(X), B(X), C(X)$) and two univariate degree $n$ polynomials (\emph{i.e.}, encoding $Z(X)$ and $D(X)$). In the 2-round online phase the prover sends a degree $n$ univariate oracle for the witness vector $(1, \mathbf{w}, \delta_1, \delta_2, \delta_3)$, a degree $n$ univariate oracle for $H(X)$, four degree $n$ univariate oracles encoding linear PCP queries, four degree $2n$ univariate oracles encoding polynomial products, and eight degree $2n$ univariate oracles for opening inner products. The total number of polynomial oracle evaluation queries is $3$ bivariate degree $n$, $8$ univariate degree $2n$, and $7$ univariate degree $n$.

\begin{theorem}[QAP Polynomial IOP]
There exists a $2$-round Polynomial IOP with preprocessing for any NP relation $\mathcal{R}$ (with multiplicative complexity $n$) that makes $7$ queries to univariate degree $n$ oracles, $8$ queries to univariate degree $2n$ oracles, and $3$ queries to bivariate degree $n$ oracles.  
\end{theorem}
 
While theoretically intriguing, compiling the QAP-based IOP with our polynomial commitments of Section~\ref{sec:protocol} is less practical than compiling the \textsf{Sonic} IOP. While the R1CS Polynomial IOP has only $15$ univariate queries (compared to \pro{Sonic}'s $27$ queries to polynomials of approximately the same degree), the $3$ bivariate polynomial oracles take quadratic time to preprocess and open. Unfortunately, our polynomial commitment scheme does not take advantage of the sparsity of these bivariate polynomials. Furthermore, ignoring prover time complexity, the size of the bivariate $\eval$ proofs are twice as large as univariate $\eval$ proofs so the number of queries is effectively equivalent to $21$ univariate degree $n$ queries. 