We additionally use two more assumptions, however both of them reduce to the Strong RSA and the Adaptive Root Assumptions.

The first assumption states that computing the order for \emph{any} element is hard. It reduces to the Adaptive Root Assumption. Interestingly, it doesn't necessarily hold for all candidate groups of unknown order as we explain below. In particular it is important to exclude elements of known order such as $-1$ from the candidate unknown order group $\ZZ_n$.

\begin{assumption}[Order Assumption]
\label{assum:order}
	The Order Assumption holds for $\ggen$ if for any efficient adversary $\adv$:
\[        
                \Pr\left[\gr{w}\neq 1 \wedge \gr{w}^{\alpha}= 1: 
                \begin{array}{l} 
                      \GG \sample \ggen(\lambda) \\ 
                      (\gr{w},\alpha) \sample \adv(\GG) \\
                      \text{where } |\alpha|<2^{\poly{}}\in \ZZ\\
                      \text{and } \gr{w}\in \GG
                \end{array} 
        \right] \leq \negl \enspace .
\]
\end{assumption}
\begin{lemma}
\label{lem:ordertoadaptive}
	The Adaptive Root Assumption implies the Order Assumption.
\end{lemma}
\begin{proof}
	We show that given an adversary $\adv_{\textsf{Ord}}$ that breaks the Order Assumption we can construct with overwhelming probability $\adv_{\textsf{AR}}$ that breaks the Adaptive Root Assumption. We run $\adv_{\textsf{Ord}}$ to get a $\gr{w}\neq 1\in \GG$ and $\alpha \in \ZZ$ such that $\gr{w}^{\alpha}=1$. To construct $\adv_{\textsf{AR}}$, $\adv_{\textsf{AR},0}$ outputs $(\gr{w},\alpha)$. The challenger generates a random challenge $\ell$. If $\gcd(\ell,\alpha)=1$ then $\adv_{\textsf{AR},1}$ can compute $\beta\gets \ell^{-1} \bmod \alpha$ and output $\gr{u}\gets\gr{w}^{\beta}$. By construction $\gr{u}^{\ell}=\gr{w}$. The probability that $\gcd(\ell,\alpha)=1$ is overwhelming because $\gcd(\ell,\alpha)\neq 1 \implies \ell \not\vert \alpha$. This happens with negligible probability as $\ell$ is picked from a set of $2^\lambda$ primes and at most $\poly$ distinct primes can divide $\alpha$.
	\end{proof}
	
	
We also define the Fractional Root Assumption, which states that for random group elements $\gr{g}$ it is hard to find a tuple $(\gr{u}\in \GG,\alpha\in \ZZ,\beta\in \ZZ)$ such that $\gr{u}^{\beta}=\gr{g}^{\alpha}$. 
We say that $(\gr{u},\alpha,\beta)$ is a \emph{fractional root} of $\gr{g}$.
%unless $\frac{\alpha}{\beta}$ is an dyadic rational, \emph{i.e.}, a rational whose denominator is a power of $2$
%In RSA groups the assumption is also conjectured to hold if $\frac{\alpha}{\beta}$ is restricted to be an integer. 
Shoup\cite{CCS:CraSho99} showed that for the unknown order group of quadratic residues in $\ZZ_n$, where $n$ is the composite of two strong primes, that the Fractional Root Assumption reduces to just the Strong RSA Assumption.

\begin{assumption}[$r$-Fractional Root Assumption]
\label{assum:fracroot}
The \defn{$r$-Fractional Root Assumption} holds for $\ggen$ for any efficient adversary $\adv$:
\[        
                \Pr\left[\gr{u}^\beta = \gr{g}^{\alpha} \wedge \frac{\beta}{\gcd(\alpha,\beta)}\neq r^k,  k \in \NN   : 
                \begin{array}{l} 
                      \GG \sample \ggen(\lambda) \\ 
                      \gr{g} \sample \GG \\
                      (\alpha, \beta, \gr{u}) \sample \adv(\GG, \gr{g}) \\
                      \quad \textnormal{where} \, |\alpha|<2^{\poly}, \\
                      \quad |\beta|<2^{\poly} \in \ZZ, \\
                      \quad \textnormal{and} \, \gr{u} \in \GG 
                \end{array} 
        \right] \leq \negl \enspace .
\]
\end{assumption}
We say $(\alpha,\beta,\gr{u})$ is a non power of $r$ fractional root of $\gr{g}$.

The Fractional Root Assumption reduces to the Order Assumption (and therefore to the Adaptive Root Assumption) and the Strong RSA Assumption.
\begin{lemma}
\label{lem:strongtofractional}
	The Adaptive Root Assumption and the $r$-Strong RSA Assumption imply the $r$-Fractional Root Assumption 
	%if groups generated by $\ggen$ have order coprime with $r$ and there exists a PPT algorithm for taking $r$th roots in these groups.
\end{lemma}
\begin{proof}
	Given an adversary $\adv_{\textsf{FR}}$ that succeeds in breaking the Fractional Root Assumption for $\ggen$ we can construct either an adversary $\adv_{RSA}$ for the Strong RSA Assumption or an adversary $\adv_{\textsf{Ord}}$ that breaks the Order Assumption for $\ggen$. As shown in Lemma \ref{lem:ordertoadaptive} the Order Assumption reduces to the Adaptive Root Assumption with overwhelming probability. 
	We first generate a group of unknown order $\GG \sample \ggen(\lambda)$.
	Then we sample $\gr{g}\sample \GG$ as done in the strong \textsf{RSA} security definition.
	
	We now run the $\adv_{\textsf{FR}}$ on input $\GG$ and $\gr{g}$ to generate a tuple $(\alpha,\beta,\gr{u})$ such that $\gr{u}^{\beta}=\gr{g}^{\alpha}$. Let $\gamma=\gcd(\alpha,\beta)$ and $\alpha'=\frac{\alpha}{\gamma}\in \ZZ$ and  $\beta'=\frac{\beta}{\gamma}\in \ZZ$. Now either $\gr{g}^{\alpha'}=\gr{u}^{\beta'}$ or $\gr{g}^{\alpha'}/\gr{u}^{\beta'}$ is a non trivial element of order $\gamma$ which would directly break the Order Assumption. In that case we constructed $\adv_{\textsf{Ord}}$ that outputs $(\gr{g}^{\alpha'}/\gr{u}^{\beta'},\gamma)$.
	
	Now assume otherwise, i.e. $\gr{g}^{\alpha'}=\gr{u}^{\beta'}$. By construction $\gcd(\alpha',\beta')=1$ and we can efficiently compute integers $a,b$ such that $a \alpha'+b \beta'=1$. By assumption on $\adv_{\textsf{FR}}$ $\beta'$ is not $r^k$. Now let $\gr{w}\gets \gr{u}^{a}\gr{g}^{b}$. Note that $\gr{w}^{\alpha'\beta'}=\gr{g}^{\alpha'}$. So either $\gr{w}^{\beta'}=\gr{g}$ or $\gr{w}^{\beta'}/\gr{g}$ is a non-trivial element of order $\alpha'$. The first case breaks the Strong RSA Assumption, as we can construct $\adv_{\textsf{RSA}}$ that outputs $(\gr{w},\beta)$, and the second breaks the Order Assumption.
\end{proof}