We present several ideas for optimizing the performance of the $\pro{Eval}$ protocol.

\paragraph{Precomputation.} The prover has to compute powers of $\gr{g}$ as large as $q^d$. While this can be done in linear time, this expense can be shifted to a preprocessing phase in which all elements $\gr{g}^{q^i}, i \in \{1, \ldots, d_{\it max}\}$ are computed. Since for coefficient $|f_i|\leq -\frac{p-1}{2}$ this allows the computation of $\gr{g}^{f(q)}$ in $O(\lambda d)$ group operations as opposed to $O(\lambda \log(d) d)$.
In addition to reducing the prover's workload, this optimization enables parallelizing it. The computation of the $\textsf{PoE}$ proofs can simiarly be parallelized. The prover can express each $Q$ as a power of $\gr{g}$ which enables pre-computation of powers of $\gr{g}$ and parallelism as described by Boneh~\emph{et al.}~\cite{C:BonBunFis19}.
%The elements $\gr{g}^{q^i}$ can themselves be accompanied by non-interactive $\mathsf{PoE}$s to establish their correct computation.

The pre-computation also enables the use of multi-exponentiation techniques~\cite{pippenger1980evaluation}. Boneh~\emph{et al.}~\cite{C:BonBunFis19} and Wesolowski~\cite{EC:Wesolowski19} showed how to use these techniques to reduce the complexity of the $\textsf{PoE}$ prover. The largest $\textsf{PoE}$ exponent $q^{\frac{d+1}{2}}$ has $O(\lambda d \log(d))$ bits. Multi-exponentiation can therefore reduce the prover work to $O(\lambda d)$ instead of $O(\lambda d \log(d))$.

%\paragraph{Early termination.} The protocol specifies the recursion ends when $d=0$, but the communication cost might be reduced if it terminates earlier. This reduction holds when the size of the fewer group elements $\gr{C}_L$ and $\gr{C}_R$ outweigh the size of the larger polynomial $f(X)$ instead of the constant $f$.

%\paragraph{Fiat-Shamir.} All the challenges of the verifier are public coin and as a result the protocol can be made non-interactive in the random oracle model with the Fiat-Shamir heuristic~\cite{C:FiaSha86}. This technique replaces each message of the verifier with the hash of all previous protocol messages, lifted to the appropriate domain. For the \textsf{PoE}s, it is beneficial to reuse the same $\ell$ across all \textsf{PoE}s and to compute this prime as the hash of the entire transcript after (dropping the $\ell$s and) replacing every instance of $\gr{Q}$ by its matching $\gr{C}_R^{q^{d'+1}}$ counterpart. This optimization requires that $\ell$ be transmitted as part of the proof so that the verifier can infer the $\gr{C}_R^{q^{d'+1}}$ and $\gr{C}_L$, and only after this inference can the verifier check that $\ell$ was computed correctly. The concrete benefit of this optimization is the reduced work for the verifier: previously he had to perform $\lceil\log(d+1)\rceil$ exponentiations of $q \bmod \ell$ to the power $d'+1$, whereas now he can do this task once and record the intermediate results.

\paragraph{Two group elements per round.} In each round the verifier has a value $\gr{C}$ and receives $\gr{C}_L$ and $\gr{C}_R$ such that $\gr{C}_L\gr{C}_R^{q^{d'+1}}=\gr{C}$. This is redundant. It suffices that the verifier sends $\gr{C}_R$. The verifier could now compute $\gr{C}_L\gets \gr{C} \cdot \gr{C}_R^{-q^{d'+1}}$ in principle, but this is expensive in practice. Instead, the verifier can infer $\gr{C}_R^{(q^{d'+1})}$ from the \textsf{PoE}: the prover's message is $\gr{Q}$ and from this value and from $\ell \in \ZZ$ and $r \in \ZZ$ the verifier computes $\gr{C}_R^{q^{d'+1}}\gets \gr{Q}^{\ell} \gr{C}_R^{r}$ as well as $\gr{C}_L \gets \gr{C}/\gr{C}_R^{q^{d'+1}}$. The security of $\textsf{PoE}$ does not require that $\gr{C}_R^{q^{d'+1}}$ be sent before the challenge $\ell$ as it is uniquely defined by $\gr{C}_R$ and $q^{d'+1}$.
The same optimization can be applied to the non-interactive variant of the protocol. 

Similarly the verifier can infer $y_L$ as $y_L\gets y-z^{d'+1} y_R$. This reduces the communication to two group elements per round and 1 field element. Additionally the prover sends $f$ which has roughly the size of $\log(d+1)$ field elements, which increases the total communication to roughly $2\log(d)$ elements in $\GG$ and $2\log(d)$ elements in $\ZZ_p$. 

%When the $\mathsf{PoE}$s are made non-interactive, the prover can get away with producing only two group elements instead of three. With a naïve application of the Fiat-Shamir heuristic, the $\mathsf{PoE}$ proof consists of $(\gr{C}_R, \gr{C}_R^\star, \gr{Q})$ where $\gr{Q}$ is determined by $\ell$, which in turn is determined by hashing all previous protocol messages: $\ell \gets \mathsf{H}(\cdot \Vert \gr{C}_R \Vert \gr{C}_R^\star)$. The optimization sends $(\gr{C}_R, \gr{Q}, \ell)$ instead. The verifier can infer $\gr{C}_R^\star = \gr{C}_R^{(q^{d'+1} \bmod \ell)}$ and then test $\mathsf{H}(\cdots \Vert \gr{C}_R \Vert \gr{C}_R^\star) \stackrel{?}{=} \ell$. This optimization is particularly compatible with the previous batching of $\mathsf{PoE}$s optimization, because while there is a unique $\gr{Q}$ for each round, there need only be one $\ell$ for the entire $\eval$ protocol.

\paragraph{Evaluation at multiple points}
The protocol and the security proof extend naturally to the evaluation in a vector of points $\boldsymbol{z}$ resulting in a vector of values $\boldsymbol{y}$, where both are members of $\mathbb{Z}_p^k$. The prover still sends $\gr{C}_L\in \GG$ and $\gr{C}_R\in \GG$ in each round and additionally $\boldsymbol{y}_L,\boldsymbol{y}_R \in \ZZ^k_p$. In the final round the prover only sends a single integer $f$ such that $\gr{g}^{f}=\gr{C}$ and $f \bmod p=y$.

This is significantly more efficient than independent executions of the protocol as the encoding of group elements is usually much larger than the encoding of elements in $\ZZ_p$. Using the optimization above, the marginal cost with respect to $k$ of the protocol is a single element in $\ZZ_p$. If $\lambda=\lceil\log_2(p)\rceil$ is $120$, then this means evaluating the polynomial at an additional point increases the proof size by only $15\log(d+1)$ bytes.

\paragraph{Joining $\mathsf{Eval}$s.} 
In many applications such as compiling polynomial IOPs to SNARKs (see Section~\ref{sec:polyiop}) multiple polynomial commitments need to be evaluated at the same point $z$. 
This can be done efficiently by taking a random linear combination of the polynomials and evaluating that combination at $z$. The prover simply sends the evaluations of the individual polynomials and then a single evaluation proof for the combined polynomials. The communication cost for evaluating $m$ polynomials at $1$ point is still linear in $m$ but only because the evaluation of each polynomial at the point is being sent. The size of the eval proof, however, is independent of $m$. 
Taking a random linear combination does increase the bound on $q$ slightly, as shown in Theorem~\ref{thm:joined} which is presented below.

\[
\mathcal{R_\textsf{JE}}(\params) = \left\lbrace
\langle (\gr{C}_1,\gr{C}_2, z, y_1,y_2,d), (f_1(X), f_2(X)) \rangle
: \\
\begin{array}{l} 
\gr{C}_1, \gr{C}_2 \in \GG \\
z, y_1, y_2 \in \mathbb{Z}_p \\
f_1(X), f_2(X) \in \ZZ(b) \\
(\gr{C}_1,z,y_1,d) \in \mathcal{R_\textsf{Eval}}(\params) \\
(\gr{C}_2,z,y_2,d) \in \mathcal{R_\textsf{Eval}}(\params)
\end{array}
\right\rbrace
\]





\begin{mdframed}
	$\pro{JoinedEval}(\crs, \gr{C}_1, \gr{C}_2, z, y_1, y_2, d; f_1(X),f_2(X)) :$ \pccomment{$f_1(X), f_2(X) \in \ZZ(\frac{p-1}{2})[X]$} \\
%	Statement: $f_1(z)=y_1\bmod p \wedge f_2(z)=y_2\bmod p \wedge \gr{g}^{f_1(q)}=\gr{C}_1$ and $\gr{g}^{f_2(q)}=\gr{C}_2$
Statement: $(\crs,\gr{C}_1,\gr{C}_2,z,y_1,y_2,b,d)\in \mathcal{R}_{\pro{JE}}$
			\begin{enumerate}[nolistsep]
        \item \verifier samples $\alpha \sample [-\frac{p-1}{2},\frac{p-1}{2}]$ and sends it to \prover
			\item \prover and \verifier compute $\gr{C}'\gets \gr{C}_1^{\alpha}\gr{C}_2$ and $y'\gets \alpha \cdot y_1 +y_2 \bmod p$
			\item \prover computes $f'(X)\gets \alpha f_1(X) +f_2(X)$
			\item \prover and \verifier run $\pro{EvalBounded}(\params,\gr{C}',z,y',d,\frac{p^2-1}{4};f(X))$
		    \end{enumerate}
\end{mdframed}

\begin{theorem}
\label{thm:joined}
The protocol $\pro{JoinedEval}$ is an interactive argument for the relation $\mathcal{R}_{\pro{JE}}$ and has perfect completeness and witness extended emulation if the $r$-Strong RSA and Order Assumption hold for $\ggen$ and if $q>(p-1)(\frac{p^2-1}{2})^{\lceil \log_2(d+1)\rceil+1}$
\end{theorem}
\begin{proof}
    Correctness is immediate and witness extended emulation requires a single application of Lemma~\ref{lem:intrandomcombine} which leads to an updated bound on $q$. In general $q$ needs to be $\frac{p^2-1}{2}$ larger for any random linear combination that is taken with $\alpha \in [-\frac{p-1}{2},\frac{p-1}{2}]$. For class groups, Lemma~\ref{lem:dyadiccombining} is used and the lower bound on $q$ grows by a factor of $(p-1)^2\frac{p+1}{2}\leq p^3$.
\end{proof}

We can additionally combine this optimization with the previous optimization of evaluating a single polynomial at different points. This allows us to evaluate $m$ polynomials at $k$ points with very little overhead. 
The prover groups the polynomials by evaluation points and first takes linear combinations of the polynomials with the same evaluation point and computes $y_1$ to $y_k$ using the same linear combinations. Then it takes another combination of the joined polynomials. In each round of the $\eval$ protocol the prover sends $y_{L,1}$ through $y_{L,k}$, i.e. one field element per evaluation point and computes $y_{R,1}$ through $y_{R,k}$. In the final step the prover sends $f$ and the verifier can check whether the final $y$ values are all equal to $f\bmod p$.
 This enables an $\eval$ proof of $m$, degree $d$ polynomials at $k$ points using only $2\log_2(d+1)$ group elements and $(1+k)\log_2(d+1)$ field elements.
 
%to a non-falsiable  but we do advise against this option. The reason for this advice is two-fold. First, in a SNARK compilation process where the commitment scheme is used as a primitive, zero-knowledge is typically added at the end anyway and it suffices to have a primitive that does not provide zero-knowledge on it own.
%Second, the natural way to add zero-knowledge is via a Pedersen-like commitment function $f(X) \mapsto \gr{g}^{f(q)}\gr{h}^r$, where $\gr{g}, \gr{h}$ are both random group elements and where $r \sample [0; 2^\lambda]$ is a randomizer with enough entropy.
%The scheme's homomorphism translates naturally to $\eval$ protocol and the only change the protocol needs is in the last step where $\gr{g}^f\gr{h}^{r'}$ must be opened to $f \bmod p$ and \emph{should not leak any other information}.
%The natural tool to achieve this is a zk-$\mathsf{PoKE}$ (zero-knowledge proof-of-knowledge of exponent)~\cite[\S A.4]{C:BonBunFis19} but unfortunately it is only provably secure in the generic group model.
\paragraph{Evaluating the polynomial over multiple fields}
The polynomial commitment scheme is highly flexible. For example it does not specify a prime field $\ZZ_p$ or a degree $d$ in the setup. It instead commits to an integer polynomial with bounded coefficients. That integer polynomial can be evaluated modulo arbitrary primes which are exponential in the security parameter $\lambda$ as the soundness error is proportional to its inverse.
Note that $q$ also needs large enough such that the scheme is secure for the given prime $p$ and degree $d$ (see Theorem \ref{thm:polycommitsecurity}). The second condition, however, can be relaxed. A careful analysis shows that the challenges $\alpha$ just need to be sampled from an exponential space, e.g. $[-2^{\lambda},2^{\lambda}]$. So as long as $q>2^{\lambda 2\lceil \log_2(d+1)\rceil+1}$ for RSA groups or  $q>2^{\lambda 3\lceil \log_2(d+1)\rceil+1}$ for Class groups one can evaluate degree $d$ polynomial with coefficients bounded by $2^\lambda$ over any prime field.

Additionally the proof elements $\gr{C}_L$, $\gr{C}_R \in \GG$ are independent from the field over which the polynomial is evaluated. This means that it is possible to evaluate a committed polynomial $f(X) \in \ZZ(b)$ over two separate fields $\ZZ_{p}$ and $\ZZ_{p'}$ in parallel using only $2\log(d+1)$ group elements. 

%This property can be used to efficiently evaluate the polynomial modulo a large integer $m$ by choosing multiple $\lambda$ bit primes $p_1,\dots p_k$ such that $\prod_{i=1}^k p_i\geq m$ and using the Chinese Remainder Theorem to simulate the evaluation modulo $m$.


