
We can extend our polynomial commitment scheme to multivariate polynomials. The idea is simply to use higher degrees of $q$ to encode the next indeterminate. The protocol is linear in the number of variables and logarithmic in the total degree of the polynomial. For simplicity we only present a protocol for $\mu$-variate polynomials where the degree in each variable is $d$. The protocol extends naturally to different degrees per variable.

\paragraph{Encoding}
Let $q_i=q^{(d+1)^i}$ then $\hat{f}(q_1,\dots,q_\mu)\in \ZZ$ is an encoding of the multivariate polynomial $f(X_1,\dots,X_\mu)$ with maximum degree $d$. We use $\dec_{Multi}(f(q),\mu,d)$ to denote the decoding of an $\mu$-variate polynomial with degree exactly $d$ in each variable. The decoding algorithm simply uses the univariate decoding algorithm described in Section \ref{sec:encoding} to decode a univariate polynomial $\hat{h}(X)$ of degree $(d+1)^{\mu}-1$. From this it is easy to compute the $\mu$-variate polynomial $f(X_1,\dots,X_\mu)$ of degree $d$ by assigning the coefficients of $X^{(d+1)^i}$ to the variable $X_i$.
\paragraph{Protocols}
 Using this encoding we can naturally derive the multivariate commitment scheme and $\eval$ protocol. The $\eval$ protocol computes the univariate polynomials $f(q_1,\dots,q_{\mu-1},X_\mu)$ and then uses the univariate eval protocol to reduce the claim from a claim about an $\mu$-variate polynomial to one about an $(\mu-1)$-variate one. At the final step the prover opens the now constant polynomial and the verifier can check the claim. For example, the protocol would reduce a bivariate (say $X$ and $Y$) cubic polynomial to a univariate one (in $Y$) in two rounds of interaction and then reduce the degree of $Y$ using another two rounds.
 
 \begin{mdframed}[userdefinedwidth=\textwidth]
\begin{minipage}{\textwidth}
	\begin{flushleft}
	$\pro{MultiSetup}(1^\secpar):$
		\begin{enumerate}[nolistsep]
			\item $ \GG \sample \ggen(\secpar)$
			\item $ \gr{g} \sample \GG$
			%\item $q \gets 2^k$ such that $q > (d+1) \cdot 2\cdot p^{\log_2(d+1)+1} $
			%\item Pick a prime $p\in \NN$ such that $\lceil\log_2(p)\rceil=\lambda$.
			%\item Pick a sufficiently large and odd $q\in \NN$ \pccomment{$q=O_\lambda(p^{\mu \cdot \log(d)})$}
			\item $\pcreturn \params = (\secpar,\GG,\gr{g})$
		\end{enumerate}
	$\pro{MultiCommit}(\params;f(X_1,\dots,X_\mu) \in \ZZ(\frac{p-1}{2})[X_1, \ldots, X_\mu]\subset \ZZ[X_1, \ldots, X_\mu]):$ 		\begin{enumerate}[nolistsep]
			\item $d\gets \deg(f)$\pccomment{For simplicity assume $f(X_1,\dots,X_n)$ has degree $d$ in each variable}
			\item $q_i\gets q^{(d+1)^{i-1}}$ for each $i\in [\mu]$
			\item $\gr{C} \gets \gr{g}^{f(q_1,\dots,q_\mu)}$
			\item $\pcreturn (\gr{C};f(X_1,\dots,X_\mu))$
		\end{enumerate}
			\end{flushleft}
\end{minipage}
\end{mdframed}
 
 \begin{mdframed}
\begin{minipage}{\textwidth}
			$\pro{MultiEval}(\params, \gr{C}\in \GG, \boldsymbol{z}\in \ZZ^\mu_p,y \in \ZZ_p, d,\mu,b \in \NN; f(X_1,\dots,X_\mu)\in \ZZ(b)[X_1, \ldots, X_\mu]) :$
			\begin{enumerate}[nolistsep]
			\item \pcif{$\mu=1$} 
			\item \pcind[1] \prover and \verifier run $\pro{EvalBounded}(\params,\gr{C},z_1,y,d,b,x;f(X_1))$ 
			\item \pcelse
			\item \pcind[1] Let $\hat{f}(X_\mu)\gets f(q_1,\dots,q_{\mu-1},X_\mu)$
			\item \pcind[1] Let $\crs_\mu \gets \{\lambda,\GG,\gr{g},p,q_\mu\}$
			\item \pcind[1] \prover and \verifier run the univariate $\pro{EvalBounded}(\params_\mu,\gr{C},z_\mu,y,d,q_\mu;\hat{f}(X))$
			\item \pcind[2] \textbf{except:} when $d=0$, $f$ is not sent; instead the protocol returns its input at this point, \emph{i.e.}, $(\gr{C}',y',b')$ along with the prover's witness $f'(X_1,\dots,X_{\mu-1})=\dec_{Multi}(f,\mu-1,d)$ (Lines~\ref{line:basestart}-\ref{line:baseend} of $\pro{EvalBounded}$). 
			\item \pcind[1]$\boldsymbol{z}'\gets (z_1,\dots,z_{\mu-1})\in \ZZ_p^{\mu-1}$
			\item \pcind[1]\prover and \verifier run $\pro{MultiEval}(\crs,C',\boldsymbol{z}',y',d,\mu-1,b';f')$
		    \end{enumerate}
      \end{minipage}
\end{mdframed}
We only proof security under the strong RSA assumption. The security proof, however, directly extends to groups where taking square roots is easy under the $2$-Strong-RSA Assumption. In that case $q>p^{3\mu \log_2(d+1)+1}$ suffices.
\begin{theorem}[Multivariate Eval]
	The polynomial commitment scheme for multi-variate polynomials consisting of protocols $(\pro{MultiSetup},\pro{MultiCommit},\pro{MultiEval})$ has perfect correctness and witness extended emulation if the Adaptive Root Assumption and the Strong RSA Assumption hold for $\ggen$ for $\mu$-variate polynomials of degree $d$ and if $d^\mu=\poly$ if $q>(p-1)^{\mu\cdot \lceil \log_2(d+1)\rceil+1}(\frac{p+1}{2})^{\mu \cdot \lceil \log_2(d+1)\rceil}<p^{2 \mu \log_2(d+1)+1}$.
\end{theorem}
\begin{proof}
	Perfect correctness follows from the correctness of the univariate scheme and the fact that the bounds $b$ are appropriately updated and remain less than $q/2$.
	
	To show witness extended emulation we use the forking lemma (Lemma \ref{lemma:GFL}) and build a polynomial time extractor algorithm $\mathcal{X}_{\pro{MultiEval}}$ that given a binary tree of transcripts of depth $\mu \cdot\lceil\log(d+1)\rceil$, extracts a witness. Each node corresponds to a different challenge $\alpha$ as described in the forking lemma. The tree consists of $\mu^{\lceil\log(d+1)\rceil}=\poly$ transcripts. 
We can therefore use Lemma~\ref{lem:poe} to replace all \textsf{PoE} executions with direct verification checks that $\gr{C}_L\gr{C}_R^{q^{d'+1}}=\gr{C}$. The lemma focuses on the univariate \pro{Eval} protocol but works identically for the multivariate protocol. The probability that an adversary can create any accepting transcript for which the $\textsf{PoE}$ can't be replaced by a direct check is negligible under the Adaptive Root Assumption.

	We now use the extractor $\mathcal{X}_{\pro{Eval'}}$ described in the proof of Theorem~\ref{thm:polycommitsecurity} to build $\mathcal{X}_{\pro{MultiEval}}$. 
	
	$\mathcal{X}_{\pro{MultiEval}}$ takes the entire transcript tree and invokes the univariate extractor on input $\gr{C}$ to extract either a degree $(d+1)^\mu-1$ polynomial univariate polynomial $\hat{f}(X)$ or an element of known order or a fractional root of $\gr{g}$. The extracted polynomial has coefficients bounded by $b=\frac{p-1}{2}(\frac{p^2-1}{2})^{\mu
	\cdot \lceil\log_2(d+1)\rceil}$. Since $q>2b$ decoding and extraction succeeds. Further using $f(X_1,\dots,X_n)\gets\dec_{Multi}(\hat{f}(q),\mu,d)$ we can extract an $\mu$-variate polynomial of degree $d$ in each variable. 
It remains to show that $f(z_1,\dots,z_\mu)=y$. Lemma~\ref{lem:intrandomcombine}, which proves the soundness of each individual extraction step, does not require the evaluation point $z$ to be the same in each round. This shows that the extracted evaluation of $f(X_1,\dots,X_\mu)$ follows exactly the honest prover's evaluation.
For the $y=\sum_{i=0}^{(d+1)^\mu-1 } \hat{f}_{i} (\prod_{j=1}^\mu  z_j^{\lceil i/(d+1)^j\rceil \bmod d+1 }) $, where $\hat{f}_i$ are the coefficients of the encoded univariate degree $(d+1)^\mu-1$. The decoding sets $\hat{f}_i$ to be the coefficient of $\prod_{j=1}^\mu  X_j^{\lceil i/(d+1)^j\rceil \bmod d+1 }$ which shows that $y=f(z_1,\dots,z_\mu)$.
Under the Strong RSA Assumption and the Adaptive Root Assumption the probability that the polynomial time adversary can produce an element of known order or a  fractional root of $\gr{g}$ is negligible. This shows that the multivariate polynomial commitment scheme has witness extended emulation.
\end{proof}
